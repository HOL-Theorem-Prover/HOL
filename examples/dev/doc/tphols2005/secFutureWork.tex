
The compiler described here has only just been completed and is still
being tested (e.g. by simulating circuits and viewing the resulting
waveforms). In the immediate future we plan to ruggedise and debug the
existing code.  In parallel with this we are already undertaking some
subtantial case studies, as described in
Section~\ref{secCaseStudy}. We plan to study the hardware we compile
and to work on making the compiler produce better designs.

At present all data-refinement (e.g. from numbers or enumerated types
to words) must be done manually, by proof in higher order logic. The
HOL4 system has some `boolification' facilities that automatically
translate higher level data-types into bit-strings, and we hope to
integrate these into the compiler.

We want to investigate using the compiler to generate test-bench
monitors that can run in parallel simulation with designs that are not
correct by construction.  Thus our hardware can act as a ``golden''
reference against which to test other implementations.

The work described here is part of a bigger project to create
hardware/software combinations by proof.  We hope to investigate the
option of creating software for ARM processors and linking it to
hardware created by our compiler (probably packaged as an ARM
co-processor). Our emphasis is likely to be on cryptographic hardware
and software, because there is a clear need for high assurance of
correct implementation in this domain.
