\documentclass{llncs}
\usepackage{epic,eepic,subfigure,alltt}
%-------------------------------------------------------------------------------
% File: notation.tex
%-------------------------------------------------------------------------------

%-------------------------------------------------------------------------------
% BNF
%-------------------------------------------------------------------------------
\newcommand{\bnfSC}[1]{\ensuremath{\mathit{\langle#1\rangle}}} % Syntax construct
\newcommand{\bnfDef}{\ensuremath{\mathit{::=}}} % ::=
\newcommand{\bnfList}[1]{\ensuremath{\{#1\}}}
\newcommand{\bnfOR}[1]{\ensuremath{|\ #1}}
\newcommand{\bnfOpt}[1]{\ensuremath{[#1]}}

%-------------------------------------------------------------------------------
% Mathematics
%-------------------------------------------------------------------------------
\newcommand{\mE}[1]{\ensuremath{\mathit{#1}}} % Math environment
\newcommand{\argm}[1]{\ensuremath{\vec{#1}}} % list of arguments or parameters
\newcommand{\VAR}[1]{\ensuremath{\mathit{#1}}} % variable
\newcommand{\OR}{\ensuremath{\vee}} % logical OR
\newcommand{\AND}{\ensuremath{\wedge}} % logical AND
\newcommand{\NOT}{\ensuremath{\neg}} % logical NOT
%\newcommand{\IF}[3]{\ensuremath{#1\, \rightarrow\, #2\, \mid\ #3}} % logical IF
\newcommand{\IF}[3]{\ensuremath{\texttt{if}~#1~\texttt{then}~#2~\texttt{else}~#3}} % logical IF
\newcommand{\TURNST}{\ensuremath{\vdash}} % turnstile
\newcommand{\IMP}{\ensuremath{\Rightarrow}} % implies
\newcommand{\FIMP}{\ensuremath{\Longrightarrow}} % implies
\newcommand{\NIMP}{\ensuremath{\not\supset}} % doesn't imply
\newcommand{\DEF}[1]{\ensuremath{\mathsf{#1}}} % definition
\newcommand{\sVAR}[2]{\ensuremath{\mE{#1}_{#2}}} % variable with subscript
\newcommand{\LAMBDA}[2]{\ensuremath{\lambda\, #1.\ #2}} % lambda abstraction
\newcommand{\COMP}[2]{\ensuremath{#1 \circ #2}} % function compositon
\newcommand{\MULT}{\ensuremath{\cdot}} % multiplication

%\newtheorem{example}{Example}

% added by MJCG
\newcommand{\TY}[1]{\ensuremath{\sl{#1}}} % type
\newcommand{\TT}[1]{\ensuremath{\texttt{#1}}} % keyword


%--------------------------------------------------------------------------
\title{A Proof-Producing Hardware Compiler for a Subset of Higher Order Logic}

\author{}
\institute{}

\begin{document}
\maketitle

\vspace*{-8mm}

\begin{center}
\begin{tabular}{ccc}
{\bf Mike Gordon, Juliano Iyoda} &\hspace*{5mm}& {\bf Scott Owens, Konrad Slind}\\
University of Cambridge          &\hspace*{5mm}& University of Utah\\
Computer Laboratory              &\hspace*{5mm}& School of Computing\\
William Gates Building           &\hspace*{5mm}& 50 South Central Campus Drive\\
JJ Thomson Avenue                &\hspace*{5mm}& Salt Lake City\\
Cambridge CB3 0FD, UK            &\hspace*{5mm}& Utah UT84112, USA
\end{tabular}


\vspace*{2mm}

({\it{authors listed in alphabetical order\/}})
\end{center}

\vspace*{-5mm}

%--------------------------------------------------------------------------
\thispagestyle{empty}

\begin{abstract}
A special purpose theorem prover for creating hardware implementations
is described.  Implementations are created as theorems
$\vdash {\it{Imp}}\Rightarrow{\it{Spec}}$, where {\it{Imp}}
describes a circuit that implements a four-phase handshake computing
a function defined in higher order logic by {\it{Spec}}.  The compiler
goes through several phases, each refining the implementation to a
more concrete form, until a representation that can be output as
synthesisable Verilog is deduced.  The compilation to hardware is
transparent and programmable. Users can tune the generated hardware
either by deductively pre-optimising the higher order logic
specifications, or by adding optimisations to the theorem proving
script that performs compilation.  The hardware designs we construct
may not have the highest performance, but we anticipate that having
guaranteed-correct implementations will interest some clients.  As a
case study we are working on producing components useful for building
cryptographic hardware.  This will result in a reference library of
``golden'' implementations.


\end{abstract}

%--------------------------------------------------------------------------
\section{Introduction}
\label{secIntroduction}
%--------------------------------------------------------------------------

We describe correct-by-construction compilation of
function definitions in higher order logic to hardware.  
Implementations are created by automated theorem proving in the form of theorems $\vdash
{\it{Imp}}\Rightarrow{\it{Spec}}$, where {\it{Imp}} is a term
describing a circuit that implements a four-phase handshake computing
a function, and the handshake is specified in higher order logic by
{\it{Spec}}.

The compiler is a specialised theorem prover
that goes through several phases, each refining the
implementation to a more concrete form, until a representation that
can be output as Verilog is deduced.

%We have in mind two uses of the work described here: generating
%implementations and generating testbench monitors.

Our goal is to generate synthesisable Verilog suitable for input to a
synthesis tool. The hardware designs we construct may not have high
performance, but we anticipate that having
guaranteed-correct implementations will interest some clients. For
example, a case study at Utah aims to produce implementations of
components useful for building cryptographic hardware.  This will
result in a reference library of ``golden'' implementations. 

The compilation to hardware is transparent and programmable. Users can
tune the generated hardware either by pre-optimising the higher order
logic specifications, or by adding optimisations to the theorem
proving script that performs compilation.

%The second use of our work is to automatically generate testbench
%infrastructure. The idea here is that one synthesises a
%guaranteed-correct representation of behaviour that can be run as a
%checker in parallel, say in a Verilog simulator, with a bespoke
%hand-crafted implementation. 
%There is an analogy with assertion based
%verification in the dynamic verification of hardware. There, one uses
%a tool like FoCs [add refs to FoCs and Charme paper] to turn
%specifications into checkers.  Traditional hardware specifications
%specifications are temporal logic properties, but if
%the specification were in HOL than our compiler could be used to make
%testbench monitors to test (rather than to be) real implementations.

In the next section we give a quick overview of the currently
implemented system using a simple (but unrealistic) example. We then
describe the specification of components in higher order logic that
underlies the work. Next we describe how function definitions in
higher order logic are translated to hardware via a sequence of steps,
ending with Verilog.  Some unfinished case studies that are in
progress are then discussed. Finally, we survey related work by
others, and outline our plans for the future. An appendix contains the
definitions of the circuit constructors that we use to build
implementations.



%--------------------------------------------------------------------------
\section{From Higher Order Logic to Verilog: an overview}
\label{secHOL2Verilog}
%--------------------------------------------------------------------------

Our system is implemented in HOL4, but the ideas could be
realised in other programmable higher order logic proof systems like
Isabelle, PVS, NuPrl and Coq. In such systems, functions are defined
recursively and then proof scripts are executed to prove properties of
the functions.  Proof scripts are programs that perform sequences of
deductions to create theorems.

The key idea of our approach to synthesis is to create theorems of the
form $\vdash {\it{Imp}}\Longrightarrow\texttt{DEV}~f$, where
{\it{Imp}} is a term describing a circuit, $\texttt{DEV}~f$ specifies
that function $f$ is computed by a four-phase handshake and the
operator $\Longrightarrow$ means that the implementation meets the
specification.  Several different representations of implementations
are created, with the lowest level currently corresponding to
synthesisable Verilog.

We illustrate the translation from higher order logic to hardware with a
simple toy example: the factorial function. This example is chosen for
clarity in illustrating our ideas and is not typical of the kind of
applications were are working on!  For a discussion of those
applications see Section~\ref{secRelatedWork}.

The factorial function \texttt{FACT} satisfies

\vspace*{-2mm}

{\baselineskip10pt\begin{verbatim}
 |- (FACT 0 = 1) /\ !n. FACT(SUC n) = SUC n * FACT n
\end{verbatim}}

\vspace*{-2mm}
We can only synthesise tail recursions, so the first
step is to define a tail-recursive function \texttt{FactIter} and show
that it can be used to compute \texttt{Fact}.

\vspace*{-2mm}

{\baselineskip10pt\begin{verbatim}
 |- FactIter(n,acc) = if n=0 then (n,acc) else FactIter(n-1,n*acc))
 |- FACT n = SND(FactIter(n,1))
\end{verbatim}}

\vspace*{-2mm}

\noindent In HOL4, \texttt{FactIter} would be defined by executing:

\vspace*{-2mm}

{\baselineskip10pt\begin{verbatim}
 Define 
  `FactIter(n,acc) = if n=0 then (n,acc) else FactIter(n-1,n*acc))`
\end{verbatim}}

\vspace*{-2mm}

\noindent to synthesise a hardware implementation one simply uses
\texttt{hwDefine} instead.

\vspace*{-2mm}

{\baselineskip10pt\begin{verbatim}
 hwDefine 
  `FactIter(n,acc) = if n=0 then (n,acc) else FactIter(n-1,n*acc))`
\end{verbatim}}

\vspace*{-2mm}

\noindent This defines \texttt{FactIter} in the logic (i.e.~does an
implicit \texttt{Define}), but it also creates a theorem:

\vspace*{-2mm}

{\baselineskip10pt\begin{verbatim}
 |- REC (DEV (\x. FST x = 0)) (DEV (\x. x))
        (PAR (DEV (\x. FST x - 1))
        (PRECEDE (\x. x) (DEV (UNCURRY $*)))) 
    ===> 
    DEV FactIter
\end{verbatim}}

\vspace*{-2mm}

\noindent The term to the left of \texttt{===>}
represents an implementation using `circuit constructors' which are
described in the appendix.  The term \texttt{DEV~FactIter} describes a
behaviour that applies the function \texttt{FactIter} under the
control of a four-phase handshake, which is specified in the
definition of the higher order function \texttt{DEV} (also described
in Section~\ref{secSpecification}). The theorem, which is proved
automatically by \texttt{hwDefine}, verifies that the circuit is a
correct implementation of \texttt{FactIter}. As it is the factorial
function \texttt{FACT} that we want to implement, we create an
implementation of it, called \texttt{Fact}, using \texttt{FactIter}:

\vspace*{-2mm}

{\baselineskip10pt\begin{verbatim}
 hwDefine `Fact n = SND(FactIter (n,1))
\end{verbatim}}

\vspace*{-2mm}

\noindent The implicit \texttt{Define} creates a
theorem corresponding to this definition. From the
theorem relating \texttt{FACT} and \texttt{FactIter} proved above
we get that \texttt{Fact} is equal to \texttt{FACT}.

\vspace*{-2mm}

{\baselineskip10pt\begin{verbatim}
 |- Fact n = SND(FactIter(n,1))
 |- Fact = FACT
\end{verbatim}}

\vspace*{-2mm}

\noindent \texttt{hwDefine} also gives us a verified implemention of
\texttt{Fact}:

\vspace*{-2mm}

{\baselineskip10pt\begin{verbatim}
 |- FOLLOW (PRECEDE (\n. (n,1)) (DEV FactIter)) SND ===> DEV Fact
\end{verbatim}}

\vspace*{-2mm}

\noindent The circuit here contains \texttt{DEV~FactIter} preceded and
followed by some combinational logic. We can refine this
implementation by replacing the specification
\texttt{DEV~FactIter} by the previously constructed implementation
(our circuit constructors are compositional).

\vspace*{-2mm}

{\baselineskip10pt\begin{verbatim}
 |- FOLLOW
     (PRECEDE (\n. (n,1))
        (REC (DEV (\x. FST x = 0)) (DEV (\x. x))
             (PAR (DEV (\x. FST x - 1))
             (PRECEDE (\x. x) (DEV (UNCURRY $*)))))) SND 
    ===> 
    DEV Fact
\end{verbatim}}

\vspace*{-2mm}

At this point, we need to decide whether multiplication (\texttt{*})
has a combinational or sequential implementation.\footnote{In the HOL term
``\texttt{UNCURRY~\$*}'' the \texttt{\$} tells the parser to
suppress \texttt{*}'s infix status and \texttt{UNCURRY} creates a
function taking a pair as an argument, which is needed as HOL's
built-in multiplication infix \texttt{*} is a curried function,
but \texttt{DEV}'s type assumes it is applied to a paired function.}

Components in a
circuit of the form \texttt{DEV~f} are specifications to compute
$f$. They can either be refined to implementations (like we just
refined \texttt{DEV~FactIter}) or they can be converted to
combinational logic (assuming we have a combinational circuit to
compute $f$ in our library of components).

Suppose we wanted to use a sequential implimentation of multiplication,
then we would synthesise a theorem:

\vspace*{-2mm}

{\baselineskip10pt\begin{verbatim}
 |- (...) ===> DEV(UNCURRY $*)
\end{verbatim}}

\vspace*{-2mm}

\noindent and then refine the occurrence of \texttt{DEV(UNCURRY~\$*)}
to \texttt{(...)}.  In this fashion we can incrementally develop a
circuit in a top-down fashion. To manage such developments we have
implemented operators for combining refinements analogous to rewriting
operators \cite{Paulson}. This is a fairly standard ML programming
method and we don't give further details here.

For simplicity, let us continue the example using a combinational
implementation of multiplication. We add \texttt{*} to the list of
combinationally-implementable functions and then our system will
automatically prove a theorem of the form shown below (we have omitted
parts of the circuit to save space, but hope what remains is enough to
give an idea).

\vspace*{-2mm}

{\baselineskip10pt\begin{verbatim}
 |- (?v0 v1 v2 v3 v4 v5 v6 v7 v8 v9 v10 v11 v12 v13 v14 v15 v16
      ...
      v51 v52 v53 v54 v55.
     CONSTANT 1 v0 /\ DEL (load,v20) /\ NOT (v20,v19) /\
     NOT (v50,v37) /\ COMB (UNCURRY $*) (v10 <> v9,v49) /\
     ...
     DEL (v49,v35) /\ DFF (v36,v38,v13) /\ DFF (v35,v37,v12) /\
     AND (v2,v8,v55) /\ AND (v55,v53,done)) 
    ==> 
    DEV FACT (load,inp,done,out) 
\end{verbatim}}

\vspace*{-2mm}

The arguments \texttt{(load,inp,done,out)} are control and data lines of the component
and are explained later. The circuit to the left of \texttt{==>} is a netlist represented in
the standard way in higher order logic \cite{Mel93} and built out of
two kinds of abstract registers: a unit delay \texttt{DEL} and
edge-triggered register \texttt{DFF}. The other components are
combinational and comprise boolean functions \texttt{NOT},
\texttt{AND}, \texttt{OR}, multiplexers \texttt{MUX}, constants
\texttt{CONSTANT~0}, \texttt{CONSTANT~1} and functional blocks
\texttt{COMB~$f$} that compute the function $f$. These components are
described in more detail in Section~\ref{secCompilingByProof}. To
create a netlist that can be converted to hardware, the occurences of
\texttt{DEL} and \texttt{DFF} need to be implemented using standard
clocked registers. This invoves converting from an abstract `cycle
based' model to a clocked synchronous model. The particular theory of doing this that we use
was developed by Melham \cite{Mel93}. We have incorporated
his theory (and even some of his old HOL88 code) to
implement a synthesis script. Applied to our factorial implementation, this yields
a theorem of the form:

\vspace*{-1mm}

{\baselineskip10pt\begin{verbatim}
|- InfRise clk
   ==>
   (?v0 v1 v2 v3 v4 v5 v6 v7 v8 v9 v10 v11 v12 v13 v14 v15 v16 v17 
     ....
     v125 v126 v127 v128 v129.
     CONSTANT 1 v0 /\ TRUE v23 /\ Dff (clk,v23,v22) /\
     COMB (UNCURRY $*) (v10 <> v9,v100) /\ TRUE v109 /\
     ...
     Dtype (clk,v12,v123) /\ AND (v77,v76,v14) /\ TRUE v129 /\
     AND (v2,v8,v126) /\ AND (v126,v124,done)) 
   ==>
   DEV FACT (load at clk,inp at clk,done at clk,out at clk) 
\end{verbatim}}

\vspace*{-2mm}

This theorem uses explicitly clocked registers \texttt{Dtype} and
\texttt{Dff} (defined in Section~\ref{secCompilingByProof}) and states
that, assuming a live clock (\texttt{InfRise~clk} means \texttt{clk}
has infinitely many rising edges), then the behaviour of the
synthesised circuit implements a handshake computing
\texttt{FACT}. The temporally abstracted signals of the form \texttt{s~at~clk} are
signals \texttt{s} abstracted on rising edges of \texttt{clk}.

We have implemented a program in Standard ML that prints a netlist of
the sort generated above as Verilog. Although the HDL created is not
formally verified (it cannot be, as Verilog has no formal semantic
specification) visual inspection and other tools can yield a high
degree of confidence that it corresponds to the verified netlist
represented in higher order logic. The Verilog we create is intended
to be synthesisable, but we have not yet tried any actual synthesis
experiments.




%--------------------------------------------------------------------------
\section{Specification}
\label{secSpecification}
%--------------------------------------------------------------------------

The circuits generated by our compiler are specified to compute a
function defined in higher order logic using a handshaking protocol,
which  works in a similar way of a function call.  

The device interface comprises the control signals \VAR{load} and
\VAR{done}.  In order to trigger the device, a positive edge (a signal
transition from low to high) must be provided on \VAR{load}. The
termination is indicated by asserting \VAR{done}.

Figure~\ref{figDev} shows a sequence of events that illustrates a
transaction in which a handshaking device computes a function $f$
starting at a time $t$ and ending at a later time $t'$ (where time
counts cycles).

\vspace*{-6mm}

\begin{figure}[htb]
   \centerline{
      \subfigure[The device is ready.]{
         \label{figDev1}\input{dev1.eepic}\hspace*{1.5cm}}
      \subfigure[There is a positive edge on \VAR{load}.]{
         \label{figDev2}\input{dev2.eepic}\hspace*{2.7cm}}}
   \hspace*{0.4cm}
   \centerline{
      \subfigure[The device is busy.]{
         \label{figDev3}\input{dev3.eepic}\hspace*{0.1cm}}
      \subfigure[The computation terminates.]{
         \label{figDev4}\input{dev4.eepic}}\hspace*{0cm}}
      \caption{\label{figDev}A handshake protocol.}
\end{figure}

\vspace*{-6mm}


At the start of a transaction (time~$t$) the device must be outputting
\DEF{T} on $done$ (to indicate it is ready) and the environment must
be asserting \DEF{F} on $load$, i.e.~in a state such that a positive
edge on $load$ can be generated (Figure~\ref{figDev1}).  A~transaction
is initiated by asserting (at time $t{+}1$) the value \DEF{T} on
$load$, i.e.~$load$ has a positive edge at time $t{+}1$. This causes
the device to read the value, $v$ say, being input on $inp$ (at time
$t{+}1$) and to de-assert $done$ (Figure~\ref{figDev2}).  The device
then becomes insensitive to inputs (Figure~\ref{figDev3}) until
\DEF{T} is next asserted on $done$, at which time (say time $t' >
t{+}1$) the value $f(v)$ computed will be output on $out$
(Figure~\ref{figDev4}).

To specifiy the behaviour of a handshaking device,
the auxiliary predicates \DEF{Posedge} and \DEF{HoldF} are defined.
A positive edge of a signal is defined as the transition of its
value from low to high or, in our case, from \DEF{F} to \DEF{T}. 
The term \mE{\DEF{HoldF}\ (t_1,t_2)\ s} says that a
signal \VAR{s} holds a low value \DEF{F} during a half-open interval
starting at \sVAR{t}{1} to just before \sVAR{t}{2}. The formal definitions are:

\vspace*{-1mm}

\[
\begin{array}{ll}
\TURNST\ \DEF{Posedge}\ s\ t~ &      = ~ \IF{~t{=}0~}{~\DEF{F}~}{~(\NOT\hspace*{0.8mm} s(t{-}1)\ \AND\ s\ t~})\\
\TURNST\ \DEF{HoldF}\ (t_1,t_2)\ s & = ~ \forall t.\ t_1 \leq t < t_2\ \IMP\ \NOT(s\ t)
\end{array}
\]



The behaviour of the handshaking device computing a function $f$ is described by the term 
$\DEF{Dev}\ f\ \VAR{(load,inp,done,out)}$ where:
\[
\begin{array}{l}
\TURNST\ \DEF{Dev}\ f\ \VAR{(load,inp,done,out)} = \\
~~\quad     (\forall t.\ \VAR{done}\ t\ \AND\ \DEF{Posedge}\ \VAR{load}\ (t{+}1)\ \\
\phantom{~~\quad     (\forall t.~} \IMP \\
\phantom{~~\quad     (\forall t.\ ~ } \exists t'.\ t' > t{+}1\ \AND\ \DEF{HoldF}\ (t{+}1,t')\ \VAR{done}\ \AND \\
\phantom{~~\quad     (\forall t.\ ~ \exists t'.\ }  \VAR{done}\ t'\ \AND\ (\VAR{out}\ t' = f (\VAR{inp}\ (t{+}1))))\ ~  \AND \\
~~\quad (\forall t.\ \VAR{done}\ t\ \AND\ \NOT(\DEF{Posedge}\ \VAR{load}\ (t{+}1))\ \IMP  \VAR{done}\ (t{+}1)) ~\AND \\
~~\quad (\forall t.\ \NOT\VAR{done}\ t\ \IMP \exists t'.\ t' > t{+}1\ \AND \VAR{done}\ (t{+}1))\\
\end{array}
\]
The first conjunct in the right-hand side describes the context presented
in Figure~\ref{figDev}. If the device is available and a positive
edge occurs on \VAR{load}, there exists a time \VAR{t'} in future
when \VAR{done} signals its termination and the output is produced.
The value of the output at time \VAR{t'} is the result
of applying \VAR{f} to the value of the input at time $\VAR{t}{+}1$.
The signal \VAR{done} holds the value \DEF{F} during the computation.
The second conjunct specifies the situation where no call
is made on \VAR{load} and the device simply remains idle.
Finally, the last conjunct states that if the device
is busy, it will eventually finish its computation
and become idle.

The synthesis tool generates theorems ${\it{Imp}}
\Rightarrow \DEF{Dev}\ f\ \VAR{(load,inp,done,out)}$, where {\it{Imp}}
is a term representing an implementation. In the next section we
describe the various forms that {\it{Imp}} takes during synthesis.


%--------------------------------------------------------------------------
\section{Steps of compilation}
\label{secImplementation}
%--------------------------------------------------------------------------
We implement functions $f$ where
$f : \sigma_1\times\cdots\times\sigma_m \rightarrow \tau_1\times\cdots\times\tau_n$
and $\sigma_1,\ldots,\sigma_m,\tau_1,\ldots,\tau_n$ are the types of
values that can be carried on busses (e.g. $n$-bit words).
The starting point of compilation is the definition of such a function $f$ by an equation of
the form: $f(x_1,\ldots,x_n)=e$, where any recursive calls of $f$ in
$e$ must be tail-recursive. Applying \texttt{hwDefine} to such a
definition will first define $f$ in higher order logic (using TFL
\cite{slind:wfrec})  and then generate a theorem $\vdash {\it{Imp_C}}
\Longrightarrow \DEF{Dev}~f$, where ${\it{Imp_C}}$ is a term
constructed using `circuit constructors'.
%\DEF{ATM}, \DEF{SEQ},
%\DEF{PAR}, \DEF{ITE} and \DEF{REC}, which are defined below.  
To support top-down development, ${\it{Imp_C}}$ can also contain terms
$\DEF{Dev}~f_1$, $\ldots$, $\DEF{Dev}~f_p$, where functions $f_1$,
$\ldots$, $f_p$ are not yet implemented. The compilation by deduction
goes through a number of steps.

\vspace*{-3mm}

\subsection*{Step 1: conversion to combinators}

The first step in translating $f(x_1,\ldots,x_n)=e$ is to encode $e$
as an applicative expression, $e_c$ say, built from the operators \DEF{Seq}
(compute in sequence),
\DEF{Par} (compute in parallel), \DEF{Ite} (if-then-else) and \DEF{Rec} (recursion), defined by:
\[
\begin{array}{ll}
\DEF{Seq}~f_1~f_2     &   = \lambda x.~f_2(f_1~x)\\
\DEF{Par}~f_1~f_2     &   = \lambda x.~(f_1~x,~f_2~x)\\
\DEF{Ite}~f_1~f_2~f_3 &   = \lambda x.~ \TT{if}~f_1~x~\TT{then}~f_2~x~\TT{else}~f_3~x\\
\DEF{Rec}~f_1~f_2~f_3 &   = \lambda x.~\TT{if}~f_1~x~\TT{then}~f_2~x~\TT{else}~\DEF{Rec}~f_1~f_2~f_3~(f_3~x)\\
\end{array}
\]
%$\DEF{Rec}~f_1~f_2~f_3$ is defined using Hilbert's
%$\epsilon$-operator, and means ``choose a function $f$ such that $f$
%satisfies the equation $f =
%\lambda x.~\TT{if}~f_1~x~\TT{then}~f_2~x~\TT{else}~f(f_3~x)$''.  
%In
%practise, $f_1$, $f_2$ and $f_3$ will be such that $f$ is uniquely
%determined.  

The encoding into an applicative expression built out of \DEF{Seq},
\DEF{Par}, \DEF{Ite} and \DEF{Rec} is performed by a proof script
coded in the theorem prover's metalanguage, Standard ML, and results
in a theorem $\vdash (\lambda(x_1,\ldots,x_n).~e)~=~e_c$, and hence
$\vdash f=e_c$.  The algorithm used is straightforward and is not
described here. As an example, the proof script deduces from:

\vspace*{-1mm}

\begin{alltt}
\( \vdash \DEF{FactIter}(n,acc) = (\texttt{if} n=0 \texttt{then} (n,acc) \texttt{else} \DEF{FactIter}(n-1,n{\times}acc)) \)
\end{alltt}

\vspace*{-1mm}

\noindent the theorem:

\vspace*{-1mm}

{\baselineskip10pt\begin{alltt}
\( \vdash \DEF{FactIter} =                                                                        \)
\(     \DEF{Rec} (\DEF{Seq} (\DEF{Par} ({\lambda}(n,acc). n) ({\lambda}(n,acc). 0)) ({=}))        \)
\(         (\DEF{Par} ({\lambda}(n,acc). n) ({\lambda}(n,acc). acc))                              \)
\(         (\DEF{Par} (\DEF{Seq} (\DEF{Par} ({\lambda}(n,acc). n) ({\lambda}(n,acc). 1)) ({-}))   \)
\(              (\DEF{Seq} (\DEF{Par} ({\lambda}(n,acc). n) ({\lambda}(n,acc). acc)) ({\times}))) \)
\end{alltt}}

\vspace*{-3mm}

\subsection*{Step 2: implementation using circuit constructors}

The next step is to replace the combinators \DEF{Seq},
\DEF{Par}, \DEF{Ite} and \DEF{Rec} with corresponding circuit constructors
\DEF{SEQ},
\DEF{PAR}, \DEF{ITE} and \DEF{REC} that build handshaking devices (see the appendix for their definitions).
The key property of these constructors are the following theorems that
enable us to compositionally deduce theorems of the form $\vdash
{\it{Imp_C}}
\Longrightarrow \DEF{Dev}~f$, where ${\it{Imp_C}}$ is a term
constructed using  the circuit constructors, and hence is a handshaking device
(the long implication symbol $\Longrightarrow$ denotes
implication lifted to functions
$f \Longrightarrow g~=~\forall x.~f(x)\Rightarrow g(x)$): 

\vspace*{-3mm}
$$\begin{array}{l}
~\TURNST\ \DEF{Dev}\ f\ \FIMP \  \DEF{Dev}\ f\\[1mm]


~\TURNST\
(P_1\ \FIMP\ \DEF{Dev}~f_1)~\AND~(P_2 \ \FIMP\ \DEF{Dev}~f_2)\\
\phantom{~\TURNST~}
 \IMP ~
(\DEF{SEQ}\ P_1\ P_2 \ \FIMP \ \DEF{Dev}\ (\DEF{Seq}~f_1~f_2))\\[1mm]


~\TURNST\ 
(P_1 \ \FIMP\ \DEF{Dev}~f_1)~\AND~(P_2 \ \FIMP\ \DEF{Dev}~f_2)\\
\phantom{~\TURNST~}
\IMP ~
(\DEF{PAR}\ P_1\ P_2\ \FIMP \ \DEF{Dev}\ (\DEF{Par}~f_1~f_2))\\[1mm]


~\TURNST\ 
(P_1 \ \FIMP\ \DEF{Dev}~f_1)~\AND~(P_2 \ \FIMP\ \DEF{Dev}~f_2)~\AND~(P_3 \ \FIMP\ \DEF{Dev}~f_3)\\
\phantom{~\TURNST~}
\IMP ~
(\DEF{ITE}\ P_1\ P_2\ P_3\  \FIMP \ \DEF{Dev}\ (\DEF{Ite}~f_1~f_2~f_3))\\[1mm]


~\TURNST\ 
\DEF{Total}(f_1,f_2,f_3)\\
\phantom{~\TURNST~}
\IMP ~
(P_1 \ \FIMP\ \DEF{Dev}~f_1)~\AND~(P_2 \ \FIMP\ \DEF{Dev}~f_2)~\AND~(P_3 \ \FIMP\ \DEF{Dev}~f_3)\\
\phantom{~\TURNST~}
\IMP ~
(\DEF{REC}\ P_1\ P_2\ P_3 \ \FIMP \ \DEF{Dev}\ (\DEF{Rec}~f_1~f_2~f_3))\\

\end{array}$$
where $\DEF{Total}(f_1,f_2,f_3)$ is a predicate ensuring that the
specified recursive function terminates.  
%there is a unique function satisfying $f = \lambda
%x.~\TT{if}~f_1~x~\TT{then}~f_2~x~\TT{else}~f(f_3~x)$.
%\[
%\DEF{Total}(f_1,f_2,f_3)~=~\exists variant.~\forall x.~\neg(f_1~x)
%                                        \IMP variant(f_3~x) < variant~x
%\]

If $e_c$ is an expression built using \DEF{Seq},\DEF{Par}, \DEF{Ite}
and \DEF{Rec} then, by suitably instantiating the predicate variables
$P_1$, $P_2$ and $P_3$, these theorems allow us to construct an
expression $e_C$ built from circuit constructors \DEF{SEQ}, 
\DEF{PAR}, \DEF{ITE} and \DEF{REC} such that $\vdash e_C \FIMP
\DEF{Dev}~e_c$. From Step~1 we have $\vdash f=e_c$, hence 
$\vdash e_C \FIMP \DEF{Dev}~f$

A function $f$ which is combinational (i.e.~can be implemented
directly with logic gates without using registers) can be packaged as
a handshaking device using a constructor \DEF{ATM}, which creates a
simple handshake interface and satisfies the refinement theorem:

\vspace*{-4mm}
$$\begin{array}{l}
%\texttt{ATM\_INTRO}\\
~\TURNST\ \DEF{ATM}\ f\ \FIMP \  \DEF{Dev}\ f\\
\phantom{~~~~~~~~~~~~~~~~~~~~~~~~~~~~~~~~~~~~~~~~~~~~~~~~~~~~~~~~~~~~~~~~~~~~~~~~~~~~~~~~~~~~~~~~~~~~~~~~~~~~~}\\[-4mm]
\end{array}$$

\noindent The circuit constructor \DEF{ATM} is defined with the other constructors in the appendix.
To avoid a proliferation  of internal handshakes, when the proof script that constructs $e_C$ from $e_c$ 
is implementing $\DEF{Seq}~f_1~f_2$, it checks to see whether $f_1$ or $f_2$ 
are compositions of combinational functions and if so introduces \DEF{PRECEDE} or \DEF{FOLLOW} instead of \DEF{SEQ},
using the theorems:

\vspace*{-4mm}
$$\begin{array}{l}
~\TURNST\
      (P\ \FIMP\ \DEF{Dev}~f_2)~\IMP~(\DEF{PRECEDE}~f_1~P\ \FIMP\ \DEF{Dev}\ (\DEF{Seq}~f_1~f_2))\\
\phantom{~~~~~~~~~~~~~~~~~~~~~~~~~~~~~~~~~~~~~~~~~~~~~~~~~~~~~~~~~~~~~~~~~~~~~~~~~~~~~~~~~~~~~~~~~~~~~~~~~~~~~~}\\[-4mm]
~\TURNST\
(P\ \FIMP\ \DEF{Dev}~f_1)~\IMP~(\DEF{FOLLOW}~P~f_2\  \FIMP\ \DEF{Dev}\ (\DEF{Seq}~f_1~f_2))\\
\end{array}$$


\noindent $\DEF{PRECEDE}~f~d$ processes inputs with $f$ before sending them to $d$ and
$\DEF{FOLLOW}~d~f$ processes outputs of $d$ with
$f$. The definitions are:

\vspace*{3mm}

$\begin{array}{l}
\DEF{PRECEDE}~f~d~(\VAR{load},inp,done,out)  =\\
~~ \exists v.~ \DEF{COMB}~f~(inp,v) ~\wedge~ d(\VAR{load},v,done,out)\\
 \\
\DEF{FOLLOW}~d~f~ (\VAR{load},inp,done,out)  =\\
~~ \exists v.~d(\VAR{load},inp,done,v) ~\wedge~ \DEF{COMB}~f~(v,out)
\end{array}$
\vspace*{3mm}

The construction $\DEF{SEQ}~d_1~d_2$ introduces a handshake between the executions
of $d_1$ and $d_2$, but $\DEF{PRECEDE}~f~d$ and $\DEF{FOLLOW}~d~f$
just `wire' f before or after $d$, respectively, without introducing a
handshake. 

\vspace*{-3mm}

\subsection*{Step 3: unfolding to a cycle-level netlist}

The result of Step~2 is a theorem
$\vdash e_C \FIMP \DEF{Dev}~f$ where $e_C$ is an expression built out of the circuit
constructors \DEF{ATM}, 
\DEF{SEQ}, \DEF{PAR}, \DEF{ITE}, \DEF{REC}, \DEF{PRECEDE} and \DEF{FOLLOW}. If the theorem
is rewritten with the definitions of these constructors (see their
definitions in the appendix) we get a circuit built out of standard
kinds of gates (\DEF{AND}, \DEF{OR}, \DEF{NOT} and \DEF{MUX}), a
generic combinational component $\DEF{COMB}~g$ (where $g$ will be a
function represented as a HOL $\lambda$-expression) and two
kinds of synchronous elements:
\DEF{DEL} and \DEF{DFF}. 

The next phase of compilation converts terms of the form $\DEF{COMB}~g~(inp,out)$
into circuits built only out of components that it is assumed can be directly realised in
hardware. Such components  currently include boolean functions (e.g. $\wedge$,
$\vee$ and $\neg$), multiplexers and simple operations  on $n$-bit words (e.g.~versions
of $+$, $-$ and $<$, various shifts etc.). 
A special purpose proof rule uses a straightforward recursive algorithm to synthesise
combinational circuits. For example:

\vspace*{-2mm}

{\begin{alltt}
\( \vdash \DEF{COMB}\ ({\lambda}(m,n). (m<n, m+1))\ (inp1\texttt{<>}inp2,out1\texttt{<>}out2) =              \)
\(     {\exists}v0. \DEF{COMB}\ ({<})\ (inp1\texttt{<>}inp2,out1) \wedge \DEF{CONSTANT} 1 v0 \wedge \)
\(         \! \DEF{COMB}\ ({+})\ (inp1\texttt{<>}v0,out2)                                              \)
\end{alltt}}
\vspace*{-2mm}

\noindent where \texttt{<>} is bus concatenation,
$\DEF{CONSTANT}~1~v0$ drives $v0$ high continuously, and
$\DEF{COMB}~{<}$ and $\DEF{COMB}~{+}$ are assumed
given components (if they were not given, then the could be
implemented explicitly, but one has to stop somewhere). 
 
The abstract registers \DEF{DEL} and \DEF{DFF} used in the definitions of
the circuit constructs are defined by:

\vspace*{-2mm}

{\begin{alltt}
\( \vdash \DEF{DEL}(inp,out) = (out 0 = inp 0) \wedge {\forall}t. out(t+1) = inp t                       \)
\( \vdash \DEF{DFF}(d,\VAR{clk},q) =                                                                           \)
\(     {\forall}t. q(t+1) = (\texttt{if} \DEF{POSEDGE} \VAR{clk} (t+1) \texttt{then} d(t+1) \texttt{else} q t) \)
\end{alltt}}

\vspace*{-2mm}

These definitions were chosen for convenience in defining
\DEF{ATM}, \DEF{SEQ}, \DEF{PAR}, \DEF{ITE} and \DEF{REC}, but they are not standard:
\DEF{DEL} is not explicitly clocked, and it is transparent at time $0$,
and
\DEF{DFF} is transparent on the positive edge of clock. 
However, we can implement \DEF{DEL} and \DEF{DFF} in terms of more
standard components,  \DEF{REG} and \DEF{REGF}, where:

{\baselineskip16pt\begin{alltt}
\( \vdash \DEF{REG}(inp,out) = {\forall}t. out(t+1) = inp t          \)
\( \vdash \DEF{REGF}(inp,out) = (out 0 = \DEF{F}) \wedge \DEF{REG}(inp,out)\)
\end{alltt}}

Both \DEF{DEL} and \DEF{DFF} can be realised by combinations of
\DEF{REG} and \DEF{REGF} (and some combinational logic). We omit
details here. The final part of Step~3 is to refine with the
definitions of \DEF{DEL}, \DEF{DFF} and $\FIMP$ to get an implication
theorem $\vdash {\it{Imp}}_C \IMP \DEF{Dev}~f~(\VAR{load},inp,done,out)$, where ${\it{Imp}}_C$
is an existentially quantified conjunction of predicates describing
primitive components (combinational components plus \DEF{REG} and
\DEF{REGF}).

\vspace*{-3mm}

\subsection*{Step 4: temporal refinement to a clocked synchronous circuit}

At the end of Step~3 one has a circuit built out of standard gates,
operations $\texttt{COMB}~op$ where $op$ has a known implementation as
a combinational circuit, and registers \DEF{REG} and \DEF{REGF}.  The
next step is to replace these abstract `cycle level' registers with
standard clocked synchronous parts. Up until now we have viewed
signals as abstract sequences of values.  We now move to a finer
temporal granularity in which values are sampled on the edge of a
clock; such values can be latched by edge-triggered registers. We
implement \DEF{DEL} and \DEF{DFF} in terms of a standard dtype
register defined by:

\vspace*{2mm}

$\DEF{Dtype}(ck,d,q) ~=~ \forall t.~q(t{+}1) = \texttt{if}~\DEF{Posedge}~ck~(t{+}1)~\texttt{then}~d~t~\texttt{else}~q~t$

\vspace*{2mm}

\noindent and a version of this register that initialises to a state holding $F$ (i.e.~$0$):

\vspace*{2mm}

$\DEF{Dff}(ck,d,q) ~ = ~ (q~0~ =~F)~\wedge~\DEF{Dtype}(ck,d,q)$

\vspace*{2mm}

If $s$ is a signal, then $s~\DEF{at}~\VAR{clk}$ is the temporal
abstraction consisting of the values of $s$ at positive edges of
$\VAR{clk}$. 
Melham shows in his monograph that:




{\baselineskip14pt\begin{alltt}
\( \vdash \DEF{InfRise} \VAR{clk} \Rightarrow {\forall}d q. \DEF{Dtype}(\VAR{clk},d,q) \Rightarrow \DEF{REG}(d \DEF{at} \VAR{clk}, q \DEF{at} \VAR{clk}) \)
\( \vdash \DEF{InfRise} \VAR{clk} \Rightarrow {\forall}d q. \DEF{Dff}(\VAR{clk},d,q) \Rightarrow \DEF{REGF}(d \DEF{at} \VAR{clk}, q \DEF{at} \VAR{clk})  \)
\end{alltt}}



By instantiating $\VAR{load}$, $inp$, $done$ and $out$ in the theorem
obtained by Step~3 to $\VAR{load}~\DEF{at}~\VAR{clk}$, $inp~\DEF{at}~\VAR{clk}$,
$done~\DEF{at}~\VAR{clk}$ and $out~\DEF{at}~\VAR{clk}$, respectively, and then
performing some deductions involving Melham's theorems (and the
monotonicity of existential quantification and conjunction with
respect to implication) we obtain a theorem

\vspace*{2mm}

$\vdash \DEF{InfRise}~\VAR{clk}~\IMP~{\it{Imp}}~\IMP~\DEF{Dev}~f~(\VAR{load}~\DEF{at}~\VAR{clk},inp~\DEF{at}~\VAR{clk},done~\DEF{at}~\VAR{clk},out~\DEF{at}~\VAR{clk})$.

\vspace*{2mm}

The term ${\it{Imp}}$ represents the compiled implementation of $\DEF{Dev}~f$ in higher order logic.
It differs from ${\it{Imp}}_C$ produced by Step~3 in having an explicit clock and using
registers $\DEF{Dtype}$ and $\DEF{Dff}$ rather than \texttt{REG} and \texttt{REGF}.

\vspace*{-3mm}

\subsection*{Step 5: translation to Verilog}

The implementation ${\it{Imp}}$ produced by Step~4 is a circuit
represented in a standard way (as described, for example, in Melham's book)
as a conjunction of predicates representing primitive components,
with internal lines existentially quantified.

This can be `pretty printed' directly into Verilog. Unfortunately
Verilog is not a semantically formal notation, so we cannot deduce
Verilog by proof. However, there is a direct and transparent
correspondence between the structure of ${\it{Imp}}$ and the Verilog
text that is printed.  Currently we have a translator to Verilog
implemented, but we are still testing and validating it.



%Steps 1 and 2 are performed automatically when
%a function is defined using the \texttt{hwDefine} command.  Steps 3
%and 4 are invoked using separate SML functions applied to the output
%of \texttt{hwDefine}. All these steps are implemented as derived proof
%rules, so that the circuit produced is formally verified by theorem
%proving.
%
%
%Typically one defines several functions using \texttt{hwDefine} in a
%top-down fashion, as illustrated in Section~\ref{secHOL2Verilog}.  The
%combination of the resulting implementations can be done
%interactively, or by writing a script (analogous to a Makefile) in
%SML. The user must explicitly declare constants that are to be
%combinational primitives, and must also supply implementations of
%these in the form of Verilog modules.
%
%
%


%--------------------------------------------------------------------------
%\section{Compiling by proof}
%\label{secCompilingByProof}
%--------------------------------------------------------------------------
%
Steps 1 and 2 in the previous section are performed automatically when
a function is defined using the \texttt{hwDefine} command.  Steps 3
and 4 are invoked using separate SML functions applied to the output
of \texttt{hwDefine}. All these steps are implemented as derived proof
rules, so that the circuit produced is formally verified by theorem
proving.


Typically one defines several functions using \texttt{hwDefine} in a
top-down fashion, as illustrated in Section~\ref{secHOL2Verilog}.  The
combination of the resulting implementations can be done
interactively, or by writing a script (analogous to a Makefile) in
SML. The user must explicitly declare constants that are to be
combinational primitives, and must also supply implementations of
these in the form of Verilog modules.



%--------------------------------------------------------------------------
\section{Case studies}
\label{secCaseStudy}
%--------------------------------------------------------------------------


We are in the midst of two case-studies: a Booth multiplier, and an implementation
of the AES encryption algorithm.

As part of a project to verify an ARM processor \cite{Fox02}, a high
level model of the muliplication algorithm used by some ARM
implementations was created in higher order logic. This is a Booth
multiplier and we are using Fox's existing specification as an example
for testing our compiler. 


A more substantial example, being done at the University of Utah, is
implementing the Advanced Encryption Standard (AES) \cite{AES}
algorithm for private-key encryption. This specifies a multi-round
algorithm with primitive computations based on finite field
operations.  Starting from an existing formalization of AES
\cite{slind:aes}, we have generated netlists and circuits for the major
components of an encryption (and decryption) round.  Although out work on AES
is incomplete, our current progress provides several illustrations of our
synthesis methodology.

The AES formalization includes a proof of functional correctness  for the
algorithm: specifically, encryption and decryption are inverse functions.
Deriving the hardware from the proven specification using logical inference
assures us that the hardware encrypter is the inverse of the hardware
decrypter.

An encryption round performs the following transformations on a 4-by-4 matrix
of input bytes:
\begin{enumerate}
\item
application of \emph{sbox}, an invertible function from bytes to bytes,
to each byte;
\item
a cyclical shift of each row;
\item
multiplication of each column by a fixed degree 3 polynomial, with coefficients
in the 256 element finite field, GF($2^8$);
\item
adding a key to the matrix with exclusive or.
\end{enumerate}

%\begin{figure}
%\begin{alltt}
%ShiftRows (b00,b01,b02,b03,b10,b11,b12,b13,b20,b21,b22,b23,b30,b31,b32,b33) =
%          (b00,b01,b02,b03,b11,b12,b13,b10,b22,b23,b20,b21,b33,b30,b31,b32)
%\end{verbatim}
%\begin{verbatim}
%|- (?v0 v1 v2.
%      DEL (load,v2) /\ NOT (v2,v1) /\ AND (v1,load,v0) /\
%      NOT (v0,done) /\ DEL (inp1,out1) /\ DEL (inp2,out2) /\
%      DEL (inp3,out3) /\ DEL (inp4,out4) /\ DEL (inp6,out5) /\
%      DEL (inp7,out6) /\ DEL (inp8,out7) /\ DEL (inp5,out8) /\
%      DEL (inp11,out9) /\ DEL (inp12,out10) /\ DEL (inp9,out11) /\
%      DEL (inp10,out12) /\ DEL (inp16,out13) /\ DEL (inp13,out14) /\
%      DEL (inp14,out15) /\ DEL (inp15,out16)) ==>
%    DEV ShiftRows (load, inp1 <> ... inp16, done, out1 <> ... <> out16) 
%\end{verbatim}
%\caption{Row shifting}
%\label{AES:shift}
%\end{figure}

Many of our specifications are not tail recursive, but formally
deriving (and verifying) tail-recursive versions was
straightforward. We were then able to generate circuits directly via
an application of the compiler.  As illustrated in
Section~\ref{secHOL2Verilog}, we explored various options for
generating components either as separate handshaking designs or expanding
them into combinational logic. We have also explored converting our
high level recursive specification of multiplication into a table
lookup. The resulting verified tables can then be stored into a RAM or
ROM device.  For synthesizing the tables directly into hardware, we
have automated the definition of a function on bytes as a balanced
\texttt{if} expression, branching on each successive bit of its input.

{\footnotesize\begin{verbatim}
0xB ** x = if WORD_BIT 7 x then
             if WORD_BIT 6 x then 
               ...
                       if WORD_BIT 0 then 0xA3 else 0xA8
               ...
           else
             if WORD_BIT 6 x then
               ...
\end{verbatim}}

Our experience so far is that we have a flexible framework for
creating optimised circuits, and the methodology of creating
implementations by deduction ensures all optimisation are valid.




%
%The discussion about multiplication below applies to the sbox implementation,
%although the sbox is a more complicated function.  The row shift operation
%translates into a simple circuit whose wiring reflects the permutation
%(Fig.~\ref{AES:shift}).  The multiplication algorithm admits several
%implementation strategies, each with important hardware
%differences.  The key addition is similar to the row shift.
%
%\begin{figure}
%\begin{verbatim}
%xtime w = w << 1 # (if MSB w then 0x1B else 0x0)
%
%b_1 ** b_2 =
%   if b_1 = 0x0 then 0x0
%   else if LSB b_1 then b2 # ((b_1 >>> 1) ** xtime b_2)
%   else                      ((b_1 >>> 1) ** xtime b_2)
%
%IterMult (b1,b2,acc) =
%   if b1 = 0w then (b1,b2,acc)
%   else IterMult (b1 >>> 1, xtime b2, if LSB b1 then (b2 # acc) else acc)
%\end{verbatim}
%\caption{AES Multipliers}
%\label{AES:mult}
%\end{figure}
%
%The column multiplication step multiplies (in GF($2^8$)) each input byte by a
%pre-specified constant, meaning that multiplications by only a small handful
%of numbers are needed (0x2 and 0x3 for encryption, and 0x9, 0xB, 0xD, and 0xE
%for decryption).  Fig.~\ref{AES:mult} gives the specification for this
%multiplication, written infix \verb+b1 ** b2+.  The \verb+<<+ and \verb+>>>+
%operators denote left shift and logical right shift; the \verb+#+ operator
%denotes exclusive or; and the \verb+MSB+ and \verb+LSB+ functions return the
%most and least significant bit, respectively.
%
%The specification cannot be used in hardware synthesis, because it is not
%tail-recursive.  The tail recursive function \verb+IterMult+
%(Fig.~\ref{AES:mult}) is easily proven equivalent to \verb+**+.  A
%straightforward application of the hardware synthesis methodology yields a
%netlist or Verilog for \verb+IterMult+.  The occurrence of \verb+xtime+ in the
%function yields a choice in hardware generation.  We can treat the \verb+xtime+
%function as a handshaking device, and refine it into \verb+IterMult+ after both
%are converted into circuit constructors.  Alternately, we can replace the
%definition of \verb+xtime+ in \verb+IterMult+ with \verb+xtime+'s body.  Since
%\verb+xtime+ is a small combinational circuit, the latter approach has the
%advantage of generating less handshaking hardware.
%
%Instead of generating an iterative circuit for the general multiplication
%function, we have the option to generate combinational circuits for only the
%six needed values of the first argument.  Partially evaluating the definition
%of \verb+**+ on its potential first arguments automatically proves six theorems
%similar to the following:
%\begin{verbatim}0xB ** x = x # xtime (x # xtime (xtime x))\end{verbatim}
%We then take these theorems as specifications for one argument multipliers and
%transforms them into hardware.  The \verb+xtime+ component can be supplied
%either before or after synthesis, as in the previous example.
%
%We can apply partial evaluation even further, and supply all of the 256
%possible second arguments as well, to derive multiplication tables, each 256
%bytes in size.
%\begin{verbatim}
%(0xB ** 0x0 = 0x0) /\ ... /\ (0xB ** 0xFF = 0xA3)
%\end{verbatim}
%These tables could be incorporated into a RAM or ROM device.  For synthesizing
%the tables directly into hardware, we have automated the definition of a
%function on bytes as a balanced \verb+if+ expression, branching on each
%successive bit of its input.
%\begin{verbatim}
%0xB ** x = if WORD_BIT 7 x then
%             if WORD_BIT 6 x then 
%               ...
%                       if WORD_BIT 0 then 0xA3 else 0xA8
%               ...
%           else
%             if WORD_BIT 6 x then
%               ...
%\end{verbatim}



%--------------------------------------------------------------------------
\section{Related work}
\label{secRelatedWork}
%--------------------------------------------------------------------------
Previous approaches to combine theorem provers 
and formal synthesis established an analogy between
the goal directed proof technique and an interactive 
design process. In LAMBDA, the user starts from the behavioural
specification and builds the circuit incrementally
by adding primitive hardware components
which automatically simplify the goal \cite{Fou89}.
Hanna {\em et al.\/} \cite{HLD89} introduce
several {\em techniques\/} (functions) that
simplify the current goal into simpler subgoals.
Techniques are adaptations to hardware design
of {\em tactics\/} in LCF.

Alternative approaches synthesise circuits
by applying semantic-preserving transformations
to their specifications. For instance,
the Digital Design Derivation (DDD) transforms
finite-state machines specified in terms of
tail-recursive lambda abstractions into hierarchical
boolean systems~\cite{Johnson90}. Lava and Hydra
are both hardware description languages embedded
in Haskell whose programs
consist of definitions of gates and their 
connections (netlists)~\cite{BCSS99,OD02}. While Lava interfaces with
external theorem provers to verify its circuits,
Hydra designers can synthesise them
via formal equational reasoning
(using definitions and lemmas from functional programming).
The functional languages $\mu$FP and Ruby
adopt similar principles in hardware design~\cite{JS90,She84}.
The circuits are defined in terms of primitive
functions over booleans, numbers and lists, and
higher-order functions, the {\em combining forms\/},
which compose hardware blocks in different
structures. Their mathematical properties provide
a calculational style in design exploration.

These approaches deal with an interactive
synthesis at the gate or state-machine level
of abstraction only. Moreover, the synthesis
and the proof of correctness require a 
substantial user guidance. Gropius and SAFL 
are two related works that address these issues.

%------ Gropius: Blumenrohr
Gropius is a hardware description language
defined as a subset of HOL~\mbox{\cite{Blu01,Gropius1}}.
Its algorithmic level
provides control structures like if-then-else,
sequential composition and while loop.
The~atomic commands are DFGs (data flow
graphs) represented by lambda abstractions.
The compiler initially combines every while loop into
a single one at the outermost level of the
program:
\[
\DEF{PROGRAM}\ \VAR{out\_default}\ (\DEF{LOCVAR}\ \VAR{vars}\ 
(\DEF{WHILE}\ c\ (\DEF{PARTIALIZE}\ \VAR{b})))
\]
The body \VAR{b} of the \DEF{WHILE} loop is an acyclic~DFG.
The list \VAR{out\_default}
provides initial values for the output variables.
The term \DEF{LOCVAR} declares the local variables
\VAR{vars} and \DEF{PARTIALIZE} converts a
non-recursive (terminating) DFG into a potentially
non-terminating command.
The compiler then synthesises a handshaking
interface which encapsulates this program.
Each of these hardware blocks are now regarded
as primitive blocks or {\em processes\/} at the system level.
Processes are connected via communication
units ({\em k-processes\/}) which implement delay,
synchronisation, duplication,
splitting and joining of a process output data
(actually there  are 10 different k-processes \cite{Blu01}).
Although the synthesis produces the proof of
correctness of each process and k-process,
the correctness of the top-level system is not generated.
The reason for that is mainly because
the top-level interface of a network of processes and 
k-processes does not match the handshaking interface pattern.

%------ SAFL: Sharp
Our compilation method is partly inspired by SAFL 
(Statically Allocated Functional Language) \cite{MS01b},
especially the ideas in Richard Sharp's PhD \cite{Sha02}. 
SAFL~is a first-order functional language whose
programs consist of a sequence of tail-recursive function 
definitions. Its high-level of abstraction allows
the exploitation of powerful program analyses
and optimisations not available in traditional
synthesis systems. 
However, the synthesis is not based on
the correct-by-construction principles
and the compiler has not been verified.

% our approach
The novelty of our approach is the 
compilation of functional programs by
composing especially designed and
pre-verified circuit constructors.
%
As each of these circuit constructors
have the key property of implementing
a device that computes precisely their
corresponding combinators, 
the verification and the compilation of 
functional programs can be done
automatically.



%--------------------------------------------------------------------------
\section{Current State and Future work}
\label{secFutureWork}
%--------------------------------------------------------------------------

The compiler described here has only just been completed and is still
being tested (e.g. by simulating circuits and viewing the resulting
waveforms). In the immediate future we plan to ruggedise and debug the
existing code.  In parallel with this we are already undertaking some
subtantial case studies, as described in
Section~\ref{secCaseStudy}. We plan to study the hardware we compile
and to work on making the compiler produce better designs.

At present all data-refinement (e.g. from numbers or enumerated types
to words) must be done manually, by proof in higher order logic. The
HOL4 system has some `boolification' facilities that automatically
translate higher level data-types into bit-strings, and we hope to
integrate these into the compiler.

We want to investigate using the compiler to generate test-bench
monitors that can run in parallel simulation with designs that are not
correct by construction.  Thus our hardware can act as a ``golden''
reference against which to test other implementations.

The work described here is part of a bigger project to create
hardware/software combinations by proof.  We hope to investigate the
option of creating software for ARM processors and linking it to
hardware created by our compiler (probably packaged as an ARM
co-processor). Our emphasis is likely to be on cryptographic hardware
and software, because there is a clear need for high assurance of
correct implementation in this domain.


%--------------------------------------------------------------------------
\bibliographystyle{plain}
\bibliography{tphols2005}
%--------------------------------------------------------------------------

%--------------------------------------------------------------------------
%\section*{Appendix: The circuit constructors}
%\label{CircuitConstructors}
%--------------------------------------------------------------------------
\newpage
%--------------------------------------------------------------------------
\section*{Appendix: The circuit constructors}
\label{CircuitConstructors}
%--------------------------------------------------------------------------
\vspace*{-3mm}

The components used in the definitions of the circuit constructors are:

\smallskip
$
\begin{array}{l}
\TURNST\ \DEF{AND}\ (\sVAR{in}{1},\sVAR{in}{2},\VAR{out}) ~ = 
   \forall t.\ \VAR{out}\ t = (\sVAR{in}{1}\ t\ \AND\ \sVAR{in}{2}\ t)\\
\TURNST\ \DEF{OR}\ (\sVAR{in}{1},\sVAR{in}{2},\VAR{out}) ~ =
   \forall t.\ \VAR{out}\ t = (\sVAR{in}{1}\ t\ \OR\ \sVAR{in}{2}\ t)\\
\TURNST\ \DEF{NOT}\ (\VAR{inp},\VAR{out}) ~ = 
   \forall t.\ \VAR{out}\ t = \NOT(\VAR{inp}\ t)\\
\TURNST\ \DEF{MUX} (\VAR{sw,in_1,in_2nout})~ =~
 \forall t.\ out~t =  \IF{sw~t}{in_1~t}{in_2~t}\\
\TURNST\ \DEF{COMB}\ f\ \VAR{(inp,out)} ~ = \forall t.\ \VAR{out}\ t = f (\VAR{inp}\ t)\\
\TURNST\ \DEF{DEL}\ (\VAR{inp},\VAR{out}) ~ = 
    (\VAR{out}\ 0 = \VAR{inp}\ 0)\ \AND\ 
    (\forall t.\ \VAR{out} (t{+}1) = \VAR{inp}\ t)\\
\TURNST\ \DEF{DFF} (\VAR{d,clk,q})~ =~
 \forall t.\ q (t{+}1) =  \IF{\DEF{Posedge}\ \VAR{clk}\ (t{+}1)}{d (t{+}1)}{q\ t}
\end{array}
$
 

$
\TURNST\ \DEF{POSEDGE} (\VAR{inp},\VAR{out}) = 
    \exists c_0~ c_1.\ \DEF{DEL} (\VAR{inp},c_0)\ \AND\
                  \DEF{NOT} (c_0, c_1)\ \AND\ \DEF{AND} (c_1,\VAR{inp},\VAR{out})
$

$
\begin{array}{l}
\TURNST\ \DEF{ATM}\ f\ (\VAR{load,inp,done,out}) =\\
\phantom{\TURNST\ ~~}    \exists c_0~ c_1.\ \DEF{POSEDGE} (\VAR{load}, c_0)\ \AND \ 
\DEF{NOT} (c_0, \VAR{done})\ \AND\\
\phantom{\TURNST\ ~~\exists c_0, c_1.\ }  \DEF{COMB}\ f\ (\VAR{inp},c_1)\ \AND\ \DEF{DEL} (c_1,\VAR{out})
\end{array}
$


$
\begin{array}{l}
\TURNST\ \DEF{SEQ}\ f\ g\ (\VAR{load,inp,done,out}) = \\
\phantom{\TURNST\ ~~} \exists c_0~ c_1~ c_2~ c_3~ \VAR{data}.\\
\phantom{\TURNST\ ~~\exists~}                      \DEF{NOT} (c_2,c_3) \ \AND \ 
                      \DEF{OR} (c_3,\VAR{load},c_0) \ \AND \  f (c_0,\VAR{inp},c_1,\VAR{data}) \ \AND \\
\phantom{\TURNST\ ~~ \exists~} 
        g (c_1,\VAR{data},c_2,\VAR{out})\ \AND\ 
        \DEF{AND} (c_1,c_2,\VAR{done}) 
\end{array}
$

$
\begin{array}{l}
\TURNST\ \DEF{PAR}\ f\ g\ (\VAR{load,inp,done,out}) = \\
\phantom{\TURNST\ ~}     \exists c_0~c_1~\VAR{start}~\sVAR{done}{1}~\sVAR{done}{2}~
                                 \sVAR{data}{1}~\sVAR{data}{2}~\sVAR{out}{1}~\sVAR{out}{2}.\\
\phantom{\TURNST\ ~~ \exists~}
       \DEF{POSEDGE} (\VAR{load},c_0)\ \AND\  
       \DEF{DEL} (\VAR{done},c_1)\ \AND \ 
       \DEF{AND} (c_0,c_1,\VAR{start})\ \AND\\
\phantom{\TURNST\ ~~ \exists~}
       f (\VAR{start},\VAR{inp},\sVAR{done}{1},\sVAR{data}{1})\ \AND \ 
       g (\VAR{start},\VAR{inp},\sVAR{done}{2},\sVAR{data}{2})\ \AND\\
\phantom{\TURNST\ ~~ \exists~}
       \DEF{DFF} (\sVAR{data}{1},\sVAR{done}{1},\sVAR{out}{1})\ \AND \  
       \DEF{DFF} (\sVAR{data}{2},\sVAR{done}{2},\sVAR{out}{2})\ \AND\\
\phantom{\TURNST\ ~~ \exists~}
       \DEF{AND} (\sVAR{done}{1},\sVAR{done}{2},done)\ \AND \ 
       (\VAR{out} = \LAMBDA{t}{(\sVAR{out}{1}\ t,\sVAR{out}{2}\ t)})
\end{array}
$

$
\begin{array}{l}
\TURNST\ \DEF{ITE}\ e\ f\ g\ (\VAR{load,inp,done,out}) =\\
\phantom{\TURNST\ ~}
   \exists c_0~c_1~c_2~\VAR{start}~\VAR{start'}~\VAR{done\_e}~\VAR{data\_e}~q~ 
                          \VAR{not\_e}~\VAR{data\_f}~\VAR{data\_g}~\VAR{sel}\\
\phantom{\TURNST\ ~\exists }
                  \VAR{done\_f}~\VAR{done\_g}~
                          \VAR{start\_f}~\VAR{start\_g}.\ \\
\phantom{\TURNST\ ~\exists ~}
           \DEF{POSEDGE} (\VAR{load},c_0)\ \AND\
           \DEF{DEL} (\VAR{done},c_1)\ \AND\ \DEF{AND} (c_0,c_1,\VAR{start})\ \AND \\
\phantom{\TURNST\ ~\exists ~}
           e (\VAR{start},\VAR{inp},\VAR{done\_e},\VAR{data\_e})\ \AND\
           \DEF{POSEDGE} (\VAR{done\_e},\VAR{start'})\ \AND \\
\phantom{\TURNST\ ~\exists ~}
           \DEF{DFF} (\VAR{data\_e},\VAR{done\_e},\VAR{sel})\ \AND\
           \DEF{DFF} (\VAR{inp},\VAR{start},q)\ \AND \\
\phantom{\TURNST\ ~\exists ~}
           \DEF{AND} (\VAR{start'},\VAR{data\_e},\VAR{start\_f})\ \AND\
           \DEF{NOT} (\VAR{data\_e},\VAR{not\_e})\ \AND \\
\phantom{\TURNST\ ~\exists ~}
           \DEF{AND} (\VAR{start'},\VAR{not\_e},\VAR{start\_g})\ \AND\
           f (\VAR{start\_f},q,\VAR{done\_f},\VAR{data\_f})\ \AND\\
\phantom{\TURNST\ ~\exists ~}
           g (\VAR{start\_g},q,\VAR{done\_g},\VAR{data\_g})\ \AND\
           \DEF{MUX} (\VAR{sel},\VAR{data\_f},\VAR{data\_g},out)\ \AND \\
\phantom{\TURNST\ ~\exists ~}
           \DEF{AND} (\VAR{done\_e},\VAR{done\_f},c_2)\ \AND\
           \DEF{AND} (c_2,\VAR{done\_g},\VAR{done}) 
\end{array}
$

$
\begin{array}{l}
\TURNST\ \DEF{REC}\ e\ f\ g\ (\VAR{load,inp,done,out}) = \\
\phantom{\TURNST\ ~}
     \exists \VAR{done\_g}~ \VAR{data\_g}~ \VAR{start\_e}~ q~ \VAR{done\_e}~ 
             \VAR{data\_e}~ \VAR{start\_f}~ \VAR{start\_g}~ \VAR{inp\_e}~ 
             \VAR{done\_f}\\[-1mm]
\phantom{\TURNST\ ~\exists }
             \sVAR{c}{0}~ \sVAR{c}{1}~ 
             \sVAR{c}{2}~ \sVAR{c}{3}~\sVAR{c}{4}~
             \VAR{start}~ \VAR{sel}~ \VAR{start'}~ \VAR{not\_e}. \\[0.5mm]
\phantom{\TURNST\ ~\exists~ }
        \DEF{POSEDGE}(\VAR{load},\sVAR{c}{0})\ \AND\
        \DEF{DEL}(\VAR{done},\sVAR{c}{1})\ \AND\
        \DEF{AND}(\sVAR{c}{0},\sVAR{c}{1},\VAR{start})\ \AND \\
\phantom{\TURNST\ ~\exists~ }
        \DEF{OR}(\VAR{start,sel,start\_e})\ \AND\
        \DEF{POSEDGE}(\VAR{done\_g,sel})\ \AND \\
\phantom{\TURNST\ ~\exists~ }
        \DEF{MUX}(\VAR{sel,data\_g,inp,inp\_e})\ \AND\
           \DEF{DFF}(\VAR{inp\_e},\VAR{start\_e},q)\ \AND \\
\phantom{\TURNST\ ~\exists~ }
           e (\VAR{start\_e},\VAR{inp\_e},\VAR{done\_e},\VAR{data\_e}) \AND\
        \DEF{POSEDGE}(\VAR{done\_e,start'})\ \AND \\
\phantom{\TURNST\ ~\exists~ }
        \DEF{AND}(\VAR{start',data\_e,start\_f})\ \AND\
        \DEF{NOT}(\VAR{data\_e,not\_e})\ \AND \\
\phantom{\TURNST\ ~\exists~ }
        \DEF{AND}(\VAR{not\_e,start',start\_g})\ \AND\
           f (\VAR{start\_f},q,\VAR{done\_f},\VAR{out})\ \AND\\
\phantom{\TURNST\ ~\exists~ }
           g (\VAR{start\_g},q,\VAR{done\_g},\VAR{data\_g})\ \AND\
        \DEF{DEL}(\VAR{done\_g},\sVAR{c}{3})\ \AND\\
\phantom{\TURNST\ ~\exists~ }
        \DEF{AND}(\VAR{done\_g},\sVAR{c}{3},\sVAR{c}{4})\ \AND\
        \DEF{AND}(\VAR{done\_f},\VAR{done\_e},\sVAR{c}{2})\ \AND\
        \DEF{AND}(\sVAR{c}{2},\sVAR{c}{4},\VAR{done})
\end{array}
$

%\fbox{Can we put the diagrams back so that they all fit onto the rest of this page?}



\end{document}
% LocalWords:  langle


