
Steps 1 and 2 in the previous section are performed automatically when
a function is defined using the \texttt{hwDefine} command.  Steps 3
and 4 are invoked using separate SML functions applied to the output
of \texttt{hwDefine}. All these steps are implemented as derived proof
rules, so that the circuit produced is formally verified by theorem
proving.


Typically one defines several functions using \texttt{hwDefine} in a
top-down fashion, as illustrated in Section~\ref{secHOL2Verilog}.  The
combination of the resulting implementations can be done
interactively, or by writing a script (analogous to a Makefile) in
SML. The user must explicitly declare constants that are to be
combinational primitives, and must also supply implementations of
these in the form of Verilog modules.

