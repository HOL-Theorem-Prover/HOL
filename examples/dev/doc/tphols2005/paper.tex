\documentclass{llncs}
\usepackage{epic,eepic,subfigure,alltt,graphicx}
%-------------------------------------------------------------------------------
% File: notation.tex
%-------------------------------------------------------------------------------

%-------------------------------------------------------------------------------
% BNF
%-------------------------------------------------------------------------------
\newcommand{\bnfSC}[1]{\ensuremath{\mathit{\langle#1\rangle}}} % Syntax construct
\newcommand{\bnfDef}{\ensuremath{\mathit{::=}}} % ::=
\newcommand{\bnfList}[1]{\ensuremath{\{#1\}}}
\newcommand{\bnfOR}[1]{\ensuremath{|\ #1}}
\newcommand{\bnfOpt}[1]{\ensuremath{[#1]}}

%-------------------------------------------------------------------------------
% Mathematics
%-------------------------------------------------------------------------------
\newcommand{\mE}[1]{\ensuremath{\mathit{#1}}} % Math environment
\newcommand{\argm}[1]{\ensuremath{\vec{#1}}} % list of arguments or parameters
\newcommand{\VAR}[1]{\ensuremath{\mathit{#1}}} % variable
\newcommand{\OR}{\ensuremath{\vee}} % logical OR
\newcommand{\AND}{\ensuremath{\wedge}} % logical AND
\newcommand{\NOT}{\ensuremath{\neg}} % logical NOT
%\newcommand{\IF}[3]{\ensuremath{#1\, \rightarrow\, #2\, \mid\ #3}} % logical IF
\newcommand{\IF}[3]{\ensuremath{\texttt{if}~#1~\texttt{then}~#2~\texttt{else}~#3}} % logical IF
\newcommand{\TURNST}{\ensuremath{\vdash}} % turnstile
\newcommand{\IMP}{\ensuremath{\Rightarrow}} % implies
\newcommand{\FIMP}{\ensuremath{\Longrightarrow}} % implies
\newcommand{\NIMP}{\ensuremath{\not\supset}} % doesn't imply
\newcommand{\DEF}[1]{\ensuremath{\mathsf{#1}}} % definition
\newcommand{\sVAR}[2]{\ensuremath{\mE{#1}_{#2}}} % variable with subscript
\newcommand{\LAMBDA}[2]{\ensuremath{\lambda\, #1.\ #2}} % lambda abstraction
\newcommand{\COMP}[2]{\ensuremath{#1 \circ #2}} % function compositon
\newcommand{\MULT}{\ensuremath{\cdot}} % multiplication

%\newtheorem{example}{Example}

% added by MJCG
\newcommand{\TY}[1]{\ensuremath{\sl{#1}}} % type
\newcommand{\TT}[1]{\ensuremath{\texttt{#1}}} % keyword


%--------------------------------------------------------------------------
\title{A Proof-Producing Hardware Compiler for a Subset of Higher Order Logic}

\author{}
\institute{}

\begin{document}
\maketitle

\vspace*{-8mm}

\begin{center}
\begin{tabular}{ccc}
{\bf Mike Gordon, Juliano Iyoda} &\hspace*{5mm}& {\bf Scott Owens, Konrad Slind}\\
University of Cambridge          &\hspace*{5mm}& University of Utah\\
Computer Laboratory              &\hspace*{5mm}& School of Computing\\
William Gates Building           &\hspace*{5mm}& 50 South Central Campus Drive\\
JJ Thomson Avenue                &\hspace*{5mm}& Salt Lake City\\
Cambridge CB3 0FD, UK            &\hspace*{5mm}& Utah UT84112, USA
\end{tabular}


\vspace*{2mm}

({\it{authors listed in alphabetical order\/}})
\end{center}

\vspace*{-5mm}

%--------------------------------------------------------------------------
\thispagestyle{empty}

\begin{abstract}
A special purpose theorem prover for creating hardware implementations
is described.  Implementations are created as theorems $\vdash
{\it{Imp}}\Rightarrow{\it{Spec}}$, where {\it{Imp}} describes a
circuit that implements a four-phase handshake computing a function
defined in higher order logic by {\it{Spec}}.  The compiler goes
through several phases, each refining the implementation to a more
concrete form, until a representation that can be output as
synthesisable Verilog is deduced.  The compilation to hardware is
transparent and programmable. Users can tune the generated hardware
either by deductively pre-optimising the higher order logic
specifications, or by adding optimisations to the theorem proving
script that performs compilation.  The hardware designs we construct
may not have the highest performance, but we hope that having
guaranteed-correct implementations will be important for some applications.
An example is one of our case studies in which we are working on
producing components for building cryptographic hardware, with
the eventual goal of developing a reference library of ``golden''
implementations.


\end{abstract}

%--------------------------------------------------------------------------
\section{Introduction}
\label{secIntroduction}
%--------------------------------------------------------------------------
One approach to formal hardware verification starts with a design and
then proves that it satisfies a specification.  In this paper we
repackage some of the ideas underlying such a `verification flow' into
an alternative `synthesis flow' in which implementations are compiled
by proof from specifications.  
%Starting from a simple functional
%specification in higher order logic, we describe a
%correct-by-construction compilation to hardware.  
The key idea is to
automatically generate theorems of the form $\vdash
{\it{Imp}}\Longrightarrow\DEF{Dev}~f$, where {\it{Imp}} is a term
describing a circuit, $\DEF{Dev}~f$ specifies that function $f$ is
computed by a four-phase handshake and the operator $\Longrightarrow$
means that the implementation meets the specification.  Several
different representations of implementations are created, with the
lowest level corresponding to synthesisable Verilog.


Our system is implemented in HOL4, but the ideas could be
realised in other programmable higher order logic proof systems like
Isabelle, PVS, NuPrl and Coq. In such systems, functions are defined
and then proof scripts are executed to prove properties of
the functions.  Proof scripts are programs that perform sequences of
deductions to create theorems.

In the next section we give a quick overview of the currently
implemented system using a simple (but unrealistic) example. We then
describe the specification of components in higher order logic that
underlies the work. Next we describe how function definitions in
higher order logic are translated to hardware via a sequence of steps,
ending with Verilog.  Some unfinished case studies that are in
progress are then discussed. Finally, we survey related work by
others, and outline our plans for the future. An appendix contains the
definitions of the circuit constructors that we use to build
implementations.

\vspace*{-3mm}



%--------------------------------------------------------------------------
\section{From Higher Order Logic to Verilog: an overview}
\label{secHOL2Verilog}
%--------------------------------------------------------------------------
\vspace*{-2mm}

We illustrate the translation from higher order logic to hardware with a
simple toy example: the factorial function. This example is chosen for
clarity and is not typical of the kind of
applications were are working on!  For a brief discussion of those
applications see Section~\ref{secCaseStudy}.
The factorial function \DEF{FACT} satisfies

\vspace*{-1mm}

{\baselineskip10pt\begin{alltt}
 \(\vdash (\DEF{FACT} 0 = 1) \wedge {\forall}n. \DEF{FACT}(\DEF{Suc} n) = \DEF{Suc} n \times \DEF{FACT} n\)
\end{alltt}}

\vspace*{-2mm}
We can only synthesise tail recursions, so the first
step is to define a tail-recursive function \DEF{FactIter} and show
that it can be used to compute \DEF{FACT}.

\vspace*{-2mm}

{\baselineskip14pt\begin{alltt}
\( \vdash \DEF{FactIter}(n,acc) = \texttt{if} n=0 \texttt{then} (n,acc) \texttt{else} \DEF{FactIter}(n-1,n{\times}acc)) \)
\( \vdash \DEF{FACT} n = \DEF{Snd}(\DEF{FactIter}(n,1)) \)
\end{alltt}}

\vspace*{-2mm}

\noindent In HOL, \DEF{FactIter} would be defined by executing:

\vspace*{-2mm}

{\baselineskip10pt\begin{alltt}
 Define
\(  `\DEF{FactIter}(n,acc) = \texttt{if} n=0 \texttt{then} (n,acc) \texttt{else} \DEF{FactIter}(n-1,n{\times}acc))`\)
\end{alltt}}

\vspace*{-2mm}

\noindent to synthesise a hardware implementation one simply uses
\texttt{hwDefine} instead.

\vspace*{-2mm}


{\baselineskip10pt\begin{alltt}
 hwDefine
\(  `\DEF{FactIter}(n,acc) = \texttt{if} n=0 \texttt{then} (n,acc) \texttt{else} \DEF{FactIter}(n-1,n{\times}acc))`\)
\end{alltt}}


\vspace*{-2mm}

\noindent This defines \DEF{FactIter} in the logic (i.e.~does an
implicit \texttt{Define}), but it also creates a theorem:

\vspace*{-2mm}

{\baselineskip10pt\begin{alltt}
\( \vdash \DEF{REC} (\DEF{Dev} ({\lambda}x. \DEF{Fst} x = 0))         \) 
\(        (\DEF{Dev} ({\lambda}x. x))                                 \)
\(        (\DEF{PAR} (\DEF{Dev} ({\lambda}x. \DEF{Fst} x - 1))        \)
\(             (\DEF{PRECEDE} ({\lambda}x. x) (\DEF{Dev}({\times})))) \)
\(    \Longrightarrow \DEF{Dev} \DEF{FactIter}                        \)
\end{alltt}}

\vspace*{-2mm}

\noindent The term to the left of $\Longrightarrow$ is a high-level
representation of an implementation using `circuit constructors' which
are described in Section~\ref{secImplementation} and formally defined in the
appendix.  The term \DEF{Dev~FactIter} describes a behaviour that
applies the function \DEF{FactIter} under the control of a four-phase
handshake. The details of the handshake are specified in the
definition of the higher order function \DEF{Dev} in
Section~\ref{secSpecification}. The theorem that is proved
automatically by
\texttt{hwDefine} verifies that the generated high-level design is a correct
implementation. In our example, we want to implement the factorial
function \DEF{FACT}, so we create an implementation of it, called
\DEF{Fact}, using \DEF{FactIter}:

\vspace*{-2mm}

{\baselineskip10pt\begin{alltt}
 hwDefine\( `\DEF{Fact} n = \DEF{Snd}(\DEF{FactIter} (n,1))`  \)
\end{alltt}}

\vspace*{-2mm}

\noindent The implicit \texttt{Define} defines \DEF{Fact} satisfying this equation, hence:

\vspace*{-2mm}

{\baselineskip14pt\begin{alltt}
\( \vdash \DEF{Fact} n = \DEF{Snd}(\DEF{FactIter}(n,1)) \)
\( \vdash \DEF{Fact} = \DEF{FACT}                       \)
\end{alltt}}

\vspace*{-2mm}

\noindent \texttt{hwDefine} also gives us a verified implementation of
\DEF{Fact}:

\vspace*{-2mm}

{\baselineskip10pt\begin{alltt}
\( \vdash \DEF{FOLLOW} (\DEF{PRECEDE} ({\lambda}n. (n,1)) (\DEF{Dev} \DEF{FactIter})) \DEF{Snd} \Longrightarrow \DEF{Dev} \DEF{Fact} \)
\end{alltt}}

\vspace*{-2mm}

\noindent The circuit here contains \DEF{Dev~FactIter} preceded and
followed by some combinational logic. We can refine this
implementation by replacing the specification
\DEF{Dev~FactIter} by the previously constructed implementation.
%(our circuit constructors are compositional).

\vspace*{-2mm}

{\baselineskip10pt\begin{alltt}
\( \vdash \DEF{FOLLOW}                                                                   \)
\(     (\DEF{PRECEDE} ({\lambda}n. (n,1))                                                \)
\(        (\DEF{REC} (\DEF{Dev} ({\lambda}x. \DEF{Fst} x = 0))                           \)
\(             (\DEF{Dev} ({\lambda}x. x))                                               \)
\(             (\DEF{PAR} (\DEF{Dev} ({\lambda}x. \DEF{Fst} x - 1))                      \)
\(                  (\DEF{PRECEDE} ({\lambda}x. x) (\DEF{Dev}({\times})))))) \DEF{Snd}   \)
\(    \Longrightarrow \DEF{Dev} \DEF{Fact}                                               \)
\end{alltt}}

\vspace*{-2mm}

At this point a design decision is needed: whether to give multiplication ($\times$)
a combinational or sequential implementation.
Components in a
circuit of the form $\DEF{Dev}~f$ are specifications to compute
$f$. They can either be refined to implementations (like we just
refined \DEF{Dev~FactIter}) or they can be converted to
combinational logic (assuming we can implement
$f$ using logic gates) and wrapped with a handshaking interface.

Suppose we wanted to use a sequential implementation of multiplication,
then we would synthesise a theorem:

\vspace*{-2mm}

{\baselineskip10pt\begin{alltt}
\( \vdash \texttt{(...)} \Longrightarrow \DEF{Dev}({\times}) \)
\end{alltt}}

\vspace*{-2mm}

\noindent and then refine the occurrence of $\DEF{Dev}(\times)$
to \texttt{(...)}.  Thus we can incrementally develop a
circuit top-down. To manage such developments we have
implemented operators for combining refinements analogous to rewriting
operators \cite{paulson83}.

For simplicity (though unrealistically), let us continue the example
using a combinational implementation of multiplication. We add
${\times}$ to the list of combinationally-implementable functions and
then our system will automatically prove a theorem of the form shown
below (we have omitted parts of the circuit to save space, but hope
what remains is enough to give an idea).

\vspace*{-2mm}

{\baselineskip10pt\begin{alltt}
\( \vdash ({\exists}v0 v1 v2 v3 v4 v5 v6 v7 v8 v9 ...                                             \)
\(      ... v51 v52 v53 v54 v55.                                                                       \)
\(     \DEF{CONSTANT} 1 v0 {\wedge} \DEF{DEL}(load,v20) {\wedge} \DEF{NOT}(v20,v19) {\wedge}       \)
\(     \DEF{NOT}(v50,v37) {\wedge} \DEF{COMB}({\times})(v10\texttt{<>}v9,v49) {\wedge}                     \)
      \(\raisebox{3mm}{\vdots}\)
\(     \DEF{DEL}(v49,v35) {\wedge} \DEF{DFF}(v36,v38,v13) {\wedge} \DEF{DFF}(v35,v37,v12) {\wedge} \)
\(     \DEF{AND}(v2,v8,v55) {\wedge} \DEF{AND}(v55,v53,done))                                     \)
\(    \Rightarrow \DEF{Dev} \DEF{FACT} (load,inp,done,out)                                         \)
\end{alltt}}

\vspace*{-2mm}

The arguments $(load,inp,done,out)$ are control and data lines of the
component and are explained in Section~\ref{secSpecification}. The
circuit to the left of $\Rightarrow$ is a netlist represented in the
standard way in higher order logic \cite{Mel93} and built out of two
kinds of abstract registers: a unit delay \DEF{DEL} and edge-triggered
register \DEF{DFF}. The other components are combinational and
comprise boolean functions \DEF{NOT},
\DEF{AND}, \DEF{OR}, multiplexers \DEF{MUX}, constants $\DEF{CONSTANT}~n$ that output a value $n$,
and functional blocks
$\DEF{COMB}~f$ that compute the function $f$ (if $f$ takes two arguments,
then the input will be $a_1\texttt{<>}a_2$, where $a_1$ and $a_2$ carry the arguments
and \texttt{<>} is bus concatenation). These components are
described in more detail in Section~\ref{secImplementation}. To
create standard hardware, the occurrences of
\DEF{DEL} and \DEF{DFF} need to be implemented using standard
clocked registers. This involves converting from an abstract `cycle
based' model to a clocked synchronous model. The particular theory of
doing this that we use was developed by Melham \cite{Mel93}. We have
incorporated his theory (and even some of his old HOL88 code) in
our compiler. Applied to our factorial implementation (actually one that uses a sequential multiplier),
we get:

%\vspace*{-1.5mm}

{\baselineskip10pt\begin{alltt}
\( \vdash \DEF{InfRise} clk                                                                        \)
\(   \Rightarrow                                                                                   \)
\(   ({\exists}v0 v1 v2 v3 v4 v5 v6 v7 v8 v9 v10 v11 ...                               \)
\(     ... v125 v126 v127 v128.                                                                   \)
\(     \DEF{CONSTANT} 1 v0 {\wedge} \DEF{FlipFlopT}(clk,load,v20) {\wedge} \DEF{NOT}(v20,v19) {\wedge}        \)
\(     \DEF{AND}(v19,load,v18) {\wedge} \DEF{Dtype}(clk,done,v17) {\wedge} \DEF{AND}(v18,v17,v16) {\wedge} \)

       \(\raisebox{3mm}{\vdots}\)
\(     \DEF{Dtype}(clk,v14,v127) {\wedge} \DEF{AND}(v14,v127,v126) {\wedge}  \)
\(     \DEF{AND} (v2,v8,v128) {\wedge} \DEF{AND}(v128,v126,done))               \)
\(   \Rightarrow \DEF{Dev} \DEF{FACT} (load \DEF{at} clk, inp \DEF{at} clk, done \DEF{at} clk, out \DEF{at} clk) \)
\end{alltt}}

%\vspace*{-2.5mm}

This theorem uses explicitly clocked registers \DEF{Dtype} and
\DEF{FlipFlopT} (defined in Section~\ref{secImplementation}) and states
that, assuming a live clock ($\DEF{InfRise}~clk$ means $clk$
has infinitely many rising edges), then the behaviour of the
synthesised circuit implements a handshake computing
\DEF{FACT}. A signal of the form $s~\DEF{at}~clk$ is the temporal abstraction
of $s$ abstracted on rising edges of $clk$.

We have implemented a program in Standard ML that prints a netlist of
the sort occurring in the theorem above as Verilog. Although the HDL created is not
formally verified (it cannot be, as Verilog has no formal semantic
specification) visual inspection and other tools can improve our
confidence that it corresponds to the verified netlist
represented in higher order logic. The Verilog we create is intended
to be synthesisable, but we have not yet tried any actual synthesis
experiments. A tiny fragment of the Verilog for the \DEF{FACT}
implementation is:



{\footnotesize\baselineskip6pt\begin{verbatim}
module FACT (clk,load,inp,done,out);
 input clk,load; input [31:0] inp;
 output done; output [31:0] out;
 wire clk,done;
 wire [31:0] v0; wire [31:0] v1; ... wire [0:0] v225;

 CONSTANT   CONSTANT_0 (v0);          defparam CONSTANT_0.size  = 31;
                                      defparam CONSTANT_0.value = 1;
 FlipFlopT  FlipFlopT_0 (clk,load,v20);
 NOT        NOT_0 (v20,v19);
 AND        AND_0 (v19,load,v18);
 Dtype      Dtype_0 (clk,done,v17);   defparam Dtype_0.size = 0;

 ...

endmodule
\end{verbatim}}

\newpage

If this is run though a Verilog simulator and the
resulting waveform viewed, then we can see that if \texttt{8}
is input on \texttt{Main.inp} then \texttt{40320} is
being output on \texttt{Main.out} when \texttt{Main.done} goes high.

\vspace*{-24mm}


\mbox{}\hspace*{-5mm}\scalebox{1.85}{\includegraphics[viewport = -30 340 250 269.5, clip = true]{gtkwave.epsi}}


%\vspace*{-16mm}
%\mbox{}\hspace*{-7mm}{\includegraphics[viewport = 0 400 350 220, clip = true]{gtkwave.epsi}}
%\newpage

%\mbox{}\hspace*{-5mm}{\includegraphics[viewport = 0 500 360 220, clip = true]{gtkwave.epsi}}
%\mbox{}\hspace*{-5mm}{\includegraphics[width=150mm,height=50mm]{gtkwave.epsi}}



\vspace*{50mm}

%--------------------------------------------------------------------------
\section{Specification}
\label{secSpecification}
%--------------------------------------------------------------------------

The circuits generated by our compiler are specified to perform a
computation using a handshaking protocol in a way similar to a
function call.

The device interface comprises the control signals \VAR{load} and
\VAR{done}.  In order to trigger the device, a positive edge (a signal
transition from low to high) must be provided on \VAR{load}. The
termination is indicated by asserting \VAR{done}.

Figure~\ref{figDev} shows a sequence of events that illustrates a
transaction in which a handshaking device computes a function $f$
starting at a time $t$ and ending at a later time $t'$ (where time
counts cycles).

\vspace*{-6mm}

\begin{figure}[htb]
   \centerline{
      \subfigure[The device is ready.]{
         \label{figDev1}\input{dev1.eepic}\hspace*{1.5cm}}
      \subfigure[There is a positive edge on \VAR{load}.]{
         \label{figDev2}\input{dev2.eepic}\hspace*{2.7cm}}}
   \hspace*{0.4cm}
   \centerline{
      \subfigure[The device is busy.]{
         \label{figDev3}\input{dev3.eepic}\hspace*{0.1cm}}
      \subfigure[The computation terminates.]{
         \label{figDev4}\input{dev4.eepic}}\hspace*{0cm}}
      \caption{\label{figDev}A handshake protocol.}
\end{figure}

\vspace*{-6mm}


At the start of a transaction (time~$t$) the device must be outputting
\DEF{T} on $done$ (to indicate it is ready) and the environment must
be asserting \DEF{F} on $load$, i.e.~in a state such that a positive
edge on $load$ can be generated (Figure~\ref{figDev1}).  A~transaction
is initiated by asserting (at time $t{+}1$) the value \DEF{T} on
$load$, i.e.~$load$ has a positive edge at time $t{+}1$. This causes
the device to read the value, $v$ say, being input on $inp$ (at time
$t{+}1$) and to de-assert $done$ (Figure~\ref{figDev2}).  The device
then becomes insensitive to inputs (Figure~\ref{figDev3}) until
\DEF{T} is next asserted on $done$, at which time (say time $t' >
t{+}1$) the value $f(v)$ computed will be output on $out$
(Figure~\ref{figDev4}).

To specify the behaviour of a handshaking device,
the auxiliary predicates \DEF{Posedge} and \DEF{HoldF} are defined.
A positive edge of a signal is defined as the transition of its
value from low to high or, in our case, from \DEF{F} to \DEF{T}. 
The term \mE{\DEF{HoldF}\ (t_1,t_2)\ s} says that a
signal \VAR{s} holds a low value \DEF{F} during a half-open interval
starting at \sVAR{t}{1} to just before \sVAR{t}{2}. The formal definitions are:

\vspace*{-1mm}

\[
\begin{array}{ll}
\TURNST\ \DEF{Posedge}\ s\ t~ &      = ~ \IF{~t{=}0~}{~\DEF{F}~}{~(\NOT\hspace*{0.8mm} s(t{-}1)\ \AND\ s\ t~})\\
\TURNST\ \DEF{HoldF}\ (t_1,t_2)\ s & = ~ \forall t.\ t_1 \leq t < t_2\ \IMP\ \NOT(s\ t)
\end{array}
\]



The behaviour of the handshaking device computing a function $f$ is described by the term 
$\DEF{Dev}\ f\ \VAR{(load,inp,done,out)}$ where:
\[
\begin{array}{l}
\TURNST\ \DEF{Dev}\ f\ \VAR{(load,inp,done,out)} = \\
~~\quad     (\forall t.\ \VAR{done}\ t\ \AND\ \DEF{Posedge}\ \VAR{load}\ (t{+}1)\ \\
\phantom{~~\quad     (\forall t.~} \IMP \\
\phantom{~~\quad     (\forall t.\ ~ } \exists t'.\ t' > t{+}1\ \AND\ \DEF{HoldF}\ (t{+}1,t')\ \VAR{done}\ \AND \\
\phantom{~~\quad     (\forall t.\ ~ \exists t'.\ }  \VAR{done}\ t'\ \AND\ (\VAR{out}\ t' = f (\VAR{inp}\ (t{+}1))))\ ~  \AND \\
~~\quad (\forall t.\ \VAR{done}\ t\ \AND\ \NOT(\DEF{Posedge}\ \VAR{load}\ (t{+}1))\ \IMP\  \VAR{done}\ (t{+}1)) ~\AND \\
~~\quad (\forall t.\ \NOT(\VAR{done}\ t)\ \IMP\ \exists t'.\ t' > t\ \AND \VAR{done}\ t')\\
\end{array}
\]
The first conjunct in the right-hand side describes the context presented
in Figure~\ref{figDev}. If the device is available and a positive
edge occurs on \VAR{load}, there exists a time \VAR{t'} in future
when \VAR{done} signals its termination and the output is produced.
The value of the output at time \VAR{t'} is the result
of applying \VAR{f} to the value of the input at time $\VAR{t}{+}1$.
The signal \VAR{done} holds the value \DEF{F} during the computation.
The second conjunct specifies the situation where no call
is made on \VAR{load} and the device simply remains idle.
Finally, the last conjunct states that if the device
is busy, it will eventually finish its computation
and become idle.

The compiler generates theorems ${\it{Imp}}
\Rightarrow \DEF{Dev}\ f\ \VAR{(load,inp,done,out)}$, where {\it{Imp}}
is a term representing an implementation. In the next section we
describe the various forms that {\it{Imp}} takes during compilation.


%--------------------------------------------------------------------------
\section{Steps of compilation}
\label{secImplementation}
%--------------------------------------------------------------------------
We implement functions $f$ where
$f : \sigma_1\times\cdots\times\sigma_m \rightarrow \tau_1\times\cdots\times\tau_n$
and $\sigma_1,\ldots,\sigma_m,\tau_1,\ldots,\tau_n$ are the types of
values that can be carried on busses (e.g. $n$-bit words).
The starting point of compilation is the definition of such a function $f$ by an equation of
the form: $f(x_1,\ldots,x_n)=e$, where any recursive calls of $f$ in
$e$ must be tail-recursive. Applying \texttt{hwDefine} to such a
definition will first define $f$ in higher order logic (using TFL
\cite{slind:wfrec})  and then generate a theorem $\vdash {\it{Imp_C}}
\Longrightarrow \DEF{Dev}~f$, where ${\it{Imp_C}}$ is a term
constructed using `circuit constructors'.
%\DEF{ATM}, \DEF{SEQ},
%\DEF{PAR}, \DEF{ITE} and \DEF{REC}, which are defined below.  
To support top-down development, ${\it{Imp_C}}$ can also contain terms
$\DEF{Dev}~f_1$, $\ldots$, $\DEF{Dev}~f_p$, where functions $f_1$,
$\ldots$, $f_p$ are not yet implemented. The compilation by deduction
goes through a number of steps.

\vspace*{-3mm}

\subsection*{Step 1: conversion to combinators}

The first step in translating $f(x_1,\ldots,x_n)=e$ is to encode $e$
as an applicative expression, $e_c$ say, built from the operators \DEF{Seq}
(compute in sequence),
\DEF{Par} (compute in parallel), \DEF{Ite} (if-then-else) and \DEF{Rec} (recursion), defined by:
\[
\begin{array}{ll}
\DEF{Seq}~f_1~f_2     &   = \lambda x.~f_2(f_1~x)\\
\DEF{Par}~f_1~f_2     &   = \lambda x.~(f_1~x,~f_2~x)\\
\DEF{Ite}~f_1~f_2~f_3 &   = \lambda x.~ \TT{if}~f_1~x~\TT{then}~f_2~x~\TT{else}~f_3~x\\
\DEF{Rec}~f_1~f_2~f_3 &   = \lambda x.~\TT{if}~f_1~x~\TT{then}~f_2~x~\TT{else}~\DEF{Rec}~f_1~f_2~f_3~(f_3~x)\\
\end{array}
\]
%$\DEF{Rec}~f_1~f_2~f_3$ is defined using Hilbert's
%$\epsilon$-operator, and means ``choose a function $f$ such that $f$
%satisfies the equation $f =
%\lambda x.~\TT{if}~f_1~x~\TT{then}~f_2~x~\TT{else}~f(f_3~x)$''.  
%In
%practise, $f_1$, $f_2$ and $f_3$ will be such that $f$ is uniquely
%determined.  

The encoding into an applicative expression built out of \DEF{Seq},
\DEF{Par}, \DEF{Ite} and \DEF{Rec} is performed by a proof script
coded in the theorem prover's metalanguage, Standard ML, and results
in a theorem $\vdash (\lambda(x_1,\ldots,x_n).~e)~=~e_c$, and hence
$\vdash f=e_c$.  The algorithm used is straightforward and is not
described here. As an example, the proof script deduces from:

\vspace*{-1mm}

\begin{alltt}
\( \vdash \DEF{FactIter}(n,acc) = (\texttt{if} n=0 \texttt{then} (n,acc) \texttt{else} \DEF{FactIter}(n-1,n{\times}acc)) \)
\end{alltt}

\vspace*{-1mm}

\noindent the theorem:

\vspace*{-1mm}

{\baselineskip10pt\begin{alltt}
\( \vdash \DEF{FactIter} =                                                                        \)
\(     \DEF{Rec} (\DEF{Seq} (\DEF{Par} ({\lambda}(n,acc). n) ({\lambda}(n,acc). 0)) ({=}))        \)
\(         (\DEF{Par} ({\lambda}(n,acc). n) ({\lambda}(n,acc). acc))                              \)
\(         (\DEF{Par} (\DEF{Seq} (\DEF{Par} ({\lambda}(n,acc). n) ({\lambda}(n,acc). 1)) ({-}))   \)
\(              (\DEF{Seq} (\DEF{Par} ({\lambda}(n,acc). n) ({\lambda}(n,acc). acc)) ({\times}))) \)
\end{alltt}}

\vspace*{-3mm}

\subsection*{Step 2: implementation using circuit constructors}

The next step is to replace the combinators \DEF{Seq},
\DEF{Par}, \DEF{Ite} and \DEF{Rec} with corresponding circuit constructors
\DEF{SEQ},
\DEF{PAR}, \DEF{ITE} and \DEF{REC} that build handshaking devices (see the appendix for their definitions).
The key property of these constructors are the following theorems that
enable us to compositionally deduce theorems of the form $\vdash
{\it{Imp_C}}
\Longrightarrow \DEF{Dev}~f$, where ${\it{Imp_C}}$ is a term
constructed using  the circuit constructors, and hence is a handshaking device
(the long implication symbol $\Longrightarrow$ denotes
implication lifted to functions
$f \Longrightarrow g~=~\forall x.~f(x)\Rightarrow g(x)$): 

\vspace*{-3mm}
$$\begin{array}{l}
~\TURNST\ \DEF{Dev}\ f\ \FIMP \  \DEF{Dev}\ f\\[1mm]


~\TURNST\
(P_1\ \FIMP\ \DEF{Dev}~f_1)~\AND~(P_2 \ \FIMP\ \DEF{Dev}~f_2)\\
\phantom{~\TURNST~}
 \IMP ~
(\DEF{SEQ}\ P_1\ P_2 \ \FIMP \ \DEF{Dev}\ (\DEF{Seq}~f_1~f_2))\\[1mm]


~\TURNST\ 
(P_1 \ \FIMP\ \DEF{Dev}~f_1)~\AND~(P_2 \ \FIMP\ \DEF{Dev}~f_2)\\
\phantom{~\TURNST~}
\IMP ~
(\DEF{PAR}\ P_1\ P_2\ \FIMP \ \DEF{Dev}\ (\DEF{Par}~f_1~f_2))\\[1mm]


~\TURNST\ 
(P_1 \ \FIMP\ \DEF{Dev}~f_1)~\AND~(P_2 \ \FIMP\ \DEF{Dev}~f_2)~\AND~(P_3 \ \FIMP\ \DEF{Dev}~f_3)\\
\phantom{~\TURNST~}
\IMP ~
(\DEF{ITE}\ P_1\ P_2\ P_3\  \FIMP \ \DEF{Dev}\ (\DEF{Ite}~f_1~f_2~f_3))\\[1mm]


~\TURNST\ 
\DEF{Total}(f_1,f_2,f_3)\\
\phantom{~\TURNST~}
\IMP ~
(P_1 \ \FIMP\ \DEF{Dev}~f_1)~\AND~(P_2 \ \FIMP\ \DEF{Dev}~f_2)~\AND~(P_3 \ \FIMP\ \DEF{Dev}~f_3)\\
\phantom{~\TURNST~}
\IMP ~
(\DEF{REC}\ P_1\ P_2\ P_3 \ \FIMP \ \DEF{Dev}\ (\DEF{Rec}~f_1~f_2~f_3))\\

\end{array}$$
where $\DEF{Total}(f_1,f_2,f_3)$ is a predicate ensuring that the
specified recursive function terminates.  
%there is a unique function satisfying $f = \lambda
%x.~\TT{if}~f_1~x~\TT{then}~f_2~x~\TT{else}~f(f_3~x)$.
%\[
%\DEF{Total}(f_1,f_2,f_3)~=~\exists variant.~\forall x.~\neg(f_1~x)
%                                        \IMP variant(f_3~x) < variant~x
%\]

If $e_c$ is an expression built using \DEF{Seq},\DEF{Par}, \DEF{Ite}
and \DEF{Rec} then, by suitably instantiating the predicate variables
$P_1$, $P_2$ and $P_3$, these theorems allow us to construct an
expression $e_C$ built from circuit constructors \DEF{SEQ}, 
\DEF{PAR}, \DEF{ITE} and \DEF{REC} such that $\vdash e_C \FIMP
\DEF{Dev}~e_c$. From Step~1 we have $\vdash f=e_c$, hence 
$\vdash e_C \FIMP \DEF{Dev}~f$

A function $f$ which is combinational (i.e.~can be implemented
directly with logic gates without using registers) can be packaged as
a handshaking device using a constructor \DEF{ATM}, which creates a
simple handshake interface and satisfies the refinement theorem:

\vspace*{-4mm}
$$\begin{array}{l}
%\texttt{ATM\_INTRO}\\
~\TURNST\ \DEF{ATM}\ f\ \FIMP \  \DEF{Dev}\ f\\
\phantom{~~~~~~~~~~~~~~~~~~~~~~~~~~~~~~~~~~~~~~~~~~~~~~~~~~~~~~~~~~~~~~~~~~~~~~~~~~~~~~~~~~~~~~~~~~~~~~~~~~~~~}\\[-4mm]
\end{array}$$

\noindent The circuit constructor \DEF{ATM} is defined with the other constructors in the appendix.
To avoid a proliferation  of internal handshakes, when the proof script that constructs $e_C$ from $e_c$ 
is implementing $\DEF{Seq}~f_1~f_2$, it checks to see whether $f_1$ or $f_2$ 
are compositions of combinational functions and if so introduces \DEF{PRECEDE} or \DEF{FOLLOW} instead of \DEF{SEQ},
using the theorems:

\vspace*{-4mm}
$$\begin{array}{l}
~\TURNST\
      (P\ \FIMP\ \DEF{Dev}~f_2)~\IMP~(\DEF{PRECEDE}~f_1~P\ \FIMP\ \DEF{Dev}\ (\DEF{Seq}~f_1~f_2))\\
\phantom{~~~~~~~~~~~~~~~~~~~~~~~~~~~~~~~~~~~~~~~~~~~~~~~~~~~~~~~~~~~~~~~~~~~~~~~~~~~~~~~~~~~~~~~~~~~~~~~~~~~~~~}\\[-4mm]
~\TURNST\
(P\ \FIMP\ \DEF{Dev}~f_1)~\IMP~(\DEF{FOLLOW}~P~f_2\  \FIMP\ \DEF{Dev}\ (\DEF{Seq}~f_1~f_2))\\
\end{array}$$


\noindent $\DEF{PRECEDE}~f~d$ processes inputs with $f$ before sending them to $d$ and
$\DEF{FOLLOW}~d~f$ processes outputs of $d$ with
$f$. The definitions are:

\vspace*{3mm}

$\begin{array}{l}
\DEF{PRECEDE}~f~d~(\VAR{load},inp,done,out)  =\\
~~ \exists v.~ \DEF{COMB}~f~(inp,v) ~\wedge~ d(\VAR{load},v,done,out)\\
 \\
\DEF{FOLLOW}~d~f~ (\VAR{load},inp,done,out)  =\\
~~ \exists v.~d(\VAR{load},inp,done,v) ~\wedge~ \DEF{COMB}~f~(v,out)
\end{array}$
\vspace*{3mm}

The construction $\DEF{SEQ}~d_1~d_2$ introduces a handshake between the executions
of $d_1$ and $d_2$, but $\DEF{PRECEDE}~f~d$ and $\DEF{FOLLOW}~d~f$
just `wire' f before or after $d$, respectively, without introducing a
handshake. 

\vspace*{-3mm}

\subsection*{Step 3: unfolding to a cycle-level netlist}

The result of Step~2 is a theorem
$\vdash e_C \FIMP \DEF{Dev}~f$ where $e_C$ is an expression built out of the circuit
constructors \DEF{ATM}, 
\DEF{SEQ}, \DEF{PAR}, \DEF{ITE}, \DEF{REC}, \DEF{PRECEDE} and \DEF{FOLLOW}. If the theorem
is rewritten with the definitions of these constructors (see their
definitions in the appendix) we get a circuit built out of standard
kinds of gates (\DEF{AND}, \DEF{OR}, \DEF{NOT} and \DEF{MUX}), a
generic combinational component $\DEF{COMB}~g$ (where $g$ will be a
function represented as a HOL $\lambda$-expression) and 
\DEF{DEL}, \DEF{DELT} and \DEF{DFF}. 

The next phase of compilation converts terms of the form $\DEF{COMB}~g~(inp,out)$
into circuits built only out of components that it is assumed can be directly realised in
hardware. Such components  currently include boolean functions (e.g. $\wedge$,
$\vee$ and $\neg$), multiplexers and simple operations  on $n$-bit words (e.g.~versions
of $+$, $-$ and $<$, various shifts etc.). 
A special purpose proof rule uses a straightforward recursive algorithm to synthesise
combinational circuits. For example:

%\vspace*{-2mm}

{\begin{alltt}
\( \vdash \DEF{COMB}\ ({\lambda}(m,n). (m<n, m+1))\ (inp1\texttt{<>}inp2,out1\texttt{<>}out2) =              \)
\(     {\exists}v0. \DEF{COMB}\ ({<})\ (inp1\texttt{<>}inp2,out1) \wedge \DEF{CONSTANT} 1 v0 \wedge \)
\(         \! \DEF{COMB}\ ({+})\ (inp1\texttt{<>}v0,out2)                                              \)
\end{alltt}}

\vspace*{-1mm}

\noindent where \texttt{<>} is bus concatenation,
$\DEF{CONSTANT}~1~v0$ drives $v0$ high continuously, and
$\DEF{COMB}~{<}$ and $\DEF{COMB}~{+}$ are assumed
given components (if they were not given, then the could be
implemented explicitly, but one has to stop somewhere). 
 
The abstract registers \DEF{DEL}, \DEF{DELT} and \DEF{DFF} used in the definitions of
the circuit constructs are defined by:

\vspace*{-1mm}

{\begin{alltt}
\( \vdash \DEF{DEL}(inp,out)  = {\forall}t. out(t+1) = inp t                       \)
\( \vdash \DEF{DELF}(inp,out) = (out 0 = \DEF{T}) \wedge \DEF{DEL}(inp,out)                       \)
\( \vdash \DEF{DFF}(d,\VAR{load},q) = {\forall}t. q(t+1) = \)
\(                        \texttt{if} \DEF{POSEDGE} \VAR{load} (t+1) \texttt{then} d(t+1) \texttt{else} q t \)
\end{alltt}}

\vspace*{-1mm}

These definitions were chosen for convenience in defining
\DEF{ATM}, \DEF{SEQ}, \DEF{PAR}, \DEF{ITE} and \DEF{REC},
however \DEF{DFF} is easily defined in terms of
\DEF{DEL}, \DEF{DELT} and some combinational logic (details omitted).

\vspace*{-3mm}

\subsection*{Step 4: temporal refinement to a clocked synchronous circuit}

At the end of Step~3 one has a circuit built out of standard gates,
operations $\texttt{COMB}~op$ (where $op$ has a known implementation as
a combinational circuit) and \DEF{DEL} and \DEF{DELT}.  The
next step is to replace these abstract `cycle level' registers with
standard clocked synchronous parts. Up until now we have viewed
signals as abstract sequences of values.  We now move to a finer
temporal granularity in which values are sampled on the edge of a
clock; such values can be latched by edge-triggered registers. We
implement \DEF{DEL} and \DEF{DELT} in terms of a standard dtype
register defined by:

\vspace*{1mm}

$\DEF{Dtype}(ck,d,q) ~=~ \forall t.~q(t{+}1) = \texttt{if}~\DEF{Posedge}~ck~(t{+}1)~\texttt{then}~d~t~\texttt{else}~q~t$

\vspace*{1mm}

\noindent and a boolean version of this register that initialises to a state holding $\DEF{T}$ (i.e.~$1$):

\vspace*{1mm}

$\DEF{FlipFlopT}(ck,d,q) ~ = ~ (q~0~ =~\DEF{T})~\wedge~\DEF{Dtype}(ck,d,q)$

\vspace*{1mm}

If $s$ is a signal, then $s~\DEF{at}~\VAR{clk}$ is the temporal
abstraction consisting of the values of $s$ at positive edges of
$\VAR{clk}$. 
Melham shows in his monograph that:

\vspace*{-1mm}


{\baselineskip14pt\begin{alltt}
\( \vdash \DEF{InfRise} \VAR{clk} \Rightarrow {\forall}d q. \DEF{Dtype}(\VAR{clk},d,q) \Rightarrow \DEF{DEL}(d \DEF{at} \VAR{clk}, q \DEF{at} \VAR{clk}) \)
\end{alltt}}

\vspace*{-1mm}

\noindent and it is easy to then derive:

\vspace*{-1mm}

{\baselineskip14pt\begin{alltt}
\( \vdash \DEF{InfRise} \VAR{clk} \Rightarrow {\forall}d q. \DEF{FlipFlopT}(\VAR{clk},d,q) \Rightarrow \DEF{DELT}(d \DEF{at} \VAR{clk}, q \DEF{at} \VAR{clk})  \)
\end{alltt}}

\vspace*{-1mm}

By instantiating $\VAR{load}$, $inp$, $done$ and $out$ in the theorem
obtained by Step~3 to $\VAR{load}~\DEF{at}~\VAR{clk}$, $inp~\DEF{at}~\VAR{clk}$,
$done~\DEF{at}~\VAR{clk}$ and $out~\DEF{at}~\VAR{clk}$, respectively, and then
performing some deductions involving Melham's theorems (and the
monotonicity of existential quantification and conjunction with
respect to implication) we obtain a theorem

\vspace*{1mm}

$\vdash \DEF{InfRise}~\VAR{clk}~\IMP~{\it{Imp}}~\IMP~\DEF{Dev}~f~(\VAR{load}~\DEF{at}~\VAR{clk},inp~\DEF{at}~\VAR{clk},done~\DEF{at}~\VAR{clk},out~\DEF{at}~\VAR{clk})$.

\vspace*{1mm}

The term ${\it{Imp}}$ represents the compiled implementation of $\DEF{Dev}~f$ in higher order logic.
It differs from ${\it{Imp}}_C$ produced by Step~3 in having an explicit clock and using
registers $\DEF{Dtype}$ and $\DEF{FlipFlopF}$ rather than \texttt{DEL} and \texttt{DELT}.

\vspace*{-3mm}

\subsection*{Step 5: translation to Verilog}

The implementation ${\it{Imp}}$ produced by Step~4 is a circuit
represented in a standard way (as described, for example, in Melham's book)
as a conjunction of predicates representing primitive components,
with internal lines existentially quantified.

This can be `pretty printed' directly into Verilog. Unfortunately
Verilog is not a semantically formal notation, so we cannot deduce
Verilog by proof. However, there is a direct and transparent
correspondence between the structure of ${\it{Imp}}$ and the Verilog
text that is printed.

%--------------------------------------------------------------------------
\section{Case studies}
\label{secCaseStudy}
%--------------------------------------------------------------------------


We are in the midst of two case-studies: a Booth multiplier, and an implementation
of the AES encryption algorithm.

As part of a project to verify an ARM processor \cite{Fox02}, a high
level model of the multiplication algorithm used by some ARM
implementations was created in higher order logic. This is a Booth
multiplier and we are using Fox's existing specification as an example
for testing our compiler. 


A more substantial example, being done at the University of Utah, is
implementing the Advanced Encryption Standard (AES) \cite{AES}
algorithm for private-key encryption. This specifies a multi-round
algorithm with primitive computations based on finite field
operations.  Starting from an existing formalization of AES
\cite{slind:aes}, we have generated netlists and circuits for the major
components of an encryption (and decryption) round.  Although out work on AES
is incomplete, our current progress provides several illustrations of our
synthesis methodology.

The AES formalization includes a proof of functional correctness  for the
algorithm: specifically, encryption and decryption are inverse functions.
Deriving the hardware from the proven specification using logical inference
assures us that the hardware encrypter is the inverse of the hardware
decrypter.

An encryption round performs the following transformations on a 4-by-4 matrix
of input bytes:
\begin{enumerate}
\item
application of \emph{sbox}, an invertible function from bytes to bytes,
to each byte;
\item
a cyclical shift of each row;
\item
multiplication of each column by a fixed degree 3 polynomial, with coefficients
in the 256 element finite field, GF($2^8$);
\item
adding a key to the matrix with exclusive or.
\end{enumerate}

%\begin{figure}
%\begin{alltt}
%ShiftRows (b00,b01,b02,b03,b10,b11,b12,b13,b20,b21,b22,b23,b30,b31,b32,b33) =
%          (b00,b01,b02,b03,b11,b12,b13,b10,b22,b23,b20,b21,b33,b30,b31,b32)
%\end{verbatim}
%\begin{verbatim}
%|- (?v0 v1 v2.
%      DEL (load,v2) /\ NOT (v2,v1) /\ AND (v1,load,v0) /\
%      NOT (v0,done) /\ DEL (inp1,out1) /\ DEL (inp2,out2) /\
%      DEL (inp3,out3) /\ DEL (inp4,out4) /\ DEL (inp6,out5) /\
%      DEL (inp7,out6) /\ DEL (inp8,out7) /\ DEL (inp5,out8) /\
%      DEL (inp11,out9) /\ DEL (inp12,out10) /\ DEL (inp9,out11) /\
%      DEL (inp10,out12) /\ DEL (inp16,out13) /\ DEL (inp13,out14) /\
%      DEL (inp14,out15) /\ DEL (inp15,out16)) ==>
%    DEV ShiftRows (load, inp1 <> ... inp16, done, out1 <> ... <> out16) 
%\end{verbatim}
%\caption{Row shifting}
%\label{AES:shift}
%\end{figure}

Many of our specifications are not tail recursive, but formally
deriving (and verifying) tail-recursive versions was
straightforward. We were then able to generate circuits directly by invoking
the compiler.  As illustrated in
Section~\ref{secHOL2Verilog}, we explored various options for
generating components either as separate handshaking designs or expanding
them into combinational logic. We have also explored converting our
high level recursive specification of multiplication into a table
lookup. The resulting verified tables can then be stored into a RAM or
ROM device.  For synthesizing the tables directly into hardware, we
have automated the definition of a function on bytes as a balanced
\texttt{if} expression, branching on each successive bit of its input.

\vspace*{-2mm}

{\footnotesize\baselineskip10pt\begin{verbatim}
0xB ** x = if WORD_BIT 7 x then
             if WORD_BIT 6 x then 
               ...
                       if WORD_BIT 0 then 0xA3 else 0xA8
               ...
           else
             if WORD_BIT 6 x then
               ...
\end{verbatim}}

\vspace*{-2mm}

Our experience so far is positive: compiling implementations by
deduction provides a secure and flexible framework for creating and
optimising designs.

%--------------------------------------------------------------------------
\section{Related work}
\label{secRelatedWork}
%--------------------------------------------------------------------------
Previous approaches to combine theorem provers 
and formal synthesis established an analogy between
the goal directed proof technique and an interactive 
design process. In LAMBDA, the user starts from the behavioural
specification and builds the circuit incrementally
by adding primitive hardware components
which automatically simplify the goal \cite{Fou89}.
Hanna {\em et al.\/} \cite{HLD89} introduce
several {\em techniques\/} (functions) that
simplify the current goal into simpler subgoals.
Techniques are adaptations to hardware design
of {\em tactics\/} in LCF.

Alternative approaches synthesise circuits
by applying semantic-preserving transformations
to their specifications. For instance,
the Digital Design Derivation (DDD) transforms
finite-state machines specified in terms of
tail-recursive lambda abstractions into hierarchical
boolean systems~\cite{Johnson90}. Lava and Hydra
are both hardware description languages embedded
in Haskell whose programs
consist of definitions of gates and their 
connections (netlists)~\cite{BCSS99,OD02}. While Lava interfaces with
external theorem provers to verify its circuits,
Hydra designers can synthesise them
via formal equational reasoning
(using definitions and lemmas from functional programming).
The functional languages $\mu$FP and Ruby
adopt similar principles in hardware design~\cite{JS90,She84}.
The circuits are defined in terms of primitive
functions over booleans, numbers and lists, and
higher-order functions, the {\em combining forms\/},
which compose hardware blocks in different
structures. Their mathematical properties provide
a calculational style in design exploration.

These approaches deal with an interactive
synthesis at the gate or state-machine level
of abstraction only. Moreover, the synthesis
and the proof of correctness require a 
substantial user guidance. Gropius and SAFL 
are two related works that address these issues.

%------ Gropius: Blumenrohr
Gropius is a hardware description language
defined as a subset of HOL~\mbox{\cite{Blu01,Gropius1}}.
Its algorithmic level
provides control structures like if-then-else,
sequential composition and while loop.
The~atomic commands are DFGs (data flow
graphs) represented by lambda abstractions.
The compiler initially combines every while loop into
a single one at the outermost level of the
program:
\[
\DEF{PROGRAM}\ \VAR{out\_default}\ (\DEF{LOCVAR}\ \VAR{vars}\ 
(\DEF{WHILE}\ c\ (\DEF{PARTIALIZE}\ \VAR{b})))
\]
The body \VAR{b} of the \DEF{WHILE} loop is an acyclic~DFG.
The list \VAR{out\_default}
provides initial values for the output variables.
The term \DEF{LOCVAR} declares the local variables
\VAR{vars} and \DEF{PARTIALIZE} converts a
non-recursive (terminating) DFG into a potentially
non-terminating command.
The compiler then synthesises a handshaking
interface which encapsulates this program.
Each of these hardware blocks are now regarded
as primitive blocks or {\em processes\/} at the system level.
Processes are connected via communication
units ({\em k-processes\/}) which implement delay,
synchronisation, duplication,
splitting and joining of a process output data
(actually there  are 10 different k-processes \cite{Blu01}).
Although the synthesis produces the proof of
correctness of each process and k-process,
the correctness of the top-level system is not generated.
The reason for that is mainly because
the top-level interface of a network of processes and 
k-processes does not match the handshaking interface pattern.

%------ SAFL: Sharp
Our compilation method is partly inspired by SAFL 
(Statically Allocated Functional Language) \cite{MS01b},
especially the ideas in Richard Sharp's PhD \cite{Sha02}. 
SAFL~is a first-order functional language whose
programs consist of a sequence of tail-recursive function 
definitions. Its high-level of abstraction allows
the exploitation of powerful program analyses
and optimisations not available in traditional
synthesis systems. 
However, the synthesis is not based on
the correct-by-construction principles
and the compiler has not been verified.

% our approach
The novelty of our approach is the 
compilation of functional programs by
composing especially designed and
pre-verified circuit constructors.
%
As each of these circuit constructors
has the key property of implementing
a device that computes precisely their
corresponding combinators, 
the verification and the compilation of 
functional programs can be done
automatically.


%--------------------------------------------------------------------------
\section{Current State and Future work}
\label{secFutureWork}
%--------------------------------------------------------------------------


A first version of the compiler described here has only just been
completed and is still being tested by simulating circuits and
viewing the resulting waveforms. We use Icarus Verilog and GTKWave,
whith are both public domain.  The source code for our compiler and
the current state of various examples (including the Booth multiplier
and AES) can be inspected at
SourceForge.\footnote{\url{http://cvs.sourceforge.net/viewcvs.py/hol/hol98/examples/dev/}}


In the immediate future we plan to ruggedise and debug the existing
code and continue and complete the case studies described in
Section~\ref{secCaseStudy}. An important part of these case studies is
examining the output of our compiler and developing improvements to
ensure that it produces reasonable quality designs.

At present all data-refinement (e.g.~from numbers or enumerated types
to words) must be done manually, by proof in higher order logic. The
HOL4 system has some `boolification' facilities that automatically
translate higher level data-types into bit-strings, and we hope to
integrate these into the compiler.

We want to investigate using the compiler to generate test-bench
monitors that can run in parallel simulation with designs that are not
correct by construction.  Thus our hardware can act as a ``golden''
reference against which to test other implementations.

The work described here is part of a bigger project to create
hardware/software combinations by proof.  We hope to investigate the
option of creating software for ARM processors and linking it to
    hardware created by our compiler (possibly packaged as an ARM
co-processor). Our emphasis is likely to be on cryptographic hardware
and software, because there is a clear need for high assurance of
correct implementation in this domain.




%--------------------------------------------------------------------------
\bibliographystyle{plain}
\bibliography{tphols2005}
%--------------------------------------------------------------------------

%--------------------------------------------------------------------------
%\section*{Appendix: The circuit constructors}
%\label{CircuitConstructors}
%--------------------------------------------------------------------------
\newpage
%--------------------------------------------------------------------------
\section*{Appendix: The circuit constructors}
\label{CircuitConstructors}
%--------------------------------------------------------------------------
\vspace*{-3mm}

The formal definitions in higher order logic of the circuit
constructors and the components used in their definitions are defined
below, followed by circuit diagrams of the constructors.

\begin{footnotesize}

\medskip
$
\begin{array}{l}
\TURNST\ \DEF{AND}\ (\sVAR{in}{1},\sVAR{in}{2},\VAR{out}) ~ = 
   \forall t.\ \VAR{out}\ t = (\sVAR{in}{1}\ t\ \AND\ \sVAR{in}{2}\ t)\\
\TURNST\ \DEF{OR}\ (\sVAR{in}{1},\sVAR{in}{2},\VAR{out}) ~ =
   \forall t.\ \VAR{out}\ t = (\sVAR{in}{1}\ t\ \OR\ \sVAR{in}{2}\ t)\\
\TURNST\ \DEF{NOT}\ (\VAR{inp},\VAR{out}) ~ = 
   \forall t.\ \VAR{out}\ t = \NOT(\VAR{inp}\ t)\\
\TURNST\ \DEF{MUX} (\VAR{sw,in_1,in_2,out})~ =~
 \forall t.\ out~t =  \IF{sw~t}{in_1~t}{in_2~t}\\
\TURNST\ \DEF{COMB}\ f\ \VAR{(inp,out)} ~ = \forall t.\ \VAR{out}\ t = f (\VAR{inp}\ t)\\
\TURNST\ \DEF{DEL}\ (\VAR{inp},\VAR{out}) ~ = \forall t.\ \VAR{out} (t{+}1) = \VAR{inp}\ t\\
\TURNST\ \DEF{DELT}\ (\VAR{inp},\VAR{out}) ~ = 
    (\VAR{out}\ 0 = \DEF{T})\ \AND\ 
    \forall t.\ \VAR{out} (t{+}1) = \VAR{inp}\ t\\
\TURNST\ \DEF{DFF} (\VAR{d,clk,q})~ =~
 \forall t.\ q (t{+}1) =  \IF{\DEF{Posedge}\ \VAR{clk}\ (t{+}1)}{d (t{+}1)}{q\ t}
\end{array}
$
 
\smallskip

$
\mbox{}\hspace*{-0.4mm}\TURNST\ \DEF{POSEDGE} (\VAR{inp},\VAR{out}) = 
    \exists c_0~ c_1.\ \DEF{DELT}(\VAR{inp},c_0) \AND
                  \DEF{NOT}(c_0, c_1) \AND \DEF{AND}(c_1,\VAR{inp},\VAR{out})
$

\smallskip


$
\begin{array}{l}
\TURNST\ \DEF{ATM}\ f\ (\VAR{load,inp,done,out}) =\\
\phantom{\TURNST\ ~~}    \exists c_0~ c_1.\ \DEF{POSEDGE} (\VAR{load}, c_0)\ \AND \ 
\DEF{NOT} (c_0, \VAR{done})\ \AND\\
\phantom{\TURNST\ ~~\exists c_0, c_1.\ }  \DEF{COMB}\ f\ (\VAR{inp},c_1)\ \AND\ \DEF{DEL} (c_1,\VAR{out})
\end{array}
$

\smallskip

$
\begin{array}{l}
\TURNST\ \DEF{SEQ}\ f\ g\ (\VAR{load,inp,done,out}) = \\
\phantom{\TURNST\ ~~} \exists c_0~ c_1~ c_2~ c_3~ \VAR{data}.\\
\phantom{\TURNST\ ~~\exists~}                      \DEF{NOT} (c_2,c_3) \ \AND \ 
                      \DEF{OR} (c_3,\VAR{load},c_0) \ \AND \  f (c_0,\VAR{inp},c_1,\VAR{data}) \ \AND \\
\phantom{\TURNST\ ~~ \exists~} 
        g (c_1,\VAR{data},c_2,\VAR{out})\ \AND\ 
        \DEF{AND} (c_1,c_2,\VAR{done}) 
\end{array}
$

\smallskip

$
\begin{array}{l}
\TURNST\ \DEF{PAR}\ f\ g\ (\VAR{load,inp,done,out}) = \\
\phantom{\TURNST\ ~}     \exists c_0~c_1~\VAR{start}~\sVAR{done}{1}~\sVAR{done}{2}~
                                 \sVAR{data}{1}~\sVAR{data}{2}~\sVAR{out}{1}~\sVAR{out}{2}.\\
\phantom{\TURNST\ ~~ \exists~}
       \DEF{POSEDGE} (\VAR{load},c_0)\ \AND\  
       \DEF{DEL} (\VAR{done},c_1)\ \AND \ 
       \DEF{AND} (c_0,c_1,\VAR{start})\ \AND\\
\phantom{\TURNST\ ~~ \exists~}
       f (\VAR{start},\VAR{inp},\sVAR{done}{1},\sVAR{data}{1})\ \AND \ 
       g (\VAR{start},\VAR{inp},\sVAR{done}{2},\sVAR{data}{2})\ \AND\\
\phantom{\TURNST\ ~~ \exists~}
       \DEF{DFF} (\sVAR{data}{1},\sVAR{done}{1},\sVAR{out}{1})\ \AND \  
       \DEF{DFF} (\sVAR{data}{2},\sVAR{done}{2},\sVAR{out}{2})\ \AND\\
\phantom{\TURNST\ ~~ \exists~}
       \DEF{AND} (\sVAR{done}{1},\sVAR{done}{2},done)\ \AND \ 
       (\VAR{out} = \LAMBDA{t}{(\sVAR{out}{1}\ t,\sVAR{out}{2}\ t)})
\end{array}
$

\smallskip

$
\begin{array}{l}
\TURNST\ \DEF{ITE}\ e\ f\ g\ (\VAR{load,inp,done,out}) =\\
\phantom{\TURNST\ ~}
   \exists c_0~c_1~c_2~\VAR{start}~\VAR{start'}~\VAR{done\_e}~\VAR{data\_e}~q~ 
                          \VAR{not\_e}~\VAR{data\_f}~\VAR{data\_g}~\VAR{sel}\\
\phantom{\TURNST\ ~\exists }
                  \VAR{done\_f}~\VAR{done\_g}~
                          \VAR{start\_f}~\VAR{start\_g}.\ \\
\phantom{\TURNST\ ~\exists ~}
           \DEF{POSEDGE} (\VAR{load},c_0)\ \AND\
           \DEF{DEL} (\VAR{done},c_1)\ \AND\ \DEF{AND} (c_0,c_1,\VAR{start})\ \AND \\
\phantom{\TURNST\ ~\exists ~}
           e (\VAR{start},\VAR{inp},\VAR{done\_e},\VAR{data\_e})\ \AND\
           \DEF{POSEDGE} (\VAR{done\_e},\VAR{start'})\ \AND \\
\phantom{\TURNST\ ~\exists ~}
           \DEF{DFF} (\VAR{data\_e},\VAR{done\_e},\VAR{sel})\ \AND\
           \DEF{DFF} (\VAR{inp},\VAR{start},q)\ \AND \\
\phantom{\TURNST\ ~\exists ~}
           \DEF{AND} (\VAR{start'},\VAR{data\_e},\VAR{start\_f})\ \AND\
           \DEF{NOT} (\VAR{data\_e},\VAR{not\_e})\ \AND \\
\phantom{\TURNST\ ~\exists ~}
           \DEF{AND} (\VAR{start'},\VAR{not\_e},\VAR{start\_g})\ \AND\
           f (\VAR{start\_f},q,\VAR{done\_f},\VAR{data\_f})\ \AND\\
\phantom{\TURNST\ ~\exists ~}
           g (\VAR{start\_g},q,\VAR{done\_g},\VAR{data\_g})\ \AND\
           \DEF{MUX} (\VAR{sel},\VAR{data\_f},\VAR{data\_g},out)\ \AND \\
\phantom{\TURNST\ ~\exists ~}
           \DEF{AND} (\VAR{done\_e},\VAR{done\_f},c_2)\ \AND\
           \DEF{AND} (c_2,\VAR{done\_g},\VAR{done}) 
\end{array}
$

\smallskip

$
\begin{array}{l}
\TURNST\ \DEF{REC}\ e\ f\ g\ (\VAR{load,inp,done,out}) = \\
\phantom{\TURNST\ ~}
     \exists \VAR{done\_g}~ \VAR{data\_g}~ \VAR{start\_e}~ q~ \VAR{done\_e}~ 
             \VAR{data\_e}~ \VAR{start\_f}~ \VAR{start\_g}~ \VAR{inp\_e}~ 
             \VAR{done\_f}\\[-1mm]
\phantom{\TURNST\ ~\exists }
             \sVAR{c}{0}~ \sVAR{c}{1}~ 
             \sVAR{c}{2}~ \sVAR{c}{3}~\sVAR{c}{4}~
             \VAR{start}~ \VAR{sel}~ \VAR{start'}~ \VAR{not\_e}. \\[0.5mm]
\phantom{\TURNST\ ~\exists~ }
        \DEF{POSEDGE}(\VAR{load},\sVAR{c}{0})\ \AND\
        \DEF{DEL}(\VAR{done},\sVAR{c}{1})\ \AND\
        \DEF{AND}(\sVAR{c}{0},\sVAR{c}{1},\VAR{start})\ \AND \\
\phantom{\TURNST\ ~\exists~ }
        \DEF{OR}(\VAR{start,sel,start\_e})\ \AND\
        \DEF{POSEDGE}(\VAR{done\_g,sel})\ \AND \\
\phantom{\TURNST\ ~\exists~ }
        \DEF{MUX}(\VAR{sel,data\_g,inp,inp\_e})\ \AND\
           \DEF{DFF}(\VAR{inp\_e},\VAR{start\_e},q)\ \AND \\
\phantom{\TURNST\ ~\exists~ }
           e (\VAR{start\_e},\VAR{inp\_e},\VAR{done\_e},\VAR{data\_e}) \AND\
        \DEF{POSEDGE}(\VAR{done\_e,start'})\ \AND \\
\phantom{\TURNST\ ~\exists~ }
        \DEF{AND}(\VAR{start',data\_e,start\_f})\ \AND\
        \DEF{NOT}(\VAR{data\_e,not\_e})\ \AND \\
\phantom{\TURNST\ ~\exists~ }
        \DEF{AND}(\VAR{not\_e,start',start\_g})\ \AND\
           f (\VAR{start\_f},q,\VAR{done\_f},\VAR{out})\ \AND\\
\phantom{\TURNST\ ~\exists~ }
           g (\VAR{start\_g},q,\VAR{done\_g},\VAR{data\_g})\ \AND\
        \DEF{DEL}(\VAR{done\_g},\sVAR{c}{3})\ \AND\\
\phantom{\TURNST\ ~\exists~ }
        \DEF{AND}(\VAR{done\_g},\sVAR{c}{3},\sVAR{c}{4})\ \AND\
        \DEF{AND}(\VAR{done\_f},\VAR{done\_e},\sVAR{c}{2})\ \AND\
        \DEF{AND}(\sVAR{c}{2},\sVAR{c}{4},\VAR{done})
\end{array}
$

\end{footnotesize}

%\fbox{Can we put the diagrams back so that they all fit onto the rest of this page?}

\begin{figure}[htb]
   \centerline{
      \subfigure[$\DEF{ATM}\ f$]{
         \label{figAtom}\input{atm.eepic}\hspace*{0cm}}
      \subfigure[$\DEF{SEQ}\ f\ g$]{
         \label{figSeq}\input{seq.eepic}\hspace*{0cm}}
      \subfigure[$\DEF{PAR}\ f\ g$]{
         \label{figPar}\input{par.eepic}}}
      \caption{\label{figSeqPar}Implementation of composite devices.}
\end{figure}
\begin{figure}[htb]
   \centerline{
       \subfigure[$\DEF{IF}\ e\ f\ g$]{
           \label{figIf}\input{ifReduced.eepic}\hspace*{0.3cm}}
       \subfigure[$\DEF{REC}\ e\ f\ g$]{
           \label{figRec}\input{recReduced.eepic}\hspace*{0cm}}}
   \caption{\label{figIfRec}The conditional and the recursive constructors.}
\end{figure}

\end{document}
% LocalWords:  langle


