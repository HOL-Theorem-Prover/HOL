\section{Theories}
\label{theories}

The result of a session with the \HOL\ system is an object called a
{\it theory\/}.  This object is closely related to what a logician
would call a theory, but there are some differences arising from the
needs of mechanical proof.  A \HOL\ theory, like a logician's theory,
contains sets of types, constants, definitions and axioms.  In
addition, however, a \HOL\ theory contains an explicit list of
theorems that have been proved from the axioms and definitions.
Logicians normally do not need to distinguish theorems that have
actually been proved from those that could be proved, hence they do
not normally consider sets of proven theorems as part of a theory;
rather, they take the theorems of a theory to be the (often infinite)
set of all consequences of the axioms and definitions.  Another
difference between logicians' theories and \HOL\ theories is that, for
logicians, theories are relatively static objects, but in \HOL\ they
can be thought of as potentially extendable. For example, the \HOL\
system provides tools for adding to theories and combining theories.
A typical interaction with \HOL\ consists in combining some existing
theories, making some definitions, proving some theorems and then
saving the resulting new theory.

The purpose of the \HOL\ system is to provide tools to enable
well-formed theories to be constructed.  All the theorems of such
theories are logical consequences of the definitions and axioms of the
theory.  The \HOL\ system ensures that only well-formed theories can
be constructed by allowing theorems to be created by {\it formal
  proof\/} only.

A theory is represented in the \HOL\ system as a collection of SML
object files, called theory files.  Each file has a name of the form
$name$\ml{Theory.}$ext$, where $name$ is a string supplied by the
user, and $ext$ is one of \ml{sig}, \ml{sml}, \ml{ui}, \ml{uo}.

Theory files are structured hierarchically to represent sequences of
extensions of an initial theory called \ml{scratch}.  Each theory file
making up a theory records some types, constants, axioms and theorems,
together with pointers to other theory files called its {\it
  parents\/}.  This collection of reachable files is called the {\it
  ancestry\/} of the theory file. Axioms, definitions and theorems are
named in the \HOL\ system by two strings: the name of the theory file
where they are stored, together with a name within that file supplied
by the user.  Specifically, axioms, definitions and theorems are named
by a pair of strings $\langle thy,name\rangle$ where $thy$ is the name
of the theory current when the item was declared and $name$ is a
specific name supplied by the user (see the functions \ml{new\_axiom},
\ml{new\_definition} \etc\ below).

A typical piece of work with the \HOL\ system consists in a number of
sessions.  A theorem-proving session consists of interactions with the
system through its ``command-line interface''.  At the same time, it's
important that the commands the user enters are saved in some sort of
file (presumably through an editor) so that these commands can be
replayed later.  For the first sessions of theory development, each
time \HOL\ is started, the saved commands from previous sessions will
need to be re-entered into the system.  This can be done with
cut-and-paste functionality, or the \ml{use} command (which takes a
file-name and reads it for input).

Eventually (possibly even after the very first session; this is a
matter of taste), the theory under development will be in a
sufficiently polished state that the user will want to save it to a
``script'' file and compile it.  This process is described further in
the examples below, but produces a theory file that can be loaded in
the same way as the system's built-in theories.

There is always a {\it current theory\/}, whose name is given by the
function \ml{current\_theory}.  This function maps the unit value,
written `\ml{()}' in \ML, to a string giving the name of the current
theory.  Thus the \ML\ expression \ml{current\_theory ()} evaluates to
a string giving the name of the current theory.  Initially \HOL\ has
its current theory called \ml{scratch}.  In the examples so far, there
hasn't been a call to \ml{new\_theory}, so the current theory is still
``scratch''.

\begin{session}\begin{verbatim}
- current_theory();
> val it = "scratch" : string
-
\end{verbatim}\end{session}

    Executing \ml{new\_theory`$thy$`} creates a new theory called
    $thy$.  The remaining example sessions will demonstrate the
    construction of a theory of prime numbers.

\begin{session}\begin{verbatim}
- new_theory "primes";
> val it = () : unit
\end{verbatim}\end{session}

\noindent
This starts a theory called \ml{primes}, which is to be made into a
theory containing definitions and theorems about the prime numbers.
This will be developed further in chapter~\ref{euclid} to include
Euclid's proof that there are an infinite number of primes.

Because we loaded \ml{arithmeticTheory} and \ml{pairTheory} to begin,
both of these theories are part of our new theory's ancestry.  In
fact, there are a number of other ancestor theories that these two
depend on in turn.

%%% Local Variables:
%%% mode: latex
%%% TeX-master: "tutorial"
%%% End:
