\chapter*{Prologue}\markboth{Prologue}{Prologue}
\label{prologue}

This volume contains a short sample of material from the upcoming
tutorial on the \HOLW{} system.  The tutorial will be one of four
documents making up the documentation for \HOLW:

\begin{myenumerate}
\item \LOGIC: a formal description of the higher order logic
  implemented by the \HOLW{} system.
\item \TUTORIAL: a tutorial introduction to \HOLW, with case studies.
\item \DESCRIPTION: a detailed user's guide for the \HOLW{} system;
\item \REFERENCE: the reference manual for \HOLW.
\end{myenumerate}

This document provides a brief and light set of examples of using \HOLW{},
as an introduction, giving a taste of how the system might be used.
Like an appetizer to a main meal, it provides just a hint of
the sustenance to come.

\section*{Getting started}

Chapter~\ref{install} explains how to get and install \HOLW.
Then the new, additional
concepts and features of the \HOLW{} logic (higher order logic 
extended with System {\it F}, kinds, and ranks)
are casually demonstrated,
in chapter \ref{chap:appetizers}.

Chapter~\ref{chap:teaser_epilogue} briefly discusses some of the
examples distributed with \holnw{} in the \ml{examples/HolOmega} directory.

%\item Chapter~\ref{tool} shows how a special purpose proof tool (a
%  conjunction normaliser) can be implemented and optimised. It
%  illustrates methods for `tuning' proof generating programs and
%  discusses trade-offs between generality and efficiency.

%\item Chapter~\ref{binomial} is a proof of the Binomial Theorem in a
%  ring.  It is a medium sized worked example whose subject matter is
%  probably more widely known than any specific piece of hardware or
%  software. The small amount of algebra and mathematical notation
%  needed to state and prove the Binomial Theorem is presented; the
%  notation is expressed in \HOLW{}, and the structure of the proof is
%  outlined.

%\end{itemize}

\vspace{1cm}

%\noindent \TUTORIAL{} has been kept short so that new users of \HOLW{} can get
%going as fast as possible. Sometimes details have been simplified. It
%is recommended that as soon as a topic in \TUTORIAL\ has been
%digested, the relevant parts of \DESCRIPTION\ and \REFERENCE\ be
%studied.

%%% Local Variables:
%%% mode: latex
%%% TeX-master: "tutorial"
%%% End:
