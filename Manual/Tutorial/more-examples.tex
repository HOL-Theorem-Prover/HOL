
\chapter{More Examples}
\label{chap:more-examples}

In addition to the examples already covered in this text, the \holn{}
distribution comes with a variety of instructive examples in the
\verb|examples| directory.  There the following examples (among
others) are to be found, using only the classic \HOL{} logic:

\begin{description}

\item [\tt autopilot.sml]

  This example is a \holn{} rendition (by Mark Staples) of a PVS
  example due to Ricky Butler of NASA. The example shows the use of
  the record-definition package, as well as illustrating some aspects
  of the automation available in \holn{}.

\item [\tt bmark]

  In this directory, there is a standard HOL benchmark: the proof of
  correctness of a multiplier circuit, due to Mike Gordon.

\item [\tt euclid.sml]

  This example is the same as that covered in
  Chapter~\ref{chap:euclid}: a proof of Euclid's theorem on the
  infinitude of the prime numbers, extracted and modified from a much
  larger development due to John Harrison. It illustrates the
  automation of \HOL{} on a classic proof.

\item[\tt ind\_def]

This directory contains some examples of an inductive definition package
in action. Featured are an operational semantics for a small imperative
programming language, a small process algebra, and combinatory logic
with its type system. The files were originally developed by Tom Melham
and Juanito Camilleri and are extensively commented.  The last is the
basis for Chapter~\ref{chap:combin}.

Most of the proofs in these theories can now be done much more easily by
using some of the recently developed proof tools, namely the simplifier
and the first order prover.

\item [\tt fol.sml]

  This file illustrates John Harrison's implementation of a
  model-elimination style first order prover.

\item[\tt lambda]

This directory develops theories of a ``de Bruijn'' style lambda calculus,
and also a name-carrying version. (Both are untyped.) The development
is a revision of the proofs underlying the paper
{\it ``5 Axioms of Alpha Conversion'',
            Andy Gordon and Tom Melham,
            Proceedings of TPHOLs'96, Springer LNCS 1125}.

\item[\tt parity]

  This sub-directory contains the files used in the parity example of
  Chapter~\ref{parity}.

\item [\tt MLsyntax]

  This sub-directory contains an extended example of a facility for
  defining mutually recursive types, due to Elsa Gunter of Bell Labs.
  In the example, the type of abstract syntax for a small but not
  totally unrealistic subset of ML is defined, along with a simple
  mutually recursive function over the syntax.

\item[\tt Thery.sml]

  A very short example due to Laurent Thery, demonstrating a cute
  inductive proof.

\item[\tt RSA]

       This directory develops some of the mathematics underlying
       the RSA cryptography scheme. The theories have been
       produced by Laurent Thery of INRIA Sophia-Antipolis.

\end{description}

In addition to the examples above, the examples covered in this
tutorial concerning the \HOLW{} logic are also present, 
as well as further developments, including the following: 

\begin{description}

\item [\tt appetizersScript.sml]

  This file gives a series of examples that in a light and easy way 
  briefly demonstrate the essential new features of the \HOLW{} logic.

\item [\tt functorScript.sml]

  This example shows how a simple version of category theory can
  be nicely realized as a shallow embedding within the new logic.
  Both functors and natural transformations are defined, and
  examples of each are demonstrated. This is similar to a development
  for HOL2P originally written by Norbert V\"{o}lker.

\item [\tt aopScript.sml]

  Building on the functor theory above, this shows several examples
  taken from {\it The Algebra of Programming}, by Richard Bird and Oege de Moor.
  These include homomorphisms, initial algebras, catamorphisms, and
  the banana split theorem.  This development was originally written
  by Norbert V\"{o}lker for HOL2P.

\item [\tt monadScript.sml]

  Also building on the functor theory above, this defines the concept
  of a monad in three different ways, and proves the three are equivalent.
  Multiple examples of monads are presented, and also how one can convert
  a monad from one of the styles of definition to another style.

\item [\tt type\_specScript.sml]

  This file contains examples of creating new types using the new definitional
  principle for type specification which has been added to the \HOLW{} theorem
  prover. In particular, this is used to create a new type by specifying it as
  the initial algebra of a signature. The example used is taken from a 1993
  paper by Tom Melham, ``The HOL Logic Extended with Quantification over Type
  Variables.''

\item [\tt packageScript.sml]

  This example shows more completely how packages and existential types
  may be created and used to hide the information about data types.
  Many of the examples are taken from and related to chapter 24 of the book
  "Types and Programmng Languages" by Benjamin C. Pierce, MIT Press, 2002.
  There is also an extended example on process scheduling queues.

\item [\tt burali\_fortiScript.sml]

  This file contains a development of the Burali-Forti paradox in the
  \HOLW{} logic, which attempts to prove false by a clever manipulation
  of the type system. This is the same as Girard's Paradox, which showed 
  that the na\"{i}ve combination of higher order logic and an advanced type 
  system was inconsistent. The development in this file demonstrates how 
  the \HOLW{} logic, which is both a higher order logic and an advanced 
  type system, prevents the inconsistency that the paradox attempts 
  to expose.  This is not a proof of the logic's consistency, but it is 
  a strong demonstration of its resilience in the face of a sophisticated 
  and subtle attack. This work is described further in an upcoming paper, 
  ``The HOL-Omega Logic and Girard's Paradox.''

\item[\tt interim]

This directory contains an extensive, worked example of a generalized
version of category theory, created by Jeremy Dawson of the Australian National
University. This generalizes the notions of functor and natural transformation
from those in \texttt{functorScript.sml}, to allow for a much richer realization
of category theory. For example, multiple categories, each with their own composition
and identity operations, may have functors defined between them.
The development of category theory is continued through the definition of adjoints,
and introduces an innovative extension of monads.
This example extensively exercises the kind structure of \HOLW, to manage
the types relating different categories and the operations among them.

\end{description}


%%% Local Variables:
%%% mode: latex
%%% TeX-master: "tutorial"
%%% End:
