\chapter{Goal Directed Proof: Tactics and Tacticals}

\label{tactics-and-tacticals}

\index{tactics!purpose of|(}
There are three primary devices that together make theorem proving practical
in \HOL. All three originate with Milner
\index{Milner, R.}\index{LCF@\LCF!Edinburgh} for Edinburgh \LCF.
The first is the theory as
a record of (among other things) facts already proved
and thence available as lemmas
\index{lemmas}
without having to be re-proved.  The second,
the subject of Chapter~\ref{derived-rules}, is the
derived rule of inference as a meta-language procedure that implements a
broad pattern of inference, but that also, at each application,
generates every primitive step of the proof. The third device is
the tactic as a means of organizing the construction of proofs;
and the use of tacticals for composing tactics.

Even with recourse to derived inference rules,
it is still surprisingly awkward to work forward,
\index{forward proof!compared to goal directed} to find a chain of
theorems that culminates in a desired theorem.  This is in part because
chains have no structure, while `proof efforts' do.  For instance, if
within one sequence, two chains of steps
are to be combined in the end by conjunction, then
one chain must follow or be interspersed with
the other in the overall sequence.  It can also be difficult to direct
the proof toward its object when starting from
only hypotheses (if any), lemmas (if any),
axioms, and theorems following from no hypotheses
(\eg\ by \ml{ASSUME} or \ml{REFL}). Likewise, it can be equally difficult
to reconstruct
\index{tactics!as documentation of proofs}
the plan of the proof effort after the fact, from the
linear sequence of theorems; the sequence is unhelpful as documentation.

The idea of goal directed proof\index{goal directed proof search!reason for} is a simple one, well known in
artificial intelligence: to organize the search\index{proof construction!as tree search} as a tree, and to reverse
the process and {\it begin\/} with the objective. The goal is then
decomposed, successively if necessary,
into what one hopes are more tractable subgoals, each decomposition
accompanied by a
plan for translating the solution of subgoals into a solution of the goal.
The choice of decomposition is an explicit way of expressing a proof
`strategy'.
\index{strategies, for proof}

Thus, for example, instead of
the linear sequencing of two branches of the proof of the conjunction,
each branch starting from scratch, the proof task is organized
as a tree search, starting with a conjunctive goal
and decomposing it into the two conjunct subgoals (undertaken in optional
order), with the intention of conjoining the two solutions when and if found.
The proof itself, as a sequence of steps, is the same however it is found;
the difference is in the search, and in the preservation, if required, of
the structured proof plan.

The representation of this idea in \LCF\ was Milner's inspiration;
the idea is similarly central to theorem proving in \HOL.
Although subgoaling theorem provers had already been built at the time,
Milner's particular contribution was in formalizing the method for
translating subgoals solutions to solutions of goals.



%The tactics and tacticals in the \HOL{} system are derived from those in the
%Cambridge \LCF\ system \cite{new-LCF-man} (which evolved
%from the ones
%in Edinburgh \LCF\ \cite{Edinburgh-LCF}).

\section{Tactics, goals and justifications}
\label{tactics}

\index{goal directed proof search!concepts of|(}
A \emph{tactic}%
\index{proof steps, as ML function applications@proof steps, as \ML{} function applications}
is an \ML{} function that when applied to a \emph{goal}
\index{goals, in HOL system@goals, in \HOL{} system}
reduces it to (i) a list\footnote{The ordering is necessary for selecting
a tree search strategy.} of
(sub)goals, along with (ii) a \emph{justification}
\index{justifications, in goal-directed proof search}
function mapping a list of theorems to a theorem.  The idea is that
the function justifies the decomposition of the goal.
%that respectively {\it achieve\/} the (sub)goals
%to a theorem that achieves the goal.
A goal is an \ML{} value whose type is isomorphic
to, but distinct from, the \ML{} abstract type \ml{thm} of theorems.
That is, a goal is
a list of terms ({\it assumptions\/})
\index{assumption list, of goal} paired with a term.
\index{term component, of goal}
These two components correspond, respectively, to the list of hypotheses
and the conclusion of a theorem. The list of assumptions is a working
record of facts that may be used in decomposing the goal.

The relation of theorems to goals is achievement:
\index{achievement, of goals}
a theorem achieves a goal if the conclusion of the theorem
is equal to the term part of the goal (up to $\alpha$-conversion), and
if each hypothesis of the theorem is equal (up to $\alpha$-conversion,
again) to some assumption of the goal. This definition assures that
the theorem purporting to satisfy a goal does not depend on
assumptions beyond the working assumptions of the goal.

A justification is (rather confusingly) called
a {\it proof\/}
\index{proofs, in HOL logic@proofs, in \HOL{} logic!as ML function applications@as \ML{} function applications}
\index{proof functions (same as justifications, validations)}
in \HOL{}, following the \LCF{} usage; it is, as mentioned,
an \ML{} function
from a theorem list to a theorem.  The \ML{} `proof' function
corresponds to a proof
\index{proofs, in HOL logic@proofs, in \HOL{} logic!as generated by tactics} in the
logical sense (of a sequence of theorems depending on inference\index{inferences, in HOL logic@inferences, in \HOL{} logic!in goal-directed proof search} rules)
only in that it must
evaluate the \ML{} function corresponding
to each inference rule on which the sequence depends
in order to compute its \ml{thm}-valued result. (`Justification', or
`validation',
\index{validations} as is sometimes used, are less confusing terms for
the \ML{} function in question.)
The proof function, or justification, returned by a tactic is intended to map the
list of theorems respectively achieving the subgoals to the
theorem achieving the original goal; it justifies the decomposition
into subgoals.

A tactic is said to {\it solve\/}
\index{solving, of goals} a goal if it reduces the goal to the
empty set of subgoals.
This depends, obviously, on there being at least one
tactic that maps a goal to the empty subgoal list.  The simplest
tactic that does this is one that can recognize when a goal is
achieved by an axiom or an existing theorem; in \HOL, the function
\ml{ACCEPT\_TAC}\index{ACCEPT_TAC@\ml{ACCEPT\_TAC}}
 does this. \ml{ACCEPT\_TAC} takes a theorem $th$
and produces a tactic that maps a value
of type \ml{thm} to the empty list of subgoals. It justifies this
`decomposition' by a proof function that maps the empty list of theorems
to the theorem $th$. The use of this technical device, or other
such tactics, ends the decomposition of subgoals, and allows the proof
to be built up.\index{tactics!purpose of|)}

Unlike theorems, goals need not be defined as an abstract type; they
are transparent and can be constructed freely. Thus, an \ML{} type
abbreviation is introduced for goals.\footnote{However, if goals were
  an abstract type, the print abbreviation could be avoided where not
  intended.}.  The operations on goals are therefore just the ordinary
pair selectors and constructor.
Likewise, type abbreviations are introduced for justifications
(proofs) and tactics. Conceptually, the following abbreviations are
made in \HOL:

\begin{hol}
\index{goal@\ml{goal}}
\index{tactic@\ml{tactic}}
\index{proof@\ml{proof}}
\index{tactics!ML type of@\ML{} type of}
\begin{verbatim}
   goal       = term list * term
   tactic     = goal -> goal list * proof
   proof      = thm list -> thm
\end{verbatim}\end{hol}

\index{subgoals@\ml{subgoals}} In fact, the type
\ml{goal~list~*~proof} is abbreviated in \ML{} to \ml{subgoals}, and
the abbreviation of \ml{tactic} made indirectly through it.  Thus, if
$T$ is a tactic and $g$ is a goal, then applying $T$ to $g$ (\ie\
evaluating the \ML\ expression $T\ g$) results in an \ML{} value of
type \ml{subgoals}, \ie\ a pair whose first component is a list of
goals and whose second component has \ML{} type \ml{proof}. (The word
`tactic' is occasionally used loosely to mean a tactic-valued
function.)

It does not follow, of course, from the type \ml{tactic} that a
particular tactic is well-behaved. For example,
suppose that
$T\ g${\small\verb% = ([%}$g_1${\small\verb%;%}$\ldots
${\small\verb%;%}$g_n${\small\verb%],%}$p${\small\verb%)%}, and that
the subgoals $g_1$ , $\dots$, $g_n$ have been solved.
That means that some
theorems $th_1$ , $\dots$, $th_n$ have been proved
such that each $th_i$ ($1\leq i\leq n$) achieves the goal $g_i$.
The justification $p$
is intended to be a
function that when applied to the list
{\small\verb%[%}$th_1${\small\verb%;%}$\ldots${\small\verb%;%}$th_n${\small\verb%]%},
succeeds in returning a theorem, $th$,
achieving the original goal $g$; but, of course, it might sometimes
not succeed. If $p$
succeeds for every list of achieving theorems, then the tactic $T$ is
said to be {\it valid\/}\index{validity, of tactics|(}. This does not guarantee, however, that
the subgoals are solvable in the first place. If, in addition
to being valid, a tactic always produces solvable subgoals from a
solvable goal, it is called {\it strongly valid\/}.
\index{strong validity, of tactics}

Tactics can be perfectly useful without being
strongly valid, or without
even being valid;
in fact, some of the most basic theorem proving strategies, expressed
as tactics, are invalid or not strongly valid.\footnote{The subgoal
  package, discussed in Section~\ref{sec:goalstack}, prevents the use of
  invalid tactics when they are liable to result in unexpected
  theorem results, but the \HOL{} system used directly allows
  it.}
An invalid tactic cannot result in the proof of false theorems;
\index{consistency, of HOL logic@consistency, of \HOL{} logic|(}
\index{security, in goal directed proof|(}
theorems in \HOL{} are always the result of performing a proof in the
basic logic, whether the proof is found by goal directed search
or forward search.\footnote{`Invalid' is perhaps a misleading term, since
  there is nothing logically amiss in the use of invalid tactics
  or the theorems produced thereby; but the term has stuck over time.}
However, an invalid tactic may produce an unintended theorem---one
that does not achieve the original goal. The typical case is when a
theorem purporting to achieve a goal actually depends on hypotheses
that extend beyond the assumptions of the goal.  The inconvenience to
the \HOL{} user in this case is that the problem may be not
immediately obvious; the default print format of theorems has
hypotheses abbreviated as dots. Invalidity may also be the result of
the failure
\index{failure, of tactics|(}
of the proof function, in the \ML{} sense of failure, when
applied to a list of theorems (if, for example, the function were
defined incorrectly); but again, no false theorems can result.
Likewise, a tactic that is not strongly valid cannot result in a false
theorem; the worst outcome of applying such a tactic is the production
of unsolvable subgoals
\index{consistency, of HOL logic@consistency, of \HOL{} logic|)}
\index{security, in goal directed proof|)}\index{validity, of tactics|)}.

Tactics are specified using the following notation:
\index{notation!for specification of tactics}

\begin{center}
\begin{tabular}{c} \\
$\mathit{goal}$ \\ \hline \hline
$\mathit{goal}_1\ \ \ \mathit{goal}_2 \ \ \ \ldots\ \ \ \mathit{goal}_n$ \\
\end{tabular}
\end{center}

\noindent For example, the
tactic for decomposing conjunctions into
two  conjunct subgoals is called {\small\verb%CONJ_TAC%}.
\index{CONJ_TAC@\ml{CONJ\_TAC}}  It is described by:

\begin{center}
\begin{tabular}{c} \\
$ t_1${\small\verb% /\ %}$t_2$ \\ \hline \hline
$t_1\ \ \ \ \ \ \ t_2$ \\
\end{tabular}
\end{center}

\noindent This indicates
that {\small\verb%CONJ_TAC%} reduces a goal of the form
{\small\verb%(%}$\Gamma${\small\verb%,%}$t_1${\small\verb%/\%}$t_2${\small\verb%)%}
to subgoals
{\small\verb%(%}$\Gamma${\small\verb%,%}$t_1${\small\verb%)%} and {\small\verb%(%}$\Gamma${\small\verb%,%}$t_2${\small\verb%)%}.
The fact that the assumptions of the original goal
are propagated unchanged to the two subgoals is indicated by the absence
of assumptions in the notation. The notation gives no indication of the
proof function.

Another example is {\small\verb%INDUCT_TAC%}\index{INDUCT_TAC@\ml{INDUCT\_TAC}}\index{induction tactics},
the tactic for performing mathematical induction
on the natural numbers:

\begin{center}
\begin{tabular}{c} \\
{\small\verb%!%}$n${\small\verb%.%}$t[n]$ \\ \hline \hline
$t[${\small\verb%0%}$]$ {\small\verb%     %} $\{t[n]\}\ t[${\small\verb%SUC %}$n]$
\end{tabular}
\end{center}

{\small\verb%INDUCT_TAC%} reduces a goal of the form
{\small\verb%(%}$\Gamma${\small\verb%,!%}$n${\small\verb%.%}$t[n]${\small\verb%)%} to a basis subgoal
{\small\verb%(%}$\Gamma${\small\verb%,%}$t[${\small\verb%0%}$]${\small\verb%)%}
and an induction step subgoal
{\small\verb%(%}$\Gamma\cup\{${\small\verb%%}$t[n]${\small\verb%%}$\}${\small\verb%,%}$t[${\small\verb%SUC %}$n]${\small\verb%)%}.
The induction assumption
is indicated in the tactic notation with set brackets.

Tactics fail
\index{failure, of tactics|)}
(in the \ML{} sense) if they are applied to
inappropriate
goals. For example, {\small\verb%CONJ_TAC%} will fail if it is applied to a goal whose
conclusion is not a conjunction. Some tactics never fail; for example
{\small\verb%ALL_TAC%}\index{ALL_TAC@\ml{ALL\_TAC}}

\begin{center}
\begin{tabular}{c} \\
$t$ \\ \hline \hline
$t$
\end{tabular}
\end{center}

\noindent is the identity tactic;
\index{identity tactic} it reduces a goal
{\small\verb%(%}$\Gamma${\small\verb%,%}$t${\small\verb%)%}
to the single
subgoal {\small\verb%(%}$\Gamma${\small\verb%,%}$t${\small\verb%)%}---\ie\
it has no effect. {\small\verb%ALL_TAC%} is useful for writing
compound tactics, as discussed later (see Section~\ref{tacticals}).

In just the way that the derived rule \ml{REWRITE\_RULE}
\index{REWRITE_TAC@\ml{REWRITE\_TAC}|(}
 is central
to forward proof (Section~\ref{avra_rewrite}), the corresponding
function \ml{REWRITE\_TAC}
\index{rewriting!main tactic for|(}
\index{rewriting!importance of, in goal directed proof|(}
is central to goal
directed proof.  Given a goal and a list of equational theorems,
\ml{REWRITE\_TAC} transforms the term component of the goal by
applying the equations as left-to-right rewrites, recursively and
to all depths, until no more changes can be made.  Unless not
required, the function includes as rewrites the same
standard set of pre-proved tautologies
\index{tautologies, in rewriting tactic}
that \ml{REWRITE\_RULE} uses.
By use of the tautologies, some subgoals can be solved
internally by rewriting, and in that case, an empty list of subgoals
is returned. The transformation of the goal is justified in each case by the
appropriate chain of inferences.
Rewriting often does a large share of the work in goal directed proof searches.\index{goal directed proof search!concepts of|)}
\index{REWRITE_TAC@\ml{REWRITE\_TAC}|)}
\index{rewriting!importance of, in goal directed proof|)}
\index{rewriting!main tactic for|)}

A simple example from list theory (Section~\ref{avra_list})
illustrates the use of tactics.
A conjunctive goal is declared, and \ml{CONJ\_TAC} applied to it:

\setcounter{sessioncount}{1}
\begin{session}\begin{verbatim}
- val g = ([]:term list,``(HD[1;2;3] = 1) /\ (TL[1;2;3] = [2;3])``);
> val g = ([], ``(HD[1;2;3] = 1) /\ (TL[1;2;3] = [2;3])``) :
  term list * term

- val (gl1,p1) = CONJ_TAC g;
> val gl1 = [([], ``HD[1;2;3] = 1``); ([], ``TL[1;2;3] = [2;3]``)] :
  (term list * term) list
  val p1 = fn : thm list -> thm
\end{verbatim}\end{session}

\noindent The subgoals are each rewritten, using the definitions of
\ml{HD} and \ml{TL}:

\begin{session}\begin{verbatim}
- open listTheory
  [...]

- HD;
> val it = |- !h t. HD (h::t) = h : thm

- TL;
> val it = |- !h t. TL (h::t) = t : thm

- val (gl1_1,p1_1) = REWRITE_TAC[HD,TL](hd gl1);
> val gl1_1 = [] : (term list * term) list
  val p1_1 = fn : thm list -> thm

- val (gl1_2,p1_2) = REWRITE_TAC[HD,TL](hd(tl gl1));
> val gl1_2 = [] : (term list * term) list
  val p1_2 = fn : thm list -> thm
\end{verbatim}\end{session}

\noindent Both of the two subgoals are now solved, so the
decomposition is complete and the proof can be built up in stages.
First the theorems achieving the subgoals are proved, then from those,
the theorem achieving the original goal:
\vfill
\newpage
\begin{session}\begin{verbatim}
- val th1 = p1_1[];
> val th1 = |- HD [1; 2; 3] = 1 : thm

- val th2 = p1_2[];
> val th2 = |- TL [1; 2; 3] = [2; 3] : thm

- p1[th1,th2];
> val it = |- (HD [1; 2; 3] = 1) /\ (TL [1; 2; 3] = [2;3]) : thm
\end{verbatim}\end{session}

\noindent Although only the theorems achieving the subgoals are `seen' here,
the proof functions of the three tactic applications together perform
the entire chain\index{goal directed proof search!generation of proofs by}\index{proofs, in HOL logic@proofs, in \HOL{} logic!as generated by tactics}
 of inferences leading to the theorem achieving the goal.
The same proof could be constructed by forward search, starting from
the definitions of \ml{HD} and \ml{TL}, but not nearly as easily.

The \HOL{} system provides a collection of pre-defined tactics (and
\ml{tactic}-valued functions) that includes
\ml{CONJ\_TAC}, \ml{INDUCT\_TAC}, \ml{ALL\_TAC}  and
\ml{REWRITE\_TAC}. The pre-defined tactics
are adequate for many applications.  In addition, there are two means of
defining new tactics.
\index{tactics!definition of new}
Since a tactic\index{proof steps, as ML function applications@proof steps, as \ML{} function applications} is an \ML{} function, the user
can define a new tactic directly in \ML.  Definitions of this sort
use \ML{} functions to construct the term part of the subgoals from
the term part of the original goal (if any transformation is required);
and they specify the justification,
which expects a list of theorems achieving the subgoals and
returns the theorem achieving (one hopes) the goal.
The proof of the theorem is encoded in the definition of the justification
 function;
that is, the means for deriving the desired theorem from the theorems
given. This typically involves references to axioms and
primitive and defined inference rules,
and is usually the more difficult part of the project.

A simple example of a tactic written in \ML\index{proofs, in HOL logic@proofs, in \HOL{} logic!as ML function applications@as \ML{} function applications}\ is afforded by \ml{CONJ\_TAC},
whose definition in \HOL{} is as follows:

\begin{hol}
\index{CONJ_TAC@\ml{CONJ\_TAC}!ML implementation of@\ML{} implementation of}
\begin{verbatim}
   fun CONJ_TAC (asl,w) =
     let val (l,r) = dest_conj w
     in
         ([(asl,l), (asl,r)],(fn [th1, th2] => CONJ th1 th2))
     end
\end{verbatim}\end{hol}

\noindent This shows how the subgoals are constructed, and how the
proof function is specified in terms of the derived rule \ml{CONJ}
(Section~\ref{avra_conj}).

The second method is to compose
\index{tactics!indirect implementation of}
\index{tactics!compound}
\index{proof construction}
existing tactics by the use
of \ML{} functions called {\it tacticals\/}.
\index{tacticals}
The tacticals provided in \HOL{} are listed in Section~\ref{tacticals}.
For example, two existing tactics can be sequenced
\index{sequencing!of tactics}
\index{tactics!sequencing of}
 by use of the
tactical \ml{THEN}:\index{THEN@\ml{THEN}}
if $T_1$ and
$T_2$ are  tactics,  then the  \ML{} expression  $T_1${\small\verb% THEN %}$T_2$
evaluates to a  tactic that first applies  $T_1$ to  a goal  and then applies
$T_2$ to each subgoal produced by $T_1$.   The  tactical {\small\verb%THEN%} is
an infixed \ML{} function. Complex and powerful tactics can be
constructed in this way; and new tacticals can also be defined, although
this is unusual.

The example from earlier
is continued, to illustrate the use of the tactical \ml{THEN}:

\begin{session}\begin{verbatim}
- val (gl2,p2) = (CONJ_TAC THEN REWRITE_TAC [HD, TL])g;
> val gl2 = [] : (term list * term) list
  val p2 = fn : thm list -> thm

- p2 [];
> val it = |- (HD [1; 2; 3] = 1) /\ (TL [1; 2; 3] = [2; 3])
\end{verbatim}\end{session}

\noindent The single tactic \ml{CONJ\_TAC THEN REWRITE\_TAC[HD;TL]}
solves the goal in one single application. The chain of inference computed,
however, is exactly the same as in the interactive proof; only the search is
different.

In general, the second method is both easier and more reliable.  It is
easier because it does not involve writing \ML{} procedures (usually
rather complicated procedures); and more reliable because
the composed tactics are valid
\index{validity, of tactics} when the constituent tactics are valid,
as a consequence of the way the tacticals are defined. Tactics written
directly in \ML{} may fail
\index{failure, of tactics!debugging}
\index{failure, of tactics|(}
\index{debugging, of tactics}
\index{tactics!debugging of}
in a variety of ways, and although, as usual, they cannot cause false
theorems to appear, the failures can be difficult to understand and
trace.\footnote{A possible extension to \HOL\ would be a `debugging
  environment' for this class of tactic.} On the other hand, there are
some proof strategies that cannot be implemented as compositions of
existing tactics, and these have to be implemented directly in \ML.
Certain sorts of inductions are an example of this; as well as tactics
to support some personal styles of proof.

Either sort of tactic can be difficult to apply by hand, as shown in
the examples above.  There can be a lot of book-keeping required to
support such an activity.  For this reason, most interactive
theorem-proving uses the subgoal or goalstack package described in
Section~\ref{sec:goalstack}.


\subsection{Details of proving theorems}
\label{using-tactics}

When a theorem is proved that the user wishes to preserve for future use,
it can be stored in the current theory
by using the function \ml{save\_thm} (see Section~\ref{theoryprims}).

To simplify the use of tactics there are three standard functions

\index{TAC_PROOF@\ml{TAC\_PROOF}|pin}
\index{prove_thm@\ml{prove\_thm}|pin}
\index{PROVE@\ml{PROVE}|pin}
\begin{holboxed}\begin{verbatim}
   TAC_PROOF : (goal * tactic) -> thm
   store_thm : (string * term * tactic) -> thm
   prove     : (term * tactic) -> thm
\end{verbatim}\end{holboxed}

\noindent \ml{TAC\_PROOF} takes a goal and a tactic, and applies the
tactic to the goal; the goal can have assumptions.  Executing
\ml{store\_thm("foo",}$t$\ml{,}$T$\ml{)} proves the goal
\ml{([],}$t$\ml{)} (\ie\ the goal with no assumptions and conclusion
$t$) using tactic $T$ and saves the resulting theorem with name
\ml{foo} in the current theory.  Executing
\ml{prove(}$t$\ml{,}$T$\ml{)} proves the goal \ml{([],}$t$\ml{)} using
$T$ and returns the result without saving it. In all cases the
evaluation fails if $T$ does not solve the goal.

In short, \HOL{} provides a very general framework in which proof
strategies can be designed, implemented, applied and tested.  Tactics
range from the very simple to the very advanced; in theory, a
conventional automatic theorem prover could be expressed as a tactic or group
of tactics.  In contrast, some users never have need to go beyond the
built in tactics of the system.  The vital support that \HOL{} provides
in all cases is the assurance that only theorems of the deductive system
can be represented as theorems of the \HOL{} system---security is
always preserved.

\section{Some tactics built into HOL}
\label{avra_builtin}

This section contains a selection of the more commonly
\index{tactics!list of some|(}
used tactics in the \HOL{} system. (see \REFERENCE\
for the complete list, with fuller explanations.)

It should be recalled that the \ML{} type \ml{thm\_tactic} abbreviates
\ml{thm->tactic}, and the type \ml{conv} abbreviates \ml{term->thm}.

\subsection{Acceptance of a theorem}


\begin{holboxed}\index{ACCEPT_TAC@\ml{ACCEPT\_TAC}|pin}
\begin{verbatim}
   ACCEPT_TAC : thm_tactic
\end{verbatim}\end{holboxed}

\begin{itemize}

\item{\bf Summary:} {\small\verb%ACCEPT_TAC %}$th$
is a tactic that solves any goal that is
achieved by $th$.

\index{forward proof!interfacing to goal directed}
\item{\bf  Use:} Incorporating forward proofs, or theorems already
proved, into goal directed proofs.
For example, one might reduce a goal $g$ to
subgoals $g_1$, $\dots$, $g_n$
using a tactic $T$ and then prove theorems $th_1$ , $\dots$, $th_n$
respectively achieving
these goals by forward proof. The tactic

\[\ml{  T THENL[ACCEPT\_TAC }th_1\ml{;}\ldots\ml{;ACCEPT\_TAC }th_n\ml{]}
\]

would then solve $g$, where \ml{THENL}
\index{THENL@\ml{THENL}} is the tactical that applies
the respective elements of the tactic list to the subgoals produced
by \ml{T} (see Section~\ref{avra_thenl}).

\end{itemize}


\subsection{Adding an assumption}

\index{ASSUME_TAC@\ml{ASSUME\_TAC}|pin}
\begin{holboxed}\begin{verbatim}
   ASSUME_TAC : thm_tactic
\end{verbatim}\end{holboxed}

\begin{itemize}

\item {\bf Summary:} {\small\verb%ASSUME_TAC |-%}$u$ adds $u$ as an assumption.
\index{assumptions!tactic for adding}

\begin{center}
\begin{tabular}{c} \\
$t$
\\ \hline \hline
$\{u\}t$
\\
\end{tabular}
\end{center}

\item{\bf Use:} Enriching the assumptions of a goal
with definitions or previously proved theorems.

\end{itemize}

\subsection{Specialization}\index{universal quantifier, in HOL logic@universal quantifier, in \HOL{} logic!tactics for}

\index{GEN_TAC@\ml{GEN\_TAC}|pin}
\begin{holboxed}\begin{verbatim}
   GEN_TAC : tactic
\end{verbatim}\end{holboxed}

\begin{itemize}

\index{specialization tactic}
\item{\bf  Summary:} Specializes a universally quantified
theorem to an arbitrary value.


\begin{center}
\begin{tabular}{c} \\
{\small\verb%!%}$x${\small\verb%.%}$t[x]$
\\ \hline \hline
$t[x']$
\\
\end{tabular}
\end{center}

\noindent where $x'$ is a variant of $x$
not free in either goal or assumptions.

\item{\bf   Use:} Solving universally quantified goals.
\ml{GEN\_TAC} is often the first step of a goal directed proof.
{\small\verb%STRIP_TAC%} (see below)
applies {\small\verb%GEN_TAC%} to universally quantified goals.
\end{itemize}

\subsection{Conjunction}

\index{CONJ_TAC@\ml{CONJ\_TAC}|pin}
\begin{holboxed}\begin{verbatim}
   CONJ_TAC : tactic
\end{verbatim}\end{holboxed}

\begin{itemize}

\index{conjunction, in HOL logic@conjunction, in \HOL{} logic!tactic for splitting of}
\item{\bf Summary:} Splits a
goal $t_1${\small\verb%/\%}$t_2$ into two
subgoals, $t_1$ and $t_2$.

\begin{center}
\begin{tabular}{c} \\
$t_1${\small\verb% /\ %}$t_2$
\\ \hline \hline
$t_1\ \ \ \ \ \ t_2$
\\
\end{tabular}
\end{center}

\item{\bf Use:} Solving conjunctive goals.
{\small\verb%CONJ_TAC%} is invoked by {\small\verb%STRIP_TAC%} (see below).

\end{itemize}

\subsection{Discharging an assumption}
\label{avradisch}

\index{implication, in HOL logic@implication, in \HOL{} logic!tactics for}
\index{DISCH_TAC@\ml{DISCH\_TAC}|pin}
\begin{holboxed}\begin{verbatim}
   DISCH_TAC : tactic
\end{verbatim}\end{holboxed}

\begin{itemize}

\item{\bf Summary:} Moves the antecedant
of an implicative goal into the assumptions, leaving the consequent
as the term component.

\begin{center}
\begin{tabular}{c} \\
$u${\small\verb% ==> %}$v$
\\ \hline \hline
$\{u\}v$
\\
\end{tabular}
\end{center}


\item{\bf Use:} Solving goals of the form
$u${\small\verb% ==> %}$v$
by assuming $u$  and then solving
$v$ under the assumption.
{\small\verb%STRIP_TAC%} (see below) invokes
{\small\verb%DISCH_TAC%} on implicative goals.
\end{itemize}

\subsection{Combined simple decompositions}

\index{STRIP_TAC@\ml{STRIP\_TAC}|pin}
\begin{holboxed}\begin{verbatim}
   STRIP_TAC : tactic
\end{verbatim}\end{holboxed}

\begin{itemize}

\item{\bf Summary:} Breaks a goal apart.
{\small\verb%STRIP_TAC%} removes one outer connective from the goal, using
{\small\verb%CONJ_TAC%}, {\small\verb%DISCH_TAC%}, {\small\verb%GEN_TAC%},
and other tactics.
If the goal has the form $t_1${\small\verb%/\%}$\cdots${\small\verb%/\%}$t_n${\small\verb% ==> %}$t$
then {\small\verb%DISCH_TAC%} makes each $t_i$ into a separate assumption.

\item{\bf Use:} Useful for splitting a goal up into manageable pieces.
Often the best thing to do first is {\small\verb%REPEAT STRIP_TAC%},
where \ml{REPEAT} is the tactical that repeatedly applies a tactic
until it fails (see Section~\ref{avra_repeat}).
\end{itemize}



\subsection{Substitution}

\index{SUBST_TAC@\ml{SUBST\_TAC}|pin}
\begin{holboxed}\begin{verbatim}
   SUBST_TAC : thm list -> tactic
\end{verbatim}\end{holboxed}

\begin{itemize}

\item{\bf Summary:}
{\small\verb%SUBST_TAC[|-%}$u_1${\small\verb%=%}$v_1${\small\verb%;%}$\ldots${\small\verb%;|-%}$u_n${\small\verb%=%}$v_n${\small\verb%]%}
changes\index{substitution, tactic for} each sub-term
$t[u_1,\ldots ,u_n]$ of the goal to
$t[v_1,\ldots ,v_n]$
by substitution.

\item{\bf Use:}
Useful in situations where {\small\verb%REWRITE_TAC%}
\index{REWRITE_TAC@\ml{REWRITE\_TAC}} does too much,
or would loop.
\end{itemize}


\subsection{Case analysis on a boolean term}
\index{case analysis, in HOL logic@case analysis, in \HOL{} logic!tactics for|(}

\index{ASM_CASES_TAC@\ml{ASM\_CASES\_TAC}|(}
\begin{holboxed}\begin{verbatim}
   ASM_CASES_TAC : term -> tactic
\end{verbatim}\end{holboxed}

\begin{itemize}

\item{\bf Summary:} \ml{ASM\_CASES\_TAC} $u$ , where $u$ is a
boolean-valued term, does case analysis on $u$.

\begin{center}
\begin{tabular}{c} \\
$t$
\\ \hline \hline
$\{u\}t\ \ \ \ \ \{${\small\verb%~%}$u\}t$
\\
\end{tabular}
\end{center}

\item{\bf Use:} Case analysis.
\end{itemize}
\index{ASM_CASES_TAC@\ml{ASM\_CASES\_TAC}|)}

\subsection{Case analysis on a disjunction}


\begin{holboxed}
\index{DISJ_CASES_TAC@\ml{DISJ\_CASES\_TAC}|pin}
\index{disjunction, in HOL logic@disjunction, in \HOL{} logic!tactic for case splits on}
\begin{verbatim}
   DISJ_CASES_TAC : thm_tactic
\end{verbatim}\end{holboxed}

\begin{itemize}

\item{\bf Summary:}
{\small\verb%DISJ_CASES_TAC |- %}$u${\small\verb% \/ %}$v$
splits a goal into two cases: one with $u$
as an assumption
and the other with $v$ as an assumption.

\begin{center}
\begin{tabular}{c} \\
$t$
\\ \hline \hline
$\{u\}t\ \ \ \ \ \{v\}t$
\\
\end{tabular}
\end{center}

\item{\bf Use:} Case analysis. The
tactic {\small\verb%ASM_CASES_TAC%} is defined in \ML{} by

{\small\begin{verbatim}
   let ASM_CASES_TAC t = DISJ_CASES_TAC(SPEC t EXCLUDED_MIDDLE)
\end{verbatim}}


\noindent where {\small\verb%EXCLUDED_MIDDLE%} is
the theorem {\small\verb%|- !t. t \/ ~t%}.

\end{itemize}
\index{case analysis, in HOL logic@case analysis, in \HOL{} logic!tactics for|)}

\subsection{Rewriting}
\label{rewrite}

\index{rewriting!main tactic for|(}
\begin{holboxed}
\index{REWRITE_TAC@\ml{REWRITE\_TAC}|pin}
\begin{verbatim}
   REWRITE_TAC : thm list -> tactic
\end{verbatim}\end{holboxed}


\begin{itemize}
\item{\bf Summary:} {\small\verb%REWRITE_TAC[%}$th_1${\small\verb%;%}$\ldots${\small\verb%;%}$th_n${\small\verb%]%}
transforms the term part of a goal by rewriting
it with the given theorems $th_1$, $\dots$, $th_n$,
and the set of pre-proved standard tautologies\index{tautologies, in rewriting tactic}.


\begin{center}
\begin{tabular}{c} \\
$\{t_1, \ldots , t_m\}t$
\\ \hline \hline
$\{t_1, \ldots , t_m\}t'$
\\
\end{tabular}
\end{center}

\noindent where $t'$ is obtained from $t$ as described.

\item{\bf Use:} Advancing goals by using definitions and
previously proved theorems (lemmas).\index{lemmas}


\item{\bf Some other rewriting tactics} (based on {\small\verb%REWRITE_TAC%}) are:
\begin{enumerate}
\item {\small\verb%ASM_REWRITE_TAC%}\index{ASM_REWRITE_TAC@\ml{ASM\_REWRITE\_TAC}}
 adds the assumptions of the goal to the list of
theorems used for rewriting.
\index{PURE_ASM_REWRITE_TAC@\ml{PURE\_ASM\_REWRITE\_TAC}}
\item {\small\verb%PURE_ASM_REWRITE_TAC%} is like {\small\verb%ASM_REWRITE_TAC%}, but it
doesn't use any built-in rewrites.
\index{PURE_REWRITE_TAC@\ml{PURE\_REWRITE\_TAC}}
\item {\small\verb%PURE_REWRITE_TAC%} uses neither the assumptions nor the built-in rewrites.
\index{FILTER_ASM_REWRITE_TAC@\ml{FILTER\_ASM\_REWRITE\_TAC}}
\item {\small\verb%FILTER_ASM_REWRITE_TAC %}$p${\small\verb% [%}$th_1${\small\verb%;%}$\ldots${\small\verb%;%}$th_n${\small\verb%]%}
simplifies the goal by rewriting
it with the explicitly given theorems $th_1$ , $\dots$, $th_n$ ,
together with those
assumptions of the goal which satisfy the predicate $p$ and also
the standard rewrites.

\end{enumerate}
\end{itemize}

\index{rewriting!main tactic for|)}


\subsection{Resolution by Modus Ponens}

\index{implication, in HOL logic@implication, in \HOL{} logic!tactics for}
\index{IMP_RES_TAC@\ml{IMP\_RES\_TAC}|pin}
\begin{holboxed}\begin{verbatim}
   IMP_RES_TAC : thm -> tactic
\end{verbatim}\end{holboxed}

\begin{itemize}

\index{resolution tactics}
\item{\bf Summary:} {\small\verb%IMP_RES_TAC %}$th$ does a limited amount of
automated theorem proving in the form of forward inference; it
`resolves' the theorem $th$ with the
assumptions of the goal
and adds any new results to the assumptions. The specification for
\ml{IMP\_RES\_TAC} is:


\begin{center}
\begin{tabular}{c} \\
$\{t_1,\ldots,t_m\}t$
\\ \hline \hline
$\{t_1,\ldots,t_m,u_1,\ldots,u_n\}t$
\\
\end{tabular}
\end{center}

\noindent  where $u_1$, $\dots$, $u_n$
are derived by `resolving' the theorem $th$ with the existing assumptions
$t_1$, $\dots$, $t_m$.
Resolution in \HOL{} is not classical resolution, but just Modus Ponens with
one-way pattern matching (not unification) and term and type instantiation. The
general case is where $th$ is of the canonical form

$\ \ \ ${\small\verb%|- !%}$x_1$$\ldots x_p${\small\verb%.%}$v_1$ {\small\verb%==>%} $v_2$ {\small\verb%==>%} $\ldots$ {\small\verb%==>%} $v_q$ {\small\verb%==>%} $v$

\noindent {\small\verb%IMP_RES_TAC %}$th$ then tries to specialize $x_1$,
$\dots$, $x_p$ in succession so that $v_1$, $\dots$, $v_q$ match members of
$\{t_1,\ldots ,t_m\}$.  Each time a match is found for some antecedent $v_i$,
for $i$ successively equal to $1$, $2$, \dots, $q$, a term and type
instantiation is made and the rule of Modus Ponens is applied.  If all the
antecedents $v_i$ (for $1 \leq i \leq q$) can be dismissed in this way, then
the appropriate instance of $v$ is added to the assumptions. Otherwise, if only
some initial sequence $v_1$, \dots, $v_k$ (for some $k$ where $1 < k < q$) of
the assumptions can be dismissed, then the remaining implication:

$\ \ \ ${\small\verb%|- %} $v_{k+1}$ {\small\verb%==>%} $\ldots$ {\small\verb%==>%} $v_q$ {\small\verb%==>%} $v$

\noindent is added to the assumptions.

For a more detailed description of resolution and \ml{IMP\_RES\_TAC}, see
\REFERENCE.  (See also the Cambridge \LCF\ Manual \cite{new-LCF-man}.)

\item{\bf Use:} Deriving new results from a previously proved implicative
theorem, in combination with the current assumptions, so that subsequent
tactics can use these new results.

\end{itemize}




\subsection{Identity}

\index{ALL_TAC@\ml{ALL\_TAC}|pin}
\begin{holboxed}\begin{verbatim}
   ALL_TAC : tactic
\end{verbatim}\end{holboxed}

\begin{itemize}
\index{identity tactic}\index{tactics!identity for}
\index{THEN@\ml{THEN}}
\item{\bf Summary:} The identity tactic for the tactical {\small\verb%THEN%}
(see Section~\ref{tactics}). Useful for writing tactics.

\item{\bf Use:}
\begin{enumerate}
\index{REPEAT@\ml{REPEAT}}
\item Writing tacticals (see description of {\small\verb%REPEAT%}
in Section~\ref{tacticals}).
\index{THENL@\ml{THENL}}
\item With {\small\verb%THENL%} (see Section~\ref{avra_thenl});
for example, if tactic $T$ produces two subgoals
$T_1$ is to be applied to the first while
nothing is to be done to the second,
then $T${\small\verb% THENL[%}$T_1${\small\verb%;ALL_TAC]%} is the
tactic required.
\end{enumerate}
\end{itemize}

\subsection{Null}

\index{NO_TAC@\ml{NO\_TAC}|pin}
\begin{holboxed}\begin{verbatim}
   NO_TAC : tactic
\end{verbatim}\end{holboxed}

\begin{itemize}
\item{\bf Summary:} Tactic that always fails.

\item{\bf Use:} Writing tacticals.
\end{itemize}


\subsection{Splitting logical equivalences}


\begin{holboxed}
\index{EQ_TAC@\ml{EQ\_TAC}|pin}
\index{equality, in HOL logic@equality, in \HOL{} logic!tactic for splitting}
\begin{verbatim}
   EQ_TAC : tactic
\end{verbatim}\end{holboxed}

\begin{itemize}

\item{\bf Summary:}
{\small\verb%EQ_TAC%}
splits an equational goal into two implications (the `if-case' and
the `only-if' case):

\begin{center}



\begin{tabular}{c} \\
$u\ \ml{=}\ v$
\\ \hline \hline
$u\ \ml{==>}\ v\ \ \ \ \ v\ \ml{==>}\ u$
\\
\end{tabular}
\end{center}

\item{\bf Use:} Proving logical equivalences, \ie\ goals of the form
``$u$\ml{=}$v$'' where $u$ and $v$ are boolean terms.

\end{itemize}

\subsection{Solving existential goals}


\begin{holboxed}
\index{EXISTS_TAC@\ml{EXISTS\_TAC}|pin}
\index{existential quantifier, in HOL logic@existential quantifier, in \HOL{} logic!tactic for}
\begin{verbatim}
   EXISTS_TAC : term -> tactic
\end{verbatim}\end{holboxed}

\begin{itemize}

\item{\bf Summary:}
{\small\verb%EXISTS_TAC "%}$u${\small\verb%"%}
reduces an existential goal {\small\verb%!%}$x${\small\verb%. %}$t[x]$
to the subgoal $t[u]$.

\begin{center}
\begin{tabular}{c} \\
$\ml{!}x\ml{.} t[x]$
\\ \hline \hline
$t[u]$
\\
\end{tabular}
\end{center}

\item{\bf Use:} Proving existential goals.

\item{\bf Comment:} \ml{EXISTS\_TAC} is a crude way of solving
existential goals, but it is the only built-in tactic for this
purpose.  A more powerful approach uses Prolog-style `logic variables'
(\ie\ meta-variables)
that can be progressively refined towards the eventual witness.
Implementing this requires goals to contain an environment giving the binding
of logic variables to terms. Details (in the context of \LCF) are given
in a  paper by Stefan Soko\l owski \cite{Stefan}.

\end{itemize}
\index{tactics!list of some|)}

\section{Tacticals}
\label{tacticals}

\index{tactics!tacticals for|(}
\index{tacticals|(}
\index{tacticals!list of some|(}
\index{tacticals!purpose of}
A {\it tactical\/} is not represented by a single \ML{} type,
but is in general
an \ML{} function that returns a tactic (or tactics) as result.
Tacticals may take parameters, and this is reflected in the variety of
\ML{} types that the built-in tacticals have.
Tacticals are used for building compound tactics.\index{compound tactics, in HOL system@compound tactics, in \HOL{} system}
\index{tactics!compound}
Some important tacticals in
the \HOL{} system
are listed below.
For a complete list of the tacticals in \HOL{} see \REFERENCE.

\subsection{Alternation}\index{alternation!of tactics|(}\index{tactics!alternation of}

\index{ORELSE@\ml{ORELSE}|pin}
\begin{holboxed}\begin{verbatim}
   ORELSE : tactic -> tactic -> tactic
\end{verbatim}\end{holboxed}


The tactical {\small\verb%ORELSE%}
is an \ML{} infix. If $T_1$ and $T_2$ are tactics,
\index{tacticals!for alternation}
then the \ML{} expression $T_1${\small\verb% ORELSE %}$T_2$
evaluates to a tactic which applies $T_1$ unless that fails;
if it fails,
it applies $T_2$. \ml{ORELSE} is defined in \ML\
as a curried infix by

\begin{hol}\begin{verbatim}
   (T1 ORELSE T2) g =  T1 g ? T2 g
\end{verbatim}\end{hol}\index{alternation!of tactics|)}


\subsection{First success}

\index{FIRST@\ml{FIRST}|pin}
\begin{holboxed}\begin{verbatim}
   FIRST : tactic list -> tactic
\end{verbatim}\end{holboxed}

The tactical \ml{FIRST} applies the first tactic, in a list
of tactics, that succeeds.

\begin{hol}\begin{alltt}
   FIRST [\(T\sb{1}\);\(T\sb{2}\);\(\ldots\);\(T\sb{n}\)] = \(T\sb{1}\) ORELSE \(T\sb{2}\) ORELSE \(\ldots\) ORELSE \(T\sb{n}\)
\end{alltt}\end{hol}



\subsection{Change detection}

\index{CHANGED_TAC@\ml{CHANGED\_TAC}|pin}
\begin{holboxed}\begin{verbatim}
   CHANGED_TAC : tactic -> tactic
\end{verbatim}\end{holboxed}


\ml{CHANGED\_TAC\ $T$\ $g$} fails if the subgoals
produced by $T$ are just \ml{[$g$]}; otherwise it is equivalent
to $T\ g$. It is defined by the following, where
{\small\verb%set_equal : * list -> * list -> bool%} tests whether two lists
denote the same set (\ie\ contain the same elements).


\begin{hol}\begin{verbatim}
   letrec CHANGED_TAC tac g =
    let gl,p = tac g in
    if set_equal gl [g] then fail else (gl,p)
\end{verbatim}\end{hol}



\subsection{Sequencing}
\index{sequencing!of tactics|(}
\index{tacticals!for sequencing|(}
\index{tactics!sequencing of|(}
\index{THEN@\ml{THEN}!ML implementation of@\ML{} implementation of|(}

\begin{holboxed}\index{THEN@\ml{THEN}|pin}
\begin{verbatim}
   THEN : tactic -> tactic -> tactic
\end{verbatim}\end{holboxed}


The tactical {\small\verb%THEN%} is an \ML{} infix. If $T_1$ and $T_2$ are tactics,
then the \ML{} expression $T_1${\small\verb% THEN %}$T_2$ evaluates to a tactic
which first applies $T_1$ and then applies $T_2$ to each subgoal produced by
$T_1$. Its definition
 in \ML{} is complex (and due to Milner)\index{Milner, R.} but worth
understanding as an exercise in \ML.  It is an \ML{} curried infix.

\begin{hol}\begin{verbatim}
   let ((T1:tactic) THEN (T2:tactic)) g =
    let gl,p = T1 g
    in
    let gll,pl = split(map T2 gl)
    in
    (flat gll, (p o mapshape(map length gll)pl));;
\end{verbatim}\end{hol}

\noindent Here are
the definitions of the \ML{} functions \ml{map}, \ml{split}, \ml{o},
\ml{length}, \ml{flat} and \ml{mapshape}:

%\begin{itemize}
\bigskip

\index{map@\ml{map}}
{\small\verb%map : (* -> **) -> * list -> ** list%}

\medskip

\begin{hol}\begin{alltt}
   map \(f\) [\(x\sb{1}\);\(\ldots\);\(x\sb{n}\)]  =  [\(f\) \(x\sb{1}\);\(\ldots\);\(f\) \(x\sb{n}\)]
\end{alltt}\end{hol}

\medskip

\index{split@\ml{split}}
{\small\verb%split : (* # **) list -> (* list # ** list)%}

\medskip

\begin{hol}\begin{alltt}\   split[(\(x\sb{1}\),\(y\sb{1}\));\(\ldots\);(\(x\sb{n}\),\(y\sb{n}\))]  =  ([\(x\sb{1}\);\(\ldots\);\(x\sb{n}\)],[\(y\sb{1}\);\(\ldots\);\(y\sb{n}\)])
\end{alltt}\end{hol}


\medskip

{\small\verb%$o : ((* -> **) # (*** -> *)) -> *** -> **$%}
 (an infix)\index{function composition, in ML@function composition, in \ML}

\medskip

\begin{hol}
\begin{alltt}
   (\(f\) o \(g\)) \(x\)  =  \(f\)(\(g\) \(x\))
\end{alltt}\end{hol}
\index{ function composition operator, in ML@{\small\verb+o+} (function composition operator, in \ML)}

\medskip

\index{length@\ml{length}}
{\small\verb%length : * list -> int%}

\medskip

\begin{hol}\begin{alltt}
   length[\(x\sb{1}\);\(\ldots\);\(x\sb{n}\)]  =  n
\end{alltt}\end{hol}

\medskip


{\small\verb%flat : (* list) list -> * list%}\index{flat@\ml{flat}}


\medskip

\begin{hol}\begin{alltt}
   flat[[\({x\sb{1}}\sb{1}\);\(\ldots\);\({x\sb{1}}\sb{m\sb{1}}\)];[\({x\sb{2}}\sb{1}\);\(\ldots\);\({x\sb{2}}\sb{m\sb{2}}\)];\(\ldots\);[\({x\sb{n}}\sb{1}\);\(\ldots\);\({x\sb{n}}\sb{m\sb{n}}\)]] =
    [\({x\sb{1}}\sb{1}\);\(\ldots\);\({x\sb{1}}\sb{m\sb{1}}\);\({x\sb{2}}\sb{1}\);\(\ldots\);\({x\sb{2}}\sb{m\sb{2}}\); \(\ldots\) ;\({x\sb{n}}\sb{1}\);\(\ldots\);\({x\sb{n}}\sb{m\sb{n}}\)]
\end{alltt}\end{hol}

\medskip

{\small\verb%mapshape : int list -> (* list -> **) list -> * list -> ** list%}\index{mapshape@\ml{mapshape}}


\medskip

\begin{hol}\begin{alltt}
   mapshape
    [\(m\sb{1}\);\(\ldots\);\(m\sb{n}\)]
    [\(f\sb{1}\);\(\ldots\);\(f\sb{n}\)]
    [\({x\sb{1}}\sb{1}\);\(\ldots\);\({x\sb{1}}\sb{m\sb{1}}\);\({x\sb{2}}\sb{1}\);\(\ldots\);\({x\sb{2}}\sb{m\sb{2}}\); \(\ldots\) ;\({x\sb{n}}\sb{1}\);\(\ldots\);\({x\sb{n}}\sb{m\sb{n}}\)] =
   [\(f\sb{1}\)[\({x\sb{1}}\sb{1}\);\(\ldots\);\({x\sb{1}}\sb{m\sb{1}}\)];\(f\sb{2}\)[\({x\sb{2}}\sb{1}\);\(\ldots\);\({x\sb{2}}\sb{m\sb{2}}\)]; \(\ldots\) ;\(f\sb{n}\)[\({x\sb{n}}\sb{1}\);\(\ldots\);\({x\sb{n}}\sb{m\sb{n}}\)]]
\end{alltt}\end{hol}

%\end{itemize}

\bigskip

Suppose \ml{$T_1\ g$ = ($gl$,$p$)} where \ml{$gl$=[$g_1$;$\ldots$;$g_n$]}.
Suppose also that
for $i$ between $1$ and $n$ it is the case that
\ml{$T_2\ g_i$ = ([${g_i}_1$;$\ldots$;${g_i}_{m_i}$],$p_i$)}.
Then \ml{split(map $T_2$ $gl$)} will evaluate to the
pair \ml{($gll$,$pl$)} of a subgoal list and a proof function, where

\bigskip

\ml{$gll$ = [[${g_1}_1$;$\ldots$;${g_1}_{m_1}$];[${g_2}_1$;$\ldots$;${g_2}_{m_2}$];
$\ \ldots\ $;[${g_n}_1$;$\ldots$;${g_n}_{m_n}$]]}

\bigskip

\noindent and
\ml{$pl$ = [$p_1$;$\ldots$;$p_n$]}. Note that

\bigskip

\ml{map length $gll$ = [$m_1$;$\ldots$;$m_n$]}

\bigskip

\noindent and that

\bigskip

\ml{flat $gll$ = [${g_1}_1$;$\ldots$;${g_1}_{m_1}$;${g_2}_1$;$\ldots$;${g_2}_{m_2}$;
$\ \ldots\ $;${g_n}_1$;$\ldots$;${g_n}_{m_n}$]}

\bigskip

Suppose now that, for $i$ between $1$ and $n$, the theorems
${th_i}_1$, $\dots$, ${th_i}_{m_i}$ achieve
the goals ${g_i}_1$, $\dots$, ${g_i}_{m_i}$, respectively.
It will follow that if $T_2$ is valid
then for $i$ between $1$ and $n$
the result of applying $p_i$ to the list of
theorems \ml{[${th_i}_1$;$\ldots$;${th_i}_{m_i}$]}
will be a theorem, $th_i$ say, which achieves
$g_i$.
Now if $T_1$ is valid then \ml{$p$[$th_1$;$\ldots$;$th_n$]}
will evaluate to a theorem,
$th$ say,
that achieves the goal $g$. Thus

\begin{hol}\begin{alltt}
    \(p\)
    (mapshape
     (map length \(gll\))
     \(pl\)
     [\({th\sb{1}}\sb{1}\);\(\ldots\);\({th\sb{1}}\sb{m\sb{1}}\);\({th\sb{2}}\sb{1}\);\(\ldots\);\({th\sb{2}}\sb{m\sb{2}}\);\(\ \ldots\ \) ;\({th\sb{n}}\sb{1}\);\(\ldots\);\({th\sb{n}}\sb{m\sb{n}}\)]) =

    \(p\)([\(p\sb{1}\)[\({th\sb{1}}\sb{1}\);\(\ldots\);\({th\sb{1}}\sb{m\sb{1}}\)];\(p\sb{2}\)[\({th\sb{2}}\sb{1}\);\(\ldots\);\({th\sb{2}}\sb{m\sb{2}}\)];\(\ \ldots\ \);\(p\sb{n}\)[\({th\sb{n}}\sb{1}\);\(\ldots\);\({th\sb{n}}\sb{m\sb{n}}\)]]) =

    \(p\)([\(th\sb{1}\);\(\ldots\);\(th\sb{n}\)]) =

    \(th\)
\end{alltt}\end{hol}

This shows that
\index{justifications, in goal-directed proof search!THEN example of@\ml{THEN} example of}
\index{proof functions (same as justifications, validations)!THEN example of@\ml{THEN} example of}
\ml{$p$ o mapshape(map length $gll$)$pl$}
is a function that, when
applied to a list of theorems respectively
achieving \ml{flat $gll$}, returns a theorem
(namely $th$) that achieves $g$.\index{sequencing!of tactics|)}
\index{tacticals!for sequencing|)}
\index{THEN@\ml{THEN}!ML implementation of@\ML{} implementation of|)}

\subsection{Selective sequencing}

\index{THENL@\ml{THENL}}
\begin{holboxed}\begin{verbatim}
   THENL : tactic -> tactic list -> tactic
\end{verbatim}\end{holboxed}
\label{avra_thenl}

\index{selective sequencing tactical}
If tactic $T$ produces $n$ subgoals and $T_1$, $\dots$,
$T_n$ are tactics
then $T${\small\verb% THENL [%}$T_1${\small\verb%;%}$\ldots${\small\verb%;%}$T_n${\small\verb%]%}
is a tactic which first applies $T$ and then
applies $T_i$ to the $i$th subgoal produced by $T$.
The tactical {\small\verb%THENL%} is useful if one wants to apply different
tactics to different subgoals.

Here is the definition of \ml{THENL}:

\begin{hol}\begin{verbatim}
      let ((T:tactic) THENL (Tl:tactic list)) g =
       let gl,p = T g
       in
       let gll,pl = (split(map (\(T,g). T g) Tgl)
                      where Tgl = combine(Tl,gl) ? failwith `THENL`)
       in
       (flat gll, (p o mapshape(map length gll)pl))
\end{verbatim}\end{hol}

\noindent The understanding of this procedure is left as an exercise!\index{tactics!sequencing of|)}

\subsection{Successive application}



\begin{holboxed}
\index{EVERY, the ML function@\ml{EVERY}, the \ML{} function|pin}
\begin{verbatim}
   EVERY : tactic list -> tactic
\end{verbatim}\end{holboxed}

\index{tacticals!for successive application}
\index{successive application!tactical for}
The tactical \ml{EVERY} applies a list of tactics one after the other.


\begin{hol}\begin{alltt}
   EVERY [\(T\sb{1}\);\(T\sb{2}\);\(\ldots\);\(T\sb{n}\)] = \(T\sb{1}\) THEN \(T\sb{2}\) THEN \(\ldots\) THEN \(T\sb{n}\)
\end{alltt}\end{hol}



\subsection{Repetition}

\begin{holboxed}\index{REPEAT@\ml{REPEAT}|pin}
\begin{verbatim}
   REPEAT : tactic -> tactic
\end{verbatim}\end{holboxed}
\label{avra_repeat}

If $T$ is a
tactic then {\small\verb%REPEAT %}$T$ is a tactic\index{tactics!repetition of}
\index{tacticals!for repetition}\index{repetition!of tactics}
that repeatedly applies
$T$ until it fails. It is defined in \ML{} by:

{\small\baselineskip\HOLSpacing\begin{verbatim}
   letrec REPEAT T g = ((T THEN REPEAT T) ORELSE ALL_TAC) g
\end{verbatim}}

\noindent (The extra argument {\small\verb%g%} is needed because \ML{} does not use
lazy evaluation.)
\index{tacticals|)}
\index{tacticals!list of some|)}
\index{tactics!tacticals for|)}

\section{Tactics for manipulating assumptions}
\label{asm-manip}

\index{tactics!for manipulating assumptions|(}
There are in general two kinds of tactics\index{tactics!term transforming}\index{tactics!assumption transforming}
 in \HOL: those that transform the
conclusion of a goal without affecting the assumptions, and those that
do (also or only) affect the assumptions.  The various tactics that
rewrite\index{rewriting!main tactic for}
 are typical of the first class; those that do `resolution'
\index{resolution tactics} belong to
the second.  Often, many of the steps of a proof in \HOL{} are carried
out `behind the scenes' on the assumptions, by tactics of the second sort.
A tactic that in some way changes the assumptions must also have a
justification that `knows how' to restore the corresponding hypotheses of
the theorem achieving the subgoal. All of this is explicit, and can be
examined by a user moving about the subgoal-proof tree.\footnote{The current
subgoal package makes this difficult, but the point still holds.}
Using these tactics in the most straightforward way, the assumptions at any
point in a goal-directed proof, \ie\ at any node in the subgoal tree,
\index{subgoal tree!in proof construction}
\index{tree of subgoals, in proof construction} form
an unordered record of every assumption made, but not yet dismissed, up to that
point.

In practice, the straightforward use of assumption-changing
\index{assumptions!role of, in goal directed proof}
tactics,
with the tools currently provided in \HOL, presents at
least two difficulties.  The first is that assumption sets can grow to an
unwieldy size, the number and/or length of terms making them difficult to
read.  In addition, forward-search tactics such as resolution often add at least
some assumptions that are never subsequently used, and these have to be
carried along with the useful assumptions; the straightforward
method provides no ready way of intercepting their arrival.
Likewise, there is no straightforward way of discarding
\index{discarding assumptions}
\index{assumptions!discarding of, in proofs}
assumptions after they have been used and are merely adding to the clutter.
Although perhaps against the straightforward spirit, this is a perfectly valid
strategy, and
requires no more than a way of denoting
the specific assumptions to be discarded. That, however,
raises the more general problem of denoting\index{assumptions!denoting of, in proofs}\index{denoting assumptions} assumptions in the first place.
Assumptions are also denoted
so that they can be
manipulated: given as parameters, combined to draw inferences, \etc\  The only
straightforward way to denote them in the existing system is to supply
their quoted text.  Though adequate, this
method may result in bulky \ML{} expressions; and it may take some effort to present the text
correctly (with necessary type information, \etc).

As always in \HOL, there are quite a few ways around the various difficulties.
One approach, of course, is the one intended in the original
design of\index{LCF@\LCF!Edinburgh} Edinburgh \LCF,
and advocates the rationale for providing a full programming language, \ML,
\index{ML@\ML!purpose of, in HOL system@purpose of, in \HOL{} system}
rather than a simple proof command set: that is for the user to
implement new tactics in \ML.  For example, resolution tactics can be adapted
by the user to add new assumptions more selectively; and case analysis tactics
to make direct replacements without adding case assumptions.
This, again, is adequate, but can involve the user in extensive amounts of
programming, and in debugging exercises for which there is no
system support.

Short of implementing new tactics, two other standard
approaches are reflected in the current system.  Both were originally
developed for Cambridge \LCF\ \cite{lcp-rewrite,new-LCF-man}; both reflect
fresh views of the assumptions; and both rely on tacticals\index{tacticals!purpose of} that transform
tactics.  The two approaches are
partly but not completely complementary.

The first
approach, described in this section, implicitly regards the assumption
set, already represented as a list, as a stack, with a {\it pop\/}
operation, so that the assumption at the top of the stack can be (i) discarded
and (ii) denoted without explicit quotation.  (The corresponding {\it push\/}
adds new assumptions at the head of the list.)
The stack can be generalized to an array to allow for access to
arbitrary assumptions.

The other approach, described in Section~\ref{tacont},
gives a way of intercepting and manipulating results without them necessarily
being added as assumptions in the first place.  The two approaches can
be combined in \HOL{} interactions.


\subsection{Theorem continuations with popping}
\label{avra_manip1}

The first proof style, that of popping assumptions
\index{popping, of assumptions}
from the assumption `stack',
\index{assumptions!as stack}
\index{stack, of assumptions}
is illustrated using its main tool: the tactical \ml{POP\_ASSUM}.

\index{POP_ASSUM@\ml{POP\_ASSUM}|pin}
\begin{holboxed}\begin{verbatim}
   POP_ASSUM : (thm -> tactic) -> tactic
\end{verbatim}\end{holboxed}

\noindent Given a function $f$\ml{:thm -> tactic}, the tactic
\ml{POP\_ASSUM}\ $f$ applies $f$ to the (assumed) first
assumption of a goal (\ie\ to the top element of the assumption stack)
and then applies the tactic created thereby to the original goal
minus its top assumption:

\begin{hol}\begin{alltt}
   POP_ASSUM \(f\) ([\(t\sb{1}\);\(\ldots\);\(t\sb{n}\)],\(t\)) = \(f\) (ASSUME \(t\sb{1}\)) ([\(t\sb{2}\);\(\ldots\);\(t\sb{n}\)],\(t\))
\end{alltt}\end{hol}

\noindent \ML{} functions such as $f$,
with type \ml{thm -> tactic}, abbreviated to \ml{thm\_tactic},
\index{thm_tactic@\ml{thm\_tactic}}
are called theorem continuations,
\index{theorem continuations} suggesting the fact that they
take theorems and then continue the proof.\footnote{There is a superficial analogy
with continuations in denotational semantics.}
The use of \ml{POP\_ASSUM}\ can be illustrated by applying it
to a particular tactic, namely \ml{DISCH\_TAC} (Section~\ref{avradisch}).

\index{DISCH_TAC@\ml{DISCH\_TAC}}
\begin{holboxed}\begin{verbatim}
   DISCH_TAC : tactic
\end{verbatim}\end{holboxed}

\noindent On a goal whose conclusion is an implication $u \imp v$,
\ml{DISCH\_TAC} reflects the natural strategy of attempting to prove
$v$ under the assumption $u$, the discharged antecedent.  For example,
suppose it were required to prove that $(n = 0) \imp (n\times n = n)$:

\setcounter{sessioncount}{1}
\begin{session}\begin{verbatim}
#g "(n = 0) ==> (n * n = n)";;
"(n = 0) ==> (n * n = n)"

() : void

#e DISCH_TAC;;
OK..
"n * n = n"
    [ "n = 0" ]
\end{verbatim}\end{session}

\noindent Application of \ml{DISCH\_TAC} to the goal produces one subgoal,
as shown, with the added assumption. To engage the assumption
as a simple substitution, the tactic \ml{SUBST1\_TAC} is useful
(see \REFERENCE\ for details).


\index{SUBST1_TAC@\ml{SUBST1\_TAC}|pin}
\begin{holboxed}\begin{verbatim}
   SUBST1_TAC : thm_tactic
\end{verbatim}\end{holboxed}

\noindent \ml{SUBST1\_TAC} expects a theorem with an equational conclusion, and
substitutes accordingly, into the conclusion of the goal. At this
point in the session, the tactical
\ml{POP\_ASSUM} is applied to
\ml{SUBST1\_TAC} to form a new tactic.
The new tactic is applied to the current subgoal.

\begin{session}\begin{verbatim}
#top_goal();;
(["n = 0"], "n * n = n") : goal

#e(POP_ASSUM SUBST1_TAC);;
OK..
"0 * 0 = 0"
\end{verbatim}\end{session}

\noindent The result, as shown, is that the assumption is used as a
substitution rule and then discarded.
\index{discarding assumptions}
\index{assumptions!discarding of, in proofs}
The one subgoal therefore has no
assumptions on its stack.  The two tactics used thus far could be combined
into one using the tactical \ml{THEN}:\index{THEN@\ml{THEN}}

\setcounter{sessioncount}{1}
\begin{session}\begin{verbatim}
#g "(n = 0) ==> (n * n = n)";;
"(n = 0) ==> (n * n = 0)"

() : void

#e(DISCH_TAC THEN POP_ASSUM SUBST1_TAC);;
OK..
"0 * 0 = 0"
\end{verbatim}\end{session}

\noindent The goal can now be solved by rewriting with a fact of arithmetic:

\begin{session}\begin{verbatim}
#e(REWRITE_TAC[MULT_CLAUSES]);;
Theorem MULT_CLAUSES autoloaded from theory `arithmetic`.
MULT_CLAUSES =
|- !m n.
    (0 * m = 0) /\
    (m * 0 = 0) /\
    (1 * m = m) /\
    (m * 1 = m) /\
    ((SUC m) * n = (m * n) + n) /\
    (m * (SUC n) = m + (m * n))

OK..
goal proved
|- 0 * 0 = 0
|- (n = 0) ==> (n * n = n)
\end{verbatim}\end{session}

\noindent A single tactic can, of course, be written to solve the goal:

\setcounter{sessioncount}{1}
\begin{session}\begin{verbatim}
#g "(n = 0) ==> (n * n = n)";;
"(n = 0) ==> (n * n = n)"

() : void

#e(DISCH_TAC THEN POP_ASSUM SUBST1_TAC THEN REWRITE_TAC[MULT_CLAUSES]);;
Theorem MULT_CLAUSES autoloaded from theory `arithmetic`.
MULT_CLAUSES =
|- !m n.
    (0 * m = 0) /\
    (m * 0 = 0) /\
    (1 * m = m) /\
    (m * 1 = m) /\
    ((SUC m) * n = (m * n) + n) /\
    (m * (SUC n) = m + (m * n))

OK..
goal proved
|- (n = 0) ==> (n * n = n)
\end{verbatim}\end{session}

This example illustrates how the tactical \ml{POP\_ASSUM} provides
access\index{assumptions!denoting of, in proofs}
\index{denoting assumptions}
to the top of the assumption `stack' (a capability that
is useful, obviously, only when the
most recently pushed assumption is the very one required).
To accomplish this access in the straightforward way would
require some more awkward
\index{assumptions!explicit} construct, with explicit assumptions:

\setcounter{sessioncount}{1}
\begin{session}\begin{verbatim}
#g "(n = 0) ==> (n * n = n)";;
"(n = 0) ==> (n * n = n)"

() : void

#e(DISCH_TAC);;
OK..
"n * n = n"
    [ "n = 0" ]

() : void

#e(SUBST1_TAC(ASSUME "n = 0"));;
OK..
"0 * 0 = 0"
    [ "n = 0" ]
\end{verbatim}\end{session}

In contrast to the above, the popping example also illustrates the
convenient disappearance of an assumption no longer required, by removing it
from the stack at the moment when it is accessed and used. This is valid
because any theorem that achieves the subgoal
will still achieve the original goal. Discarding\index{discarding assumptions}\index{assumptions!discarding of, in proofs} assumptions
is a separate issue from accessing them;
there could, if one liked, be another
tactical that produced a similar tactic on a theorem continuation
to \ml{POP\_ASSUM} but which did not pop the
stack.

Finally, \ml{POP\_ASSUM} $f$ induces case splits where $f$ does.  To prove
$(n=0 \disj n=1) \imp (n\times n = n)$, the function \ml{DISJ\_CASES\_TAC}
can be used. The tactic

\ \ \ml{DISJ\_CASES\_TAC\ |- $p$}{\small\verb% \/ %}\ml{$q$}

\noindent splits a goal into two subgoals that have
$p$ and $q$, respectively, as new assumptions.


\setcounter{sessioncount}{1}
\begin{session}\begin{verbatim}
#g "((n = 0) \/ (n = 1)) ==> (n * n = n)";;
"(n = 0) \/ (n = 1) ==> (n * n = n)"

() : void

#e DISCH_TAC;;
OK..
"n * n = n"
    [ "(n = 0) \/ (n = 1)" ]

() : void

#backup();;
"(n = 0) \/ (n = 1) ==> (n * n = n)"
\end{verbatim}\end{session}
\vfill
\newpage
\begin{session}\begin{verbatim}
#e(DISCH_TAC THEN POP_ASSUM DISJ_CASES_TAC);;
OK..
2 subgoals
"n * n = n"
    [ "n = 1" ]

"n * n = n"
    [ "n = 0" ]

() : void

#backup();;
"(n = 0) \/ (n = 1) ==> (n * n = n)"

() : void

#e(DISCH_TAC THEN POP_ASSUM DISJ_CASES_TAC THEN POP_ASSUM SUBST1_TAC);;
OK..
2 subgoals
"1 * 1 = 1"

"0 * 0 = 0"
\end{verbatim}\end{session}

As noted earlier, \ml{POP\_ASSUM} is useful when an assumption
is required that is still at the top of the stack,
as in the examples.  However, it is often
necessary to access assumptions made at arbitrary previous times, in order to
give them as parameters, combine them, \etc\ The stack approach can be
extended to such cases by re-conceiving the stack as an array\index{assumptions!as array}\index{array, of assumptions}, and by
use of the tactical \ml{ASSUM\_LIST}:\index{ASSUM_LIST@\ml{ASSUM\_LIST}|(}

\begin{hol}\begin{verbatim}
   ASSUM_LIST : (thm list -> tactic ) -> tactic
\end{verbatim}\end{hol}

\noindent where

\begin{hol}\begin{alltt}
   ASSUM_LIST \m{f} ([\m{t\sb{1}};...;\m{t\sb{n}}],\m{t}) = \m{f}([ASSUME \m{t\sb{1}};...;ASSUME \m{t\sb{n}}])
\end{alltt}\end{hol}

\noindent That is, given a function $f$, \ml{ASSUM\_LIST}$\ f$ forms a new tactic
by applying $f$ to the list of (assumed) assumptions of a goal, then applies
the resulting tactic to the goal.  For example, a tactic of the form
{\small\verb%ASSUM_LIST (\thl.%}$\ f\ $\ml{(el\ $i$\ thl))} applies the
function $f$ to the $i$th assumption of a goal to produce a new tactic, then
applies the new tactic to the goal.
Again, \ml{ASSUM\_LIST REWRITE\_TAC} is a tactic that engages all of the
current assumptions as rewrite rules.
In this way, the array approach
enables arbitrary assumptions to be accessed; and in particular,
specific assumptions to be accessed by location using the function \ml{el}.

To illustrate the use of \ml{ASSUM\_LIST}, suppose it were required to prove
something different: \index{ASSUM_LIST@\ml{ASSUM\_LIST}|)}
that $(\forall m.\ m + n = m) \imp (n \times n = n)$.  Suppose
also that the arithmetic fact \ml{ADD\_INV\_{0}} is already known: namely, that
$\forall m\ n.\ (m + n = m) \imp (n = 0)$. After discharging the assumption,
the conclusion of the theorem \ml{ADD\_INV\_{0}} is imported as an
assumption, occupying first place in the array.

\setcounter{sessioncount}{1}
\begin{session}\begin{verbatim}
#g "(!m. m + n = m) ==> (n * n = n)";;
"(!m. m + n = m) ==> (n * n = n)"

() : void

#e(DISCH_TAC);;
OK..
"n * n = n"
    [ "!m. m + n = m" ]

() : void

#e(ASSUME_TAC ADD_INV_0);;
Theorem ADD_INV_0 autoloaded from theory `arithmetic`.
ADD_INV_0 = |- !m n. (m + n = m) ==> (n = 0)

OK..
"n * n = n"
    [ "!m. m + n = m" ]
    [ "!m n. (m + n = m) ==> (n = 0)" ]
\end{verbatim}\end{session}

\noindent The problem is now to combine the two assumptions to produce the
obvious conclusion. That requires denoting
\index{denoting assumptions}
\index{assumptions!denoting of, in proofs} them, for which \ml{ASSUM\_LIST}
provides the means. Finally,
\ml{ASSUME\_TAC} places the conclusion of the new result in the assumptions.
(The \ML{} function \ml{el: int -> * list -> *} is used here to select a
numbered element of a list.)

\begin{session}\begin{verbatim}
#e(ASSUM_LIST(\thl. ASSUME_TAC
                       (MP (SPECL ["m:num";"n:num"] (el 1 thl))
                           (SPEC "m:num"(el 2 thl)))));;
##OK..
"n * n = n"
    [ "!m. m + n = m" ]
    [ "!m n. (m + n = m) ==> (n = 0)" ]
    [ "n = 0" ]
\end{verbatim}\end{session}

\noindent The goal can now be solved as in the previous example.

To access the
two particular assumptions in the straightforward way would again require quoting
their text. To access all of them (to pass to \ml{REWRITE\_TAC}, for
instance) would require quoting all of them.

\ml{ASSUM\_LIST} addresses the issue of accessing assumptions,
but not the issue of discarding them.  A related function generalizes
\ml{POP\_ASSUM} to discard them as well:

\begin{hol}\begin{verbatim}
   POP_ASSUM_LIST : (thm list -> tactic ) -> tactic
\end{verbatim}\end{hol}

\noindent \ml{POP\_ASSUM\_LIST}
\index{POP_ASSUM_LIST@\ml{POP\_ASSUM\_LIST}}
resembles \ml{ASSUM\_LIST} except in removing
all of the old assumptions of the subgoal, the way that \ml{POP\_ASSUM}
removes the most recent.  (Thus \ml{POP\_ASSUM} is no more than a special case
of \ml{POP\_ASSUM\_LIST} that selects the first element of those supplied
and re-assumes the others.)

\begin{hol}\begin{alltt}
   POP_ASSUM_LIST \(f\) ([\(t\sb{1}\);\(\ \ldots\ \);\(t\sb{n}\)],\(t\)) =  \(f\) [ASSUME \(t\sb{1}\);\(\ \ldots\ \);ASSUME \(t\sb{n}\)] ([],t)
\end{alltt}\end{hol}

\noindent This is used when the existing assumptions have served
their purpose and can be discarded, as in the current example:

\begin{session}\begin{verbatim}
#backup();;
"n * n = n"
    [ "!m. m + n = m" ]
    [ "!m n. (m + n = m) ==> (n = 0)" ]

() : void

#e(POP_ASSUM_LIST(\thl. ASSUME_TAC
                           (MP (SPECL ["m:num";"n:num"] (el 1 thl))
                               (SPEC "m:num"(el 2 thl)))));;
##OK..
"n * n = n"
    [ "n = 0" ]
\end{verbatim}\end{session}

\noindent This leaves only the one assumption vital to solving the goal,
as before. In some contexts, the new result is required as an assumption,
but here it can be used immediately:

\begin{session}\begin{verbatim}
#backup();;
"n * n = n"
    [ "!m. m + n = m" ]
    [ "!m n. (m + n = m) ==> (n = 0)" ]

() : void

#e(POP_ASSUM_LIST(\thl. SUBST1_TAC
                           (MP (SPECL ["m:num";"n:num"] (el 1 thl))
                               (SPEC "m:num"(el 2 thl)))));;
##OK..
"0 * 0 = 0"
\end{verbatim}\end{session}

\noindent \ml{POP\_ASSUM\_LIST} can, of course,
take any function of appropriate
type, but is in fact often used in conjunction with the element-selecting
functions. Function composition occasionally allows a more
compact expression to be written.

The array view (of which the stack view is a special case)
gives a way in which unnecessary assumptions can
be dropped, and assumptions can be accessed, individually if necessary,
using tacticals.
Although this approach can be effective, as illustrated, it does
tend to rely on the ordering of the representation of the assumption
\index{assumptions!importance of ordering of} set.
(That is, \ml{POP\_ASSUM} necessarily does, while the other two provide the
temptation!) A minor drawback of this reliance is that tactics are then
sensitive to changes that alter the order or composition of the assumptions;
for example, changes in the implementation of \HOL, modifications of
existing tactics, and so on.
However, that sensitivity is not so serious in any one incarnation of \HOL;
there is a logical viewpoint that regards the assumptions (sequents) as
ordered anyway.
A more serious problem is that order-sensitive tactics are meaningful
only during interactive sessions; to reconstruct the assumptions from
the \ML{} text and the original goal alone is generally difficult,
and more so when assumptions are denoted by location.
This means that (i) the resulting tactics cannot easily be generalized
for use in other contexts, and (ii) the \ML{} text does not supply
useful documentation
\index{tactics!as documentation of proofs} of the solution of the goal.
Also, as shown in the last example, it slightly unsatisfactory
to push and subsequently pop assumptions, especially in immediate succession,
where this could be avoided.

Two other tacticals that can be used to manipulate the assumption list are
{\small\verb%FIRST_ASSUM%} and {\small\verb%EVERY_ASSUM%}.
These are characterized by:

\index{FIRST_ASSUM@\ml{FIRST\_ASSUM}}
\index{EVERY_ASSUM@\ml{EVERY\_ASSUM}}
\begin{hol}\begin{alltt}
   FIRST_ASSUM \(f\) ([\(t\sb{1}\); \(\ldots\) ;\(t\sb{n}\)], \(t\))  =
    (\(f\)(ASSUME \(t\sb{1}\)) ORELSE \(\ldots\) ORELSE \(f\)(ASSUME \(t\sb{n}\))) ([\(t\sb{1}\); \(\ldots\) ;\(t\sb{n}\)], \(t\))

   EVERY_ASSUM \(f\) ([\(t\sb{1}\); \(\ldots\) ;\(t\sb{n}\)], \(t\))  =
    (\(f\)(ASSUME \(t\sb{1}\)) THEN \(\ldots\) THEN  \(f\)(ASSUME \(t\sb{n}\))) ([\(t\sb{1}\); \(\ldots\) ;\(t\sb{n}\)], \(t\))
\end{alltt}\end{hol}


\subsection{Theorem continuations without popping}
\label{tacont}

The idea of the second approach is suggested by the way the array-style
tacticals\index{tacticals!purpose of} supply a list of theorems (the assumed assumptions)
to a function.  These tacticals use the function to
infer new
results from the list of theorems, and then to do something with the
results. In some cases,
\eg\ the last example, the assumptions need never have been made in the
first place, which suggests a different use of tacticals.
The original example for \ml{POP\_ASSUM}
illustrates this: namely, to show that $(n = 0) \imp (n\times n = n)$.  Here,
instead of discharging the antecedent by applying
\ml{DISCH\_TAC} to the goal, which adds the antecedent as an assumption
and returns the consequent as the conclusion,
and {\it then\/} supplying the (assumed) added assumption to the
theorem continuation \ml{SUBST1\_TAC} and
discarding it at the same time,
a tactical called \ml{DISCH\_THEN} is applied to \ml{SUBST1\_TAC} directly.
\ml{DISCH\_THEN} transforms \ml{SUBST1\_TAC} into
a new tactic: one that applies \ml{SUBST1\_TAC} directly to the (assumed)
antecedent, and the resulting tactic to a subgoal with no new
assumptions and the consequent as its conclusion:
\vfill
\newpage

\setcounter{sessioncount}{1}
\begin{session}\begin{verbatim}
#DISCH_THEN;;
- : (thm_tactic -> tactic)

#DISCH_THEN SUBST1_TAC;;
- : tactic

#g "(n = 0) ==> (n * n = n)";;
"(n = 0) ==> (n * n = n)"

() : void

#e(DISCH_THEN SUBST1_TAC);;
OK..
"0 * 0 = 0"
\end{verbatim}\end{session}

\noindent This gives the same result as the stack method, but more
directly, with a more compact \ML{} expression,
and with the attractive feature that the term
$n=0$ is never an assumption, even for an interval of one step.
This technique is often used at the moment when results are available;
as above, where the result produced by discharging the antecedent can be
immediately passed to substitution. If the result were only needed
later, it {\it would\/} have to be held as an assumption. However, results
can be manipulated when they are available, and their results
either held as assumptions or used immediately.
For example, to prove $(0=n) \imp (n \times n = n)$,
the result $n=0$ could be reversed
immediately:

\setcounter{sessioncount}{1}
\begin{session}\begin{verbatim}
#g "(0 = n) ==> (n * n = n)";;
"(0 = n) ==> (n * n = n)"

() : void

#e(DISCH_THEN(SUBST1_TAC o SYM));;
OK..
"0 * 0 = 0"
\end{verbatim}\end{session}

The justification of \ml{DISCH\_THEN SUBST1\_TAC} is easily constructed
from the justification of \ml{DISCH\_TAC} composed with the justification of
\ml{SUBST1\_TAC}. \index{assumptions!internal|(}
The term $n=0$ is assumed, to yield the
theorem that is passed to the theorem continuation \ml{SUBST1\_TAC},
and it is accordingly discharged during the construction of the
actual proof; but the assumption happens
only internally
\index{assumptions!internal|)} to the tactic \ml{DISCH\_THEN SUBST1\_TAC}, and not
as a step in the tactical proof.  In other words, the subgoal tree here
has one node fewer than before, when an explicit step (\ml{DISCH\_TAC})
reflected the assumption.

On the goal with the disjunctive antecedent, this method again
provides a compact tactic:

\setcounter{sessioncount}{1}
\begin{session}\begin{verbatim}
#g "((n = 0) \/ (n = 1)) ==> (n * n = n)";;
"(n = 0) \/ (n = 1) ==> (n * n = n)"

() : void

#e(DISCH_THEN(DISJ_CASES_THEN SUBST1_TAC));;
OK..
2 subgoals
"1 * 1 = 1"

"0 * 0 = 0"
\end{verbatim}\end{session}

\noindent This avoids the repeated popping and pushing of the stack
solution, and likewise, gives a shorter \ML{} expression. Both give
a shorter expression than the direct method, which is:

\begin{hol}\begin{verbatim}
   DISCH_TAC
    THEN DISJ_CASES_TAC(ASSUME "(n = 0) \/ (n = 1)")
    THENL[SUBST1_TAC(ASSUME "n = 0");
          SUBST1_TAC(ASSUME "n = 1")]
\end{verbatim}\end{hol}

To summarize, there are so far at least five ways to solve a goal
(and these are often combined in one interaction):
directly, using the stack view of the assumptions,
using the array view with or without discarding assumptions, and using a
tactical to intercept an assumption step.  All of the following work
\index{assumptions!compared methods of handling}
on the goal $(n=0) \imp (n \times n = n)$:

\begin{hol}\index{ASSUM_LIST@\ml{ASSUM\_LIST}}
\begin{verbatim}
   DISCH_TAC
    THEN SUBST1_TAC(ASSUME "n = 0")
    THEN REWRITE_TAC[MULT_CLAUSES]

   DISCH_TAC
    THEN POP_ASSUM SUBST1_TAC
    THEN REWRITE_TAC[MULT_CLAUSES]

   DISCH_TAC
    THEN ASSUM_LIST (SUBST1_TAC o el 1)
    THEN REWRITE_TAC[MULT_CLAUSES]

   DISCH_TAC
    THEN POP_ASSUM_LIST (SUBST1_TAC o el 1)
    THEN REWRITE_TAC[MULT_CLAUSES]

   DISCH_THEN SUBST1_TAC
    THEN REWRITE_TAC[MULT_CLAUSES]
\end{verbatim}\end{hol}

\noindent Furthermore, all five induce the
same sequence of inferences leading to
the desired theorem; internally, no inference steps are saved by the
economies in the \ML{} text or the subgoal tree.  In this sense,
the choice is entirely one of style and taste;
of how to organize the decomposition into subgoals.
The first expression illustrates the verbosity of denoting
assumptions by text (the goal with the
disjunctive antecedent gave a clearer
example); but also
the intelligibility of the resulting expression, which, of course, is all
that is saved of the interaction, aside from the final theorem.
The last expression
illustrates both the elegance and the inscrutibility of
using functions to manipulate intermediate results directly, rather than
as assumptions.
The middle three expressions
show how results can be used as assumptions (discarded when
redundant, if desired); and how
assumptions can be denoted without
recourse to their text.
It is a strength of the \LCF\ approach
\index{LCF@\LCF} to
theorem proving that many different proof styles are supported,
(all in a secure way) and indeed, can be studied in their own
right.

\HOL{} provides several other theorem continuation functions analogous to
\ml{DISCH\_THEN} and \ml{DISJ\_CASES\_THEN}.
(Their names always end with
`\ml{\_THEN}', `\ml{\_THENL} or `\ml{\_THEN2}'.)
Some of these do convenient inferences for the user.
For example:

\index{CHOOSE_THEN@\ml{CHOOSE\_THEN}|pin}
\begin{holboxed}\begin{verbatim}
   CHOOSE_THEN : thm_tactical
\end{verbatim}\end{holboxed}

\noindent Where \ml{thm\_tactical} abbreviates
{\small\verb%thm_tactic -> tactic%}.
\ml{CHOOSE\_THEN\ $f$\ (|-\ ?$x$.$t[x]$)}
is a tactic that, given a goal, generates the subgoal
obtained
by applying $f$ to \ml{($t[x]$|-$t[x]$)}.  The intuition is that if
\ml{|-\ ?$x$.$t[x]$} holds then \ml{|-\ $t[x]$}
holds for some value of $x$ (as long as the
variable $x$ is not free elsewhere in the theorem or current goal).
%(The choice of the witness is `understood' by the justification function.)
This gives an easy way of using existentially quantified theorems,
something that is otherwise awkward.

The new method has other applications as well, including as an
implementation technique.
For example,
\index{tactics!indirect implementation of}
taking \ml{DISJ\_CASES\_THEN} as basic, \ml{DISJ\_CASES\_TAC}
can be defined by:

\begin{hol}\begin{verbatim}
   let DISJ_CASES_TAC = DISJ_CASES_THEN ASSUME_TAC
\end{verbatim}\end{hol}

\noindent Similarly, the method is useful for modifying existing tactics
(\eg\ resolution tactics) without
having to re-program them in \ML.  This avoids the danger of
introducing tactics
whose justifications may fail,
\index{failure, of tactics} a particularly difficult problem to
track down; it is also much easier than starting from scratch.

The main theorem continuation functions in the system are:

\begin{hol}\begin{verbatim}
   ANTE_RES_THEN
   CHOOSE_THEN      X_CHOOSE_THEN
   CONJUNCTS_THEN   CONJUNCTS_THEN2
   DISJ_CASES_THEN  DISJ_CASES_THEN2   DISJ_CASES_THENL
   DISCH_THEN
   IMP_RES_THEN
   RES_THEN
   STRIP_THM_THEN
   STRIP_GOAL_THEN
\end{verbatim}\end{hol}

\noindent See \REFERENCE\ for full details.
\index{tactics!for manipulating assumptions|)}





%%% Local Variables:
%%% mode: latex
%%% TeX-master: "description"
%%% End:
