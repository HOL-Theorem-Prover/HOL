\documentclass[10pt]{scrartcl}

\usepackage[utf8]{inputenc}
\usepackage[english]{babel}

\usepackage[backend=biber]{biblatex}
\addbibresource{references.bib}

\usepackage{alltt}

\usepackage{booktabs}

\usepackage[scaled=0.8]{beramono}  % our monospace font
\usepackage[T1]{fontenc}  % this is necessary for beramo to work

\usepackage{listings}
\usepackage{upquote}

\lstdefinelanguage{hol4}{
  %alsoletter={@=>},
  morekeywords={
     Definition, End, Theorem, Proof, QED},
  %  fun, let, val, in, end, if, then, else, case, of},
  %sensitive=true,
  %morecomment=[l]{//},
  morecomment=[s]{(*}{*)},
  commentstyle=\color{gray},
  showstringspaces=false,
  columns=fullflexible,
  mathescape=true,
  numberstyle=\tiny,
  basicstyle=\ttfamily,
  numbersep=5pt,
  stepnumber=2,
  numbers=none,                   % where to put the line-numbers
  morestring=[b]"
}
\lstset{language=hol4, breaklines=true}
\usepackage{listings}

\usepackage{hyperref}
\usepackage{todonotes}
\newcommand{\heiko}[1]{\todo[inline,author=Heiko,bordercolor=red!20,color=orange!20]{#1}}
\newcommand{\IG}[1]{\todo[inline,author=Ivan,bordercolor=blue!20,color=orange!20]{#1}}
\newcommand{\ekey}[1]{\texttt{#1}}

\addto\extrasenglish{
  \def\sectionautorefname{Section}
  \def\subsectionautorefname{Section} % intentional
}


\title{A Programmer's Guide to Proving Theorems with HOL4}
\author{Anonymous}
\date{}

\begin{document}
\maketitle{}

\paragraph*{Summary}
Interactive theorem proving is a method to perform trustworthy, rigorous
mathematical proofs checked by computers.
From a programmer's point of view, learning an ITP system is similar to
learning a new programming language.
The main difference is that an ITP system allows to first implement a function,
and then prove theorems about it.
This guide is written for experienced programmers, that have seen pen-and-paper
proofs before, but never worked with an ITP system.
If you ever wanted to replace that nice, cozy thought of \emph{this function
should work as I intend it to} with the high-assurance of a rigorous
mathematical proof, this guide is for you\footnote{If you are asking yourself the question how this is even possible, the same reasoning applies.}.
%
\section{Introduction/Setup}
%
As with any programming language, picking up an ITP system first requires
installing the system.
In this guide we use the HOL4~\cite{HOL4web} theorem prover.
The guide also uses a tool called \emph{Lassie}, equipping HOL4 with training
wheels.

To follow this guide, \texttt{polyML}, \texttt{Apache Ant}, and \texttt{ruby}
must be installed.
For \texttt{polyML}, we recommend downloading the latest version from git
(\url{https://github.com/polyml/polyml}).
Both \texttt{Apache Ant}, and \texttt{ruby} can be installed through the system
package manager on mainline Linux distributions.
On Mac OS they should be available through \texttt{brew}.
We provide a Debian setup script that makes sure all the dependencies are
installed,
and downloads and installs the latest versions of HOL4 and Lassie with this
guide (\autoref{sec:script}).
Running the setup script temporarily sets the variables \texttt{HOLDIR}, and
\texttt{LASSIEDIR} to the directories where HOL4 and Lassie have been installed.
To use both tools later, these variables must be be configured permanently
before running HOL4.

\subsection{Running HOL4}

We will explain how to interact using HOL4 using the emacs editor.
If you are not familiar with using emacs in general, we recommend the
built-in tutorial which is started by pressing \ekey{Control} together with
\ekey{h}, then \ekey{t}. An online version is available at
\url{https://www.gnu.org/software/emacs/tour/}.
There also exist plugins for interacting with HOL4 using the vim, and Sublime
text editors, and the overall interaction is the same, only the keybindings may
change.

When emacs is started, it normally tries to load a Lisp program from an
initialization file, or init file for short. This file, if it exists, specifies
how to initialize Emacs for you. Emacs looks for your init file using the
filenames \verb!~/.emacs!, \verb!~/.emacs.el!, and
\verb!~/.emacs.d/init.el!; you can choose to use any one of these three names.
Here, \verb!~/! stands for your home directory
(\url{https://www.gnu.org/software/emacs/manual/html_node/emacs/Init-File.html}).

To interact with HOL4 within emacs, it suffices to append the following lines to
the emacs configuration file:
\begin{lstlisting}
(load-file "<HOL install dir>/tools/hol-unicode.el")
(load-file "<HOL install dir>/tools/hol-input.el")
(load-file "<HOL install dir>/tools/holscript-mode.el")
(load-file "<HOL install dir>/tools/hol-mode.el")
\end{lstlisting}
\noindent Here, \lstinline{<HOL install dir>} is the same path as the one
to which \texttt{HOLDIR} is set.

After loading the files, respectively restarting emacs, HOL4 files will by
default open in the HOL4 script mode.

In general, HOL4 files always end with the suffix \texttt{Script.sml}.
For example, a theory about natural numbers would be called
\texttt{naturalNumbersScript.sml}
In HOL4 speak, files ending with \texttt{Script.sml} are called \emph{script files}.

To interact with HOL4 we must first start a read-eval-print loop (REPL), similar
to interpreter of programming languages like ruby or python.
After opening a script file in emacs, HOL4 is started by presssing
first the \texttt{Alt} key together with \texttt{h}, then \texttt{H}.
In emacs this keybinding is abbreviated as \ekey{M-h H}, where \ekey{M} refers
to the \texttt{Alt} key\footnote{The \ekey{M} originates from the fact that
  emacs users refer to the \ekey{Alt} key as \emph{Meta}.}.
After pressing the keybinding, emacs interactively prompts for a position of
the HOL4 REPL (the prompt reads \lstinline{HOL buffer position:}).
Possible options are \texttt{vertical}, \texttt{horizontal}, and
\texttt{new-frame}.
The first two split the current either vertically or horizontally to show the
HOL4 REPL.
Option \texttt{new-frame} will open the REPL in a separate window (as
\emph{frame} is ``emacs speak'' for windows).
For users with a single monitor, we recommend \texttt{horizontal}, and if two
or more monitors are available, \texttt{new-frame}.

A closing remark on using emacs:
If you ever find yourself stuck, pressing \ekey{C-g} (\ekey{Control} and
\ekey{g} at the same time) will abort any emacs process and get you back to
editing text, and pressing \ekey{M-h} {C-c} (\ekey{Alt} and \ekey{h} followed by
\ekey{Control} and \ekey{c}) aborts any running HOL4 REPL process that may have
diverged.
%%% Local Variables:
%%% mode: latex
%%% TeX-master: "main"
%%% End:

%
\section{Interacting with the HOL4 REPL}

The HOL4 REPL is an extended version of the Poly/ML~\cite{polymlweb} REPL, and
behaves like the REPL's of other interpreted languages.
In general, it is recommend to first open a script file before starting
the REPL.

\textbf{Important setup note:} We strongly recommend opening a file
\texttt{tutorialScript.sml} in the folder \texttt{\$LASSIEDIR/examples}, to make
sure that the below code can be run easily.
This is done by either manually creating an empty file named \texttt{tutorialScript.sml} in \texttt{\$LASSIEDIR/examples}
and opening it with the toolbar buttons in emacs, or using \ekey{C-x C-f} (\ekey{Control} and \ekey{x}, then \ekey{Control} and \ekey{f})
to open it within emacs, by typing in the path.

We will explain in \autoref{sec:libraries} how the file should start to make its
contents reusable. For now we simply use it as a scratchpad.
The REPL is started by pressing \ekey{M-h H} (\ekey{Alt} and \ekey{h}, then \ekey{H}),
and HOL4 code is send to the REPL with the keybinding \ekey{M-h M-r}
(pressing \ekey{Alt} and \ekey{h}, then \ekey{Alt} and \ekey{r}).

For example, type
\begin{lstlisting}
  3 + 5;
\end{lstlisting}

anywhere in the currently opened script file.
Sending is then done by first highlighting the code with emacs using \ekey{C-space}.
Here, \ekey{C} stands for \ekey{Control}, so to start marking text, press
\ekey{Control} together with the \ekey{space} key.
The arrow keys are used for marking the code to be send to the REPL, and once it
has been completely selected, press \ekey{M-h M-r}.
Alternatively, holding down the shift key while moving the cursor will also select text.

The REPL should print:
\begin{lstlisting}[frame=single]
> 3+5;
val it = 8: int
\end{lstlisting}

All of the functionality of the Poly/ML REPL, and in general, the Standard ML
basis library (see e.g. \url{https://smlfamily.github.io/Basis/} for a reference)
are available in the HOL4 REPL.
Thus HOL4 supports creating and manipulating lists, strings, options, and
simple I/O.

It is strongly recommended to type or copy the code snippets from this tutorial
into an actual script file and sending them to the REPL with the keybinding \ekey{M-h M-r}.
This makes sure that the code can be experimented with and commands that have
been entered are not lost in the limbo of the REPL printouts.

As a quick point of reference, \autoref{tbl:keybindings} gives a short,
executive summary of the most commonly used keybindings, taken from the
documentation of the HOL4 emacs mode (\url{https://hol-theorem-prover.org/hol-mode.html}).

\begin{table}
  \centering
\begin{tabular}{@{}cll@{}}
  \toprule
  Keybinding & \multicolumn{1}{c}{Effect} & \multicolumn{1}{c}{Remark}\\
  \midrule
  \ekey{M-h H} & Start a new HOL4 session & \\
  \ekey{M-h M-r} & Send marked text to REPL & \\
  \ekey{M-h g} & Start a new proof & Must be within a \texttt{Theorem}, \texttt{Proof} block\\
  \ekey{M-h e} & Applies a tactic & Marked SML code must have type `tactic'\\
  \ekey{M-h d} & Stop current interactive proof \\
  \bottomrule
\end{tabular}
  \caption{Most common HOL4-mode keybindings}\label{tbl:keybindings}
\end{table}
%%% Local Variables:
%%% mode: latex
%%% TeX-master: "lassie-tutorial"
%%% End:

%
\section{A First (Pen and Paper Style) Proof}\label{sec:hol_ex1}
%
The HOL4 features presented so far are exactly those of an interpreted
programming language.
Next, we will define our first function in HOL4, and prove a first theorem about
it.

%A common mathematical notation is $\sum_{i=0}^n f (i)$,
%summing the numbers from $0$ to $n$, and applying function $f$.
%We define a specialized version for $f (x) = x$ in HOL4:
The function we will define is a sum of natural numbers up to $n$,
$\sum_{i=0}^n i$.
A HOL4 definition looks like this:

\begin{lstlisting}
Definition sumFun_def:
  sumFun (n:num) = if (n = 0) then 0 else n + sumFun (n-1)
End
\end{lstlisting}

The \lstinline{Definition} and \lstinline{End} keyword tell the REPL that we
define a HOL4 function and mark its end.
In the REPL, \lstinline{sumFun_def} is the name of the definition, under which it
can be accessed.
As a convention, when defining function $f$ in HOL4, its definition should be
named \lstinline{f_def}.
The type annotation \lstinline{n:num} tells the HOL4 REPL to parse variable \texttt{n}
as a natural number.
To load the definition into the HOL4 REPL, mark it completely, including the \lstinline{Definition}
keyword and the \lstinline{End} keyword, and send it to the REPL with \ekey{M-h M-r}.

Alternatively, HOL4 also supports defining a function by a system of equations,
moving the \texttt{case} expression of the \lstinline{sum} function to the
outside:
%
\begin{lstlisting}
Definition sum_def:
  sum 0 = 0 /\
  sum (n:num) = n + sum (n-1)
End
\end{lstlisting}

To avoid a name clash we have renamed the function into \lstinline{sum}.
As for the definition of \lstinline{sumFun_def}, to send the definition to the REPL,
mark it and send it with \ekey{M-h M-r}.
Choosing one definition over the other has different benefits and downsides.
As a rule of thumb, it is recommended to choose the latter version, giving a
system of equations if the function requires a top-level \texttt{case}
expression.
We will do the proof for the function \texttt{sum} here, and give a general
guideline on when to prefer which version later in \autoref{subsec:tipsAndTricks}.

As a simple, first example, we will prove a closed form for \lstinline{sum n}:
\[
  \sum_{i=0}^{n} i = \frac{n * (n + 1)}{2}
\]

In HOL4 this theorem is stated as

\begin{lstlisting}
Theorem closed_form_sum:
  ! n. sum n = n * (n + 1) DIV 2
Proof
QED
\end{lstlisting}

Again, \lstinline{Theorem}, \lstinline{Proof}, and \lstinline{QED} are the
keywords marking a theorem statement in the REPL and the indented line is
the statement that we want to prove.
Similar to a definition, the name \lstinline{closed_form_sum} is an identifier
which is used later to refer to the theorem statement proven in other proofs.
This makes theorems first class citizens of the HOL4 REPL, also allowing
functions to manipulate and inspect their statements.

When proving a theorem for the first time in HOL4, the proof is usually done
interactively.
Starting an interactive proof is as simple as marking the indented line
(\lstinline{! n. sum n = n * (n + 1) DIV 2}) and pressing \ekey{M-h g}.
The HOL4 REPL prints

\begin{lstlisting}[mathescape=true, frame=single]
> val it =
   Proof manager status: 1 proof.
   1. Incomplete goalstack:
        Initial goal:
        $\forall$ n. sum n = n * (n + 1) DIV 2
   : proofs
\end{lstlisting}

In HOL4, theorems are proven by applying so-called \emph{tactics} to the current
goal.
These tactics are a group of SML functions, implemented in the HOL4
distribution, and filled in between the \lstinline{Proof} and the \lstinline{QED}
keywords.
In this tutorial, we decouple learning how the theorem prover works from
learning the syntax of the tactics language by performing interactive proofs
with Lassie using natural language.

To load Lassie and the natural language descriptions required for the proof,
run
\begin{lstlisting}
open LassieLib arithTacticsLib logicTacticsLib arithmeticTheory;
val _ = LassieLib.loadJargon "Arithmetic";
val _ = LassieLib.loadJargon "Logic";
\end{lstlisting}
interactively.

The closed form is the standard example for proofs by induction in math classes.
Following this example, we start the proof with
\begin{lstlisting}
nltac `Induction on 'n'.`
\end{lstlisting}

Here, \lstinline{nltac} is a Lassie function that parses natural language and
translates it into a HOL4 tactic.
The parameter \lstinline{`Induction on 'n'.`} is the natural language
description of the tactic used.
To apply the tactic, the line must be marked and run with \ekey{M-h e}.
After running the code, the HOL4 REPL shows
\begin{lstlisting}
> OK..
2 subgoals:
val it =

    0.  sum n = n * (n + 1) DIV 2
   ------------------------------------
        sum (SUC n) = SUC n * (SUC n + 1) DIV 2

   sum 0 = 0 * (0 + 1) DIV 2

2 subgoals
   : proof
\end{lstlisting}

The line \lstinline{2 subgoals} tells us that we must prove two separate goals
to finish the proof.
As HOL4 keeps track of these subgoals for us, we need not manage them manually
to make sure that the proof remains error-free.
Note that in the induction step, HOL4 automatically adds the inductive
hypothesis as an assumption (labeled with \lstinline{0}) above a dashed line.

First, we prove the base case \lstinline{sum 0 = 0 * (0 + 1) DIV 2}, then we
show the induction step \lstinline{sum (n + 1) = (n + 1) * (n + 2) DIV 2}.
Function \lstinline{SUC} is the HOL4 version of Peano's successor function.
Intuitively \lstinline{SUC n} refers to the natural number after \lstinline{n},
i.e. \lstinline{n + 1}.

As for a pen-and-paper proof, the base case of the induction is trivial, and
solved with the simple statement \lstinline{nltac `use [sum_def] to simplify.`},
leaving us only with the induction step from above.
In contrast to a pen-and-paper proof, we have to explicitly state that we
simplify with the definition of our summation function (\lstinline{sum_def}).
This is part of the enforced rigour required by the theorem prover\footnote{
We will show in \autoref{subsec:tipsAndTricks} how one can get rid of this in certain cases.}.

As for a pen-and-paper proof, the first step on the induction step is to
simplify:
\begin{lstlisting}
nltac `use [sum_def, GSYM ADD_DIV_ADD_DIV] to simplify.`
\end{lstlisting}

Here, \lstinline{ADD_DIV_ADD_DIV} is a theorem from the HOL4 standard library
used to enrich the simplifier with the additional knowledge.
To find out its statement, mark the theorem name only, and send it to the
REPL with \ekey{M-h M-r}.
Sending \lstinline{GSYM} to the REPL shows that the function has type
\lstinline{:thm -> thm}, meaning that it takes a theorem as input and returns a
theorem.
Function \lstinline{GSYM} is a polyML function rotating an equality theorem,
replacing equality $a = b$ with equality $b = a$.
Sending \lstinline{GSYM ADD_DIV_ADD_DIV} and \lstinline{ADD_DIV_ADD_DIV} to the REPL, one can
observe its effect easily.

Using \lstinline{GSYM} can be useful from time to time as rewriting in HOL4 is
directed from left-to-right.
If we have a theorem showing $f x = b$, HOL4 will rewrite any occurence of
$f x$ into an occurence of $b$, but it will never replace occurences of $b$
with occurences of $f x$.

After applying the tactic, the REPL will show the subgoal that remains to be proven:
\begin{lstlisting}
> OK..
1 subgoal:
val it =

    0.  sum n = (n * (n + 1)) DIV 2
   ------------------------------------
        (2 * SUC n + n * (n + 1)) DIV 2 = SUC n * (SUC n + 1) DIV 2

   : proof
\end{lstlisting}

Applying the following tactics step-by-step closes the proof:

\begin{lstlisting}
nltac `'2 * SUC n + n * (n + 1) = SUC n * (SUC n + 1)' suffices to show the goal.`
nltac `show 'SUC n * (SUC n + 1) = (SUC n + 1) + n * (SUC n + 1)' using (simplify with [MULT_CLAUSES]).`
nltac `simplify.`
nltac `show 'n * (n + 1) = SUC n * n' using (trivial using [MULT_CLAUSES, MULT_SYM]).`
nltac `'2 * SUC n = SUC n + SUC n' follows trivially.`
nltac `'n * (SUC n + 1) = SUC n * n + n' follows trivially.`
nltac `rewrite assumptions. simplify.`
\end{lstlisting}

\begin{sloppypar}
The natural language tactic \lstinline{`show 'SUC n * (SUC n + 1) = (SUC n + 1) + n * (SUC n + 1)' using (simplify with [MULT_CLAUSES]).`}
shows another feature of HOL4:
We can extend the list of assumptions with the theorem mentioned after
\lstinline{show}.
Before running the tactic, the state of the goal is
\end{sloppypar}
\begin{lstlisting}
> OK..
1 subgoal:
val it =

    0.  sum n = n * (n + 1) DIV 2
   ------------------------------------
        2 * SUC n + n * (n + 1) = SUC n * (SUC n + 1)

   : proof
\end{lstlisting}

and after running the tactic, the subgoal becomes
\begin{lstlisting}
> OK..
1 subgoal:
val it =

    0.  sum n = n * (n + 1) DIV 2
    1.  SUC n * (SUC n + 1) = SUC n + 1 + n * (SUC n + 1)
   ------------------------------------
        2 * SUC n + n * (n + 1) = SUC n * (SUC n + 1)

   : proof
\end{lstlisting}

Running all tactics, one after another, the REPL shows that the proof is finished by printiting
\begin{lstlisting}
> OK..

val it =
   Initial goal proved.
   $\vdash$ $\forall$ n. sum n = n * (n + 1) DIV 2: proof
\end{lstlisting}

To reuse the theorem later, and to make it automatically checkable by HOL4, we
have to put the natural language into a single call to \lstinline{nltac}.
The full code for the theorem is given in \autoref{fig:gaussProof}.
%
\begin{figure}[t]
\begin{lstlisting}[mathescape=true]
Theorem closed_form_sum:
  $\forall$ n. sum n = (n * (n + 1)) DIV 2
Proof
  nltac `
   Induction on 'n'.
   use [sum_def] to simplify.
   use [sum_def, GSYM ADD_DIV_ADD_DIV] to simplify.
   use [sum_def, GSYM ADD_DIV_ADD_DIV] to simplify.
   '2 * SUC n + n * (n + 1) = SUC n * (SUC n + 1)' suffices to show the goal.
   show 'SUC n * (SUC n + 1) = (SUC n + 1) + n * (SUC n + 1)' using (simplify with [MULT_CLAUSES]).
   simplify.
   show 'n * (n + 1) = SUC n * n' using (trivial using [MULT_CLAUSES, MULT_SYM]).
   '2 * SUC n = SUC n + SUC n' follows trivially.
   'n * (SUC n + 1) = SUC n * n + n' follows trivially.
   rewrite assumptions. simplify.
QED
\end{lstlisting}
\caption{Complete proof of the closed form for summing natural numbers until $n$ using Lassie}\label{fig:gaussProof}
\end{figure}

Marking the complete statement, and running it with \ekey{M-h M-r} will save the
theorem under the name \lstinline{gaussian_sum}.
%%% Local Variables:
%%% mode: latex
%%% TeX-master: "main"
%%% End:

%
\chapter{Librerie}\label{HOLlibraries}

% LaTeX macros in HOL manuals
%
% \holtxt{..}     for typewriter text that is HOL types or terms.  To
%                 produce backslashes, for /\, \/ and \x. x + 1, use \bs
% \ml{..}         for typewriter text that is ML input, including the
%                 names of HOL API functions, such as mk_const
% \theoryimp{..}  for names of HOL theories.

% text inside \begin{verbatim} should be indented three spaces, unless
% the verbatim is in turn inside a \begin{session}, in which case it
% should be flush with the left margin.


\newcommand{\simpset}{simpset}
\newcommand{\Simpset}{Simpset}

 \newcommand{\term}      {\mbox{\it term}}
 \newcommand{\vstr}      {\mbox{\it vstr}}

Una \emph{libreria} � un'astrazione intesa fornire un livello pi� alto
di organizzazione per applicazioni \HOL{}. In generale, una libreria pu�
contenere una collezione di teorie, procedure di dimostrazione, e materiale
di supporto, come la documentazione. Alcune librerie forniscono semplicemente
procedure di dimostrazione, come \ml{simpLib}, mentre altre forniscono teorie e
procedure di dimostrazioni, tale che \ml{intLib}. Le librerie possono includere altre
librerie.

Nel sistema \HOL{}, le librerie sono tipicamente rappresentate da strutture
\ML{} nominate seguendo la convenzione che la libreria \emph{x}
si trover� nella struttura \ML{} \ml{xLib}. Caricare questa struttura
dovrebbe caricare tutte le sotto-componenti rilevanti della libreria e settare
qualunque parametro di sistema che sia appropriato per l'uso della libreria.

Quando il sistema \HOL{} � invocato nella sua configurazione normale, alcune
utili librerie sono caricate automaticamente. La libreria \HOL{}
pi� di base � \ml{boolLib}, che supporta le definizioni della logica
\HOL{}, che si trova nella teoria \theoryimp{bool}, e fornisce un'utile
suite di strumenti di definizione e ragionamento.

Un'altra libreria usata in modo pervasivo si trova nella struttura \ml{Parse}
(il lettore pu� vedere che non siamo strettamente fedeli alla nostra
convenzione circa le denominazioni delle librerie). La libreria parser fornisce supporto
per il parsing e il `pretty-printing' dei tipi, i termini, e
i teoremi \HOL{}.

La libreria \ml{boss} fornisce una collezione base di teorie
standard e di procedure di dimostrazione di alto livello, e serve come una piattaforma
standard su cui lavorare. Essa � pre-caricata e aperta quando il sistema
\HOL{} si avvia. Essa include \ml{boolLib} e
\ml{Parse}. Le teorie fornite includono \theoryimp{pair},
\theoryimp{sum}, \theoryimp{option}; le teorie aritmetiche
\theoryimp{num}, \theoryimp{prim\_rec}, \theoryimp{arithmetic},
e \theoryimp{numeral}; e \theoryimp{list}. Altre librerie
incluse in \ml{bossLib} sono \ml{goalstackLib}, che fornisce
un gestore di dimostrazione per dimostrazioni basate su tattiche; \ml{simpLib}, che fornisce
una variet� di semplificatori; \ml{numLib}, che fornisce una procedura
di decisione per l'aritmetica; \ml{Datatype}, che fornisce
supporto di alto-livello per definire tipi di dato algebrici; e \ml{tflLib},
che fornisce supporto per definire funzioni ricorsive.


\section{Parsing e Prettyprinting}
\label{sec:parsing-printing}

Ogni tipo e termine in \HOL{} � in definitiva costruito per applicazione dei
costruttori (astratti) primitivi per i tipi e i termini. Tuttavia, al
fine di ospitare un'ampia variet� di espressioni matematiche, \HOL{}
fornisce un'infrastruttura flessibile per il parsing e il prettyprinting dei tipi
e dei termini attraverso la struttura \ml{Parse}.

Il parser dei termini supporta l'inferenza di tipo, l'overloading, i binders, e
varie dichiarazioni di fixity (infisso, prefisso, suffisso, e
combinazioni). Ci sono anche dei flag per controllare il comportamento
del parser. Inoltre, la struttura del parser � esposta cos� che
possano essere costruiti rapidamente nuovi parser per supportare applicazioni utente.

Il parser � parametrizzato da grammatiche per tipi e termini. Il
comportamento del parser e del prettyprinter di conseguenza � di solito alterato
da manipolazioni di grammatica.
%
\index{parsing, della logica HOL@parsing, della logica \HOL{}!grammatiche per}
%
Queste possono essere di due generi: \emph{temporanee} o \emph{permanenti}.
I cambiamenti temporanei dovrebbero essere usati nelle implementazioni di librerie, o nei
file di script per quei cambiamenti che l'utente non vuole far
persistere nelle teorie discendenti da quella attuale. I cambiamenti permanenti
sono appropriati per l'uso in file di script, e saranno forzati in tutte
le teorie discendenti. Le funzioni che fanno cambiamenti temporanei sono denotate
da un prefisso \ml{temp\_} nei loro nomi.

\subsection{Parsing dei tipi}
\index{types, nella logica HOL@types, nella logica \HOL{}!parsing of|(}

Il linguaggio dei tipi � semplice. Una grammatica astratta per il
linguaggio � presentata nella Figura~\ref{fig:abstract-type-grammar}. La
grammatica attuale (con i valori concreti per i simboli infissi e gli operatori
di tipo) pu� essere ispezionata usando la funzione \ml{type\_grammar}.
\begin{figure}[tbhp]
\newcommand{\nt}[1]{\mathit{#1}}
\newcommand{\tok}[1]{\texttt{\bfseries #1}}
\renewcommand{\bar}{\;\;|\;\;}
\[
\begin{array}{lcl}
\tau &::=& \tau \odot \tau \bar \nt{vtype} \bar \nt{tyop} \bar
           \tok{(} \;\nt{tylist}\;\tok{)} \;\nt{tyop}\bar \tau \;\nt{tyop}
           \bar \tok{(}\;\tau\;\tok{)} \bar \tau\tok{[}\tau\tok{]}\\
\odot &::=& \tok{->} \bar \tok{\#} \bar \tok{+} \bar \cdots\\
\nt{vtype} &::=& \tok{'a} \bar \tok{'b} \bar \tok{'c} \bar \cdots\\
\nt{tylist} &::=& \tau \bar \tau \;\tok{,}\;\nt{tylist}\\
\nt{tyop} &::=& \tok{bool} \bar \tok{list} \bar \tok{num} \bar
           \tok{fun} \bar \cdots
\end{array}
\]
\caption{Una grammatica astratta per i tipi \HOL{} ($\tau$).  Gli infissi ($\odot$)
  legano sempre pi� debolmente dei gli operatori di tipo~($\nt{tyop}$) (e
  e dei sottoscritti di tipo~($\tau\tok{[}\tau\tok{]}$)), cos� che
  $\tau_1 \,\odot\, \tau_2 \,\nt{tyop}$ � sempre parsato come $\tau_1\, \odot\,
  (\tau_2 \,\nt{tyop})$.  Infissi differenti possono avere priorit�
	differenti, e gli infissi a diversi livelli di priorit� possono associare
	in modo differente (alla sinistra, alla destra, o non associare del tutto). Gli utenti possono
	estendere le categorie $\odot$ e $\nt{tyop}$ facendo nuove definizioni
	di tipo, e manipolando la grammatica direttamente.}
\label{fig:abstract-type-grammar}
\end{figure}

\paragraph{Infissi di tipo}
Gli infissi possono essere introdotti con la funzione \ml{add\_infix\_type}.
Questa imposta un mapping da un simbolo infisso (come \texttt{->}) al
nome di un operatore di tipo esistente (come \texttt{fun}). E' necessario
dare al simbolo binario un livello di precedenza e
un'associativit�. Si veda \REFERENCE{} per maggiori dettagli.

\paragraph{Abbreviazioni di tipo}
\index{abbreviazioni di tipo}

Gli utenti possono abbreviare pattern di tipo comuni con delle \emph{abbreviazioni}.
Questo � fatto con la funzione \ML{} \ml{type\_abbrev}:
\begin{hol}
\begin{verbatim}
   type_abbrev : string * hol_type -> unit
\end{verbatim}
\end{hol}
Un'abbreviazione � un nuovo operatore di tipo, di qualsiasi numero di argomenti,
che espande in un tipo esistente. Per esempio, si potrebbe sviluppare una
teoria leggera di numeri estesi con un infinito, in cui
tipo rappresentante fosse \holtxt{num option} (\holtxt{NONE}
rappresenterebbe il valore infinito). Si potrebbe impostare un'abbreviazione
\holtxt{infnum} che si espanderebbe a questo tipo sottostante.
Sono supportati anche i pattern polimorfici. Per esempio, come descritto
nella Sezione~\ref{sec:theory-of-sets}, l'abbreviazione \holtxt{set}, di
un unico argomento, � tale che \holtxt{:'a set} espande nel tipo
\holtxt{:'a -> bool}, per qualsiasi tipo \holtxt{:'a}.

Quando i tipi devono essere stampati, l'espansione delle abbreviazioni fatta dal parser � invertita.
Per maggiori informazioni, si veda la voce \ml{type\_abbrev} in \REFERENCE.
\index{tipi, nella logica HOL@tipi, nella logica \HOL{}!parsing dei|)}

\subsection{Parsing dei termini}

Il parser dei termini fornisce un'infrastruttura basata sulla grammatica per supportare
la sintassi concreta per le formalizzazioni. Di solito, la grammatica \HOL{} viene
estesa quando viene fatta una nuova definizione o una specifica di costante. (L'introduzione
di una nuova costante � discussa
nelle Sezioni~\ref{sec:constant-definitions} e \ref{conspec}.) Tuttavia,
qualsiasi identificatore pu� avere assegnato ad esso in ogni momento uno stato di parsing.
In quello che segue, esploriamo alcune delle capacit� del
parser dei termini \HOL{}.


\subsubsection{Architettura del parser}
\label{sec:parser:architecture}

Il parser trasforma le stringhe in termini. Fa questo nella seguente
serie di fasi, ciascuna delle quali � influenzata dalla grammatica fornita.
Di solito questa grammatica � la grammatica globale di default, ma gli utenti possono
organizzarsi per usare grammatiche differenti se lo desiderano. %
\index{parsing, della logica HOL@parsing, della logica \HOL{}!grammatiche per}
%
Correttamente, il parsing avviene dopo che il lexing ha diviso l'input in una
serie di token. Per maggiori informazioni sul lexing, si veda la Sezione~\ref{HOL-lex}.
\begin{description}
\item[Sintassi Concreta:] Caratteristiche come gli infissi, i binder e le forme
	mix-fix sono tradotte, creando una forma di ``sintassi
	astratta'' intermedia (tipo ML \ml{Absyn}). Le fixity possibili sono
	discusse nella Sezione~\ref{sec:parseprint:fixities} di sotto. Le forme
	di sintassi concreta sono aggiunte alla grammatica con funzioni come
	\ml{add\_rule} e \ml{set\_fixity} (per le quali, si veda \REFERENCE).
	L'azione di questa fase di parsing � incarnata nella funzione
	\ml{Absyn}.

 Il data type \ml{Absyn} � costruito usando i costruttori \ml{AQ}
	(un antiquote, si veda la Sezione~\ref{sec:quotation-antiquotation}); %
%
  \index{antiquotation, nei termini della logica HOL@antiquotation, nei termini della logica \HOL{}}%
%
  \ml{IDENT} (un identificatore); \ml{QIDENT} (un identificatore qualificato,
	dato come \holtxt{thy\$ident}); \ml{APP} (un'applicazione di una forma
	a un'altra); \ml{LA} (un'astrazione di una variabile su un corpo),
	e \ml{TYPED} (una forma accompagnata da un vincolo di tipo\footnote{I
		tipi nei vincoli \ml{Absyn} non sono tipi HOL completi, ma valori
		da un'altro tipo intermedio \ml{Pretype}.}, si veda
	la Sezione~\ref{sec:parseprint-type-constraints}). A questo stadio della
	traduzione, non viene fatta alcuna distinzione tra costanti e
	variabili: bench� le forme \ml{QIDENT} debbano essere delle costanti, gli utente sono
	anche in grado di riferirsi alle costanti dando loro dei nomi nudi.

  E' possibile che i nomi che occorrono nel valore \ml{Absyn} siano
	differenti da qualsiasi token che appariva nell'input
	originale. Per esempio, l'input
\begin{verbatim}
   ``if P then Q else R``
\end{verbatim} si trasformer� in
\begin{verbatim}
   APP (APP (APP (IDENT "COND", IDENT "P"), IDENT "Q"), IDENT "R")
\end{verbatim}
  (Questo � un output leggermente semplificato: i vari costruttori per
	\ml{Absyn}, incluso \ml{APP}, prendono anche parametri di localizzazione.)

  La grammatica standard include una regola che associa la speciale
	forma miz-fix per l'espressioni if-then-else con il sottostante
	``nome'' \holtxt{COND}. E' \holtxt{COND}. che sar� alla fine
	risolto come la costante \holtxt{bool\dol{}COND}.

  Se si usa la sintassi di ``quotation'' con un dollaro nudo,%
%
\index{ escape, nel parser della logica HOL@\ml{\$} (escape, nel parser della logica \HOL{})}%
\index{token!sopprimere il comportamento di parsing dei|(}
%
allora questa fase del parser non tratter� le stringhe come parte di una
forma speciale. Per esempio, \holtxt{\holquote{\dol{}if~P}} si trasforma
nella forma \ml{Absyn}
\begin{verbatim}
   APP(IDENT "if", IDENT "P")
\end{verbatim}
  \emph{non} in una forma che coinvolge \holtxt{COND}.

  Pi� tipicamente, spesso si scrive qualcosa come
	\holtxt{\holquote{\$+~x}}, che genera la sintassi astratta
\begin{verbatim}
   APP(IDENT "+", IDENT "x")
\end{verbatim}
  Senza il segno di dollaro, il parser della sintassi concreta si lamenterebbe
	del fatto che l'infisso pi� non ha un argomento
	sinistro. Quando il risultato di successo del parsing � passato alla
	fase successiva, il fatto che ci sia una costante chiamata \holtxt{+}
	dar� all'input il suo significato desiderato.

  Si pu� eseguire l'``escape'' dei simboli racchiudendoli in parentesi.
	Cos�, quello di sopra si potrebbe scrivere \holtxt{\holquote{(+)~x}} per avere
	lo stesso effetto.
\index{token!sopprimere il comportamento di parsing dei|)}

  L'utente pu� inserire a questo punto nel processo di parsing delle funzioni
	di trasformazione intermedia sviluppate per proprio conto . Questo � fatto
	con la funzione
\begin{verbatim}
   add_absyn_postprocessor
\end{verbatim}
  La funzione dell'utente sar� di tipo \ml{Absyn~->~Absyn} e potr�
	eseguire qualunque cambiamento sia appropriato. Come tutti gli altri aspetti del
	parsing, queste funzioni sono parte di una grammatica: se l'utente non vuole
	vedere usata una particolare funzione, pu� organizzarsi in modo che il parsing
	sia fatto rispetto a una grammatica differente.

\item[Risoluzione dei Nomi:] Le forme nude \ml{IDENT} nel valore \ml{Absyn}
	sono risolte come variabili libere, nomi legati o costanti.
	Questo processo risulta in un valore del tipo di dati \ml{Preterm}, che
	ha costruttori simili a quelli in \ml{Absyn} eccetto che con le forme
	per le costanti. %
	\index{parsing, della logica HOL@parsing, della logica \HOL{}!preterms}%
	Una stringa pu� essere convertita direttamente in un \ml{Preterm} per mezzo della
	funzione \ml{Preterm}.

  Un nome legato � il primo argomento a un costruttore \ml{LAM}, un
	identificatore che occorro sul lato sinistro di una freccia di
	una espressione case, o un identificatore che occorre all'interno di un pattern
	di comprensione d'insieme. Una costante � una stringa che � presente nel dominio
	dell'``overload map'' della grammatica. Le variabili libere sono tutti gli altri identificatori.
	Variabili libere dello stesso nome in un termine avranno lo stesso
	tipo. Gli identificatori sono testati per vedere se sono legati, e poi per
	vedere se sono costanti. Cos� � possibile scrivere
\begin{verbatim}
   \SUC. SUC + 3
\end{verbatim}
  ed avere la stringa \holtxt{SUC} trattata come un numero nel
	contesto dell'astrazione data, piuttosto che come la costante
	successore.

\index{parsing, della logica HOL@parsing, della logica \HOL{}!overloading}
\index{overloading|see{parsing, of \HOL{} logic, overloading}}
  L'``overloap map'' � una mappa da stringhe a liste di termini. I
	termini di solito sono solo costanti, ma possono essere termini arbitrari (dando
	origine a ``macro sintattiche'' o ``pattern'').\index{macro sintattiche}
	Questa infrastruttura � usata per permettere ad un nome come \holtxt{+} di mappare a
	costanti di addizione differenti in teorie come
	\theoryimp{arithmetic}, \theoryimp{integer}, e
  \theoryimp{words}. In questo modo i nomi ``reali'' delle costanti si possono
	slegare da ci� che l'utente digita. Nel caso dell'addizione, il
	pi� sui numeri naturali � di fatto chiamato \holtxt{+} (strettamente,
	\holtxt{arithmetic\$+}); ma sugli interi, �
	\holtxt{int\_add}, e su word � \holtxt{word\_add}. (Si noti
	che poich� ciascuna costante arriva da una teoria differente e cos� da un
	namespace differenti, esse potrebbero avere tutte il nome \holtxt{+}.)

  \index{inferenza di tipo!nel parser HOL@nel parser \HOL{}}
  Quando la risoluzione dei nomi determina che un identificatore dovrebbe essere trattato
	come una costante, esso � mappato a una forma pretermine che elenca tutte le
	possibilit� per quella stringa. Successivamente, poich� i termini nel
	range della mappa di overload tipicamente avranno tipi differenti,
	l'inferenza di tipo spesso eliminer� le possibilit� dalla lista. Se
	rimangono pi� possibilit� dopo che l'inferenza di tipo � stata
	eseguita, allora sar� stampato un warning, e sar� scelta una
	delle possibilit�. (Gli utenti possono controllare quali termini sono
	selezionati quando si presenta questa situazione.)

  Quando un termine nella mappa di overload � scelto come l'opzione migliore, �
	sostituito nel termini alla posizione appropriata. Se il termine � una
	lambda astrazione, allora sono fatte tante $\beta$-riduzioni quante
	possibili, usando qualsiasi argomento a cui il termine sia stato applicato. E'
	in questo modo che un pattern sintattico pu� elaborare gli argomenti. (Si veda
	anche la Sezione~\ref{sec:parser:syntactic-patterns} per maggiori informazioni sui
	pattern sintattici.)
\item[Inferenza di Tipo:] %
  \index{inferenza di tipo!nel parser HOL@nel parser \HOL{}}%
  Tutti i termini nella logica \HOL{} sono ben tipizzati. Il kernel rafforza
	questo attraverso le API per il tipo di dato \ml{term}. (In particolare,
	la funzione \ml{mk\_comb} %
	\index{mk_comb@\ml{mk\_comb}}%
	controlla che il tipo del primo argomento sia una funzione il cui
	dominio � uguale al tipo del secondo argomento.) Il lavoro del
	parser � di trasformare le stringhe fornite dall'utente in termini. Per convenienza,
	� vitale che l'utente non debba fornire tipi per tutti gli
	identificatori che digita. (Si veda la Sezione~\ref{sec:parser:type-inference}
	di sotto.)

  In presenza di identificatori sottoposti a overload, l'inferenza di tipo pu� non essere
	in grado di assegnare un tipo unico a tutte le costanti. Se esistono
	pi� possibilit�, ne sar� scelto uno quando il \ml{Preterm} � infine
	convertito in un termine genuino.
\item[Conversione a un Termine:]%
  Quando � stato completato il controllo di tipo di un \ml{Preterm}, la conversione finale da
	quel tipo al tipo del \ml{term} � per lo pi� semplice. L'utente
	pu� pure inserire ulteriore elaborazione a questo punto, cos� che una
	funzione fornita dall'utente modifichi il risultato prima che il parser
	lo restituisca.
\end{description}

\subsubsection{Caratteri unicode}
\label{sec:parser:unicode-characters}

\index{Unicode}\index{UTF-8}
\index{parsing, della logica HOL@parsing, della logica \HOL{}!Caratteri unicode|(}
E' possibile fare in modo che l'infrastruttura \HOL{} di parsing e stampa
utilizzi caratteri Unicode (scritti nella codifica UTF-8). Questo rende
possibile scrivere e leggere termini come
\begin{alltt}
   \(\forall\)x. P x \(\land\) Q x
\end{alltt}
piuttosto che
\begin{alltt}
   !x. P x /\bs{} Q x
\end{alltt}
Se lo desiderano, gli utenti possono semplicemente definire costanti che hanno caratteri
Unicode nei loro nomi, e lasciare le cose come stanno. Il problema con
questo approccio � che gli strumenti standard probabilmente creeranno file
di teoria che includono binding \ML{} (illegali) come \ml{val $\rightarrow$\_def =
  \dots}. Il risultato saranno file \ml{...Theory.sig} e
\ml{...Theory.sml} che falliranno di compilare, anche se la chiamata a
\ml{export\_theory} pu� avere successo. Questo problema pu� essere manovrato attraverso
l'uso di funzioni come \ml{set\_MLname}, probabilmente � una pratica
migliore usare solamente caratteri alfanumerici nei nomi delle costanti, e
usare poi funzioni come \ml{overload\_on} e \ml{add\_rule} per creare
la sintassi unicode per la costante sottostante.

Se gli utenti hanno a disposizione dei font con il repertorio di caratteri appropriati per
mostrare la loro sintassi, e confidano nel fatto che anche ognuno degli utenti delle loro
teorie li hanno, allora questo � perfettamente ragionevole. Tuttavia, se
gli utenti vogliono mantenere la retro-compatibilit� con la pura sintassi
ASCII, possono farlo definendo prima una sintassi ASCII pura. Dopo aver fatto
questo, possono creare una versione Unicode della sintassi con la
funzione \ml{Unicode.unicode\_version}. Quindi, quando la variabile
di traccia \ml{"Unicode"}  � $0$, la sintassi ASCII sar� usata per
il parsing e la stampa. Se la traccia � impostata a $1$, allora funzioner�
anche la sintassi unicode nel parse, e il pretty-priunter
la preferir� quando i termini sono stampati.

Per esempio, in \ml{boolScript.sml}, il carattere Unicode per l'and
logico (\texttt{$\land$}), � impostato come un'alternativa unicode per
\texttt{/\bs} con la chiamata
\begin{verbatim}
   val _ = unicode_version {u = UChar.conj, tmnm = "/\\"};
\end{verbatim}
(In questo contesto, � stata aperta la struttura \ml{Unicode},
dando accesso anche alla struttura \ml{UChar} che contiene i binding
per l'alfabeto Greco, e alcuni altri simboli matematici.)

L'argomento a \ml{unicode\_version} � un record con campi \ml{u}
e \ml{tmnm}. Entrambe sono stringhe. Il campo \ml{tmnm} pu� essere o
il nome di una costante, o un token che appare in una regola di sintassi concreta
(che possibilmente mappa a qualche altro nome). Se il \ml{tmnm} � solo il
nome di una costante, allora, con la variabili di traccia abilitata, la stringa
\ml{u} sar� sottoposta a overloading con lo stesso nome. Se il \ml{tmnm} � lo
stesso di un token di una regola di sintassi concreta, allora il comportamento � di
creare una nuova regola che mappa allo stesso nome, ma con la stringa \ml{u}
usata come il token.

\paragraph{Regole di lexing con caratteri Unicode}
%
\index{token!Caratteri unicode}
\index{identificatori, nella logica HOL@identificatori, nella logica \HOL{}!caratteri che non associano}
%
In parole povere, \HOL{} considera i caratteri divisi in tre
classi: alfanumerici, simboli che non associano e simboli. Questo
influenza il comportamento del lexer quando incontra delle stringhe di
caratteri. A meno che ci sia gi� nella grammatica uno specifico token ``misto'',
i token si dividono quando la classe di caratteri cambia. Cos�, nella
stringa
\begin{verbatim}
   ++a
\end{verbatim}
il lexer vedr� due token, \holtxt{++} e \holtxt{a}, perch�
\holtxt{+} � un simbolo e \holtxt{a} � un alfanumerico. La
classificazione dei caratteri Unicode extra � molto
semplicistica: tutt le lettere greche eccetto \holtxt{$\lambda$} sono alfanumerici;
il simbolo di negazione logica \holtxt{$\neg$} non associa; e
ogni altra cosa � simbolica. (L'eccezzione per \holtxt{$\lambda$} � per
permettere a stringhe come \holtxt{$\lambda$x.x} di essere divise dal lexer in \emph{quattro} token.)

\index{parsing, della logica HOL@parsing, della logica \HOL{}!Caratteri unicode|)}

\subsubsection{Pattern sintattici (``macro'')}
\label{sec:parser:syntactic-patterns}
\index{parsing, della logica HOL@parsing, della logica \HOL{}!overloading}
\index{parsing, della logica HOL@parsing, della logica \HOL{}!pattern sintattici|(}

La ``mappa di overload'' menzionata in precedenza � di fatto una combinazione di
mappe, una per il parsing, e una per la stampa. La mappa di parsing � da
nomi a liste di termini, e determina come i nomi che appaiono in un
\ml{Preterm} saranno tradotti in termini. In sostanza, i nomi legati sono tradotti
in variabili legate, nomi non legati che non sono nel dominio della mappa sono tradotti
in variabili libere, e nomi non legati nel dominio della mappa sono tradotti
in uno degli elementi dell'insieme associato con il nome dato.
Ogni termine nell'insieme delle possibilit� pu� avere un tipo differente, cos�
l'inferenza di tipo sceglier� da quelli che hanno tipi coerenti con
il resto del termine dato. Se la lista risultante contiene pi� di
un elemento, allora il termine che appare prima nella lista sar�
scelto.

Il caso d'uso pi� comune per la mappa di overload � di avere nomi mappati a
costanti. In questo modo, per esempio, le varie teorie aritmetiche possono
mappare la stringa \ml{"+"} alla nozione rilevante di addizione, ognuna delle
quali � una costante differente. Tuttavia, il sistema ha una flessibilit�
extra perch� i nomi possono mappare termini arbitrari. Per esempio, �
possibile mappare a istanze di costanti con specifici tipi. Cos�, la
stringa \ml{"<=>"} mappa l'eguaglianza, ma dove gli argomenti sono forzati
essere di tipo \holquote{:bool}.

Inoltre, se il termine mappato � una lambda-astrazione (cio�, della
forma $\lambda x.\;M$), allora il parser eseguir� tutte le $\beta$-riduzioni
possibili per quel termine e gli argomenti che lo accompagnano. per
esempio, in \theoryimp{boolTheory} e nei suoi discendenti, la stringa
\ml{"<>"} � sottoposta a overload rispetto al termine \holquote{\bs{}x~y.~\td{}(x~=~y)}.
Inoltre, \ml{"<>"} � impostato come un infisso al livello della sintassi
concreta. Quando l'utente inserisce \holquote{x~\lt\gt~y}, il valore
\ml{Absyn} risultante �
\begin{verbatim}
   APP(APP(IDENT "<>", IDENT "x"), IDENT "x")
\end{verbatim}
The \ml{"x"}  and \ml{"y"} identifiers will map to free variables, but
the \ml{"<>"} identifier maps to a list containing
\holquote{\bs{}x~y.~\td(x~=~y)}. This term has type
\begin{verbatim}
   :'a -> 'a -> bool
\end{verbatim}
e le variabili polimorfiche sono generalizzabili, permettendo all'inferenza
di tipo di dare i tipi (identici) appropriati a \ml{x} e \ml{y}.
Assumendo che questa opzione sia l'unico oveloading per il \ml{"<>"} lasciato
dopo l'inferenza di tipo, allora il termine risultante sar�
\holtxt{\td(x~=~y)}. Meglio, bench� questa sar� la struttura
sottostante del termine in memoria, esso sar� di fatto stampato come
\holquote{x~\lt\gt~y}.

Se il termine mappato nella mappa di overload contiene qualsiasi variabili libere,
queste variabili non saranno istanziate in alcun modo. In particolare,
se queste variabili hanno tipi polimorfici, allora le variabili di tipo in
questi tipi saranno costanti: non soggette a istanziazione dall'inferenza
di tipo.

\paragraph{Il pretty-printing e i pattern sintattici} La seconda parte della ``mappa di overload'' � una mappa da termini a stringhe, che specifica come
i termini dovrebbero essere trasformati indietro in identificatori. (Bench� di fatto
non costruisca un valore \ml{Absyn}, questo processo inverte la fase del parsing
di risoluzione dei nomi, producendo qualcosa che � poi stampato
secondo la parte di sintassi concreta della grammatica data.)

Poich� il parsing pu� mappare nomi singoli in complicate strutture di termine,
la stampa deve essere in grado di ricondurre una complicata struttura di termine a
un nome singolo. Esso fa questo eseguendo
un term matching.%
\index{matching!nel pretty-printing dei termini}%
%
\footnote{Il matching eseguito � del primo ordine; per contro il matching di ordine superiore
	� fatto nel semplificatore.}
Se pi� pattern matchano con lo stesso termine, allora il printer sceglie il
match pi� specifico (quello che richiede l'istanziazione minima delle
variabili del pattern.) Se questo risulta ancora in pi� possibilit�, ugualmente
specifiche, ha la precedenza il pattern aggiunto
pi� recentemente. (Gli utenti possono cos� manipolare le preferenze del printer
eseguendo delle chiamate altrimenti ridondanti alla funzione \ml{overload\_on}.)

Nell'esempio di sopra dell'operatore non-uguale-a, il pattern sar�
\holtxt{\~{}(?x = ?y)}, dove i punti interrogativi, indicano variabili del pattern
istanziabili. Se un pattern include delle variabili libere (si ricordi che
la \ml{x} e la \ml{y} in questo esempio erano legate da un'astrazione),
allora queste non saranno istanziabili.

Non c'� alcun'altra finezza nell'uso di questa infrastruttura: matching
``pi� grandi'', che coprono pi� di un termine hanno la precedenza. La difficolt�
che questo pu� causare � illustrata nel pattern \holtxt{IS\_PREFIX} dalla teoria
\theoryimp{rich\_listTheory}. Per questioni di retro-compatibilit�
questo identificatore mappa a
\begin{verbatim}
   \x y. isPREFIX y x
\end{verbatim}
dove \holtxt{isPREFIX} � una costante da \theoryimp{listTheory}.
(La questione � che \holtxt{IS\_PREFIX} si aspetta i suoi argomenti in
ordine inverso a quello che si aspetta \holtxt{isPREFIX}.) Ora, quando questa
macro � impostata la mappa di overload contiene gi� un mapping dalla
stringa \holtxt{"isPREFIX"} alla costante \holtxt{isPREFIX} (questo
accade con ogni definizione di costante). Ma dopo la chiamata
che stabilisce il nuovo pattern per \holtxt{IS\_PREFIX}, la
forma \holtxt{isPREFIX} non sar� pi� stampata. N� � sufficiente,
ripetere la chiamata
\begin{verbatim}
   overload_on("isPREFIX", ``isPREFIX``)
\end{verbatim}
Piuttosto (supponendo che \holtxt{isPREFIX} sia di fatto la forma
di stampa preferita), la chiamata deve essere
\begin{verbatim}
   overload_on("isPREFIX", ``\x y. isPREFIX x y``)
\end{verbatim}
cos� che il pattern di \ml{isPREFIX} � lungo quanto quella di \ml{IS\_PREFIX}.
\index{parsing, della logica HOL@parsing, della logica \HOL{}!pattern sintattici|)}


\subsubsection{Vincoli di tipo}
\label{sec:parseprint-type-constraints}

\index{vincoli di tipo!nel parser HOL@nel parser \HOL{}}
Un termine pu� essere vincolato ad essere un certo tipo. Per esempio,
\holtxt{X:bool} vincola la variabile \holtxt{X} ad avere il tipo
\holtxt{bool}. Un tentativo di vincolare un
termine in modo inappropriato sollever� un'eccezione: per esempio,
\begin{hol}
\begin{verbatim}
   if T then (X:ind) else (Y:bool)
\end{verbatim}
\end{hol}
fallir� perch� entrambi i rami di un condizionale dovono essere dello stesso
tipo. I vincoli di tipo possono essere visti come un suffisso che lega pi�
strettamente di qualunque altra cosa eccetto l'applicazione di funzione. Cos� $\term\
\ldots\ \term \ : \type$ � uguale a $(\term\ \ldots\ \term)\ :
\type$, ma $x < y:\holtxt{num}$ � un vincolo legittimo sulla sola
variabile $y$.


L'inclusione di \holtxt{:} negli identificatori simbolici significa che pu�
essere necessario separare qualche vincolo con degli spazi vuoti. Per esempio,
\begin{hol}
\begin{verbatim}
   $=:bool->bool->bool
\end{verbatim}
\end{hol}
sar� suddiviso dal lexer \HOL{} come
\begin{hol}
\begin{verbatim}
   $=: bool -> bool -> bool
\end{verbatim}
\end{hol}
ed elaborato dal parser come un'applicazione del'identificatore simbolico \holtxt{\$=:} alla
lista di termini argomento [\holtxt{bool}, \holtxt{->}, \holtxt{bool},
\holtxt{->}, \holtxt{bool}]. Uno spazio messo al punto giusto eviter� questo problema:
\begin{hol}
\begin{verbatim}
   $= :bool->bool->bool
\end{verbatim}
\end{hol}
� elaborato da parser come l'identificatore simbolico ``='' vincolato a un tipo.
Al posto del \holtxt{\$}, si possono anche usare le parentesi per rimuovere dai lessemi
un comportamento di parsing speciale:
\begin{hol}
\begin{verbatim}
   (=):bool->bool->bool
\end{verbatim}
\end{hol}

\subsubsection{Inferenza di tipo}
\label{sec:parser:type-inference}

\index{inferenza di tipo!nel parser HOL@nel parser \HOL{}|(}
Si consideri il termine \holtxt{x = T}: esso (e tutti i suoi sottotermini)
ha un tipo nella logica \HOL{}. Ora, \holtxt{T} ha il tipo \holtxt{bool}. Questo
significa che la costante \holtxt{=} ha tipo \holtxt{xty -> bool -> bool},
per qualche tipo \holtxt{xty}. Dal momento che lo schema di tipo per \holtxt{=} �
\holtxt{'a -> 'a -> bool}, sappiamo che \holtxt{xty} di fatto deve essere
\holtxt{bool} perch� l'istanza di tipo sia ben formata. Sapendo
questo, possiamo dedurre che il tipo di \holtxt{x} deve essere \holtxt{bool}.

Trascurando il gergo (``schema'' e ``istanza'') nel precedente
paragrafo, abbiamo condotto un assegnamento di tipo alla struttura di termine,
finendo con un termine ben formato. Sarebbe molto noioso per gli utenti
condurre un tale assegnamento a mano per ciascun termine immesso ad \HOL{}.
Cos�, \HOL{} usa un adattamento dell'algoritmo d'inferenza di tipo di Milner
per l'\ML{} quando si costruiscono dei termini attraverso il parsing. Alla fine dell'inferenza
di tipo, alle variabili di tipo non vincolate sono assegnati dei nomi da parte del sistema.
Di solito, questa assegnazione fa la cosa giusta. Tuttavia, a volte, il
tipo pi� generale non � ci� che si desidera e l'utente deve aggiungere dei vincoli
di tipo ai sotto termini rilevanti. Per situazioni complicate, si pu� assegnare la
variabili globale \ml{show\_types}. Quando � impostato questo flag,
i prettyprinter per i termini e i teoremi mostreranno come i tipi
sono stati assegnati ai sottotermini. Se non si vuole che il sistema assegni
delle variabili di tipo per proprio conto, si pu� impostare la variabile globale
\ml{guessing\_tyvars} a \ml{false}, nel qual caso
l'esistenza delle variabili di tipo non assegnate alla fine dell'inferenza di tipo
solleveranno un'eccezzione.
\index{type inference!nel parser HOL@nel parser \HOL{}|)}


\subsubsection{Overloading}
\label{sec:parsing:overloading}

\index{parsing, della logica HOL@parsing, della logica \HOL{}!overloading|(}
Una misura limitata di risoluzione di overloading � eseguita dal parser
del termine. Per esempio, il simbolo `tilde' (\holtxt{\~{}})
denota la negazione booleana nella teoria iniziale di \HOL, e denota anche
l'invero additivo nelle teorie \ml{integer} e
\ml{real}. Se carichiamo la teoria \ml{integer}
e immettiamo un termine ambiguo con \holtxt{\~{}}, il
sistema ci informer� che � stata eseguita la risoluzione dell'overloading.

\setcounter{sessioncount}{0}
\begin{session}
\begin{verbatim}
- load "integerTheory";
> val it = () : unit

- Term `~~x`;
<<HOL message: more than one resolution of overloading was possible.>>
> val it = `~~x` : term

- type_of it;
> val it = `:bool` : hol_type
\end{verbatim}
\end{session}

Per risolvere pi� scelte possibili � usato un meccanismo di priorit�.
Nell'esempio, \holtxt{\~{}} potrebbe essere coerentemente scelto avere il tipo
\holtxt{:bool -> bool} o \holtxt{:int -> int}, e il
meccanismo ha scelto il primo. Per un controllo pi� fine, si possono usare dei
vincoli di tipo espliciti. Nella seguente sessione, il
\holtxt{\~{}\~{}x} nella prima quotation ha il tipo \holtxt{:bool},
mentre nella seconda, un vincolo di tipo assicura che \holtxt{\~{}\~{}x} ha
il tipo \holtxt{:int}.

\begin{session}
\begin{verbatim}
- show_types := true;
> val it = () : unit

- Term `~(x = ~~x)`;
<<HOL message: more than one resolution of overloading was possible.>>
> val it = `~((x :bool) = ~~x)` : term

- Term `~(x:int = ~~x)`;
> val it = `~((x :int) = ~~x)` : term
\end{verbatim}
\end{session}

Si noti che il simbolo \holtxt{\~{}} sta per due costanti differenti nella
seconda quotation; la sua prima occorrenza � la negazione booleana, mentre
le altre due occorrenze sono l'operazione d'inverso additivo per
gli interi.
\index{parsing, della logica HOL@parsing, della logica \HOL{}!overloading|)}

\subsubsection{Le fixity}
\label{sec:parseprint:fixities}

Al fine di fornire qualche flessibilit� notazionale, le costanti sono disponibili in varie forme o {\it fixity}: oltre a essere costanti ordinarie (senza alcuna fixity), le costanti possono essere anche dei {\it binder}, {\it prefissi}, {\it suffissi}, {\it infissi}, o {\it closefix}.
Pi� in generale, i termini possono anche essere rappresentati usando specifiche {\it mixfix} ragionevolmente arbitrarie.
Il grado in cui i termini legano i loro argomenti associati � conosciuto come precedenza.
Pi� grande � questo numero, pi� stretto il binding.
Per esempio, nel momento in cui � introdotto, \verb-+- ha una precedenza di 500,  mentre il pi� stretto binder moltiplicazione (\verb+*+) ha una precedenza di 600.

\paragraph{I binder}

Un binder � un costrutto che lega una variabile; per esempio, il
quantificatore universale. In \HOL, questo � rappresentato usando un trucco che
risale ad Alnozo Church: un binder � una costante che prende una lambda
astrazione come suo argomento. Il binding lambda � usato per implementare
il binding del costrutto. Questa � una soluzione elegante ed uniforme.
Cos� la sintassi concreta \verb+!v. M+ � rappresentata
dall'applicazione della costante \verb+!+ all'astrazione \verb+(\v. M)+.

I binder pi� comuni sono \verb+!+, \verb+?+, \verb+?!+, e
\verb+@+. A volte si vogliono iterare applicazione dello stesso
binder, ad esempio,
\begin{alltt}
   !x. !y. ?p. ?q. ?r. \term.
\end{alltt}
Questo pu� invece essere reso come
\begin{alltt}
   !x y. ?p q r. \term.
\end{alltt}

\paragraph{Infissi}

Le costanti infisse possono associare in tre modi differenti: a destra,
a sinistra o non associare del tutto. (Se \holtxt{+} fosse non-associativo, allora
il parser non riuscirebbe ad elaborare \holtxt{3 + 4 + 5}; si dovrebbe scrivere
\holtxt{(3 + 4) + 5} o \holtxt{3 + (4 + 5)} a seconda del significato
desiderato.) L'ordine di precedenza per l'insieme iniziale di infissi �
\holtxt{/\bs}, \holtxt{\bs/}, \holtxt{==>}, \holtxt{=},
\begin{Large}\holtxt{,}\end{Large} (la virgola\footnote{Quando
	� stata caricata \theoryimp{pairTheory}.}). Inoltre, tutte queste
costanti sono associative a destra. Cos�
\begin{hol}
\begin{verbatim}
   X /\ Y ==> C \/ D, P = E, Q
\end{verbatim}
\end{hol}
%
� uguale a
%
\begin{hol}
\begin{verbatim}
   ((X /\ Y) ==> (C \/ D)), ((P = E), Q).
\end{verbatim}
\end{hol}
%
\noindent Un'espressione
\[
\term \; \holtxt{<infix>}\; \term
\]
� rappresentata internamente come
\[
((\holtxt{<infix>}\; \term)\; \term)
\]

\paragraph{Prefissi}

Mentre gli infissi appaiono tra i loro argomenti, i prefissi appaiono prima di essi.
Questo potrebbe inizialmente apparire la stessa cosa di quanto accade con la normale applicazione di funzione dove il simbolo alla sinistra semplicemente non ha alcuna fixity: $f$ in $f(x)$ non si comporta forse come un prefisso?
Di fatto tuttavia, in un termine come $f(x)$, dove $f$ e $x$ non hanno fixity, la sintassi � trattata come se ci fosse una invisibile applicazione di funzione infissa tra i due token: $f\cdot{}x$.
Questo operatore infisso lega pi� strettamente, cos� che quando si scrive $f\,x + y$, il risultato del parser � $(f\cdot{}x) + y$\footnote{Ci sono operatori infissi che legano pi� strettamente, il punto nella selezione di campo fa s� che $f\,x.fld$ sia elaborato dal parser come $f\cdot(x.fld)$.\index{tipi record!notazione di selezione del campo}}.
E' quindi utile permettere dei prefissi genuini cos� che gli operatori possano vivere a livelli di precedenza differenti rispetto all'applicazione di funzione.
Un esempio di questo � \verb+~+, la negazione logica.
Questo � un prefisso con una precedenza pi� bassa dell'applicazione di funzione.
Normalmente
\[
   f\;x\; y\qquad \mbox{� elaborata dal parser come}\qquad (f\; x)\; y
\] ma \[
  \holtxt{\~{}}\; x\; y\qquad\mbox{� elaborata dal parser come}\qquad
  \holtxt{\~{}}\; (x\; y)
\]
poich� la precedenza di \verb+~+ � pi� bassa di quella dell'applicazione di funzione.
Il simbolo unario di negazione sarebbe anche tipicamente definito come un prefisso, se non altro per permettere di scrivere \[
  {\it negop}\,{\it negop}\,3
\]
(qualunque cosa sia {\it negop}) senza bisogno di parentesi extra.

Dall'altro lato, la sintassi \holtxt{univ} per l'insieme universale (si veda la Sezione~\ref{sec:theory-of-sets}\index{universal set}) � un esempio di un operatore prefisso che lega pi� strettamente dell'applicazione.
Questo significa che \holtxt{f\,univ(:'a)} � elaborato dal parser come \holtxt{f(univ(:'a))}, non come \holtxt{(f univ)(:'a)} (su cui il parser fallirebbe il controllo di tipo).

\paragraph{Suffissi}

I suffissi appaiono dopo il loro argomenti. Non ci sono suffissi
introdotti nelle teorie standard disponibili in \HOL{}, ma gli utenti
sono sempre in grado di introdurli per loro conto se lo scelgono. I suffissi sono
associati con una precedenza esattamente come lo sono gli infissi e i prefissi.
Se \holtxt{p} � un prefisso, \holtxt{i} un infisso, e \holtxt{s} un
suffisso, allora ci sono sei ordinamenti possibili per i tre differenti
operatori, sulla base delle loro precedenze, dando cinque risultati per il parsing di
$\holtxt{p}\; t_1\; \holtxt{i}\; t_2\; \holtxt{s}$ a seconda delle
relative precedenze:
\[
\begin{array}{cl}
\mbox{\begin{tabular}{c}Precedenze\\(dalla pi� bassa alla pi� alta)\end{tabular}} &
\multicolumn{1}{c}{\mbox{Risultato del parsing}}\\
\hline
p,\;i,\;s & \holtxt{p}\;(t_1\;\holtxt{i}\;(t_2\;\holtxt{s}))\\
p,\;s,\;i & \holtxt{p}\;((t_1\;\holtxt{i}\;t_2)\;\holtxt{s})\\
i,\;p,\;s & (\holtxt{p}\;t_1)\;\holtxt{i}\;(t_2\;\holtxt{s})\\
i,\;s,\;p & (\holtxt{p}\;t_1)\;\holtxt{i}\;(t_2\;\holtxt{s})\\
s,\;p,\;i & (\holtxt{p}\;(t_1\;\holtxt{i}\;t_2))\;\holtxt{s}\\
s,\;i,\;p & ((\holtxt{p}\;t_1)\;\holtxt{i}\;t_2)\;\holtxt{s}\\
\end{array}
\]

\paragraph{I closefix}

I termini closefix sono operatori che racchiudono completamente gli argomenti.
Un esempio che si potrebbe usare nello sviluppo di una teoria della
semantica denotazionale sono le parentesi semantiche. Cos�, le infrastrutture di parsing
di \HOL{} possono essere configurate in modo da permettere di scrivere \holtxt{denotation x}
come \holtxt{[| x |]}. I closefix non sono associati con delle precedenze
perch� non possono competere per gli argomenti con altri operatori.


\subsubsection{Trucchi e magia del parser}

Qui descriviamo come ottenere alcuni effetti utili con il
parser in \HOL{}.

\begin{description}

\item[Aliasing] Se si vuole che una sintassi speciale sia un ``alias'' per una
	forma \HOL{} normale, questo � facile da ottenere; entrambi gli esempi fatti finora
	di fatto hanno fatto proprio questo. Tuttavia, se si vuole avere soltanto una
	normale sostituzione uno-a-uno di una stringa per un'altra, non si pu�
	usare la fase grammatica/sintassi per parsing per fare questo. Piuttosto, si
	pu� usare il meccanismo di overloading. Per esempio, sia
	\texttt{MEM} un alias per \texttt{IS\_EL}. Abbiamo bisogno della funzione
	\texttt{overload\_on} per impostare l'overload della constante originale sul nuovo
	nome:
\begin{verbatim}
   val _ = overload_on ("MEM", Term`IS_EL`);
\end{verbatim}

\item[Rendere l'addizione associativa a destra] Se si ha un numero di vecchi
	script che assumono che l'addizione sia associativa a destra perch� questo �
	come era una volta \HOL{}, potrebbe essere troppo penoso convertire tutto. Il trucco
	� di rimuovere tutte le regole al livello dato della grammatica, e
	rimetterle come infissi che associano sulla destra. il modo pi� semplice per riconoscere
	quali regole sono nella grammatica � per ispezione (usando
	\ml{term\_grammar()}). Con la sola \ml{arithmeticTheory}
	caricata, gli unici infissi al livello 500 sono \holtxt{+} and
  \holtxt{-}. Cos�, rimuoviamo le loro regole:
\begin{verbatim}
   val _ = app temp_remove_rules_for_term ["+", "-"];
\end{verbatim}
  \noindent E poi le rimettiamo con l'associativit�
	appropriata:
\begin{verbatim}
   val _ = app (fn s => temp_add_infix(s, 500, RIGHT)) ["+", "-"];
\end{verbatim}
\noindent Si noti che usiamo le versioni \ml{temp\_} di queste due
funzioni cos� che altre teorie che dipendono da questa non saranno
influenzate. Si noti inoltre che non possiamo avere due infissi allo stesso
livello di precedenza con differenti associativit�, cos� dobbiamo
rimuovere entrambi gli operatori, non solo l'addizione.

\item[Sintassi mix-fix per {\it if-then-else}:]
\index{condizionali, nella logica HOL@condizionali, nella logica \HOL{}!stampa dei}
%
Il primo passo per andare in questa direzione � di guardare all'aspetto generale
delle espressioni di questa forma. In questo caso, sar�:
%
\[
  \holtxt{if}\;\; \dots \;\;\holtxt{then}\;\;\dots\;\;
  \holtxt{else}\;\;\dots
  \]
%
 Dal momento che ci deve essere un termine ``a penzoloni'' sulla destra, la
	fixity appropriata � \ml{Prefix}. Sapendo che il termine costante
	sottostante � chiamato \holtxt{COND}, il modo pi� semplice per ottenere
	la sintassi desiderata �:
\begin{verbatim}
val _ = add_rule
   {term_name = "COND", fixity = Prefix 70,
    pp_elements = [TOK "if", BreakSpace(1,0), TM, BreakSpace(1,0),
                   TOK "then", BreakSpace(1,0), TM, BreakSpace(1,0),
                   TOK "else", BreakSpace(1,0)],
    paren_style = Always,
    block_style = (AroundEachPhrase, (PP.CONSISTENT, 0))};
\end{verbatim}
\noindent La regola effettiva � leggermente un p� pi� complicata, e
si pu� trovare nei sorgenti della teoria \theoryimp{bool}.

\item[Sintassi mix-fix sintassi per la sostituzione di termini:]

Qui ci� che si desidera � di essere in grado di scrivere qualcosa come:
\[
  \mbox{\texttt{[}}\,t_1\,\mbox{\texttt{/}}\,t_2\,\mbox{\texttt{]}}\,t_3
\]
denotando la sostituzione di $t_1$ per $t_2$ in $t_3$, magari
traducendolo in \holtxt{SUB $t_1$ $t_2$ $t_3$}. Questo sembra
come ci dovesse essere un altro \ml{Prefix}, ma la scelta delle
parentesi quadre (\holtxt{[} e \holtxt{]}) come delimitatori sarebbe
in conflitto con la sintassi concreta per i letterali lista se si facesse questo.
Dato che i letterali lista sono di fatto della classe
\ml{CloseFix}, la nuova sintassi deve essere della stessa classe. Questo � abbastanza semplice
da fare: impostiamo la sintassi
\[
\holtxt{[}\,t_1\,\holtxt{/}\,t_2\,\holtxt{]}
\]
in modo che mappi a \holtxt{SUB $t_1$ $t_2$}, un valore di un tipo
funzionale, che quando applicato a un terzo argomento apparir�
corretto\footnote{Si noti che facendo la stessa cosa per
	l'esempio \textit{if-then-else} di sopra sarebbe
	inappropriato, dal momento che permetterebbe di scrivere
\[ \holtxt{if}\;P\;\holtxt{then}\;Q\;\holtxt{else} \]
senza l'argomento finale}.
La regola per questo � cos�:
\begin{verbatim}
  val _ = add_rule
           {term_name = "SUB", fixity = Closefix,
            pp_elements = [TOK "[", TM, TOK "/", TM, TOK "]"],
            paren_style = OnlyIfNecessary,
            block_style = (AroundEachPhrase, (PP.INCONSISTENT, 2))};
\end{verbatim}

\end{description}

\subsubsection{Nascondere le costanti}
\label{hidden}

\index{parsing, della logica HOL@parsing, della logica \HOL{}!nascondere lo status di costante|(}
\index{sistema HOL@sistema \HOL{}!nascondere le costanti nel|(}
\index{costanti, nella logica HOL@costanti, nella logica \HOL{}!nascondere lo status delle}
\index{parsing, della logica HOL@parsing, della logica \HOL{}!overloading}
%
La seguente funzione pu� essere usata per nascondere lo status di costante di un
nome dal parser delle quotation.

\begin{holboxed}
\index{hide@\ml{hide}|pin}
\begin{verbatim}
  val hide   : string -> ({Name : string, Thy : string} list *
                          {Name : string, Thy : string} list)
\end{verbatim}
\end{holboxed}

\noindent La valutazione di \ml{hide "$x$"}
fa s� che il parser delle quotation tratti $x$ come una variabile (purch� le regole
lessicali lo permettano), anche se $x$ � il nome di una costante nella teoria attuale
(le costanti e le variabili possono avere lo stesso nome).
Questo � utile se si vogliono usare delle variabili
%
\index{variabili, nella logica HOL@variabili, nella logica \HOL{}!con nomi di costante}
%
con lo stesso nome di costanti dichiarate in precedenza (o incorporate)
(ad esempio \ml{o}, \ml{I}, \ml{S} \etc). Il nome $x$ � ancora una costante
per i costruttori, le teorie, ecc; \ml{hide} influisce solo sul parsing e
la stampa rimuovendo il nome dato dalla ``mappa di overload'' descritta
di sopra nella Sezione~\ref{sec:parser:architecture}. Si noti che l'effetto
di \ml{hide} � \emph{temporaneo}; i suoi effetti non persistono nelle
teorie discendenti da quella attuale. Si veda la voce \ml{hide} in
\REFERENCE{} per maggiori dettagli, inclusa una spiegazione del tipo
restituito.

La funzione

\begin{holboxed}
\index{reveal@\ml{reveal}|pin}
\begin{verbatim}
   reveal : string -> unit
\end{verbatim}
\end{holboxed}

\noindent annulla il nascondimento.

La funzione

\begin{holboxed}
\index{hidden@\ml{hidden}|pin}
\begin{verbatim}
   hidden : string -> bool
\end{verbatim}
\end{holboxed}

\noindent controlla se una stringa � il nome di una costante nascosta.
\index{sistema HOL@sistema \HOL{}!adattamento dell'interfaccia utente del}
\index{sistema HOL@sistema \HOL{}!nascondere le costanti nel|)}
\index{parsing, della logica HOL@parsing, della logica \HOL{}!nascondere lo status di costante nel|)}

\subsubsection{Adattare la profondit� del pretty-print}
\index{stampa, nella logica HOL@stampa, nella logica \HOL{}!adattamento della profondit� strutturale nella}

La seguente reference \ML{} pu� essere usata per impostare la profondit� massima
della stampa

\begin{holboxed}
\index{max_print_depth@\ml{max\_print\_depth}|pin}
\begin{verbatim}
   max_print_depth : int ref
\end{verbatim}
\end{holboxed}

\index{profondit� di stampa di default, per la logica HOL@profondit� di stampa di default, per la logica \HOL{}|(}

\noindent La profondit� di default della stampa � $-1$ che � intesa significare
nessun massimo. I sotto termini annidati pi� profondamente della profondit�
massima di stampa sono stampati come \holtxt{...}. Per esempio:

\setcounter{sessioncount}{0}
\begin{session}
\begin{verbatim}
- ADD_CLAUSES;
> val it =
    |- (0 + m = m) /\ (m + 0 = m) /\ (SUC m + n = SUC (m + n)) /\
       (m + SUC n = SUC (m + n)) : thm

- max_print_depth := 3;
> val it = () : unit
- ADD_CLAUSES;
> val it = |- (... + ... = m) /\ (... = ...) /\ ... /\ ... : thm
\end{verbatim}
\end{session}
\index{profondit� di stampa di default, per la logica HOL@profondit� di stampa di default, per la logica \HOL{}|)}

\subsection{Quotation e antiquotation}
\label{sec:quotation-antiquotation}

\index{quotation, nella logica HOL@quotation, nella logica \HOL{}!parser per}
\index{parsing, della logica HOL@parsing, della logica \HOL{}!della sintassi di quotation|(}
La sintassi correlata alla logica nel sistema HOL � tipicamente passato al
parser in forme speciali conosciute come \emph{quotation}. Una quotation di base
� delimitata da singoli accenti grave (cio�, \ml{`}, carattere ASCII~96). Quando
i valori quotation sono stampati dal loop interattivo ML, appaiono
piuttosto brutti a causa  dello speciale filtro che � fatto di questi
valori ancor prima che l'interprete li veda:
\setcounter{sessioncount}{0}
\begin{session}
\begin{verbatim}
- val q = `f x = 3`;
> val 'a q = [QUOTE " (*#loc 1 11*)f x = 3"] : 'a frag list
\end{verbatim}
\end{session}
Quotations (Moscow ML prints the type as \ml{'a frag list}) are the
raw input form expected by the various HOL parsers.  They are also
polymorphic (to be explained below).  Thus the function
\ml{Parse.Term} function takes a (term) quotation and returns a term,
and is thus of type \[ \ml{term quotation -> term}
\]

I parser dei termini e dei tipi possono essere chiamati anche implicitamente usando
i doppi accenti acuti come delimitatori. Per il parser dei tipi, il primo
carattere non spazio dopo il delimitatore principale deve essere un segno di deu punti.
Cos�:
\begin{session}
\begin{verbatim}
- val t = ``p /\ q``;
> val t = ``p /\ q`` : term

- val ty = ``:'a -> bool``;
> val ty = ``:'a -> bool`` : hol_type
\end{verbatim}
\end{session}

L'espressione legata alla variabile ML \ml{t} di sopra di fatto � espansa
a un'applicazione della funzione \ml{Parse.Term} alla quotation
argomento \ml{`p /\bs{} q`}. Analogamente, la seconda espressione si espande
in un'applicazione di \ml{Parse.Type} alla quotation \ml{`:'a -> bool`}.

Il vantaggio importante delle quotation rispetto a normali stringhe \ML{} �
che esse possono includere caratteri di nuova riga e backslash senza
richiedere caratteri speciali di escape. Le nuove righe occorrono ogni volta che i termini vanno oltre
una dimensione banale, mentre i backslash occorrono non solo nella
rappresentazione di $\lambda$, ma anche nella sintassi per la congiunzione e
la disgiunzione.

Se una quotation deve includere un carattere di accento grave, allora questo dovrebbe
essere fatto usando il carattere di escape proprio della sintassi della quotation, il
caret (\ml{\^}, carattere ASCII~94). Per avere un semplice caret, le cose diventano
leggermente pi� complicate. Se una sequenza di caret � seguita dallo
spazio vuoto (incluso un carattere di nuova riga), allora quella sequenza di caret �
passata al parser di HOL senza modifiche. Altrimenti, un singolo caret si pu�
ottenere scrivendone due in una riga. (L'ultima regola � analoga al
modo in cui in \ML{} la sintassi delle stringhe tratta il backslash.) Cos�:
\begin{session}
\begin{verbatim}
- ``f ^` x ``;
<<HOL message: inventing new type variable names: 'a, 'b, 'c>>
> val it = ``f ` x`` : term

- ``f ^ x``;
<<HOL message: inventing new type variable names: 'a, 'b, 'c>>
> val it = ``f ^ x`` : term
\end{verbatim}
\end{session}

La regola per i caret non seguiti da uno spazio vuoto � illustrata qui,
includendo un esempio che accade quando non si segue la regola
per il quoting:
\begin{session}
\begin{verbatim}
- ``f ^^+ x``;
<<HOL message: inventing new type variable names: 'a, 'b, 'c>>
> val it = ``f ^+ x`` : term

- ``f ^+ x``;
! Toplevel input:
! (Parse.Term [QUOTE " (*#loc 2 3*)f ", ANTIQUOTE (+),
!              QUOTE " (*#loc 2 7*) x"]);
!                                                  ^
! Ill-formed infix expression
\end{verbatim}
\end{session}

L'uso principale del caret � d'introdurre le \emph{quntiquotation} (come
suggerito nell'ultimo esempio di sopra). All'interno di una quotation, l'espressioni
della forma {\small\verb+^(+}$t${\small\verb+)+}
%
\index{ antiquotation, nella logica HOL@{\small\verb+^+} (antiquotation, nella logica \HOL{})}
%
(dove $t$ � un'espressione \ML\ di tipo
%
\index{controllo di tipo, nella logica HOL@controllo di tipo, nella logica \HOL{}!antiquotation nel}
%
\ml{term} o \ml{type}) sono chiamate antiquotation.
%
\index{termini, nella logica HOL@termini, nella logica \HOL{}!antiquotation}
\index{antiquotation, nei termini della logica HOL@antiquotation, nei termini della logica \HOL{}}
%
Una quotation \holtxt{\^{}($t$)} � valutata al
valore \ML{} di $t$. Per esempio, {\small\verb+``x \/ ^(mk_conj(``y:bool``, ``z:bool``))``+}
� valutata allo stesso termine di {\small\verb+``x \/ (y /\ z)``+}. L'uso
pi� comune dell'antiquotation � quando il termine $t$ � legato a una variabile
\ML\ $x$. In questo caso {\small\verb+^(+}$x${\small\verb+)+} pu� essere
abbreviato da {\small\verb+^+}$x$.

La seguente sessione illustra l'antiquotation.

\setcounter{sessioncount}{0}
\begin{session}
\begin{verbatim}
- val y = ``x+1``;
> val y = ``x + 1`` : term

val z = ``y = ^y``;
> val z = ``y = x + 1`` : term

- ``!x:num.?y:num.^z``;
> val it = ``!x. ?y. y = x + 1`` : term
\end{verbatim}
\end{session}

\noindent Anche i tipi possono essere sottoposti all'antiquotation:

\begin{session}
\begin{verbatim}
- val pred = ``:'a -> bool``;
> val pred = ``:'a -> bool`` : hol_type

- ``:^pred -> bool``;
> val it = ``:('a -> bool) -> bool`` : hol_type
\end{verbatim}
\end{session}

\noindent Le quotation sono polimorfiche, e la variabile di tipo di una
quotation corrisponde al tipo dell'entit� che pu� essere sottoposta all'antiquotation
in quella quotation. Dal moemnto che il parser dei termini si aspetta solo termini
sottoposti ad antiquotation, l'antiquotation di un tipo all'interno di una quotation di termine richiede l'uso di
\holtxt{ty\_antiq}. Per esempio,%
%
\index{ty_antiq@\ml{ty\_antiq}}

\begin{session}
\begin{verbatim}
- ``!P:^pred. P x ==> Q x``;

! Toplevel input:
! Term `!P:^pred. P x ==> Q x`;
!           ^^^^
! Type clash: expression of type
!   hol_type
! cannot have type
!   term

- ``!P:^(ty_antiq pred). P x ==> Q x``;
> val it = `!P. P x ==> Q x` : term
\end{verbatim}
\end{session}
%
\index{parsing, della logica HOL@parsing, della logica \HOL{}!della sintassi delle quotation|)}



\subsection{Retro-compatibilit� della sintassi}

Questa sezione del manuale documenta il cambiamento (esteso) fatto al
parsing di \HOL{} dei termini e dei tipi nella release Taupo (una delle
release HOL3) e al di l� del punto di vista di un utente che non
vuole sapere come usare le nuove strutture, ma vuole essere sicuro
che il proprio vecchio codice continui a funzionare in modo pulito.

I cambiamenti che possono far s� che i vecchi termini falliscano il parsing sono:
\begin{itemize}
\newcommand\condexp{\holtxt{$p$ => $q$ | $r$}}
\item La precedenza delle annotazioni di tipo � completamente cambiata. Ora
	� un suffisso molto stretto (bench� con una precedenza pi� debole di quella
	associata con l'applicazione di funzione), invece di uno debole.
	Questo significa che \mbox{\tt (x,y:bool \# bool)} ora dovrebbe essere scritto
	come \mbox{\tt (x,y):bool \# bool}. La forma precedente sar� ora
	parsata come un'annotazione di tipo che si applica solo a \verb+y+. Questo
	cambiamento porta la sintassi delle logica pi� vicina a quella dell'SML e
	dovrebbe rendere in genere pi� facile annotare le tuple, dal momento che ora
	si pu� scrivere \[ (x\,:\,\tau_1,\;y\,:\,\tau_2,\dots z\,:\,\tau_n)
  \] al posto di \[
  (x\,:\,\tau_1, \;(y\,:\,\tau_2, \dots (z\,:\,\tau_n)))
  \] dove le parentesi extra si sono dovute aggiungere solo per permettere di
	scrivere una forma di vincolo che occorre frequentemente.
\item La maggior parte degli operatori aritmetici ora sono associativi a sinistra piuttosto che
	a destra. In particolare, $+$, $-$, $*$ e {\tt DIV} sono
	associativi a sinistra. In modo simile, gli analoghi operatori nelle altre
	teoria aritmetiche come {\tt integer} e {\tt real} sono anche associativi
	a sinistra. Questo porta il parser di \HOL{} in linea con la pratica
	matematica standard.
\item Il binding dell'eguaglianza nell'espressioni {\tt let} � trattata esattamente
	nello stesso modo dell'eguaglianze negli altri contesti. Nelle versioni precedenti
	di \HOL, l'eguaglianze in questo contesto hanno una precedenza di bindig differente
	pi� debole.
\item La vecchia sintassi per l'espressioni condizionali � stata
	rimossa. Cos� la stringa \holquote{\condexp} ora deve essere
	scritta
	$\holquote{\texttt{if}\;p\;\texttt{then}\;q\;\texttt{else}\;r}$.
\item Alcune categorie lessicali sono sorvegliate pi� strettamente. I letterali
	stringa (le stringhe all'interno dei doppi apici) e quelli numerici non possono essere usati
	a meno che le teorie rilevanti non siano state caricate. Inoltre questi
	letterali non possono essere usati come variabili all'interno di scopi di binding.
\end{itemize}


\section{Un Semplice Gestore di Dimostrazione Interattivo}\label{sec:goalstack}

Il \emph{goal stack} fornisce una semplice interfaccia di dimostrazione interattiva
basata sulle tattiche. Quando si vogliono usare le tattiche per decomporre una dimostrazione, sorgono
molti stati intermedi; il goalstack si prende cura del necessario mantenimento
di queste informazioni. L'implementazione dei goalstack qui riportati � un
ridisegno della concezione originale di Larry Paulson.

La libreria goalstack � caricata automaticamente quando \HOL{} si avvia.

I tipi astratti \ml{goalstack} e \ml{proofs} sono il
punto focale delle operazioni di dimostrazione all'indietro. il tipo \ml{proofs} pu� essere
considerato come una lista di goalstack indipendenti. La maggior parte delle operazioni agiscono sulla
testa della lista dei goalstack; ci sono anche operazioni cos� che il
punto focale pu� essere cambiato.

\subsection{Avviare un goalstack di dimostrazione}

\begin{hol}
\begin{verbatim}
   g        : term quotation -> proofs
   set_goal : goal -> proofs
\end{verbatim}
\end{hol}

Si ricordi che il tipo \ml{goal} � un'abbreviazione per
\ml{term list * term}. Per partire su un nuovo goal, si da a
\ml{set\_goal} un goal. Questa crea un nuovo goalstack e lo rende il
punto focale di ulteriori operazioni.

Un'abbreviazione per \ml{set\_goal} � la funzione \ml{g}: essa
invoca il parser automaticamente, e non permette al goal di
avere alcuna assunzione.

La chiamata a \ml{set\_goal}, o \ml{g}, aggiunge un nuovo tentativo di dimostrazione a
quelli esistenti, \textit{cio�}, al posto di sovrascrivere il tentativo
di dimostrazione attuale, il nuovo tentativo � impilato in cima.

\subsection{Applicare una tattica a un goal}

\begin{hol}
\begin{verbatim}
   expandf : tactic -> goalstack
   expand  : tactic -> goalstack
   e       : tactic -> goalstack
\end{verbatim}
\end{hol}

Come si fa dunque di fatto a fare una dimostrazione goalstack? Nella maggior parte dei casi,
l'applicazione delle tattiche al goal attuale � fatto con la funzione
\verb+expand+. Nel raro caso in cui si voglia applicare una
tattica {\it invalida\/}, allora � usata \verb+expandf+. (Per una
spiegazione delle tattiche invalide, si veda il Capitolo 24 di \& Melham.)
Per espandere una tattica si pu� anche usare l'abbreviazione \verb+e+.


\subsection{Undo}

\begin{hol}
\begin{verbatim}
   b          : unit -> goalstack
   rd         : unit -> goalstack
   drop       : unit -> proofs
   dropn      : int  -> proofs
   backup     : unit -> goalstack
   redo       : unit -> goalstack
   restart    : unit -> goalstack
   set_backup : int  -> unit
\end{verbatim}
\end{hol}

Spesso (siamo tentati di dire {\it di solito}!) si prende una strada sbagliata
nel fare una dimostrazione, o si fa un errore nell'impostare un goal. Per annullare un passo
nel goalstack, sono usate la funzione \ml{backup} e la sua abbreviazione
\ml{b}. Questo ripristiner� il goalstack al suo stato precedente.
Per rifare un passaggio nel goalstack, si usa la funzione \ml{redo},
abbreviata come \ml{rd}.


Per eseguire il backup completo al goal originale, pu� essere usata
la funzione \ml{restart}. Ovviamente, � anche importante liberarsi
dei tentativi di dimostrazione che sono sbagliati; per questo c'� \ml{drop},
che si sbarazza del tentativo di dimostrazione corrente, e \ml{dropn}, che
elimina i primi $n$ tentativi di dimostrazione.


Ogni tentativo di dimostrazione ha la sua \emph{lista-di-annullamento} degli stati
precedenti. La lista di annullamento per ciascun tentativo � di dimensione fissata (inzialmente
12). Se si vuole impostare questo valore per il tentativo corrente di dimostrazione, si pu�
usare la funzione \ml{set\_backup}. Se la dimensione della lista di
backup � impostata essere pi� piccola di quanto sia attualmente, la lista di annullamente sar�
immediatamente troncata. Non si pu� annullare un'operazione ``proofs-level'', come
\ml{set\_goal} o \ml{drop}.

\subsection{Visualizzare lo stato del proof manager}

\begin{hol}
\begin{verbatim}
   p            : unit -> goalstack
   status       : unit -> proofs
   top_goal     : unit -> goal
   top_goals    : unit -> goal list
   initial_goal : unit -> goal
   top_thm      : unit -> thm
\end{verbatim}
\end{hol}

Per visualizzare lo stato del proof manager in qualsiasi momento, si possono
usare le funzioni \ml{p} e \ml{status}. La prima mostra solo
i subgoal in cima al goalstack corrente, mentre la seconda da una
sintesi di ogni tentativo di dimostrazione.

To get the top goal or goals of a proof attempt, use \ml{top\_goal}
and \ml{top\_goals}. To get the original goal of a proof attempt,
use \ml{initial\_goal}.

Per ottenere il o i top goal di un tentativo di dimostrazione, si usi \ml{top\_goal}
e \ml{top\_goals}. Per ottenere il goal originale di un tentativo di dimostrazione,
si usi \ml{initial\_goal}.

Una volta che un teorema � stato dimostrato il goalstack che � stato usato per derivarlo
continua ad esistere (e anche la sua lista-di-annullamento): il suo compito principale ora � quello di
mantenere il teorema. Questo teorema pu� essere estratto con
\ml{top\_thm}.

\subsection{Spostare il fuoco su un differente subgoal o tentativo di dimostrazione}

\begin{hol}
\begin{verbatim}
   r             : int -> goalstack
   R             : int -> proofs
   rotate        : int -> goalstack
   rotate_proofs : int -> proofs
\end{verbatim}
\end{hol}

Spesso vogliamo spostare la nostra attenzione a un differente goal nella dimostrazione
attuale, o a una dimostrazione differente. Le funzioni che fanno questo sono
\ml{rotate} e \ml{rotate\_proofs}, rispettivamente. Le abbreviazioni
\ml{r} e \ml{R} sono pi� semplici da digitare.

\section{Dimostrazione di Alto Livello---\texttt{bossLib}}
% would use \ml{boss} above but it puts LaTeX into fits
\label{sec:bossLib}
\newcommand\bossLib{\ml{bossLib}}

\index{bossLib@\ml{bossLib}}
La libreria \bossLib\ introduce alcuni degli strumenti di dimostrazione di teoremi
pi� ampiamente utilizzati in \HOL{} e li fornisce di un'interfaccia conveniente
per l'interazione. La libreria attualmente si concentra su tre cose:
definizione di datatype e funzioni; operazioni interattive di dimostrazione
di alto livello, e composizione di ragionatori automatici. Il caricamento di \bossLib\
impegna a lavorare in un contesto che fornisce gi� le teorie
dei booleani, le coppie, le somme, il tipo option, l'aritmetica, e le liste.


\subsection{Supporto per passi di dimostrazione di alto livello}
\label{sec:high-level-proof-steps}

Le seguenti funzioni usano informazione nel database per facilitare
l'applicazione delle funzionalit� di \HOL{} sottostanti.

\index{Induct_on (tattica ML d'induzione)@\ml{Induct\_on} (tattica \ML{} d'induzione)}
\index{Cases_on (tattica ML di case-split)@\ml{Cases\_on} (tattica \ML{} di case-split)}
\begin{verbatim}
   type_rws     : hol_type -> thm list
   Induct       : tactic
   Cases        : tactic
   Cases_on     : term quotation -> tactic
   Induct_on    : term quotation -> tactic
\end{verbatim}

\index{type_rws@\ml{type\_rws}}
\index{TypeBase@\ml{TypeBase}}
%
La funzione \ml{type\_rws} cercher� per il tipo dato nel
database sottostante \ml{TypeBase} e restituir� utili regole di riscrittura per
quel tipo. Le regole di riscrittura del datatype sono costruite a partire dai
teoremi di iniettivit� e distinzione, insieme con la definizione di costante
case. Le tattiche di semplificazione \ml{RW\_TAC}, \ml{SRW\_TAC},
e il \simpset{} \ml{(srw\_ss())} includono automaticamente questi
teoremi. Altre tattiche usate con altri \simpset{} avranno bisogno di questi
teoremi per essere aggiunte manualmente.

\index{teoremi d'induzione, nella logica HOL@teoremi d'induzione, nella logica \HOL{}!per tipi di dato algebrici}
%
La tattica \ml{Induct} rende conveniente invocare l'induzione. Quando
� applicata a un goal, � esaminato il quantificatore universale principale;
se il suo tipo � quello di un datatype conosciuto, � estratta e applicata
l'appropriata tattica d'induzione strutturale.

The \ml{Cases} tactic makes it convenient to invoke case
analysis. The leading universal quantifier in the goal is examined; if
its type is that of a known datatype, the appropriate structural
case analysis theorem is extracted and applied.

La tattica \ml{Cases\_on} prende una quotation, che �
parsata a un termine $M$, e poi in $M$ viene effettuata una ricerca per il goal. Se $M$
� una variabile, allora si cerca per una variabile con lo stesso nome. Una volta
che si conosce il termine su cui effettuare lo split, il suo tipo e i fatti associati sono
ottenuti dal database sottostante e usati per eseguire il case
split. Se alcune delle variabili libere di $M$ sono legate nel goal, � fatto un tentativo
per rimuovere i quantificatori (universali) cos� che il case split abbia
vigore. Infine, $M$ non ha bisogno di apparire nel goal, bench� dovrebbe almeno
contenere alcune delle variabili libere che compaiono gi� nel goal. Si noti
che la tattica \ml{Cases\_on} � pi� generale di \ml{Cases}, ma
richiede che gli sia dato un termine esplicito.

\index{Induct_on (tattica ML d'induzione)@\ml{Induct\_on} (tattica \ML{} d'induzione)}
La tattica \ml{Induct\_on} prende una quotation, che � parsata in un
termine $M$, e poi si cerca in $M$ il goal. Se $M$ � una
variabile, allora si cerca per una variabile con lo stesso nome. Una volta che il
termine su cui effettuare l'induzione � conosciuto, il suo tipo e i fatti associati sono
ottenuti dal database sottostante e usati per eseguire
l'induzione. Se $M$ non � una variabile, � creato una nuova variabile $v$
che non occorre gi� nel goal, ed � usata per costruire un termine $v = M$
a cui viene subordinato il goal prima che sia eseguita
l'induzione. Prima tuttavia, tutti i termini che contengono variabili libere da $M$
sono spostate dalle assunzioni alla conclusione del goal, e tutte
le variabili libere dei $M$ sono quantificate universalmente. \ml{Induct\_on} �
pi� generale di \ml{Induct}, ma richiede che le venga dato un termine
esplicito.

Sono stati forniti tre entry-point supplementari per induzioni pi�
esotiche:
\begin{description}
\item [\ml{completeInduct\_on}] esegue un'induzione completa sul
	termine denotato dalla quotazione data. L'induzione completa permette
	un'ipotesi d'induzione apparentemente\footnote{L'induzione completa e l'induzione
		matematica ordinaria sono entrambe derivabili l'una dall'altra.} pi� forte
	rispetto all'induzione matematica ordinaria: vale a dire, quando
	si esegue l'induzione su $n$, � permesso assumere che la propriet� valga per
	\emph{tutti} gli $m$ pi� piccoli di $n$. Formalmente: $\forall P.\ (\forall x.\
  (\forall y.\ y < x \supset P\, y) \supset P\,x) \supset \forall x.\
  P\,x$. Questo permette di usare l'ipotesi d'induzione pi� di
	una volta, e permette anche d'istanziare l'ipotesi d'induzione
	in modo diverso dal predecessore.

\item [\ml{measureInduct\_on}] prende una quotation, e la suddivide
	per trovare un termine e una funzione misura con cui indurre.
	Per esempio, se si volesse fare un'induzione sulla lunghezza di una lista
	\holtxt{L}, l'invocazione \ml{measureInduct\_on~`LENGTH L`}
	sarebbe appropriata.

\item [\ml{recInduct}] prende un teorema d'induzione generato da
	\ml{Define} o \ml{Hol\_defn} e lo applica al goal attuale.

\end{description}


\subsection{Ragionatori Automatici}
\label{sec:automated-reasoners}

\ml{bossLib} riunisce i pi� potenti ragionatori in \HOL{} e
prova a rendere facile comporli in un modo semplice. Prendiamo i nostri ragionatori
base da \ml{mesonLib}, \ml{simpLib}, e \ml{numLib},
ma il punto di \ml{bossLib} � di fornire un livello di astrazione cos�
che l'utente debba sapere solo pochi entry-point\footnote{Nella met� degli anni 1980
	Graham Birtwistle ha sostenuto un tale approccio, chiamandolo `HOL in Dieci
	Tattiche}. (Quest librerie sottostanti, e altre che forniscono strumenti analogamente
potenti sono descritte nel dettaglio nelle sezioni di sotto.)
\begin{hol}
\begin{verbatim}
   PROVE      : thm list -> term -> thm
   PROVE_TAC  : thm list -> tactic

   METIS_TAC  : thm list -> tactic
   METIS_PROVE: thm list -> term -> thm

   DECIDE     : term quotation -> thm
   DECIDE_TAC : tactic
\end{verbatim}
\end{hol}
La regola d'inferenza \texttt{PROVE} (e la tattica corrispondente
\texttt{PROVE\_TAC}) prende una lista di teoremi e un termine, e tenta
di dimostrare il termine usando un ragionatore al primo ordine. Le due funzioni
\ml{METIS} eseguono la stessa funzionalit� ma usano un metodo di dimostrazione
sottostante differente. Gli entry-point \texttt{PROVE} si riferiscono alla
libreria \texttt{meson}, che � ulteriormente descritta nella
Sezione~\ref{sec:mesonLib} di sotto. Il sistema \ml{METIS} � descritto
nella Sezione~\ref{sec:metisLib}. La regola d'inferenza \texttt{DECIDE}
(e la tattica corrispondente \texttt{DECIDE\_TAC}) applica una procedura
di decisione che (al meno) gestisce enunciati dell'aritmetica lineare.

\begin{hol}
\begin{verbatim}
   RW_TAC   : simpset -> thm list -> tactic
   SRW_TAC  : ssfrag list -> thm list -> tactic
   &&       : simpset * thm list -> simpset  (* infix *)
   std_ss   : simpset
   arith_ss : simpset
   list_ss  : simpset
   srw_ss   : unit -> simpset
\end{verbatim}
\end{hol}
%
\index{RW_TAC@\ml{RW\_TAC}} La tattica di riscrittura \ml{RW\_TAC} lavora
prima aggiungendo i teoremi dati nel \simpset dato; poi
semplifica il goal quanto pi� possibile; quindi esegue dei case split
sull'espressioni condizionali nel goal; poi ripetutamente (1)
elimina tutte le ipotesi della forma $v = M$ o $M = v$ dove $v$ �
una variabile che non occorre in $M$, (2) rompe qualsiasi equazione tra
termini costruttore ovunque nel goal. Infine,
\ml{RW\_TAC} solleva le espressioni-\holtxt{let} all'interno del goal cos� che
l'equazioni binding appaiono come
abbreviazioni\index{abbreviazioni!dimostrazione basata-su-tattche} nelle
assunzioni.

\index{SRW_TAC@\ml{SRW\_TAC}} La tattica \ml{SRW\_TAC} � analoga a
\ml{RW\_TAC}, ma lavora rispetto a un \simpset{} sottostante
(accessibile attraverso la funzione \ml{srw\_ss}) che viene aggiornato ogni volta che viene caricato
un nuovo contesto. Questo \simpset{} pu� essere aumentato attraverso
l'aggiornamento di ``frammenti \simpset{} '' (valori \ml{ssfrag}) e
teoremi. Nelle situazioni in cui ci sono grandi tipi archiviati nel
sistema, le performance di \ml{RW\_TAC} ne possono risentire perch�
aggiunge ripetutamente tutti i teoremi di riscrittura per i tipi conosciuti in un
\simpset{} prima di attaccare il goal. Dall'altro lato,
\ml{SRW\_TAC} carica le riscritture nel \simpset{} al di sotto di
\ml{srw\_ss()} una volta sola, rendendo l'operazione pi� veloce in questa
situazione.

\ml{bossLib} fornisce un numero d'insiemi di semplificazione. Il
simpset per la logica pura, le somme, le coppie e il tipo \ml{option} �
chiamato \ml{std\_ss}. Il simpset per l'aritmetica � chiamato
\ml{arith\_ss}, e il simpset per le liste � chiamato \ml{list\_ss}.
I simpset forniti da \bossLib{} aumentano strettamente di forza:
\ml{std\_ss} � contenuto in \ml{arith\_ss}, e \ml{arith\_ss} �
contenuto in \ml{list\_ss}. Il combinatore infisso \ml{\&\&} � usato
per costruire un nuovo \simpset{} da un \simpset{} e una lista di
teoremi dati. La tecnologia di semplificazione di \HOL{} � ulteriormente descritta nella
Sezione~\ref{sec:simpLib} di sotto e nelle \REFERENCE.

\begin{hol}
\begin{verbatim}
   by : term quotation * tactic -> tactic (* infix 8 *)
   SPOSE_NOT_THEN : (thm -> tactic) -> tactic
\end{verbatim}
\end{hol}
La funzione \ml{by} � un operatore infisso che prende una quotation
e una tattica $tac$. La quotation � parsata in un termine $M$. Quando
l'invocazione ``\ml{$M$ by $\mathit{tac}$}'' � applicata a un goal
$(A,g)$, � creato un nuovo subgoal $(A,M)$ e ad esso � applicata la tattica $tac$.
Se il goal � dimostrato, il teorema risultante � de-costruito e aggiunto
alle assunzioni del goal originale; cos� la dimostrazione procede con
il goal $((M::A), g)$. (Si noti tuttavia, che avverr� uno split dei casi
se la de-costruzione di $\ \vdash M$ espone delle disgiunzioni.) Cos�
\ml{by} permette di mischiare un utile stile di ragionamento `asserzionale' o `Mizar-like'
all'ordinaria dimostrazione basata sulle tattiche\footnote{Le dimostrazioni nel
	sistema Mizar sono documenti leggibili, diversamente dalla maggior parte
	delle dimostrazioni basate su tattiche.}

L'entry-point \ml{SPOSE\_NOT\_THEN} inizia una dimostrazione per
contraddizione assumendo la negazione del goal e spostando la
negazione all'interno dei quantificatori. Essa fornisce il teorema
risultante come un argomento della funzione fornite, che user� il
teorema per costruire e applicare una tattica.

\section{Dimostrazione al Primo Ordine---\texttt{mesonLib} e \texttt{metisLib}}
\label{sec:first-order-proof}
\index{procedure di decisione!logica del primo ordine}

La dimostrazione del primo ordine � una potente tecnica di dimostrazione di teoremi che pu�
sbrigare goal complicati. Diversamente da strumenti come il semplificatore, o
dimostra un goal completamente, o fallisce. Non pu� trasformare un goal
in una forma differente (e pi� utile).

\subsection{Model elimination---\texttt{mesonLib}}
\label{sec:mesonLib}

\index{meson (model elimination) procedura@\ml{meson} (model elimination) procedura}
\index{metodo model elimination per la logica del primo ordine}

La libreria \ml{meson} � un'implementazione del
metodo model-elimination per trovare dimostrazioni di goal nella logica
del primo ordine. Ci sono tre entry-point principali:
\begin{hol}
\begin{verbatim}
   MESON_TAC     : thm list -> tactic
   ASM_MESON_TAC : thm list -> tactic
   GEN_MESON_TAC : int -> int -> int -> thm list -> tactic
\end{verbatim}
\end{hol}

Ciascuna di quest tattiche tenta di dimostrare il goal. Esse o avranno
successo nel fare questo, o falliranno con un eccezione ``depth exceeded''. Se
il fattore di ramificazione nello spazio di ricerca � alto, le tattiche
\texttt{meson} possono richiedere anche molto tempo per raggiungere la profondit� massima.

Tutte le tattiche \texttt{meson} prendono una lista di teoremi. Questi
fatti extra sono usati dalla procedura di decisione per aiutare a dimostrare il goal.
\texttt{MESON\_TAC} ignora le assunzioni del goal; gli altri due
entry-point includono le assunzioni come pare del sequente da
dimostrare.

I parametri extra a \ml{GEN\_MESON\_TAC} forniscono un controllo extra del
comportamento dell'aumentare iterativo della profondit� che � al centro della
ricerca per una dimostrazione. In ogni iterazione data, l'algoritmo ricerca
per una dimostrazione di profondit� non pi� alta di un parametro $d$. Il
comportamento di default per \ml{MESON\_TAC} e \ml{ASM\_MESON\_TAC} � di iniziare $d$
a 0, incrementarlo di uno ogni volta che una ricerca fallisce, e di fallire se
$d$ eccede il valore archiviato nel valore della reference
\ml{mesonLib.max\_depth}. Per contro,
\ml{GEN\_MESON\_TAC~min~max~step} inizia $d$ a \ml{min}, lo incrementa
di \ml{step}, e rinuncia quando $d$ eccede \ml{max}.

La funzione \ml{PROVE\_TAC} da \ml{bossLib} esegue qualche
normalizzazione, prima di passare un goal e le sue assunzioni a
\ml{ASM\_MESON\_TAC}. A causa di questa normalizzazione, nella maggior parte
delle circostanze, si dovrebbe preferire \ml{PROVE\_TAC}
a \ml{ASM\_MESON\_TAC}.

\subsection{Risoluzione---\texttt{metisLib}}
\label{sec:metisLib}

\index{procedura metis (risoluzione)@procedura \ml{metis} (risoluzione)}
\index{metodo di risoluzione per la logica del primo ordine}

La libreria \ml{metis} � un'implementazione del metodo di risoluzione
per trovare dimostrazioni di goal nella logica del primo ordine. Ci sono due
entry-point principali:

\begin{hol}
\begin{verbatim}
   METIS_TAC   : thm list -> tactic
   METIS_PROVE : thm list -> term -> thm
\end{verbatim}
\end{hol}

Entrambe le funzioni prendono una lista di teoremi, e questi sono usati come lemmi
nella dimostrazione. \texttt{METIS\_TAC} � una tattica, e o avr� successo
nel dimostrare il goal, o se non ha successo o fallir� o continuer� a ciclare
all'infinito. \texttt{METIS\_PROVE} prende un termine $t$ e prova a dimostrare un
teorema con conclusione $t$: se ha successo, � restituito il teorema
$\vdash t$. Come per \texttt{METIS\_TAC}, potrebbe fallire o ciclare all'infinito se
la ricerca della dimostrazione non ha successo.

La famiglia \texttt{metisLib} di strumenti di dimostrazione implementano la risoluzione
ordinata e il calcolo di paramodulazione ordinata per la logica del primo ordine,
che di solito li rende pi� adatti a goal che richiedono ragionamenti di eguaglianza
non banali rispetto alle tattiche in \texttt{mesonLib}.


\section{Semplificazione---\texttt{simpLib}}
\label{sec:simpLib}
\index{semplificazione|(}

Il semplificatore � il motore di riscrittura pi� sofisticato in \HOL{}. E'
raccomandato come un cavallo di battaglia di scopo generale durante la dimostrazione di teoremi
interattiva. Come strumenti di riscrittura, il ruolo generale del semplificatore
� di applicare teoremi della forma generale
\[
\vdash l = r
\]
a termini, rimpiazzando le istanze di $l$ nel termine con $r$. Cos�, la
ruotine base di semplificazione � una \emph{conversione}, che prende un termine
$t$, e restituisce un teorema $\vdash t = t'$, o l'eccezione
\ml{UNCHANGED}.

La conversione di base �
\begin{hol}
\begin{verbatim}
   simpLib.SIMP_CONV : simpLib.simpset -> thm list -> term -> thm
\end{verbatim}
\end{hol}
Il primo argomento, un \simpset, � il modo standard di fornire una
collezione di regole di (e altri dati, che saranno spiegati di sotto) al
semplificatore. Ci sono dei \simpset{} che accompagnano le teorie principali
di \HOL{}. Per esempio, il \simpset{} \ml{bool\_ss}
in \ml{boolSimps} incorpora tutti i teoremi di riscrittura usuali desiderabili su formule
booleane:
\setcounter{sessioncount}{0}
\begin{session}
\begin{verbatim}
- SIMP_CONV bool_ss [] ``p /\ T \/ ~(q /\ r)``;
> val it = |- p /\ T \/ ~(q /\ r) = p \/ ~q \/ ~r : thm
\end{verbatim}
\end{session}
Oltre alla riscrittura con i teoremi ovvi, \ml{bool\_ss} �
anche capace di eseguire semplificazioni che non sono esprimibili come
teoremi semplici:
\begin{session}
\begin{verbatim}
- SIMP_CONV bool_ss [] ``?x. (\y. P (f y)) x /\ (x = z)``;
> val it = |- (?x. (\y. P (f y)) x /\ (x = z)) = P (f z) : thm
\end{verbatim}
\end{session}
In questo esempio, il semplificatore ha eseguito una $\beta$-riduzione nel
primo congiunto sotto il quantificatore esistenziale, e poi ha fatto una
riduzione ``unwinding'' o ``one-point'', riconoscendo che l'unico
valore possibile per la variabile quantificata \holtxt{x} era il valore
\holtxt{z}.

Il secondo argomento a \ml{SIMP\_CONV} � una lista di teoremi da
aggiungere al \simpset fornito, e da aggiungere come regole di riscrittura addizionali.
In questo modo, gli utenti possono aumentare temporaneamente i \simpset{} con
le loro proprie riscritture. Se un particolare insieme di teoremi � usato spesso come
un tale argomento, allora � possibile costruire un valore \simpset{} per
incorporare queste nuove riscritture.

Per esempio, la riscrittura \ml{arithmeticTheory.LEFT\_ADD\_DISTRIB}, che
afferma che $p(m + n) = pm + pn$, non fa parte di alcun \simpset{} standard
di \HOL{}. Questo perch� pu� causare un aumento poco attraente nella
dimensione del termine (ci sono due occorrenze di $p$ al lato destro
del teorema). Ci� nonostante, � chiaro che questo teorema pu� pu�
essere appropriato occasionalmente:
\begin{session}
\begin{verbatim}
- SIMP_CONV bossLib.arith_ss [LEFT_ADD_DISTRIB] ``p * (n + 1)``;
> val it = |- p * (n + 1) = p + n * p : thm
\end{verbatim}
\end{session}
Si noti come il \simpset{} \ml{arith\_ss} non ha solamente semplificato il
termine intermedio \ml{(p * 1)}, ma ha anche riordinato l'addizione per
mettere il termine pi� semplice sulla sinistra, e ordinato gli argomenti
della moltiplicazione.


\subsection{Tattiche di semplificazione}
\label{sec:simplification-tactics}
\index{semplificazione!tattiche}

Il semplificatore � implementato intorno alla conversione \ml{SIMP\_CONV},
che � una funzione per `convertire' i termini in teoremi. Per applicare
il semplificatore ai goal (alternativamente, per eseguire dimostrazioni basate su tattiche
con il semplificatore), \HOL{} fornisce cinque tattiche, ognuna delle quali �
disponibile in \ml{bossLib}.

\subsubsection{\ml{SIMP\_TAC : simpset -> thm list -> tactic}}
\index{SIMP_TAC@\ml{SIMP\_TAC}}

\ml{SIMP\_TAC} � la tattica di semplificazione pi� semplice: essa tenta di
semplificare il goal attuale (ignorando le assunzioni) usando il \simpset{}
dato e i teoremi aggiuntivi. Non � niente di pi� che il
sollevamento della sottostante conversione \ml{SIMP\_CONV} al livello
di tattica attraverso l'uso della funzione standard \ml{CONV\_TAC}.

\subsubsection{\ml{ASM\_SIMP\_TAC : simpset -> thm list -> tactic}}
\index{ASM_SIMP_TAC@\ml{ASM\_SIMP\_TAC}}

Come \ml{SIMP\_TAC}, \ml{ASM\_SIMP\_TAC} semplifica il goal attuale
(lasciando le assunzioni intatte), ma include le assunzioni
del goal come regole di riscrittura extra. Cos�:
\begin{session}
\begin{verbatim}
1 subgoal:
> val it =
    P x
    ------------------------------------
      x = 3
     : goalstack

- e (ASM_SIMP_TAC bool_ss []);
OK..
1 subgoal:
> val it =
    P 3
    ------------------------------------
      x = 3
     : goalstack
\end{verbatim}
\end{session}
\noindent
In questo esempio, \ml{ASM\_SIMP\_TAC} ha usato \holtxt{x = 3} come una
regola di riscrittura addizionale, e ha sostituito la \holtxt{x} di \holtxt{P x}
con \holtxt{3}. Quando un'assunzione � usata da \ml{ASM\_SIMP\_TAC} essa
� convertita in regole di riscrittura nello stesso modo dei teoremi passati nella
lista data come secondo argomento della tattica. Per esempio,
un'assunzione \holtxt{\~{}P} sar� trattata come la riscrittura \holtxt{|- P = F}.

\subsubsection{\ml{FULL\_SIMP\_TAC : simpset -> thm list -> tactic}}
\index{FULL_SIMP_TAC@\ml{FULL\_SIMP\_TAC}}

\noindent
La tattica \ml{FULL\_SIMP\_TAC} semplifica non solo la conclusione di
un goal ma anche le sue assunzioni. Essa procede semplificando
ciascuna assunzione una alla volta, usando inoltre le assunzioni precedenti nella
semplificazione delle assunzioni successive. Dopo essere stata semplificata, ciascuna
assunzione � ri-aggiunta alla lista di assunzioni del goal con la
tattica \ml{STRIP\_ASSUME\_TAC}. Questo significa che le assunzioni che
diventano congiunzioni avranno ciascuno dei congiunti assunti separatamente.
Le assunzioni che diventano disgiunzioni faranno s� che un nuovo sotto goal sia
creato per ciascun disgiunto. Se un'assunzione � semplificata a falso,
questo risolver� il goal.

\ml{FULL\_SIMP\_TAC} attacca le assunzioni nell'ordine in cui
appaiono nella lista dei termini che rappresentano le assunzioni
del goal. Tipicamente quindi, la prima assunzione da semplificare
sar� l'assunzione aggiunta pi� di recente. Vista alla luce della
stampa dei goal di \ml{goalstackLib}, \ml{FULL\_SIMP\_TAC} si fa
strada lungo l'elenco delle assunzioni, dal basso verso l'alto.

La seguente sessione dimostra un uso semplice di \ml{FULL\_SIMP\_TAC}:
\begin{session}
\begin{verbatim}
    x + y < z
    ------------------------------------
      0.  f x < 10
      1.  x = 4
     : goalstack

- e (FULL_SIMP_TAC bool_ss []);
OK..
1 subgoal:
> val it =
    4 + y < z
    ------------------------------------
      0.  f 4 < 10
      1.  x = 4
     : goalstack
\end{verbatim}
\end{session}
In questo esempio, l'assunzione \holtxt{x = 4} ha fatto s� che la \holtxt{x}
nell'assunzione \holtxt{f x < 10} sia stata rimpiazzata da \holtxt{4}. La
\holtxt{x} nel goal � stata sostituita in modo analogo. Se le assunzioni fossero
apparse nell'ordine opposto, solo la \holtxt{x} del goal sarebbe
cambiata.

La prossima sessione dimostra un comportamento ancora pi� interessante.
\begin{session}
\begin{verbatim}
> val it =
    f x + 1 < 10
    ------------------------------------
      x <= 4
     : goalstack

- e (FULL_SIMP_TAC bool_ss [arithmeticTheory.LESS_OR_EQ]);
OK..
2 subgoals:
> val it =
    f 4 + 1 < 10
    ------------------------------------
      x = 4

    f x + 1 < 10
    ------------------------------------
      x < 4
     : goalstack
\end{verbatim}
\end{session}
In questo esempio, il goal � stato riscritto con il teorema che afferma
\[
\vdash x \leq y \equiv x < y \lor x = y
\]
Sostituendo l'assunzione con una disgiunzione che risulta in due sotto goal.
Nel secondo di questi, l'assunzione \holtxt{x = 4} ha ulteriormente
semplificato il resto del goal.

\subsubsection{\ml{RW\_TAC : simpset -> thm list -> tactic}}
\index{RW_TAC@\ml{RW\_TAC}}

Nonostante il suo tipo sia lo stesso delle tattiche di semplificazioni gi�
descritte, \ml{RW\_TAC} � una tattica ``aumentata''. Essa � aumentata in
due modi:
\begin{itemize}
\item Quando si semplifica il goal, il \simpset{} fornito � aumentato
	non solo con i teoremi passati esplicitamente nel secondo argomento,
	ma anche con tutte le regole di riscrittura dalla \ml{TypeBase}, e
	anche con le assunzioni del goal.
%
  \index{TypeBase@\ml{TypeBase}}
\item \ml{RW\_TAC} also does more than just perform simplification.
  It also repeatedly ``strips'' the goal.  For example, it moves the
  antecedents of implications into the assumptions, splits
  conjunctions, and case-splits on conditional expressions.  This
  behaviour can rapidly remove a lot of syntactic complexity from
  goals, revealing the kernel of the problem.  On the other hand, this
  aggressive splitting can also result in a large number of
  sub-goals.  \ml{RW\_TAC}'s augmented behaviours are intertwined with
  phases of simplification in a way that is difficult to describe.
\end{itemize}

\subsubsection{\ml{SRW\_TAC : ssfrag list -> thm list -> tactic}}
\index{SRW_TAC@\ml{SRW\_TAC}}

La tattica \ml{SRW\_TAC} ha un tipo differente dalle altre
tattiche di semplificazione. Non prende un \simpset{} come un argomento.
Piuttosto la sua operazione si fonda sempre sul \simpset{} incorporato
\ml{srw\_ss()} (ulteriormente descritto nella Sezione~\ref{sec:srw_ss}). I
teoremi forniti come il secondo argomento di \ml{SRW\_TAC} sono trattati nello
stesso modo delle altre tattiche di semplificazione.  Infine, la
lista dei frammenti \simpset{} sono incorporati nel \simpset{}
sottostante, permettendo all'utente di fondere capacit� di semplificazione
aggiuntive se lo desidera.

Per esempio, per includere la procedura di decisione Presburger, si potrebbe
scrivere
\begin{hol}
\begin{verbatim}
   SRW_TAC [ARITH_ss][]
\end{verbatim}
\end{hol}
I frammenti \Simpset{} sono descritti di sotto nella
Sezione~\ref{sec:simpset-fragments}.

La tattica \ml{SRW\_TAC} esegue la stessa combinazione di semplificazione e
goal-splitting che fa \ml{RW\_TAC}. Le principali differenze tra le
due tattiche risiedono nel fatto che la seconda pu� essere inefficiente quando
si lavora con una grande \ml{TypeBase}, e nel fatto che lavorare con
\ml{SRW\_TAC} risparmia dal dover costruire esplicitamente
dei \simpset{} che includano tutte le riscritture ``appropriate'' del contesto
attuale. Il secondo ``vantaggio'' � basato sull'assunzione che
\ml{(srw\_ss())} non include mai riscritture inappropriate. La presenza
di riscritture non utilizzate non � mai un problema: la presenza di riscritture che
fanno la cosa sbagliata pu� essere causa di maggiore irritazione.

\subsection{I \simpset{} standard}
\label{sec:standard-simpsets}

\HOL{} � fornito con un numero di \simpset{} standard. Ognuno di questi �
accessibile dall'interno di \ml{bossLib}, bench� alcuni si originano in altre
strutture.

\subsubsection{\ml{pure\_ss} and \ml{bool\_ss}}
\label{sec:purebool-ss}
%
\index{pure_ss@\ml{pure\_ss}}
%
Il \simpset{} \ml{pure\_ss} (definito nella struttura \ml{pureSimps})
non contiene del tutto teoremi di riscrittura, e gioca il ruolo di una tabula
rasa all'interno dello spazio dei \simpset{} possibili. Quando si costruisce un
\simpset{} completamente nuovo, \ml{pure\_ss} � un punto di partenza possibile.
Il \simpset{} \ml{pure\_ss} ha solo due componenti: regole di congruenza
per specificare come traversare i termini. e una funzione che trasforma
i teoremi in regole di riscrittura. Le regole di congruenza sono ulteriormente descritte
nella Sezione~\ref{sec:advanced-simplifier}; la generazione di regole
di riscrittura da teoremi � descritta nella
Sezione~\ref{sec:simplifier-rewriting}.

\index{bool_ss (insieme di semplificazione)@\ml{bool\_ss} (insieme di semplificazione)}
%
Il \simpset{} \ml{bool\_ss} (definito nella struttura \ml{boolSimps}) �
spesso usato quando altri \simpset{} potrebbero essere troppo. Esso contiene
regole di riscritture per i connettivi booleani, e poco altro. Esso
contiene tutti i teoremi di de~Morgan per spostare le negazioni tra i
connettivi (congiunzione, disgiunzione, implicazione e espressioni
condizionali), incluse le regole dei quantificatori che hanno $\neg(\forall
x.\,P(x))$ e $\neg(\exists x.\,P (x))$ sui loro lati sinistri. Esso
contiene anche le regole che specificano il comportamento dei connettivi
quando le costanti \holtxt{T} e \holtxt{F} appaiono come loro
argomenti. (Una di queste regole � \holtxt{|- T /\bs{} p = p}.)

Come nell'esempio di sopra, \ml{bool\_ss} esegue anche
delle $\beta$-riduzioni e svolgimenti di un solo punto. Questi ultimi trasformano termini
della forma \[
\exists x.\;P(x)\land\dots (x = e) \dots\land Q(x)
\]
in
\[
P(e) \land \dots \land Q(e)
\]
Analogamente, lo svolgimento trasformer� $\forall x.\;(x = e)
\Rightarrow P(x)$ in $P(e)$.

Infine, \ml{bool\_ss} include anche regole di congruenza che permettono
al semplificatore di fare delle assunzioni aggiuntive quando sono semplificate
implicazioni ed espressioni condizionali. Questa caratteristica � spiegata
ulteriormente nella Sezione~\ref{sec:simplifier-rewriting} di sotto, ma pu� essere
illustrata da qualche esempio (il primo dimostra anche lo svolgimento
sotto un quantificatore universale):
\begin{session}
\begin{verbatim}
- SIMP_CONV bool_ss [] ``!x. (x = 3) /\ P x ==> Q x /\ P 3``;
> val it = |- (!x. (x = 3) /\ P x ==> Q x /\ P 3) = P 3 ==> Q 3 : thm

- SIMP_CONV bool_ss [] ``if ~(x = 3) then P x else Q x``;
> val it = |- (if ~(x = 3) then P x else Q x) =
              (if ~(x = 3) then P x else Q 3) : thm
\end{verbatim}
\end{session}

\subsubsection{\ml{std\_ss}}
%
\index{std_ss (insieme di semplificazione)@\ml{std\_ss} (insieme di semplificazione)}
%
Il \simpset{} \ml{std\_ss} � definito in \ml{bossLib}, e aggiunge
regole di riscrittura pertinenti ai tipi di somme, coppie, option e
numeri naturali a \ml{bool\_ss}.
\begin{session}
\begin{verbatim}
- SIMP_CONV std_ss [] ``FST (x,y) + OUTR (INR z)``;
<<HOL message: inventing new type variable names: 'a, 'b>>
> val it = |- FST (x,y) + OUTR (INR z) = x + z : thm

- SIMP_CONV std_ss [] ``case SOME x of NONE => P | SOME y => f y``;
> val it = |- (case SOME x of NONE => P | SOME v => f v) = f x : thm
\end{verbatim}
\end{session}

Con i numeri naturali, il \simpset{} \ml{std\_ss} pu� calcolare
con valori ground, e anche includere una suite di ``riscritture ovvie''
per formule che includono variabili.
\begin{session}
\begin{verbatim}
- SIMP_CONV std_ss [] ``P (0 <= x) /\ Q (y + x - y)``;
> val it = |- P (0 <= x) /\ Q (y + x - y) = P T /\ Q x : thm

- SIMP_CONV std_ss [] ``23 * 6 + 7 ** 2 - 31 DIV 3``;
> val it = |- 23 * 6 + 7 ** 2 - 31 DIV 3 = 177 : thm
\end{verbatim}
\end{session}

\subsubsection{\ml{arith\_ss}}
%
\index{arith_ss (insieme di semplificazione)@\ml{arith\_ss} (insieme di semplificazione)}
%
Il \simpset{} \ml{arith\_ss} (definito in \ml{bossLib}) estende
\ml{std\_ss} aggiungendo la capacit� di decidere formule dell'aritmetica
Presburger, e per normalizzare espressioni aritmetiche (raccogliendo
coefficienti , e ri-ordinazione di addendi). La sottostante procedura di decisione
per i numeri naturali � quella descritta nella
Sezione~\ref{sec:numLib} di sotto.

Questi due aspetti del \simpset{} \ml{arith\_ss} sono dimostrati
qui:
\begin{session}
\begin{verbatim}
- SIMP_CONV arith_ss [] ``x < 3 /\ P x ==> x < 20 DIV 2``;
> val it = |- x < 3 /\ P x ==> x < 20 DIV 2 = T : thm

- SIMP_CONV arith_ss [] ``2 * x + y - x + y``;
> val it = |- 2 * x + y - x + y = x + 2 * y : thm
\end{verbatim}
\end{session}
Si noti che la sottrazione su numeri naturali funziona in modi che possono
sembrare non intuitivi. In particolare, la normalizzazione del coefficiente non pu�
occorrere quando atteso prima:
\begin{session}
\begin{verbatim}
- SIMP_CONV arith_ss [] ``2 * x + y - z + y``;
! Uncaught exception:
! UNCHANGED
\end{verbatim}
\end{session}
Sui numeri naturali, l'espressione $2 x + y - z + y$  non �
uguale a $2 x + 2 y - z$. In particolare, queste espressioni non sono
uguali quando  $2x + y < z$.

\subsubsection{\ml{list\_ss}}
%
\index{list_ss (insieme di semplificazione)@\ml{list\_ss} (insieme di semplificazione)}
%
L'ultimo valore \simpset{} puro in \ml{bossLib}, \ml{list\_ss} aggiunge
teoremi di riscrittura circa il tipo delle liste a \ml{arith\_ss}. Queste
riscritture includono i fatti ovvi circa i costruttori del tipo lista
\holtxt{NIL} e \holtxt{CONS}, come il fatto che \holtxt{CONS} �
iniettivo:
\begin{hol}
\begin{verbatim}
   (h1 :: t1 = h2 :: t2) = (h1 = h2) /\ (t1 = t2)
\end{verbatim}
\end{hol}
Opportunamente, \ml{list\_ss} include anche delle riscritture per le funzioni
definite per ricorsione primitiva sulle liste. Esempi includono
\holtxt{MAP}, \holtxt{FILTER} e \holtxt{LENGTH}. Cos�:
\begin{session}
\begin{verbatim}
- SIMP_CONV list_ss [] ``MAP (\x. x + 1) [1;2;3;4]``;
> val it = |- MAP (\x. x + 1) [1; 2; 3; 4] = [2; 3; 4; 5] : thm

- SIMP_CONV list_ss [] ``FILTER (\x. x < 4) [1;2;y + 4]``;
> val it = |- FILTER (\x. x < 4) [1; 2; y + 4] = [1; 2] : thm

- SIMP_CONV list_ss [] ``LENGTH (FILTER ODD [1;2;3;4;5])``;
> val it = |- LENGTH (FILTER ODD [1; 2; 3; 4; 5]) = 3 : thm
\end{verbatim}
\end{session}
Questi esempi dimostrano come il semplificatore pu� essere usato come un valutatore
simbolico di scopo generale per termini che assomigliano in grande misura a quelli
che appaiono in un linguaggio di programmazione funzionale. Si noti che
questa funzionalit� � fornita anche da \ml{computeLib} (si veda
la Sezione~\ref{sec:computeLib} di sotto); \ml{computeLib} � pi�
efficiente, ma meno generale del semplificatore. Per esempio:
\begin{session}
\begin{verbatim}
- EVAL ``FILTER (\x. x < 4) [1;2;y + 4]``;
> val it =
    |- FILTER (\x. x < 4) [1; 2; y + 4] =
       1::2::(if y + 4 < 4 then [y + 4] else []) : thm
\end{verbatim}
\end{session}

\subsubsection{Il \simpset{} ``stateful''---\ml{srw\_ss()}}
\label{sec:srw_ss}
\index{srw_ss (insieme di semplificazione)@\ml{srw\_ss} (insieme di semplificazione)}

L'ultimo \simpset{} esportato da \ml{bossLib} � nascosto dietro una
funzione. Il valore \ml{srw\_ss} ha il tipo \ml{unit -> simpset}, cos�
che si deve digitare \ml{srw\_ss()}  per ottenere un valore \simpset{}.
Questo uso di un tipo funzione permette al \simpset{} sottostante di essere
archiviato in una reference \ML{}, e permette ad esso di essere aggiornato
dinamicamente. In questo modo, la trasparenza referenziale � deliberatamente
spezzata. Tutti gli altri \simpset{} si comporteranno sempre in modo identico:
\ml{SIMP\_CONV~bool\_ss} � la stessa routine di semplificazione ovunque
e ogni volta che � chiamata.

Per contro, \ml{srw\_ss} � progettata per essere aggiornata. Quando una teoria �
caricata, quando un nuovo tipo � definito. il valore dietro \ml{srw\_ss()}
cambia, e il comportamento di \ml{SIMP\_CONV} applicato a
\ml{(srw\_ss())} cambia con esso. La filosofia di sviluppo dietro
\ml{srw\_ss} � che essa dovrebbe essere sempre una ragionevole prima scelta in
tutte le situazioni dove il semplificatore � utilizzato.

Questa versatilit� � illustrata nel seguente esempio:
\begin{session}
\begin{verbatim}
- Hol_datatype `tree = Leaf | Node of num => tree => tree`;
<<HOL message: Defined type: "tree">>
> val it = () : unit

- SIMP_CONV (srw_ss()) [] ``Node x Leaf Leaf = Node 3 t1 t2``;
<<HOL message: Initialising SRW simpset ... done>>
> val it =
    |- (Node x Leaf Leaf = Node 3 t1 t2) =
       (x = 3) /\ (Leaf = t1) /\ (Leaf = t2) : thm

- load "pred_setTheory";
> val it = () : unit

- SIMP_CONV (srw_ss()) [] ``x IN { y | y < 6}``;
> val it = |- x IN {y | y < 6} = x < 6 : thm
\end{verbatim}
\end{session}
%
Gli utenti possono aumentare il \simpset{} stateful da soli con la funzione
%
\begin{holboxed}
\index{export_rewrites@\ml{export\_rewrites}}
\begin{verbatim}
   BasicProvers.export_rewrites : string list -> unit
\end{verbatim}
\end{holboxed}
Le stringhe passate a \ml{export\_rewrites} sono i nomi di teoremi
nell'attuale segmento di teoria (quelli che saranno esportati quando
\ml{export\_theory} � chiamata). Non solo questi teoremi sono aggiunti
al \simpset{} sottostante nella sessione attuale, ma essi saranno
aggiunti nelle sessioni future quando la teoria � ricaricata.
\begin{session}
\begin{verbatim}
- val tsize_def = Define`
  (tsize Leaf = 0) /\
  (tsize (Node n t1 t2) = n + tsize t1 + tsize t2)
`;
Definition has been stored under "tsize_def".
> val tsize_def =
    |- (tsize Leaf = 0) /\
       !n t1 t2. tsize (Node n t1 t2) = n + tsize t1 + tsize t2 : thm

- val _ = BasicProvers.export_rewrites ["tsize_def"];

- SIMP_CONV (srw_ss()) [] ``tsize (Node 4 (Node 6 Leaf Leaf) Leaf)``;
> val it = |- tsize (Node 4 (Node 6 Leaf Leaf) Leaf) = 10 : thm
\end{verbatim}
\end{session}

Come regola generale, \ml{(srw\_ss())} include tutte le ``riscritture ovvie''
del suo contesto, cos� come codice per fare calcoli standard
(come l'aritmetica esegita nell'esempio di sopra). Non
include procedure di decisione che possono esibire performance occasionalmente
povere, cos� i frammenti \simpset{} che contengono queste procedure
dovrebbero essere aggiunte manualmente a quelle invocazioni di semplificazione che ne
hanno bisogno.

\subsection{Frammenti \simpset{}}
\label{sec:simpset-fragments}
\index{semplificazione!frammenti simpset}

Il frammento \simpset{} � il blocco base di costruzione usato per
costruire valori \simpset{}. C'� una funzione base che
esegue questa costruzione:
\begin{hol}
\begin{verbatim}
   op ++  : simpset * ssfrag -> simpset
\end{verbatim}
\end{hol}
dove \ml{++} � un infisso. In generale, � meglio costruire sopra
il \simpset{} \ml{pure\_ss} o uno dei sui discendenti al fine di
selezionare la funzione ``filtro'' di default per convertire teoremi in
regole di riscrittura. (Questo processo di filtro � descritto di sotto nella
Sezione~\ref{sec:generating-rewrite-rules}.)

Per le teorie principali (o gruppi di esse), una collezione di
frammenti \simpset{} rilevanti si trova di solito nel modulo \ml{<thy>Simps},
dove \ml{<thy>} � il nome della teoria. Per esempio, i frammenti
\simpset{}  per la teoria dei numeri naturali si trovano in
\ml{numSimps}, e i frammenti per le liste si trovano in \ml{listSimps}.

Alcuni frammenti \simpset{} standard della distribuzione sono descritti
nella Tabella~\ref{table:ssfrags}. Questi ed altri frammenti \simpset{}
sono descritti in maggior dettaglio nelle \REFERENCE.

\begin{table}[htbp]
\begin{center}
\renewcommand{\arraystretch}{1.2}
\begin{tabular}{lp{0.65\textwidth}}
\ml{BOOL\_ss} &
Riscritture standard per gli operatori booleani
(congiunzione, negazione \&c), cos� come una conversione per eseguire
$\beta$-riduzioni.  (In \ml{boolSimps}.)
\\
\ml{CONG\_ss} & Regole di congruenza per l'implicazione e le espressioni
condizionali. (In \ml{boolSimps}.)
\\
\ml{ARITH\_ss} &
La procedura di decisione sui numeri naturali
per l'aritmetica universale Presburger. (In \ml{numSimps}.)
\\
\ml{PRED\_SET\_AC\_ss} & Normalizzazione-AC per unioni e intersezioni
su insiemi. (In \ml{pred\_setSimps}.)
\end{tabular}
\end{center}
\caption{Alcuni dei frammenti \simpset{} standard di \HOL{}}
\label{table:ssfrags}
\end{table}

I frammenti \simpset{} in definitiva sono costruiti con il costruttore
\ml{SSFRAG}:
\begin{hol}
\begin{verbatim}
   SSFRAG : {
     convs  : convdata list,
     rewrs  : thm list,
     ac     : (thm * thm) list,
     filter : (controlled_thm -> controlled_thm list) option,
     dprocs : Traverse.reducer list,
     congs  : thm list,
     name   : string option
   } -> ssfrag
\end{verbatim}
\end{hol}
Una descrizione completa per i vari campi del record passato a
\ml{SSFRAG}, e il loro significato � dato in \REFERENCE. La
funzione \ml{rewrites} fornisce una strada semplice per costruire un
frammento che include solo una lista di riscritture:
\begin{hol}
\begin{verbatim}
   rewrites : thm list -> ssfrag
\end{verbatim}
\end{hol}

\subsection{Riscrittura con il semplificatore}
\label{sec:simplifier-rewriting}

La riscrittura � l'``operazione core'' del semplificatore. Questa sezione
descrive l'azione di riscrittura in maggior dettaglio.


\subsubsection{Riscrittura di base}
\label{sec:basic-rewriting}

Data una regola di riscrittura della forma \[
\vdash \ell = r
\]
il semplificatore eseguir� una scansione dall'alto verso il basso del termine di input $t$,
cercando per dei \emph{match}~(si veda la Sezione~\ref{sec:simp-homatch} di sotto)
di $\ell$ all'interno di $t$. Questo match occorrer� in un sotto termine di $t$
(chiamiamolo $t_0$) e restituir� un'istanziazione. Quando questa
istanziazione � applicata alla regola di riscrittura, il risultato sar� una nuova
equazione della forma \[
\vdash t_0 = r'
\]
Poich� il sistema a quel punto ha un teorema che esprime un'equivalenza per
$t_0$ pu� creare la nuova equazione \[
  \vdash \underbrace{(\dots t_0\dots)}_t = (\dots r' \dots)
\]
L'attraversamento del termine da semplificare � ripetuto fino a quando non si trova
alcun match ulteriore per le regole di riscrittura del semplificatore. La
strategia di attraversamento �
\begin{enumerate}
\item \label{enum:simp-traverse-toplevel}%
  Mentre c'� un qualsiasi match per le regole di riscrittura archiviate a questo livello,
	si continui ad applicarle. \emph{Non} si pu� fare affidamento sull'ordine in cui le regole
	di riscrittura sono applicate, eccetto quando un \simpset{}
	include due riscritture con esattamente gli stessi lati sinistri. la
	riscrittura aggiunta per ultima sar� quella preferita per match. (Questo permette una
	certa misura di overloading della riscrittura nella costruzione dei
	\simpset{}.)
%% may not wish to own up to above detail
\item \label{enum:simp-traverse-recurse}%
  Eseguire una ricorsione sui sotto termini del termine. Il modo in cui i termini sono
	attraversati a questo passo pu� essere controllato da \emph{regole di
		congruenza}~(una caratteristica avanzata, si veda la Sezione~\ref{sec:simp-congruences}
	di sotto)
\item Se il passo~\ref{enum:simp-traverse-recurse} ha cambiato il termine
	completamente, si provi un'altra fase di riscrittura a questo livello. Se questo fallisce,
	o se non c'� stato alcun cambiamento dall'attraversamento dei sotto-termin, si provi
	qualsiasi procedura di decisione incorporata (si veda
	la Sezione~\ref{sec:simp-embedding-code}). Se la fase di riscrittura o
	qualsiasi delle procedure di decisione ha alterato il termine, ri ritorni al
	passo~\ref{enum:simp-traverse-toplevel}. Altrimenti di termini.
\end{enumerate}

\subsubsection{Riscrittura condizionale}
\index{semplificazione!riscrittura condizionale}

La descrizione di sopra � una leggera semplificazione dello stato reale delle
cose. Una caratteristica particolarmente potente del semplificatore � che
essa realmente usa regole di riscrittura \emph{condizionali}. Questi sono teoremi
della forma
\[
\vdash P \Rightarrow (\ell = r)
\]
Quando il semplificatore trova un match per il termine $\ell$ durante il suo attraversamento
del termine, tenta di scaricare la condizione $P$. Se il
semplificatore pu� semplificare il termine $P$ a vero, allora l'istanza di
$\ell$ nel termine attraversato pu� essereo sostituito dall'appropriata
istanziazione di $r$.

Quando si semplifica $P$ (un termine che non necessariamente nemmeno occorre
nell'originale), il semplificatore si pu� trovare ad applicare un'altra
regola di riscrittura condizionale. Per fermare eccessive applicazioni
ricorsive, il semplificatore tiene traccia di uno stack di tutte le
condizioni collaterali su cui sta lavorando. Il semplificatore smette la dimostrazione
su una condizione collaterale se nota una ripetizione in questo stack.
C'� anche una variabile accessibile dall'utente, \ml{Cond\_rewr.stack\_limit}
che specifica la dimensione massima dello stack che il semplificatore pu�
di usare.

Le riscritture condizionali possono essere estremamente utili. Per esempio, i teoremi
circa la divisione e il modulo sono spesso accompagnate da condizioni
che richiedono che il divisore non sia zero. Il semplificatore pu� spesso
scaricare queste, in particolare se include una procedura di decisione
aritmetica. Per esempio, il teorema \ml{MOD\_MOD} dalla teoria
\ml{arithmetic} afferma
\[
\vdash 0 < n \;\Rightarrow \; (k\,\textsf{MOD}\,n)\,\textsf{MOD}\,n = k
\,\textsf{MOD}\,n
\]
Il semplificatore (in modo specifico, \ml{SIMP\_CONV~arith\_ss~[MOD\_MOD]})
pu� usare questo teorema per semplificare il termine
\holtxt{(k~MOD~(x~+~1))~MOD~(x~+~1)}: la procedura di decisione
aritmetica pu� dimostrare che \holtxt{0 < x + 1}, giustificando la riscrittura.

Bench� le regole riscrittura condizionali siano potenti, non ogni teorema della
forma descritta di sopra � una scelta appropriata. Una riscrittura scelta male
pu� causare un considerevole degrado delle performance del semplificatore, dal momento
che perde tempo nel tentare di dimostrare condizioni collaterali impossibili. Per
esempio, il semplificatore non � molto bravo a trovare testimoni
esistenziali. Questo significa che la riscrittura condizionale \[
\vdash x < y \land y < z \Rightarrow (x < z = \top)
\]
non funzioner� come si potrebbe sperare. In generale, il semplificatore non � un
buon strumento per eseguire ragionamenti di transitivit�. (Si provi invece uno strumento
al primo ordine come \ml{PROVE\_TAC})

\subsubsection{Generare regole di riscrittura per i teoremi}
\label{sec:generating-rewrite-rules}
\index{teoremi equazionali, nella logica HOL@teoremi equazionali, nella logica \HOL{}!uso dei ... nel semplificatore}

Ci sono due strade per cui un teorema per la riscrittura pu� essere passato al
semplificatore: o come un argomento esplicito a una delle funzioni
\ML{} (\ml{SIMP\_CONV}, \ml{ASM\_SIMP\_TAC} ecc) che prendono liste di
teoremi come argomenti, o includendoli in un frammento \simpset{}
che � incorporato in un \simpset. In entrambi i casi, questi teoremi sono
trasformati prima di essere usati. Le trasformazioni applicate sono
progettate per rendere l'uso interattivo quanto pi� conveniente possibile.

In particolare, non � necessario passare al semplificatore teoremi
che sono esattamente della forma
\[
\vdash P \Rightarrow (\ell = r)
\]
Piuttosto, il semplificatore trasformer� i suoi teoremi di input per estrarre
le riscritture di questa stessa forma. L'esatta trasformazione eseguita �
dipendente dal \simpset{} utilizzato: ciascun \simpset{} contiene la sua
propriat funzione ``filtro'' che � applicata ai teoremi che sono aggiunti ad
esso. La maggior parte dei \simpset{} usano la funzione filtro dal \simpset{}
\ml{pure\_ss} (si veda la Sezione~\ref{sec:purebool-ss}). Tuttavia, quando un
frammento \simpset{} � aggiunto a un simpset completo, il frammento pu�
specificare un componente filtro addizionale. Se specificata, questa funzione
� di tipo \ml{controlled\_thm~->~controlled\_thm~list}, ed � applicata
ad ognuno dei teoremi prodotti dal filtro del \simpset{} esistente.
%
\index{semplificazione!garantire la terminazione}
(Un teorema ``controllato'' � uno che � accompagnato da un pezzo di
dati di ``controllo'' che esprimono il limite (se ne esiste uno) sul numero di volte
che pu� essere applicato. Si veda la Sezione~\ref{sec:simp-special-rewrite-forms}
per come gli utenti possono introdurre questi limiti. Il tipo di ``controllo''
appare nel modulo \ML{} \ml{BoundedRewrites}.)

Il filtro che produce riscritture in \ml{pure\_ss} elimina
le congiunzioni, le implicazioni e le quantificazioni universali fino a quando o ha
un teorema di eguaglianza, o qualche altra forma booleana. Per esempio,
il teorema \ml{ADD\_MODULUS} afferma
\[
\vdash
\begin{array}[t]{l}
(\forall n\;x.\;\;0 < n \Rightarrow ((x + n)\,\textsf{MOD}\,n =
 x\,\textsf{MOD}\,n)) \;\;\land\\
(\forall n\;x.\;\;0 < n \Rightarrow ((n + x)\,\textsf{MOD}\,n =
 x\,\textsf{MOD}\,n))
\end{array}
\]
Questo teorema si trasforma in due regole di riscrittura \[
\begin{array}{l}
\vdash 0 < n \Rightarrow ((x + n)\,\textsf{MOD}\,n = x\,\textsf{MOD}\,n)\\
\vdash 0 < n \Rightarrow ((n + x)\,\textsf{MOD}\,n = x\,\textsf{MOD}\,n)
\end{array}
\]

Se guardando a un'eguaglianza dove ci sono delle variabili sul
lato destro che non occorrono sul lato sinistro, il
semplificatore trasforma questo nella regola \[
\vdash (\ell = r) = \top
\]
Analogamente, se � una negazione booleana $\neg P$, diventa la regola \[
\vdash P = \bot
\]
e altre formule booleane $P$ diventano \[
\vdash P = \top
\]

Infine, se guardando a un'eguaglianza il cui lato sinistro � esso stesso
un'eguaglianza, e dove il lato destro non � a sua volta un'eguaglianza,
il semplificatore trasforma $(x = y) = P$ nelle due regole
\[
\begin{array}{l}
\vdash (x = y) = P\\
\vdash (y = x) = P
\end{array}
\]
Questo � generalmente utile. Per esempio, un teorema come
\[
\vdash \neg(\textsf{SUC}\,n = 0)
\]
far� s� che il semplificatore riscriva entrambi $(\textsf{SUC}\,n = 0)$ e
$(0 = \textsf{SUC}\,n)$ a falso.

La restrizione che il lato destro di una tale regola non sia esso stesso
un'eguaglianza � una semplifice euristica che previene alcune forme di looping.


\subsubsection{Regole di riscrittura di matching}
\label{sec:simp-homatch}

Dato un teorema di riscrittura $\vdash \ell = r$, il primo passo di
esecuzione di una riscrittura � determinare se $\ell$ pu� o meno essere
istanziata cos� da renderla uguale al termine che �
riscritto. Questo processo � conosciuto come matching. Per esempio, se $\ell$
� il termine $\textsf{SUC}(n)$, allora farne il matching verso il termine
$\textsf{SUC}(3)$ avr� successo, e trover� l'istanziazione $n\mapsto
3$. Contrariamente all'unificazione, il matching non � simmetrico: un
pattern $\textsf{SUC}(3)$ non matcher� il termine $\textsf{SUC}(n)$.

\index{matching di ordine superiore} \index{matching!di ordine superiore} Il semplificatore
usa una speciale forma di matching di ordine superiore. Se un pattern
include una variabile di qualche tipo funzione (diciamo $f$), e quella variabile
� applicata a un argomento $a$ che non include alcuna variabile eccetto quelle
che sono legate da un'atrazione ad uno spopo pi� alto. allora il termine
combinato $f(a)$ matcher� qualsiasi termine del tipo appropriato fino a quando le
sole occorrenze delle variabili legate in $a$ sono in sotto-termini
che matchano $a$.

Assumiamo per i seguenti esempi che la variabile $x$ sia legata a uno
scopo pi� alto. Allora, se $f(x)$ deve matchare $x + 4$, la variabile $f$
sar� istanziata a $(\lambda x.\; x + 4)$. Se $f(x)$ deve matchare
$3 + z$, allora $f$ sar� istanziata a $(\lambda x.\;3 + z)$.
Inoltre $f(x + 1)$ matcha $x + 1 < 7$, ma non matcha $x + 2 <
7$.

Un matching di ordine superiore di questo genere rende pi� facile esprimere i risultati
dei movimenti dei quantificatori come regole di riscrittura, e avere queste regole applicate dal
semplificatore. Per esempio, il teorema
\[
\vdash (\exists x. \;P(x)\lor Q(x)) = (\exists x.\;P(x)) \lor (\exists
x.\;Q(x))
\]
ha due variabili di un tipo funzione ($P$ e $Q$), ed entrambe sono
applicate alla variabile legata $x$. Questo significa che quando applicato
all'input \[
\exists z. \;z < 4 \lor z + x = 5 * z
\]
il matcher trover� l'istanziazione \[
\begin{array}{l}
P \mapsto (\lambda z.\;z < 4)\\
Q \mapsto (\lambda z.\;z + x = 5 * z)
\end{array}
\]

Eseguendo questa istanziazione, e poi facendo qualche $\beta$-riduzione
sulla regola di riscrittura, si produce il teorema \[
\vdash (\exists z. \;z < 4 \lor z + x = 5 * z) =
(\exists z. \;z < 4) \lor (\exists z.\;z + x = 5 * z)
\]
come richiesto.

Un altro esempio di una regola che il semplificatore user� con successo �
\[
\vdash f \circ (\lambda x.\; g(x)) = (\lambda x.\;f(g(x)))
\]
La presenza dell'astrazione sul lato sinistro della regola
richiede che un'astrazione appaia nel termine da matchare, cos� questa
regola pu� essere vista come un'implementazione di un metodo per muovere le astrazioni
al di sopra delle composizioni di funzione.

Un esempio di un possibile lato sinistro che \emph{non} matchare come
in generale si potrebbe desiderare � $(\exists x.\;P(x + y))$. Questo �
perch� il predicato $P$ � applicato a un argomento che include la
variabile libera $y$.

\subsection{Caratteristiche avanzate}
\label{sec:advanced-simplifier}

Questa sezione descrive alcune delle caratteristiche avanzate del semplificatore.

\subsubsection{Regole di congruenza}
\label{sec:simp-congruences}
\index{semplificazione!regole di congruenza}
\index{regole di congruenza!nella semplificazione}
Le regole di congruenza controllano il modo in cui il semplificatore traversa un termine.
Esse forniscono anche un meccanismo per mezzo del quale possono essere
aggiunte al contesto del semplificatore assunzioni aggiuntive, che rappresentano informazione circa il
contesto contenitore. Le regole di congruenza pi� semplici sono incorporate nel
simpset \ml{pure\_ss}. Esse specificano come traversare termini applicazione e
astrazione. A questo livello fondamentale, queste regole di congruenza
sono poco di pi� che le regole d'inferenza \ml{ABS}
\[
\frac{\Gamma \turn t_1 = t_2}
{\Gamma \turn (\lquant{x}t_1) = (\lquant{x}t_2)}
\]
(dove $x\not\in\Gamma$) e \ml{MK\_COMB}
\[
\frac{\Gamma \turn f = g \qquad \qquad \Delta \turn x = y}
{\Gamma \cup \Delta \turn f(x) = g(y)}
\]
Quando si specifica l'azione del semplificatore, queste regole dovrebbero essere
lette verso l'alto. Con \ml{ABS}, per esempio, la regola dice ``quando
si semplifica un'astrazione, si semplifichi il corpo $t_1$ a qualche nuovo $t_2$,
e poi il risultato � formato ri-astraendo con la variabile
legata~$x$.''

Ulteriori regole di congruenza dovrebbero essere aggiunte al semplificatore nella forma
di teoremi, attraverso il campo \ml{congs} dei record passati al
costruttore \ml{SSFRAG}. Tali regole di congruenza dovrebbero essere della forma
\[
\mathit{cond_1} \Rightarrow \mathit{cond_2} \Rightarrow \dots (E_1 =
E_2)
\]
dove $E_1$ � la forma da riscrivere. Ciascun $\mathit{cond}_i$ pu�
essere o una formula booleana arbitraria (nel qual caso � trattata come
una condizione collaterale da scaricare) o un'equazione della forma generale
\[
\forall \vec{v}. \;\mathit{ctxt}_1 \Rightarrow \mathit{ctxt}_2
\Rightarrow \dots (V_1(\vec{v}) = V_2(\vec{v}))
\]
dove la variabile $V_2$ deve occorrere libera in $E_2$.

Per esempio, la forma teorema di \ml{MK\_COMB} sarebbe
\[
\vdash (f = g) \Rightarrow (x = y) \Rightarrow (f(x) = g(y))
\]
e la forma teorema di \ml{ABS} sarebbe
\[
\vdash (\forall x. \;f (x) = g (x)) \Rightarrow (\lambda x. \;f(x)) = (\lambda
x.\;g(x))
\]
La forma per \ml{ABS} dimostra come � possibile per le regole
di congruenza gestire variabili legate. Dal momento che le regole di congruenza sono
matchate con il match di ordine superiore della Sezione~\ref{sec:simp-homatch},
questa regola matcher� tutti i possibili termini astrazione.

Questi semplici esempi non hanno ancora dimostrato l'uso delle
condizioni $\mathit{ctxt}$ su sotto-equazioni. Un esempio di questo �
la regola di congruenza (che si trova in \ml{CONG\_ss}) per le implicazioni. Questa
afferma
\[
\vdash (P = P') \Rightarrow (P' \Rightarrow (Q = Q')) \Rightarrow
(P \Rightarrow Q = P' \Rightarrow Q')
\]
Questa regola dovrebbe essere letta: ``Quando si semplifica $P\Rightarrow Q$, prima
si semplifichi $P$ a $P'$. Poi si assuma $P'$, e si semplifichi $Q$ a $Q'$.
Poi il risultato � $P' \Rightarrow Q'$.''

La regola per espressioni condizionali �
\[
\vdash \begin{array}[t]{l}
  (P = P') \Rightarrow (P' \Rightarrow (x = x')) \Rightarrow
  (\neg P' \Rightarrow (y = y')) \;\Rightarrow\\
       (\textsf{if}\;P\;\textsf{then}\;x\;\textsf{else}\;y =
       \textsf{if}\;P'\;\textsf{then}\;x'\;\textsf{else}\;y')
\end{array}
\]
Questa regola permette di assumere la guardia quando si semplifica il
ramo-vero del condizionale, e di assumere la sua negazione quando si
semplifica il ramo-falso.

Le assunzioni contestuali da regole di congruenza sono trasformate in
riscritture usando il meccanismo descritto nella
Sezione~\ref{sec:generating-rewrite-rules}.

Le regole di congruenza possono essere usate per ottenere un numero di interessanti
effetti. Per esempio, una congruenza pu� specificare che i sotto-termini
\emph{non} siano semplificati se lo si desidera. Questo potrebbe essere usato per impedire
la semplificazione dei rami delle espressioni condizionali:
\[
\vdash (P = P') \Rightarrow
       (\textsf{if}\;P\;\textsf{then}\;x\;\textsf{else}\;y =
       \textsf{if}\;P'\;\textsf{then}\;x\;\textsf{else}\;y)
\]
Se aggiunta al semplificatore, questa regola prender� la precedenza su qualsiasi
altra regola per le espressioni condizionali (mascherando quella sopra
\ml{CONG\_ss}, diciamo) e far� s� che il semplificatore di scendere solo
nella guardia. Con le riscritture standard (da \ml{BOOL\_ss}):
\[
\begin{array}{l}
\vdash \;\textsf{if}\;\top\;\textsf{then}\;x\;\textsf{else}\;y \,\;=\,\; x\\
\vdash \;\textsf{if}\;\bot\;\textsf{then}\;x\;\textsf{else}\;y \,\;=\,\; y
\end{array}
\]
gli utenti possono scegliere che il semplificatore ignori
i rami di un condizionali fino a quando la guarda di quel condizionale � semplificata
o a vero o a falso.


\subsubsection{Normalizzazione-AC}
\index{simplification!AC-normalisation}

The simplifier can be used to normalise terms involving associative
and commutative constants.  This process is known as
\emph{AC-normalisation}.  The simplifier will perform AC-normalisation
for those constants which have their associativity and commutativity
theorems provided in a constituent \simpset{} fragment's \ml{ac}
field.

For example, the following \simpset{} fragment will cause
AC-normalisation of disjunctions
\begin{hol}
\begin{verbatim}
   SSFRAG {ac = [(DISJ_ASSOC, DISJ_COMM)],
           rewrs = [], filter = NONE, convs = [],
           dprocs = [], congs = []}
\end{verbatim}
\end{hol}
The pair of provided theorems must state
\begin{eqnarray*}
x \oplus y &=& y \oplus x\\
x \oplus (y \oplus z) &=& (x \oplus y) \oplus z
\end{eqnarray*}
for a constant $\oplus$.  The theorems may be universally quantified,
and the associativity theorem may be oriented either way.  Further,
either the associativity theorem or the commutativity theorem may be
the first component of the pair.  Assuming the \simpset{} fragment
above is bound to the \ML{} identifier \ml{DISJ\_ss}, its behaviour is
demonstrated in the following example:
\begin{session}
\begin{verbatim}
- SIMP_CONV (bool_ss ++ DISJ_ss) [] ``p /\ q \/ r \/ P z``;
<<HOL message: inventing new type variable names: 'a>>
> val it = |- p /\ q \/ r \/ P z = r \/ P z \/ p /\ q : thm
\end{verbatim}
\end{session}

\index{arith_ss (insieme di semplificazione)@\ml{arith\_ss} (insieme di semplificazione)}
L'ordine degli operandi nella forma normale-AC in cui lavora la normalizzazione-AC
del semplificatore non � specificato. Tuttavia, la forma
normale � sempre associata a destra. Si noti inoltre che il \simpset{}
\ml{arith\_ss}, e il frammento \ml{ARITH\_ss} che ne � la base, ha
le sue procedure di normalizzazione su misura per l'addizione sui
numeri naturali. Combinare la normalizzazione-AC, come descritta qui, con
\ml{arith\_ss} pu� far s� che il semplificatore entri in un loop infinito.

I teoremi AC possono essere aggiunti ai \simpset{} anche attraverso la lista di teoremi parte
dell'interfaccia della tattica e della conversione, usando la forma speciale di riscrittura
\ml{AC}:
\begin{session}
\begin{verbatim}
- SIMP_CONV bool_ss [AC DISJ_ASSOC DISJ_COMM] ``p /\ q \/ r \/ P z``;
<<HOL message: inventing new type variable names: 'a>>
> val it = |- p /\ q \/ r \/ P z = r \/ P z \/ p /\ q : thm
\end{verbatim}
\end{session}
Si veda la Sezione~\ref{sec:simp-special-rewrite-forms} per maggiori informazioni su speciali
forme di riscrittura.

\subsubsection{Incorporate codice}
\label{sec:simp-embedding-code}

Il semplificatore dispone di due modi diversi in cui il codice dell'utente pu� essere
incorporato nel suo attraversamento e semplificazione dei termini di input. Incorporando
il loro proprio codice gli utenti possono personalizzare il comportamento del
semplificatore in una misura significativa.

\paragraph{Conversioni utente}
Il pi� semplice dei due metodi permente al semplificatore di includere
conversioni fornite dall'utente. Queste sono aggiunte ai \simpset{} nel
campo {convs} dei frammenti \simpset{}. Questo campo prende liste di
valori di tipo
\begin{hol}
\begin{verbatim}
   { name: string,
    trace: int,
      key: (term list * term) option,
     conv: (term list -> term -> thm) -> term list -> term -> thm}
\end{verbatim}
\end{hol}

I campo \ml{name} e \ml{trace} sono usati quando il tracing del semplificatore
� acceso. Se la conversione � applicata, e se il livello di traccia del
semplificatore � maggiore di o uguale al campo \ml{trace}, allora sar�
emesso un messagio circa l'applicazione della conversione
(che include il suo \ml{name}).

Il campo \ml{key} del record di sopra � usato per specificare i
sotto-termini a cui la conversione dovrebbe essere applicata. Se il valore �
\ml{NONE}, allora la conversione sar� provata ad ogni posizione.
Altrimenti, la conversione � applicata a posizioni di termine che matchano il
pattern fornito. Il primo componente del pattern � una lista di
variabili che dovrebbero essere trattate come costanti quando si cercano match
del pattern. Il secondo componente � il pattern di termine stesso. Il matching
verso questa componente \emph{non} � fatto dal match di ordine superiore della
Sezione~\ref{sec:simp-homatch}, ma da una ``term-net'' di ordine superiore.
Questa forma di matching non cerca di essere precisa; essa � usata per
eliminare in modo efficiente match chiaramente impossibili. Non controlla
i tipi, e non controlla binding multipli. Questo significa che la
conversione non sar� applicata a termini che sono match esatti
per il pattern fornito.

Infine, la conversione stessa. Molti usi di questa infrastruttura sono per aggiungere
normali conversioni \HOL{} (di tipo \ml{term->thm}), e questo pu� essere
fatto ignorando i primi due parametri del campo \ml{conv} Per una
conversione \ml{myconv}, l'idioma standard � di scrivere
\ml{K~(K~myconv)}. Se l'utente lo desidera, tuttavia, il suo codice
\emph{pu�} riferirsi ai primi due parametri. Il secondo parametro �
lo stack delle condizioni collaterali che sono state tentate fino ad ora. Il
primo permette al codice dell'utente di richiamare il semplificatore, passando
lo stack delle condizioni collaterali, e una nuova condizione per risolvere.
L'argomento \ml{term} deve essere di tipo \holtxt{:bool}, e la chiamata
ricorsiva lo semplificher� a vero (e richiamer� \ml{EQT\_ELIM} per trasformare un termine
$t$ nel teorema $\vdash t$). Questa restrizione pu� essere sollevata in una
futura versione di \HOL{} ma per come stanno le cose ora, la chiamata ricorsiva pu�
essere usata \emph{solo} per scaricare la condizione collaterale. Si noti anche che
� responsabilit� dell'utente passare uno stack appropriatamente aggiornato di
condizioni collaterali all'invocazione ricorsiva del semplificatore.

Una conversione fornita dall'utente non dovrebbe mai restituire l'identit� riflessiva
(un'istanza di $\vdash t = t$). Questo far� andare in loop il
semplificatore. Piuttosto che restituire un tale risultato, si sollevi un \ml{HOL\_ERR} o
un'eccezione \ml{Conv.UNCHANGED}. (Entrambe sono trattate allo stesso modo dal semplificatore.)

\paragraph{Procedure di decisione consapevoli del contesto}
Un altro metodo, pi� complicato, per incorporare codice utente nel
semplificatore � \emph{attraverso} il campo \ml{dprocs} della struttura
frammento \simpset{}. Questo metodo � pi� generale rispetto ad aggiungere
conversioni, e permette anche al codice utente di costruire e mantenere i suoi
propri contesti logici costruiti su misura.

Il campo \ml{dprocs} richiede liste di valori di tipo
\ml{Traverse.reducer}. Questi valori sono costruiti con il
costruttore \ml{REDUCER}:
\begin{hol}
\begin{verbatim}
   REDUCER : {initial : context,
              addcontext : context * thm list -> context,
              apply : {solver : term list -> term -> thm,
                       context : context,
                       stack : term list} -> term -> thm}
          -> reducer
\end{verbatim}
\end{hol}
Il tipo \ml{context} � un alias per il tipo \ML{} incorporato
\ml{exn}, quello delle eccezioni. Le eccezioni sono qui usate come un
``tipo universale'', capace di archiviare dati di qualsiasi tipo. Per esempio,
se il dato desiderato � una coppia di un intero e un booleano, allora si
pu� fare la seguente dichiarazione:
\begin{hol}
\begin{verbatim}
   exception my_data of int * bool
\end{verbatim}
\end{hol}
Non � necessario rendere questa dichiarazione visibile con uno scopo
ampio. Infatti, solo le funzioni che accedono e creano contesti di questa
forma hanno bisogno di vederla. Per esempio:
\begin{hol}
\begin{verbatim}
  fun get_data c = (raise c) handle my_data (i,b) => (i,b)
  fun mk_ctxt (i,b) = my_data(i,b)
\end{verbatim}
\end{hol}

Quando crea un valore di tipo \ml{reducer}, l'utente deve fornire un
contesto iniziale, e due funzioni. La prima, \ml{addcontext}, �
chiamata dal meccanismo di attraversamento del semplificatore per dare ad ogni
procedura di decisione incorporata accesso ai teoremi che rappresentano la nuova informazione
di contesto. Per esempio, questa funzione � chiamata con i teoremi dalle
assunzioni attuali in \ml{ASM\_SIMP\_TAC}, e con i teoremi
dagli argomenti lista di teoremi a tutt le varie funzioni di
semplificazione. Mentre un termine � attraversato, le regole di congruenza che governano
questo attraversamento possono fornire teoremi addizionali; questi saranno passati
anche alla funzione \ml{addcontext}. (Naturalmente, � del tutto a carico
della funzione \ml{addcontext} come questi teoremi saranno
gestiti; essi possono persino essere ignorati completamente.)

Quando un riduttore � applicato a un termine, � richiamata la funzione
\ml{apply} fornita. Oltre che al termine da trasformare, la
funzione \ml{apply} � passata anche a un record che contiene un
risolutore di condizione collaterale, l'attuale contesto della procedura di decisione. e
allo stack delle condizioni collaterali tentate fino ad ora. Lo stack e il risolutore
sono gli stessi degli argomenti addizionali dati alle conversioni fornite
dall'utente. La potenza dell'astrazione del riduttore � di avere accesso a
un contesto che pu� essere costruito in modo appropriato per ogni procedura di decisione.

Le procedure di decisione sono applicate per l'ultima volta quando un termine � incontrato dal
semplificatore. Inoltre, esse sono applicate \emph{dopo} che il semplificatore ha
gia� eseguito un ricorsione in ogni sotto-termine e ha tentato di fare quante pi� riscritture
possibili. Questo significa che bench� la riscrittura del semplificatore avvenda in un maniera che va
dall'alto verso il basso, le procedure di decisione saranno applicate dal basso verso l'alto e
solo come ultima risorsa.

Come per le conversioni-utente, le procedure di decisione devono sollevare un'eccezione
piuttosto che restituire istanze di riflessivit�.

\subsubsection{Forme speciali di riscrittura}
\label{sec:simp-special-rewrite-forms}

Si pu� accedere ad alcune delle caratteristiche del semplificatore in un modo
relativamente semplice usando funzioni \ML{} per costruire speciali forme
di teorema. Questi teoremi speciali possono poi essere passati negli
argomenti lista-di-teoremi delle tattiche di semplificazione.

A due delle caratteristiche avanzate del semplificatore, la normalizzazione-AC e
le regole di congruenza si pu� accedere in questo modo. Piuttosto che costruire un
frammento \simpset{} custom che include le regole AC o di congruenza
richieste, l'utente pu� usare piuttosto le funzioni \ml{AC} o \ml{Cong}:
\begin{hol}
\begin{verbatim}
   AC : thm -> thm -> thm
   Cong : thm -> thm
\end{verbatim}
\end{hol}
Per esempio, se il valore teorema
\begin{hol}
\begin{verbatim}
   AC DISJ_ASSOC DISJ_COMM
\end{verbatim}
\end{hol}
appare tra i teoremi passati a una tattica di semplificazionie, allora
il semplificatore eseguir� una normalizzazione-AC di disgiunzioni. La
funzione \ml{Cong} fornisce un'interfaccia analoga per l'aggiunta di
nuove regole di congruenza.

\index{semplificazione!garantire la terminazione}
\index{Once (controllo delle applicazioni di riscrittura)@\ml{Once} (controllo delle applicazioni di riscrittura)|pin}
\index{Ntimes (controllo delle applicazioni di riscrittura)@\ml{Ntimes} (controllo delle applicazioni di riscrittura)|pin}
Altre due funzioni forniscono un meccanismo crudo per controllare il
numero di volte che una riscrittura individuale sar� applicata.
\begin{hol}
\begin{verbatim}
   Once : thm -> thm
   Ntimes : thm -> int -> thm
\end{verbatim}
\end{hol}
Un teorema ``avvolto'' nella funzione \ml{Once} sar� applicato
una sola volta quando il semplificatore � applicato a un termine dato. Un teorema
avvolto in \ml{Ntimes} sar� applicato tante volte quante sono date nel
parametro intero.

\paragraph{Semplificare a particolari sotto-termini}
\index{semplificazione!a particolari sotto-termini}
Abbiamo gi� visto (Sezione~\ref{sec:simp-congruences} di sopra) che
la tecnologia di congruenza del semplificatore pu� essere usata per forzare il
semplificatore ad ignorare termini particolari. L'esempio nella sezione
di sopra ha discusso come un regola di congruenza potrebbe essere usata per assicurare che
solo le guardie delle espressioni condizionali dovrebbero essere semplificate.

In molte dimostrazioni, � comune voler riscrivere solo un lato o
l'altro di un connettivo binario (spesso, questo connettivo �
un'eguaglianza). Per esempio, questo capita quando si riscrive con equazioni
da complicate definizioni ricorsive che non sono soltanto ricorsioni
strutturali. In tali definizioni, il lato sinistro dell'equazioni
avr� un simbolo funzione attaccato a una sequenza di variabili, ad esempio:
\begin{hol}
\begin{verbatim}
   |- f x y = ... f (g x y) z ...
\end{verbatim}
\end{hol}
Teoremi di una forma analoga sono anche restituiti come i teoremi
``casi'' dalle definizioni induttive.

Qualunque sia la loro origine, tali teoremi sono il classico esempio di
qualcosa a cui si vorrebbe attaccare il qualificatore \ml{Once}.
Tuttavia, questo pu� non essere sufficiente: si pu� desiderare di dimostrare un risultato
come
\begin{hol}
\begin{verbatim}
   f (constructor x) y = ... f (h x y) z ...
\end{verbatim}
\end{hol}
(Con le relazioni, il goal pu� spesso rappresentare un'implicazione al posto di
un'eguaglianza.) In questa situazioni, spesso si vuole semplicemente espandere
l'istanza di \holtxt{f} sulla sinistra, lasciando l'altra occorrenza
da sola. Usare \ml{Once} espander� solo una di esse, ma senza
specificare quale deve essere espansa.

La soluzione a questo problema � di usare speciali regole di congruenza,
costruite come forme speciali che possono essere passate come teoremi come
\ml{Once}. Le funzioni
\begin{hol}
\begin{verbatim}
   SimpL : term -> thm
   SimpR : term -> thm
\end{verbatim}
\end{hol}
costruiscono regole di congruenza per forzare la riscritture alla sinistra o alla destra di
termini particolari. Per esempio, se \holtxt{opn} � un operatore binario,
\ml{SimpL~\holquote{(opn)}} restituisce \ml{Cong} applicato al teorema
\begin{hol}
\begin{verbatim}
   |- (x = x') ==> (opn x y = opn x' y)
\end{verbatim}
\end{hol}
\index{SimpLHS@\ml{SimpLHS}|pin}\index{SimpRHS@\ml{SimpRHS}|pin}
Dal momento che il caso eguaglianza � cos� comune, gli speciali valori
\ml{SimpLHS} e \ml{SimpRHS} sono forniti per forzare
la semplificazione sulla sinistra o sulla destra di un'eguaglianza rispettivamente.
Queste sono definite essere solo applicazioni di \ml{SimpL} e \ml{SimpR}
all'eguaglianza.

Si noti che queste regole si applicano per tutto un termine, non soltanto
all'occorrenza pi� alta di un operatore. Inoltre, l'operatore pi� alto nel
termine non ha bisogno di essere quello della regola di congruenza. Questo comportamento �
una conseguenza automatica dell'implementazione in termini di regole
di congruenza.

\subsubsection{Limitare la semplificazione}
\label{sec:limit-simpl}

\index{semplificazione!garantire la terminazione}
Oltre alle forme-di-teoremi \ml{Once} and \ml{Ntimes} appena
discusse, che limitano il numero di volte che una particolare riscrittura �
applicata, il semplificatore pu� anche essere limitato nel numero totale di
riscritture che esso esegue. La funzione \ml{limit} (in \ml{simpLib} e
\ml{bossLib})
\begin{hol}
\begin{verbatim}
   limit : int -> simpset -> simpset
\end{verbatim}
\end{hol}
registra un limite numerico in un \simpset{}. Quando un \simpset{} limitato
poi lavora su un termine, non applicher� mai pi� del numero dato
di riscritture a quel termine. Quando sono usate riscritture condizionali, la
riscrittura fatta nello scaricamento delle condizioni collaterali pesano negativamente sul
limite, finch� la riscrittura � infine applicata. Anche l'applicazione
delle regole di congruenza, delle conversioni e delle procedure
di decisione fornite dall'utente pesano tutte negativamente sul limite.

Quando il semplificatore cede il controllo a una conversione o a una procedura di
decisione fornita dall'utente non pu� garantire che queste funzioni restituiranno
alla fine un risultato (e inoltre esse possono prendere un tempo arbitrariamente lungo per terminare, spesso una preoccupazione
con le procedure di decisione aritmetica), ma l'uso di \ml{limit} �
altrimenti un buon metodo per assicurare che la semplificazione termina.

\subsubsection{Riscrittura con pre-ordini arbitrari}
\label{sec:preorder-rewriting}
\index{semplificazione!con i pre-ordini}

Oltre a semplificare rispetto all'eguaglianza, � anche
possibile usare il semplificatore per ``riscrivere'' rispetto a una relazione
che � riflessiva e transitiva (un \emph{preordine}). Questo pu� essere un
modo molto potente di lavorare con relazioni di transizione nella semantica
operazionale.

{\newcommand{\bred}{\ensuremath{\rightarrow^*_\beta}}

  Si immagini, per esempio, che si debba impostare un ``profondo incorporamento'' del
	$\lambda$-calcolo. Questo implicher� la definizione di un nuovo tipo
	(diciamo, \texttt{lamterm}) all'interno della logica, cos� come le definizioni delle
	funzioni appropriate (ad esempio, la sostituzione) e le relazioni su
	\texttt{lamterm}. E' probabile che si lavori con la chiusura riflessiva
	e transitiva della $\beta$-riduzione (\bred). Questa relazione ha
	regole di congruenza come
\[
\begin{array}{c@{\qquad\qquad}c}
\infer{M_1 \,N\;\bred\;M_2\,N}{M_1 \;\bred\;M_2} &
\infer{M \,N_1\;\bred\;M\,N_2}{N_1 \;\bred\;N_2}\\[3mm]
\multicolumn{2}{c}{\infer{(\lambda v.M_1)\;\bred\;(\lambda v.M_2)}{M_1\;\bred M_2}}
\end{array}
\] e un'importante riscrittura
\[
\infer{(\lambda v. M)\,N \;\bred\; M[v := N]}{}
\]
Dovendo applicare queste regole manualmente mostrare che un
dato termine iniziale si pu� ridurre a una particolare destinazione � di solito
molto doloroso, coinvolgendo molte applicazioni, non solo quelle dei teoremi
di sopra, ma anche quelle dei teoremi che descrivono la chiusura riflessiva e
transitiva (si veda la Sezione~\ref{relation}).

Bench� il $\lambda$-calcolo sia non-deterministico, esso � anche confluente, cos�
vale il seguente teorema:
\[
\infer{
  M_1 \;\bred\;N\;\;=\;\;M_2\;\bred\; N
}{
  \beta\textrm{-nf}\;N & M_1 \;\bred\;M_2
}
\]
Questo � il teorema critico che giustifica il cambio dalla riscrittura
con l'eguaglianza alla riscrittura con \bred. Esso dice che se si ha un termine
$M_1\bred N$, dove $N$ � una forma $\beta$-normale, e se $M_1$ riscrive a
$M_2$ sotto \bred, allora il termine originale � uguale a $M_2\bred N$.
Avendo fortuna, $M_2$ sar� sintatticamente identico a $N$, e
la riflessivit� di \bred{} dimostrer� il risultato desiderato. I teoremi
come questi, che giustificano il cambio da una relazione di riscrittura a
un'altra sono conosciuti come \emph{congruenze d'indebolimento}.

Quando aggiustato in modo appropriato, il semplificatore pu� essere modificato per sfruttare
i cinque teoremi di sopra, e dimostrare automaticamente risultati come
\[
u ((\lambda f\,x. f (f\,x)) v) \bred u (\lambda x. v(v\,x))
\]
(sotto le assunzioni che i termini $u$ e $v$ siano variabili del
$\lambda$-calcolo, rendendo il risultato come una forma $\beta$-normale).

Inoltre, si avranno probabilmente vari teoremi di riscrittura
che si vorranno usare oltre a quelli specificati di sopra. Per
esempio, se in precedenza si � dimostrato un teorema come
\[
K\,x\,y \bred x
\]
allora il semplificatore pu� prendere anche questo in considerazione.

La funzione che ottiene tutto questo �
\index{add_relsimp@\ml{add\_relsimp}}
\begin{verbatim}
   simpLib.add_relsimp  : {trans: thm, refl: thm, weakenings: thm list,
                           subsets: thm list, rewrs : thm list} ->
                          simpset -> simpset
\end{verbatim}
I campo del record che � il primo argomento sono:
\begin{description}
\item[\texttt{trans}] Il teorema che afferma che la relazione �
	transitiva, nella forma $\forall x y z. R\,x\,y \land R\,y\,z \Rightarrow R x z$.
\item[\texttt{refl}] Il teorema che afferma che la relazione �
	riflessiva, nella forma $\forall x. R\,x\,x$.
\item[\texttt{weakenings}] Una lista di congruenze d'indebolimento, della
	forma generale $P_1 \Rightarrow P_2 \Rightarrow \cdots (t_1 = t_2)$, dove almeno una delle
	$P_i$ menzioner� presumibilmente la nuova relazione $R$ applicata a una
	variabile che appare in $t_1$. Altri
	antecedenti possono essere condizioni collaterali come il requisito
	nell'esempio di sopra che il termine $N$ sia in forma $\beta$-normale.
\item[\texttt{subsets}] Teoremi della forma $R'\,x\,y \Rightarrow R\,x\,y$.
	Questi sono usati per aumentare il risultante ``filtro'' del \simpset{} cos� che
	i teoremi nel contesto che menziona $R'$ deriveranno utili riscritture
	che coinvolgono $R$. Nell'esempio della $\beta$-riduzione, si potrebbe avere anche
	una relazione $\rightarrow_{wh}^*$ per la riduzione weak-head. Qualsiasi riduzione
	weak-head � anche una $\beta$-riduzione, cos� pu� essere utile anche avere che
	il semplificatore ``promuova'' automaticamente i fatti circa la riduzione weak-head
	a fatti circa la $\beta$-riduzione, e poi li usi come riscritture.
\item[\texttt{rewrs}] Possibilmente riscritture condizionali, presumibilmente la maggior parte
	della forma $P \Rightarrow R\,t_1\,t_2$. Qui possono anche essere
	incluse riscritture sull'eguaglianza, che permettono di includere utili fatti aggiuntivi. Per
	esempio, quando si lavora con il $\lambda$-calcolo, si potrebbe includere sia
	la riscrittura per $K$ di sopra, cos� come la definizione della
	sostituzione.
\end{description}
} % end of block defining \bred

L'applicazione di questa funzione a un \simpset{} \texttt{ss}
produrr� un \texttt{ss} aumentato che ha tutti i comportamenti del
\texttt{ss} esistente, cos� come la capacit� di riscrivere con la relazione
data.


\index{semplificazione|)}

\section{Efficient Applicative Order Reduction---\texttt{computeLib}}
\label{sec:computeLib}

La Sezione~\ref{sec:datatype} e la Sezione~\ref{TFL} mostrano la capacit� di
\HOL{} di rappresentare molti dei costrutti standard della programmazione
funzionale. Se si vuole quindi `eseguire' programmi funzionali su
argomenti, ci sono molte scelte. Primo, si pu� applicare il
semplificatore, come mostrato nella Sezione~\ref{sec:simpLib}. Questo permette
di esercitare tutto il potere del processo di riscrittura,
inclusa, per esempio, l'applicazione delle procedure di decisione per
dimostrare vincoli sulle regole di riscrittura condizionale. Secondo, si potrebbe
scrivere il programma, e tutti i programmi da cui dipende in modo transitivo,
in un file in una sintassi concreta adatta, e invocare un compilatore o
un interprete. Questa funzionalit� � disponibile in \HOL{} attraverso l'uso di
\ml{EmitML.exportML}.

Terzo, si pu� usare \ml{computeLib}. Questa libreria supporta una valutazione
call-by-value delle funzioni \HOL{} per passi deduttivi. In altre parole, �
molto simile ad avere un interprete \ML{} all'interno della logica \HOL{},
lavorando per inferenza in avanti. Quando usati in questo modo, i programmi
funzionali possono essere eseguiti pi� velocemente che usando il semplificatore.

Gli entry-point pi� accessibili per usare la libreria \ml{computeLib}
sono la conversione \ml{EVAL} e la sua tattica corrispondente
\ml{EVAL\_TAC}. Queste dipendono su un database interno che archivia
definizioni di funzione. Nel seguente esempio, caricare \ml{sortingTheory}
aumenta questo database con le definizioni rilevanti, in particolare quella
di Quicksort (\holtxt{QSORT}), e poi possiamo valutare
\holtxt{QSORT} su una lista concreta.
%
\setcounter{sessioncount}{0}
\begin{session}
\begin{verbatim}
  - load "sortingTheory";

  - EVAL ``QSORT (<=) [76;34;102;3;4]``;
  > val it = |- QSORT $<= [76; 34; 102; 3; 4] = [3; 4; 34; 76; 102] : thm
\end{verbatim}
\end{session}
Spesso, l'argomento a una funzione non ha variabili: in quel caso
l'applicazione di \ml{EVAL} dovrebbe restituire un risultato ground,
come nell'esempio di sopra. Tuttavia, \ml{EVAL} pu� anche valutare funzioni su
argomenti con variabili---la cosiddetta valutazione \emph{simbolica}---e
in quel caso, il comportamento di \ml{EVAL} dipende dalla struttura
dell'equazioni di ricorsione. Per esempio, nella seguente sessione, se c'�
sufficiente informazione nell'input, la valutazione simbolica pu� restituire
un risultato interessante. Tuttavia, se nell'input non c'� informazione
sufficiente a permettere all'algoritmo alcuna presa, non avr� luogo
alcuna espansione
%
\begin{session}
\begin{verbatim}
  - EVAL ``REVERSE [u;v;w;x;y;z]``;
  > val it = |- REVERSE [u; v; w; x; y; z] = [z; y; x; w; v; u] : thm

  - EVAL ``REVERSE alist``;
  > val it = |- REVERSE alist = REVERSE alist : thm
\end{verbatim}
\end{session}
%

\subsection{Trattare con la divergenza}

La difficolt� maggiore con l'uso di \ml{EVAL} � la terminazione. Troppo
spesso, la valutazione simbolica con \ml{EVAL} diverger�, o generer�
termini enormi. Il caso usuale sono i condizionali con le variabili nel
test. Per esempio, la seguente definizione � probabilmente uguale a \holtxt{FACT},
%
\begin{session}
\begin{verbatim}
  Define `fact n = if n=0 then 1 else n * fact (n-1)`;
  > val it = |- fact n = (if n = 0 then 1 else n * fact (n - 1)) : thm
\end{verbatim}
\end{session}
%
Ma le due definizioni sono valutate in modo completamente differente.
%
\begin{session}
\begin{verbatim}
  EVAL ``FACT n``;
  > val it = |- FACT n = FACT n : thm

  - EVAL ``fact n``;
  <.... interrupt key struck ...>
  > Interrupted.
\end{verbatim}
\end{session}
%
La definizione ricorsiva primitiva di \holtxt{FACT} non si espande
per nulla, mentre la ricorsione stile-decostruttore di \holtxt{fact} non smette mai
di espandere. Un rudimentale strumento di monitoraggio mostra il comportamento, prima
su un argomento ground, poi su un argomento simbolico.
%
\begin{session}
\begin{verbatim}
  - val [fact] = decls "fact";
  - computeLib.monitoring := SOME (same_const fact);

  - EVAL ``fact 4``;
  fact 4 = (if 4 = 0 then 1 else 4 * fact (4 - 1))
  fact 3 = (if 3 = 0 then 1 else 3 * fact (3 - 1))
  fact 2 = (if 2 = 0 then 1 else 2 * fact (2 - 1))
  fact 1 = (if 1 = 0 then 1 else 1 * fact (1 - 1))
  fact 0 = (if 0 = 0 then 1 else 0 * fact (0 - 1))
  > val it = |- fact 4 = 24 : thm

  - EVAL ``fact n``;
  fact n = (if n = 0 then 1 else n * fact (n - 1))
  fact (n - 1) = (if n - 1 = 0 then 1 else (n - 1) * fact (n - 1 - 1))
  fact (n - 1 - 1) =
  (if n - 1 - 1 = 0 then 1 else (n - 1 - 1) * fact (n - 1 - 1 - 1))
  fact (n - 1 - 1 - 1) =
  (if n - 1 - 1 - 1 = 0 then
     1
   else
     (n - 1 - 1 - 1) * fact (n - 1 - 1 - 1 - 1))
     .
     .
     .
\end{verbatim}
\end{session}
%
In ogni espansione ricorsiva, il testo coinvolge una variabile, e di conseguenza
non pu� essere ridotta a \holtxt{T} o a \holtxt{F}. Cos�, l'espansione
non si ferma mai.

Some simple remedies can be adopted in trying to deal with
non-terminating symbolic evaluation.

Si possono adottare alcuni semplici rimedi nel provare a trattare con
una valutazione simbolica che non termina.
\begin{itemize}
\item \ml{RESTR\_EVAL\_CONV} si comporta come \ml{EVAL} eccetto
	che prende una lista extra di costanti. Durante
	la valutazione, se si incontra una delle costanti fornite, essa
	non sar� espansa. Questo permette di valutare gi� fino a un livello specificato.
	e pu� essere usato per tagliare alcune valutazioni circolari.
\item anche \ml{set\_skip} pu� essere usata per controllare
	la valutazione. Si veda la voce per \ml{CBV\_CONV} in \REFERENCE{} per
	una discussione di \ml{set\_skip}

\end{itemize}

\paragraph{Valutatori custom}

Per alcuni problemi, � desiderabile costruire un valutatore
personalizzato, specializzato su un insieme fissato di definizioni. Il tipo
\ml{compset} che si trova in \ml{computeLib} � il tipo dei database di definizione. Le
funzioni \ml{new\_compset}, \ml{bool\_compset}, \ml{add\_funs}, e
\ml{add\_convs} forniscono il modo standard per costruire tali
database. Un altro \holtxt{compset} piuttosto utile �
\ml{reduceLib.num\_compset}, che pu� essere usato per valutare
termini con numeri e booleani. Dato un \ml{compset}, la funzione
\ml{CBV\_CONV} genera un valutatore: � usata per implementare \ml{EVAL}.
Si veda \REFERENCE{} per maggiori dettagli.

\paragraph{Trattare con Funzioni sui Numeri di Peano}

Le funzioni definite per pattern-matching su numeri nello stile di Peano non sono
nel formato giusto per \ml{EVAL}, dal momento che i calcoli saranno
asintoticamente inefficienti. Piuttosto, la stessa definizione dovrebbe essere
usata su numerali, che � una notazione posizionale descritta nella
Sezione~\ref{sec:numerals}. Tuttavia, � preferibile per le dimostrazioni
lavorare su numeri di Peano. Per colmare questa lacuna, la funzione
\ml{numLib.SUC\_TO\_NUMERAL\_DEFN\_CONV} � usata per convertire una funzione
su numeri di Peano in una su numerali, che � il formato che
\ml{Eval} preferisce. \ml{Define} chiamer� automaticamente
\ml{SUC\_TO\_NUMERAL\_DEFN\_CONV} sul suo risultato.

\paragraph{Archiviare le definizioni}

\ml{Define} aggiunge automaticamente la sua definizione al compset globale
usato da \ml{EVAL} e \ml{EVAL\_TAC}. Tuttavia, quando \ml{Hol\_defn} �
usata per definire una funzione, le sue equazioni di definizione non sono aggiunte al
compset globale fino a quando \ml{tprove} � usata per dimostrare i vincoli
di terminazione. Inoltre, \ml{tprove} non aggiunge la definizione
in modo persistente nel compset globale. Di conseguenza, si deve usare
\ml{add\_persistent\_funs} in una teoria per essere sicuri che le definizioni
fatte da \ml{Hol\_defn} siano disponibili a \ml{Eval} nelle teorie
discendenti. Un altro punto: si deve chiamare \ml{add\_persistent\_funs}
prima di chiamare \ml{export\_theory}.


\section{Le Librerie Aritmetiche---\texttt{numLib}, \texttt{intLib} and \texttt{realLib}}
\label{sec:numLib}
\index{procedure di decisione!Aritemtica Presburger su numeri naturali}

Ognuna delle librerie aritmetiche di \HOL{} fornisce una
suite di definizioni e teoremi cos� come il supporto per l'inferenza automatica.

\paragraph{numLib}

I numeri pi� di base in \HOL{} sono i numeri naturali. La
libreria \ml{numLib} comprende le teorie \ml{numTheory},
\ml{prim\_recTheory}, \ml{arithmeticTheory}, e \ml{numeralTheory}.
Questa libreria incorpora anche un valutatore per espressioni numeriche
da \ml{reduceLib} e una procedura di decisione per l'aritmetica lineare
\ml{ARITH\_CONV}. Il valutatore e la procedura di decisione sono
integrati nel simpset \ml{arith\_ss} usato dal semplificatore.
Allo stesso modo, la procedura di decisione dell'aritmetica lineare pu� essere invocata
direttamente attraverso \ml{DECIDE} e \ml{DECIDE\_TAC}, che si trovano entrambe in
\ml{bossLib}.


\index{procedure di decisione!Aritemtica Presburger su interi}
\paragraph{intLib}

La libreria \ml{intLib} comprende \ml{integerTheory}, un'estesa
teoria degli interi, pi� due procedure di decisione
per la completa aritmetica Presburger. Queste sono disponibili come
\ml{intLib.COOPER\_CONV} e \ml{intLib.ARITH\_CONV}. Queste
procedure di decisione sono in grado di trattare con l'aritmetica lineare
sugli interi e i numeri naturali, cos� come di trattare
con un'alternanza arbitraria di quantificatori. La procedura
\ml{ARITH\_CONV} � un'implementazione dell'Omega Test, e sembra
in generale avere migliori performance rispetto all'algoritmo di Cooper. Ci sono
comunque problemi per cui questo non � vero, cos� � utile avere disponibili
entrambe le procedure.

\paragraph{realLib}

La libreria \ml{realLib} fornisce uno sviluppo fondazionale
dei numeri reali e dell'analisi. Si veda la Sezione~\ref{reals}
per una rapida descrizione delle teorie.
E' anche fornita una teoria dei polinomi, in \theoryimp{polyTheory}.
Una procedura di decisione per l'aritmetica lineare sui numeri reali
� anche fornita da \ml{realLib}, sotto il nome \ml{REAL\_ARITH\_CONV}
e \ml{REAL\_ARITH\_TAC}.

\section{Libreria Bit Vector---\texttt{wordsLib}}

La libreria \theoryimp{wordsLib} fornisce uno strumento di supporto per i bit-vectors, questo include infrastrutture per: la valutazione, il parsing, il pretty-printing e la semplificazione.

\subsection{Valutazione}

La libreria \theoryimp{wordsLib} dovrebbe essere caricata quando si valutano termini bit-vector ground. Questa libreria fornisce un \emph{compset} \ml{words\_compset}, che
pu� essere usato nella costruzione di \emph{compese} e conversioni personalizzati.
\setcounter{sessioncount}{0}
\begin{session}
\begin{verbatim}
- load "wordsLib";
> val it = () : unit

- EVAL ``8w + 9w:word4``;
> val it = |- 8w + 9w = 1w : thm
\end{verbatim}
\end{session}
Si noti che qui � usata un'annotazione di tipo per designare la dimensione word. Quando la dimensione word � rappresentata da una variabile di tipo (cio� per word i lunghezza arbitraria), la valutazione
pu� dare risultati parziali o insoddisfacenti.

\subsection{Parsing e pretty-printing}

I word possono essere parsati in binario, decimale e esadecimale. Per esempio:
\begin{session}
\begin{verbatim}
- ``0b111010w : word8``;
> val it = ``58w`` : term

- ``0x3Aw : word8``;
> val it = ``58w`` : term
\end{verbatim}
\end{session}
E' possibile fare il parsing di numeri ottali, ma questo deve essere prima abilitato impostando la reference \ml{base\_tokens.allow\_octal\_input} a true. Per esempio:
\begin{session}
\begin{verbatim}
- ``072w : word8``;
> val it = ``72w`` : term

- base_tokens.allow_octal_input:=true;
> val it = () : unit

- ``072w : word8``;
> val it = ``58w`` : term
\end{verbatim}
\end{session}

I word possono essere stampati usando le basi numeriche standard. Per esempio, la funzione
\ml{wordsLib.output\_words\_as\_bin} selezioner� il formato binario:
\begin{session}
\begin{verbatim}
- wordsLib.output_words_as_bin();
> val it = () : unit

- EVAL ``($FCP ODD):word16``;
> val it = |- $FCP ODD = 0b1010101010101010w : thm
\end{verbatim}
\end{session}
La funzione \ml{output\_words\_as} � pi� flessibile e permette alla base numerica di variare a seconda della
lunghezza del word e del valore numerico. Il pretty-printer di default (installato quando si carica \theoryimp{wordsLib}) stampa valori piccoli in decimale e valori grandi in esadecimale.
La funzione \ml{output\_words\_as\_oct} abiliter� automaticamente il parsing per i numeri ottali.

La variabile di traccia \ml{"word printing"} fornisce un metodo alternativo per cambiare la base numerica di output --- � particolarmente adatta per selezionare temporaneamente una base numerica, per esempio:
\begin{session}
\begin{verbatim}
- Feedback.trace ("word printing", 1) Parse.print_term ``32w``;
<<HOL message: inventing new type variable names: 'a>>
0b100000w> val it = () : unit
\end{verbatim}
\end{session}
Le scelte sono come segue: 0 (default) -- numeri piccoli in decimale, numeri grandi in esadecimale: 1 -- binario; 2 -- ottale; 3 -- decimale; e 4 -- esadecimale.

\subsubsection{Tipi}

Si pu� aver notato che \ty{:word4} e \ty{:word8} sono stati usati come convenienti abbreviazioni di parsing per \ty{:\bool[4]} e \ty{:\bool[8]} --- questa agevolazione � disponibile per molte dimensioni standard di word. Gli utenti che desiderano usare questa notazione per dimensioni di non-standard di word possono usare la funzione \ml{wordsLib.mk\_word\_size}:
\begin{session}
\begin{verbatim}
- ``:word15``;
! Uncaught exception:
! HOL_ERR

- wordsLib.mk_word_size 15;
> val it = () : unit

- ``:word15``;
> val it = ``:bool[15]`` : hol_type
\end{verbatim}
\end{session}

\subsubsection{Overloading degli operatori}

I simboli per le operazioni aritmetiche standard (addizione, sottrazione e moltiplicazione) sono sottoposte a overload con operatori per altre teorie standard, cio� per i numeri naturali, interi, razionali e reali. In molti casi l'inferenza di tipo risolver� l'overloading, tuttavia, in alcuni casi questo non � possibile. La scelta dell'operatore dipender� dall'ordine in cui le teorie sono caricate. Per cambiare questo comportamento sono fornite le funzioni \ml{wordsLib.deprecate\_word} e \ml{wordsLib.prefer\_word}. Per esempio, nella seguente sessione, la selezione degli operatori word � deprecata:
\begin{session}
\begin{verbatim}
- type_of ``a + b``;
<<HOL message: more than one resolution of overloading was possible>>
<<HOL message: inventing new type variable names: 'a>>
> val it = ``:bool['a]`` : hol_type

- wordsLib.deprecate_word();
> val it = () : unit

- type_of ``a + b``;
<<HOL message: more than one resolution of overloading was possible>>
> val it = ``:num`` : hol_type
\end{verbatim}
\end{session}
Di sopra, l'addizione tra numeri naturali � scelta al posto dell'addizione word. Al contrario, i word sono preferiti rispetto agli interi di sotto:
\begin{session}
\begin{verbatim}
- load "intLib"; ...

- type_of ``a + b``;
<<HOL message: more than one resolution of overloading was possible>>
> val it = ``:int`` : hol_type

- wordsLib.prefer_word();
> val it = () : unit
- type_of ``a + b``;
<<HOL message: more than one resolution of overloading was possible>>
<<HOL message: inventing new type variable names: 'a>>
> it = ``:bool['a]`` : hol_type
\end{verbatim}
\end{session}
Naturalmente, potrebbero essere state aggiunte annotazioni di tipo per evitare questo problema completamente. Si noti che, diversamente da \ml{deprecate\_int}, la funzione \ml{deprecate\_word} non rimuove gli overloading, semplicemente abbassa la loro precedenza.

\subsubsection{Indovinare le lunghezze word}

Pu� essere una seccatura aggiungere annotazioni di tipo quando si specifica il tipo di ritorno per operazioni come: \holtxt{word\_extract}, \holtxt{word\_concat}, \holtxt{concat\_word\_list} e \holtxt{word\_replicate}. Questo perch� c'� spesso una lunghezza ``standard'' che potrebbe essere indovinata, ad esempio la concatenazione di solito somma le lunghezze dei word costituenti. Un'agevolazione per indovinare la lunghezza word � controllata dalla reference \ml{wordsLib.guessing\_word\_lengths}, che � falsa di default. Le congetture sono eseguite durante un passo di post-processing che avviene dopo l'applicazione di \ml{Parse.Term}. Questo � mostrato di sotto.
\begin{session}
\begin{verbatim}
- wordsLib.guessing_word_lengths:=true;
> val it = () : unit

- ``concat_word_list [(4 >< 1) (w:word32); w2; w3]``;
<<HOL message: inventing new type variable names: 'a, 'b>>
<<HOL message: assigning word length: 'a <- 4>>
<<HOL message: assigning word length: 'b <- 12>>
> val it =
    ``concat_word_list [(4 >< 1) w; w2; w3]``
     : term
\end{verbatim}
\end{session}
Nell'esempio di sopra, le congetture sulla lunghezza dei word sono attivate. Sono fatte due congetture: ci si aspetta che l'estrazione dia un word di quattro bit, e che la concatenazione dia un word di dodici bit ($3 \times 4$). Se sono richieste lunghezze numeriche non standard allo si possono aggiungere delle annotazioni di tipo per evitare che siano fatte delle congetture. Quando la congettura � disattivata i tipi risultanti rimangono come variabili di tipo inventate, cio� come gli alfa e i beta di sopra.

\subsection{Semplificazione e conversioni}

Sono forniti i seguenti frammenti \emph{simpset}:
\begin{description}
\item[\ml{SIZES\_ss}] valuta un gruppo di funzioni che operano su tipi numerici, come \holtxt{dimindex} e \holtxt{dimword}.
\item[\ml{BIT\_ss}] prova a semplificare le occorrenze della funzione\holtxt{BIT}.
\item[\ml{WORD\_LOGIC\_ss}] semplifica operazioni logiche bitwise.
\item[\ml{WORD\_ARITH\_ss}] semplifica operazioni aritmetiche word. La sottrazione � sostituita con la moltiplicazione da -1.
\item[\ml{WORD\_SHIFT\_ss}] semplifica operazioni shift.
\item[\ml{WORD\_ss}] contiene tutti i frammenti di sopra, e fa anche una valutazione estra dei termini ground. Questo frammento � aggiunto a \ml{srw\_ss}.
\item[\ml{WORD\_ARITH\_EQ\_ss}] semplifica \holtxt{``a = b``} to \holtxt{``a - b = 0w``}.
\item[\ml{WORD\_BIT\_EQ\_ss}] espande in modo aggressivo operazioni bit-vector non-aritmetiche in espressioni booleane. (Dovrebbe essere usata con attenzione -- include \ml{fcpLib.FCP\_ss}.)
\item[\ml{WORD\_EXTRACT\_ss}] semplificazione per una variet� di operazioni: conversioni da word a word; concatenazione; shift e estrazione bit-field.  Pu� essere usata in  situationi dove \ml{WORD\_BIT\_EQ\_ss} non � adatta.
\item[\ml{WORD\_MUL\_LSL\_ss}] semplifica la moltiplicazione con un letterale word in una somma di prodotti parziali.
\end{description}
Molti di questi frammenti \emph{simpset} hanno delle conversioni corrispondenti. Per esempio, la conversione \ml{WORD\_ARITH\_CONV} � basata si \ml{WORD\_ARITH\_EQ\_ss}, tuttavia, fa del lavoro extra per assicurare che \holtxt{``a = b``} e \holtxt{``b = a``} si convertano nella stessa espressioni. Di conseguenza, questa conversione � adatta per ragionare circa l'eguaglianza delle espressioni aritmetiche word.

Il comportamento dei frammenti elencati di sopra � mostrato usando la seguente funzione:
\begin{session}
\begin{verbatim}
- fun conv ss = SIMP_CONV (pure_ss++ss) [];
> val conv = fn : ssfrag -> term -> thm
\end{verbatim}
\end{session}
La seguente sessione mostra \ml{SIZES\_ss}:
\begin{session}
\begin{verbatim}
- conv wordsLib.SIZES_ss ``dimindex(:12)``;
> val it = |- dimindex (:12) = 12 : thm

- conv wordsLib.SIZES_ss ``FINITE univ(:32)``;
> val it = |- FINITE univ(:32) <=> T : thm
\end{verbatim}
\end{session}
Il frammento \ml{BIT\_ss} converte \holtxt{BIT} in un test di appartenenza su un insieme di posizioni bit (pi� alte):
\begin{session}
\begin{verbatim}
- conv wordsLib.BIT_ss ``BIT 3 5``;
> val it = |- BIT 3 5 <=> (3 = 0) \/ (3 = 2) : thm

- conv wordsLib.BIT_ss ``BIT i 123``;
> val it = |- BIT i 123 <=> i IN {0; 1; 3; 4; 5; 6} :
  thm
\end{verbatim}
\end{session}
Questa semplificazione fornisce supporto per il ragionamento circa le operazioni bitwise su lunghezze word arbitrarie. I frammenti aritmetico, logico e shift aiutano a ripulire espressioni word di base:
\begin{session}
\begin{verbatim}
- conv wordsLib.WORD_LOGIC_ss ``a && 12w || 11w && a``;
<<HOL message: inventing new type variable names: 'a>>
> val it =
    |- a && 12w || 11w && a = 15w && a :
  thm

- conv wordsLib.WORD_ARITH_ss ``3w * b + a + 2w * b - a * 4w:word2``;
> val it =
    |- 3w * b + a + 2w * b - a * 4w = a + b
     : thm

- conv wordsLib.WORD_SHIFT_ss ``0w << 12 + a >>> 0 + b << 2 << 3``;
<<HOL message: inventing new type variable names: 'a>>
> val it =
    |- 0w << 12 + a >>> 0 + b << 2 << 3 = 0w + a + b << (2 + 3)
     : thm
\end{verbatim}
\end{session}

I frammenti rimanenti non sono inclusi in \ml{wordsLib.WORD\_ss} or \ml{srw\_ss}. Il frammento eguaglianza bit � mostrato di sotto.
\begin{session}
\begin{verbatim}
- SIMP_CONV (std_ss++wordsLib.WORD_BIT_EQ_ss) [] ``a && b = ~0w : word2``;
> val it =
    |- (a && b = ~0w) <=> (a ' 1 /\ b ' 1) /\ a ' 0 /\ b ' 0
     : thm
\end{verbatim}
\end{session}
Il frammento esatto � utile per il ragionamento circa operazioni bit-field ed � usato meglio in combinazione con \ml{wordsLib.SIZES\_ss} o \ml{wordsLib.WORD\_ss}, per esempio:
\begin{session}
\begin{verbatim}
- SIMP_CONV (std_ss++wordsLib.SIZES_ss++wordsLib.WORD_EXTRACT_ss) []
   ``(4 -- 1) ((a:word3) @@ (b:word2)) : word5``;
> val it =
    |- (4 -- 1) (a @@ b) = (2 >< 0) a << 1 || (1 >< 1) b
     : thm
\end{verbatim}
\end{session}
Infine, il frammento \ml{WORD\_MUL\_LSL\_ss} � mostrato di sotto.
\begin{session}
\begin{verbatim}
- conv wordsLib.WORD_MUL_LSL_ss ``5w * a : word8``;
> val it = |- 5w * a = a << 2 + a : thm
\end{verbatim}
\end{session}
Riscrivere con il teorema \ml{wordsTheory.WORD\_MUL\_LSL} fornisce un mezzo per annullare questa semplificazione, per esempio:
\begin{session}
\begin{verbatim}
- SIMP_CONV (std_ss++wordsLib.WORD_ARITH_ss) [wordsTheory.WORD_MUL_LSL]
    ``a << 2 + a : word8``;
> val it = |- a << 2 + a = 5w * a : thm
\end{verbatim}
\end{session}
Ovviamente, senza aggiungere delle garanzie, questo teorema di riscrittura non pu� essere dispiegato in combinazione con il frammento \ml{WORD\_MUL\_LSL\_ss}.

\subsubsection{Procedure di decisione}

Una procedura di decisione per i word � fornita nella forma di
\ml{blastLib.BBLAST\_PROVE}. Questa procedura usa il \emph{bit-blasting} ---
convertire espressioni word in proposizioni e poi usare il risolutore SAT per
decidere il goal\footnote{Questo approccio permette di dare contro-esempi
quando la negazione di un goal � insoddisfacibile.}. Questo approccio � ragionevolmente generale e
pu� affrontare un ampia gamma di problemi bit-vector. Tuttavia, ci sono alcune
limitazioni: l'approccio funziona solo per lunghezze word costanti, artimetica
lineare (moltiplicazione per letterali) e per shift e estrazioni
bit-field rispetto a valori letterali. Si noti inoltre che alcuni problemi saranno
potenzialmente lenti da dimostrare, ad esempio quando le dimensioni dei word sono grandi e/o quando
ci sono molte addizioni annidate (magari attraverso la moltiplicazione).


I seguenti esempi mostrano \ml{BBLAST\_PROVE} in uso:
\begin{session}
\begin{verbatim}
- blastLib.BBLAST_PROVE ``a + 2w <+ 4w = a <+ 2w \/ 13w <+ a :word4``;
> val it =
    |- a + 2w <+ 4w <=> a <+ 2w \/ 13w <+ a
     : thm

- blastLib.BBLAST_PROVE ``w2w (a:word8) <+ 256w : word16``;
> val it = |- w2w a <+ 256w : thm
\end{verbatim}
\end{session}
La procedura di decisione \ml{BBLAST\_PROVE} � basata sulla conversione
\ml{BBLAST\_CONV}. Questa conversione pu� essere usata per convertire problemi bit-vector
in una forma proposizionale; per esempio:
\begin{session}
\begin{verbatim}
- blastLib.BBLAST_CONV ``(((a : word16) + 5w) << 3) ' 5``;
> val it =
   |- ((a + 5w) << 3) ' 5 <=> (~a ' 2 <=> ~(a ' 1 /\ a ' 0))
   : thm
\end{verbatim}
\end{session}
Ci sono anche tattiche bit-blasting: \ml{BBLAST\_TAC} e \ml{FULL\_BBLAST\_TAC}; dove solo la seconda fa uso delle assunzioni del goal.

\section{La libreria \texttt{HolSat}}\label{sec:HolSatLib}
\input{HolSat.tex}


\section{La libreria \texttt{HolQbf}}\label{sec:HolQbfLib}
\input{HolQbf.tex}


\section{La libreria \texttt{HolSmt}}\label{sec:HolSmtLib}
\index{HolSmtLib|(}
\index{SMT solvers|see {HolSmtLib}}
\index{decision procedures!SMT}

\setcounter{sessioncount}{0}

The purpose of \ml{HolSmtLib} is to provide a platform for
experimenting with combinations of interactive theorem proving and
Satisfiability Modulo Theories~(SMT) solvers.  \ml{HolSmtLib} was
developed as part of a research project on {\it Expressive
  Multi-theory Reasoning for Interactive Verification} (EPSRC grant
EP/F067909/1) from 2008 to~2011.  It is loosely inspired by
\ml{HolSatLib} (Section~\ref{sec:HolSatLib}), and has been described
in parts in the following publications:
\begin{itemize}
\item Tjark Weber: {\it SMT Solvers: New Oracles for the HOL Theorem
  Prover}.  To appear in International Journal on Software Tools for
  Technology Transfer (STTT), 2011.
\item Sascha B{\"o}hme, Tjark Weber: {\it Fast LCF-Style Proof
  Reconstruction for Z3}.  In Matt Kaufmann and Lawrence C.\ Paulson,
  editors, Interactive Theorem Proving, First International
  Conference, ITP 2010, Edinburgh, UK, July 11--14, 2010.
  Proceedings, volume 6172 of Lecture Notes in Computer Science, pages
  179--194.  Springer, 2010.
\end{itemize}
\ml{HolSmtLib} uses external SMT solvers to prove instances of SMT
tautologies, \ie, formulas that are provable using (a combination of)
propositional logic, equality reasoning, linear arithmetic on integers
and reals, and decision procedures for bit vectors and arrays.  The
supported fragment of higher-order logic varies with the SMT solver
used, and is discussed in more detail below.  At least for Yices, it
is a superset of the fragment supported by \ml{bossLib.DECIDE} (and
the performance of \ml{HolSmtLib}, especially on big problems, should
be much better).

\subsection{Interface}

The library currently provides four tactics to invoke different SMT
solvers, namely \ml{CVC\_ORACLE\_TAC}, \ml{YICES\_TAC},
\ml{Z3\_ORACLE\_TAC}, and \ml{Z3\_TAC}.  These tactics are defined in
the \ml{HolSmtLib} structure, which is the library's main entry point.
Given a goal~$(\Gamma, \varphi)$ (where $\Gamma$ is a list of
assumptions, and $\varphi$ is the goal's conclusion), each tactic
returns (i)~an empty list of new goals, and (ii)~a validation function
that returns a theorem~$\Gamma' \vdash \varphi$ (with $\Gamma'
\subseteq \Gamma$). These tactics fail if the SMT solver cannot prove
the goal.\footnote{Internally, the goal's assumptions and the
\emph{negated} conclusion are passed to the SMT solver.  If the SMT
solver determines that these formulas are unsatisfiable, then the
(unnegated) conclusion must be provable from the assumptions.}  In
other words, these tactics solve the goal (or fail).  As with other
tactics, \ml{Tactical.TAC\_PROOF} can be used to derive functions of
type \ml{goal -> thm}.

For each tactic, the \ml{HolSmtLib} structure additionally provides a
corresponding function of type \ml{term -> thm}.  These functions are
called \ml{CVC\_ORACLE\_PROVE}, \ml{YICES\_PROVE},
\ml{Z3\_ORACLE\_PROVE}, and \ml{Z3\_PROVE}, respectively.  Applied to
a formula $\varphi$, they return the theorem $\emptyset \vdash
\varphi$ (or fail).

Furthermore, the \ml{HolSmtLib} structure provides three more tactics,
namely \ml{cvco\_tac}, \ml{z3o\_tac} and \ml{z3\_tac}. These tactics
are equivalent to \ml{CVC\_ORACLE\_TAC}, \ml{Z3\_ORACLE\_TAC} and
\ml{Z3\_TAC} (respectively), but additionally take a list of theorems
which are used as lemmas in the proof, similarly to \ml{METIS\_TAC}.

\paragraph{Oracles vs.\ proof reconstruction}

\ml{CVC\_ORACLE\_TAC}, \ml{YICES\_TAC}, \ml{Z3\_ORACLE\_TAC},
\ml{cvco\_tac} and \ml{z3o\_tac} use the SMT solver (cvc5, Yices, Z3,
cvc5 and Z3, respectively) as an oracle: the solver's result is
trusted. Bugs in the SMT solver or in \ml{HolSmtLib} could potentially
lead to inconsistent theorems. Accordingly, the derived theorem is
tagged with an oracle tag.

\ml{Z3\_TAC} and \ml{z3\_tac}, on the other hand, perform proof
reconstruction.  They request a detailed proof from Z3, which is then
checked in \HOL{}. One obtains a proper \HOL{} theorem; no
(additional) oracle tags are introduced. However, Z3's proofs do not
always contain enough information to allow efficient checking in
\HOL{}; therefore, proof reconstruction may be slow or fail.

\paragraph{Supported subsets of higher-order logic}

\ml{YICES\_TAC} employs a translation into Yices's native input
format.  The interface supports types \holtxt{bool}, \holtxt{num},
\holtxt{int}, \holtxt{real}, \holtxt{->} (\ie, function types),
\holtxt{prod} (\ie, tuples), fixed-width word types, inductive data
types, records, and the following terms: equality, Boolean connectives
(\holtxt{T}, \holtxt{F}, \holtxt{==>}, \holtxt{/\bs}, \holtxt{\bs /},
negation, \holtxt{if-then-else}, \holtxt{bool-case}), quantifiers
(\holtxt{!}, \holtxt{?}), numeric literals, arithmetic operators
(\holtxt{SUC}, \holtxt{+}, \holtxt{-}, \holtxt{*}, \holtxt{/}, unary
minus, \holtxt{DIV}, \holtxt{MOD}, \holtxt{ABS}, \holtxt{MIN},
\holtxt{MAX}), comparison operators (\holtxt{<}, \holtxt{<=},
\holtxt{>}, \holtxt{>=}, both on \holtxt{num}, \holtxt{int}, and
\holtxt{real}), function application, lambda abstraction, tuple
selectors \holtxt{FST} and \holtxt{SND}, and various word operations.

cvc5 and Z3 are integrated via a more restrictive translation into
SMT-LIB~2 format, described below.  Therefore, Yices is typically the
solver of choice at the moment (unless you need proof reconstruction,
which is available for Z3 only).  However, there are a few operations
(\eg, specific word operations) that are supported by the SMT-LIB
format, but not by Yices.  See \ml{selftest.sml} for further details.

Terms of higher-order logic that are not supported by the respective
target solver/\allowbreak translation are typically treated in one of
three ways:
\begin{enumerate}
\item Some unsupported terms are replaced by equivalent suppported
  terms during a pre-processing step.  For instance, all tactics first
  generalize the goal's conclusion by stripping outermost universal
  quantifiers, and attempt to eliminate certain set expressions by
  rewriting them into predicate applications: \eg, \holtxt{y IN \{x |
    P x\}} is replaced by \holtxt{P y}.  The resulting term is
  $\beta$-normalized.  Depending on the target solver, further
  simplifications are performed.
\item Remaining unsupported constants are treated as uninterpreted,
  \ie, replaced by fresh variables.  This should not affect soundness,
  but it may render goals unprovable and lead to spurious
  counterexamples.  To see all fresh variables introduced by the
  translation, you can set \ml{HolSmtLib}'s tracing variable (see
  below) to a sufficiently high value.
\item Various syntactic side conditions are currently not enforced by
  the translation and may result in invalid input to the SMT solver.
  For instance, Yices only allows \emph{linear} arithmetic (\ie,
  multiplication by constants) and word-shifts by numeric literals
  (constants).  If the goal is outside the allowed syntactic fragment,
  the SMT solver will typically fail to decide the problem.
  \ml{HolSmtLib} at present only provides a generic error message in
  this case.  Inspecting the SMT solver's output might provide further
  hints.
\end{enumerate}

\begin{session}
\begin{verbatim}
- load "HolSmtLib"; open HolSmtLib;
(* output omitted *)
> val it = () : unit

- show_tags := true;
> val it = () : unit

- CVC_ORACLE_PROVE ``(a ==> b) /\ (b ==> a) <=> (a <=> b)``;
> val it = [oracles: DISK_THM, HolSmtLib] [axioms: ] []
           |- (a ==> b) /\ (b ==> a) <=> (a <=> b) : thm

- YICES_PROVE ``(a ==> b) /\ (b ==> a) <=> (a <=> b)``;
> val it = [oracles: DISK_THM, HolSmtLib] [axioms: ] []
           |- (a ==> b) /\ (b ==> a) <=> (a <=> b) : thm

- Z3_ORACLE_PROVE ``(a ==> b) /\ (b ==> a) <=> (a <=> b)``;
> val it = [oracles: DISK_THM, HolSmtLib] [axioms: ] []
           |- (a ==> b) /\ (b ==> a) <=> (a <=> b) : thm

- Z3_PROVE ``(a ==> b) /\ (b ==> a) <=> (a <=> b)``;
> val it = [oracles: DISK_THM] [axioms: ] []
           |- (a ==> b) /\ (b ==> a) <=> (a <=> b) : thm
\end{verbatim}
\end{session}

\paragraph{Support for the SMT-LIB 2 file format}

SMT-LIB (see \url{https://smtlib.cs.uiowa.edu/}) is the standard input
format for SMT solvers.  \ml{HolSmtLib} supports (a subset of)
version~2.0 of this format.  A translation of \HOL{} terms into
SMT-LIB~2 format is available in \ml{SmtLib.sml}, and a parser for
SMT-LIB~2 files (translating them into \HOL{} types, terms, and
formulas) can be found in \ml{SmtLib\_Parser.sml}, with auxiliary
functions in \ml{SmtLib\_\{Logics,Theories\}.sml}.

The SMT-LIB~2 translation supports types \holtxt{bool}, \holtxt{num},
\holtxt{int} and \holtxt{real}, fixed-width word types, and the
following terms: equality, Boolean connectives, quantifiers, numeric
literals, arithmetic operators, comparison operators, function
application, and various word operations.  Notably, the SMT-LIB
interface does \emph{not} support data types or records, and
higher-order formulas.  See the files mentioned above and the examples
in \ml{selftest.sml} for further details.

\paragraph{Tracing}

Tracing output can be controlled via \ml{Feedback.set\_trace
  "HolSmtLib"}.  See the source code in \ml{Library.sml} for
possible values.

Communication between \HOL{} and external SMT solvers is via temporary
files.  These files are located in the standard temporary directory,
typically {\tt /tmp} on Unix machines.  The actual file names are
generated at run-time, and can be shown by setting the above tracing
variable to a sufficiently high value.

The default behavior of \ml{HolSmtLib} is to delete temporary files
after successful invocation of the SMT solver.  This also can be
changed via the above tracing variable.  If there is an error, files
are retained in any case (but note that the operating system may
delete temporary files automatically, \eg, when \HOL{} exits).

\subsection{Installing SMT solvers}

\ml{HolSmtLib} has been tested with cvc5~1.3.0, Yices~1.0.40, and
Z3~4.15.3. Later versions may or may not work.  (Yices~2 is not
supported.)  To use \ml{HolSmtLib}, you need to install at least one
of these SMT solvers on your machine.  As mentioned before, Yices
supports a larger fragment of higher-order logic than Z3, but proof
reconstruction has been implemented only for Z3.

cvc5 is available for various platforms from
\url{https://cvc5.github.io/}. After installation, you must set the
environment variable {\tt \$HOL4\_CVC\_EXECUTABLE} to the pathname of
the cvc5 executable, \eg, {\tt /usr/bin/cvc5}, before you invoke \HOL.

Yices is available for various platforms from
\url{https://yices.csl.sri.com/}.  After installation, you must set
the environment variable {\tt \$HOL4\_YICES\_EXECUTABLE} to the
pathname of the Yices executable, \eg, {\tt /bin/yices}, before you
invoke \HOL.

Z3 is available for various platforms from
\url{https://github.com/Z3Prover/z3}. After installation, you must set
the environment variable {\tt \$HOL4\_Z3\_EXECUTABLE} to the pathname
of the Z3 executable, \eg, {\tt /bin/z3}, before you invoke \HOL.

It should be relatively straightforward to integrate other SMT solvers
that support the SMT-LIB~2 input format as oracles.  However, this
will involve a (typically small) amount of Standard ML programming,
\eg, to interpret the solver's output.  See \ml{CVC.sml} and
\ml{Z3.sml} for some relevant code.

\subsection{Wishlist}

The following features have not been implemented yet.  Please submit
additional feature requests (or code contributions) via
\url{https://github.com/HOL-Theorem-Prover/HOL}.

\paragraph{Counterexamples}

For satisfiable input formulas, SMT solvers typically return a
satisfying assignment.  This assignment could be displayed to the
\HOL{} user as a counterexample.  It could also be turned into a
theorem, similar to the way \ml{HolSatLib} treats satisfying
assignments.

\paragraph{Proof reconstruction}

Several other SMT solvers can also produce proofs, and it would be
nice to offer \HOL{} users more choice.  However, in the absence of a
standard proof format for SMT solvers, it is perhaps not worth the
implementation effort.

\paragraph{Support for cvc5's and Z3's SMT-LIB extensions}

cvc5 and Z3 support extensions of the SMT-LIB language, \eg, data
types, sets, multisets/bags and strings. \ml{HolSmtLib} does not
utilize these extensions yet when calling these SMT solvers. This
would require the translation for these solvers to be distinct from
the generic SMT-LIB translation.

\paragraph{SMT solvers as a web service}

The need to install an SMT solver locally poses an entry barrier.  It
would be much more convenient to have a web server running one (or
several) SMT solvers, roughly similar to the ``System on TPTP''
interface that G.~Sutcliffe provides for first-order theorem provers
(\url{http://www.cs.miami.edu/~tptp/cgi-bin/SystemOnTPTP}).  For
Isabelle/HOL, such a web service has been installed by S.~B{\"o}hme in
Munich, but unfortunately it is not publicly available.  Perhaps the
SMT-EXEC initiative (\url{http://www.smtexec.org/}) could offer
hardware or implementation support.

\index{HolSmtLib|)}

%%% Local Variables:
%%% mode: latex
%%% TeX-master: "description"
%%% End:


\section{La libreria \texttt{Quantifier Heuristics}}\label{sec:QuantHeuristicsLib}
\index{quantHeuristicsLib|(}
\index{Quantifier Instantiation|see {quantHeuristicsLib}}

\setcounter{sessioncount}{0}

\subsection{Motivation}

Often interactive proofs can be simplified by instantiating
quantifiers. The \ml{Unwind} library\index{Unwind}, which is part of the simplifier, allows
instantiations of ``trivial'' quantifiers:

\[ \forall x_1\ \ldots x_i \ldots x_n.\ P_1 \wedge \ldots \wedge x_i = c \wedge \ldots \wedge P_n \Longrightarrow Q \]
and
\[ \exists x_1\ \ldots x_i \ldots x_n.\ P_1 \wedge \ldots \wedge x_i =
c \wedge \ldots \wedge P_n \] can be simplified by
instantiating $x_i$ with $c$. Because unwind-conversions are
part of \holtxt{bool\_ss}, they are used with nearly every call of the simplifier
and often simplify proofs considerably. However, the \ml{Unwind} library can only handle these common cases. If the term structure is
only slightly more complicated, it fails. For example, $\exists x.\ P(x) \Longrightarrow (x = 2) \wedge Q(x)$
cannot be tackled.

There is also the \ml{Satisfy} library\index{Satisfy}, which uses
unification to show existentially quantified formulas. It can handle
problems like $\exists x.\ P_1(x,c_1)\ \wedge \ldots P_n(x,c_n)$ if
given theorems of the form $\forall x\ c.\ P_i(x, c)$. This is often
handy, but still rather limited.

The quantifier heuristics library (\ml{quantHeuristicsLib}) provides more power
and flexibility. A few simple examples of what it can do
are shown in Table~\ref{table-qh-examples}. Besides the power demonstrated
by these examples, the library is highly flexible as well.  At its
core, there is a modular, syntax driven search for instantiation.
This search consists of a collection of interleaved heuristics.  Users
can easily configure existing heuristics and add own ones. Thereby, it
is easy to teach the library about new predicates, logical connectives
or datatypes.

\newcommand{\mytablehead}[1]{\\\multicolumn{2}{l}{\textit{#1}}\\}
\begin{table}[h]
\centering %\scriptsize (too small, no need to save page spaces here)
\begin{tabular}{lll}
\textbf{Problem} & \textbf{Result} \\\hline

\mytablehead{basic examples}
$\exists x.\ x = 2 \wedge P (x)$ & $P(2)$ \\
$\forall x.\ x = 2 \Longrightarrow P (x)$ & $P(2)$ \\

\mytablehead{solutions and counterexamples}
$\exists x.\ x = 2$ & $\textit{true}$ \\
$\forall x.\ x = 2$ & $\textit{false}$ \\

\mytablehead{complicated nestings of standard operators}
$\exists x_1. \forall x_2.\ (x_1 = 2) \wedge P(x_1, x_2)$ &
$\forall x_2.\ P(2, x_2)$ \\

$\exists x_1, x_2.\ P_1(x_2) \Longrightarrow (x_1 = 2) \wedge P(x_1, x_2)$ &
$\exists x_2.\ P_1(x_2) \Longrightarrow P(2, x_2)$ \\
$\exists x.\ ((x = 2) \vee (2 = x)) \wedge P(x)$ & $P(2)$ \\

\mytablehead{exploiting unification}
$\exists x.\ (f (8 + 2) = f (x + 2)) \wedge P (f(10))$ & $P (f(10))$ \\
$\exists x.\ (f (8 + 2) = f (x + 2)) \wedge P (f(x + 2))$ & $P (f(8 + 2))$ \\
$\exists x.\ (f (8 + 2) = f (x + 2)) \wedge P (f(x))$ & - (\textrm{no instantiation found}) \\

\mytablehead{partial instantiation for datatypes}
$\forall p.\ c = \textsf{FST}(p) \Longrightarrow P(p)$ & $\forall p_2.\ P(c, p_2)$ \\

$\forall x.\ \textsf{IS\_NONE}(x) \vee P(x)$ & $\forall x'.\ P (\textsf{SOME}(x'))$ \\

$\forall l.\ l \neq [\,] \Longrightarrow P(l)$ & $\forall \textit{hd}, \textit{tl}.\
P(\textit{hd} :: \textit{tl})$ \\

\mytablehead{context}
$P_1(c) \Longrightarrow \exists x.\ P_1(x) \vee P_2(x)$ & \textit{true} \\
$P_1(c) \Longrightarrow \forall x.\ \neg P_1(x) \wedge P_2(x)$ & $\neg P_1(c)$ \\

$(\forall x.\ P_1(x) \Rightarrow (x = 2)) \Longrightarrow (\forall x.\ P_1(x) \Rightarrow P_2(x))$ &
$(\forall x.\ P_1(x) \Rightarrow (x = 2)) \Rightarrow (P_1(2) \Rightarrow P_2(2))$ \\

$\big((\forall x.\ P_1(x) \Rightarrow P_2(x)) \wedge P_1(2)\big) \Longrightarrow \exists x.\ P_2(x)$ &
\textit{true} \\
\hline
\end{tabular}
\caption{Examples}
\label{table-qh-examples}
\end{table}

\subsection{User Interface}\label{sec:qh-interface}

The quantifier heuristics library can be found in the sub-directory
\holtxt{src/quantHeuristics}.  The entry point to the framework is the
library \holtxt{quantHeuristicsLib}.

\subsubsection{Conversions}
Usually the library is used for
converting a term containing quantifiers to an equivalent one. For this,
the following high level entry points exists:
\bigskip

\noindent
\begin{tabular}{@{}ll}
\texttt{QUANT\_INSTANTIATE\_CONV} & \texttt{: quant\_param list -> conv} \\
\texttt{QUANT\_INST\_ss} & \texttt{: quant\_param list -> ssfrag} \\
\texttt{QUANT\_INSTANTIATE\_TAC} & \texttt{: quant\_param list -> tactic} \\
\texttt{ASM\_QUANT\_INSTANTIATE\_TAC} & \texttt{: quant\_param list -> tactic}
\end{tabular}
\bigskip

All these functions get a list of \emph{quantifier heuristic parameters} as arguments. These
parameters essentially configure, which heuristics are used during the guess-search. If
an empty list is provided, the tools know about the standard Boolean combinators, equations and context.
\texttt{std\_qp} adds support for common datatypes like pairs or lists.
Quantifier heuristic parameters are explained in more detail in
Section~\ref{quantHeu-subsec-qps}.

So, some simple usage of the quantifier heuristic library looks like:

\begin{session}
\begin{verbatim}
- QUANT_INSTANTIATE_CONV [] ``?x. (!z. Q z /\ (x=7)) /\ P x``;
> val it = |- (?x. (!z. Q z /\ (x = 7)) /\ P x) <=> (!z. Q z) /\ P 7: thm

- QUANT_INSTANTIATE_CONV [std_qp] ``!x. IS_SOME x ==> P x``
> val it = |- (!x. IS_SOME x ==> P x) <=> !x_x'. P (SOME x_x'): thm
\end{verbatim}
\end{session}

Usually, the quantifier heuristics library is used together with the
simplifier using \holtxt{QUANT\_INST\_ss}. Besides interleaving
simplification and quantifier instantiation, this has the benefit of
being able to use context information collected by the simplifier:

\begin{session}
\begin{verbatim}
- QUANT_INSTANTIATE_CONV [] ``P m ==> ?n. P n``
Exception- UNCHANGED raised

- SIMP_CONV (bool_ss ++ QUANT_INST_ss []) [] ``P m ==> ?n. P n``
> val it = |- P m ==> (?n. P n) <=> T: thm
\end{verbatim}
\end{session}

It's usually best to use \holtxt{QUANT\_INST\_ss}
together with e.\,g.\ \holtxt{SIMP\_TAC} when using the library with tactics.
However, if free variables of the goal should be instantiated, then
\holtxt{ASM\_QUANT\_INSTANTIATE\_TAC} should be used:

\begin{session}
\begin{verbatim}
P x
------------------------------------
  IS_SOME x
  : proof

- e (ASM_QUANT_INSTANTIATE_TAC [std_qp])
> P (SOME x_x') : proof
\end{verbatim}
\end{session}

There is also \holtxt{QUANT\_INSTANTIATE\_TAC}. This tactic does not
instantiate free variables. Neither does it take assumptions into consideration.
It is just a shortcut for using \holtxt{QUANT\_INSTANTIATE\_CONV} as a tactic.


\subsubsection{Unjustified Guesses}

Most heuristics justify the guesses they produce and therefore allow to
prove equivalences of e.\,g.\ the form $\exists x.\ P(x) \Leftrightarrow P(i)$.
However, the implementation also supports unjustified guesses, which may be bogus.
Let's consider e.\,g.\ the formula $\exists x.\ P(x) \Longrightarrow (x = 2)\ \wedge\ Q(x)$.
Because nothing is known about $P$ and $Q$, we can't find a safe instantiation for $x$ here.
However, $2$ looks tempting and is probably sensible in many situations. (Counterexample:
$P(2)$, $\neg Q(2)$ and $\neg P(3)$ hold)

\texttt{implication\_concl\_qp} is a quantifier parameter that looks for valid guesses in the conclusion of an implication.
Then, it assumes without justification that these guesses are probably sensible for the whole implication as well.
Because these guesses might be wrong, one can either use implications or
expansion theorems like $\exists x.\ P(x)\ \Longleftrightarrow (\forall x.\ x \neg c \Rightarrow \neg P(x)) \Rightarrow P(c)$.

\begin{session}
\begin{verbatim}
- QUANT_INSTANTIATE_CONV [implication_concl_qp]
     ``?x. P x ==> (x = 2) /\ Q x``
Exception- UNCHANGED raised

- QUANT_INSTANTIATE_CONSEQ_CONV [implication_concl_qp]
     CONSEQ_CONV_STRENGTHEN_direction
     ``?x. P x ==> (x = 2) /\ Q x``
> val it =
   |- (P 2 ==> Q 2) ==> ?x. P x ==> (x = 2) /\ Q x: thm

- EXPAND_QUANT_INSTANTIATE_CONV [implication_concl_qp]
    ``?x. P x ==> (x = 2) /\ Q x``
> val it = |- (?x. P x ==> (x = 2) /\ Q x) <=>
              (!x. x <> 2 ==> ~(P x ==> (x = 2) /\ Q 2)) ==> P 2 ==> Q 2

- SIMP_CONV (std_ss++EXPAND_QUANT_INST_ss [implication_concl_qp]) []
    ``?x. P x ==> (x = 2) /\ Q x``
> val it =
   |- (?x. P x ==> (x = 2) /\ Q x) <=>
      (!x. x <> 2 ==> P x) ==> P 2 ==> Q 2: thm
\end{verbatim}
\end{session}

The following entry points should be used to exploit unjustified guesses:
\bigskip

\noindent
\begin{tabular}{@{}ll}
\texttt{QUANT\_INSTANTIATE\_CONSEQ\_CONV} & \texttt{: quant\_param list -> directed\_conseq\_conv} \\
\texttt{EXPAND\_QUANT\_INSTANTIATE\_CONV} & \texttt{: quant\_param list -> conv} \\
\texttt{EXPAND\_QUANT\_INST\_ss} & \texttt{: quant\_param list -> ssfrag} \\
\texttt{QUANT\_INSTANTIATE\_CONSEQ\_TAC} & \texttt{: quant\_param list -> tactic}
\end{tabular}


\subsubsection{Explicit Instantiations}

A special (degenerated) use of the framework, is turning guess search off completely and
providing instantiations explicitly. The tactic \holtxt{QUANT\_TAC} allows this. This means that
it allows to partially instantiate quantifiers at subpositions
with explicitly given terms. As such, it can be seen as
a generalisation of \holtxt{EXISTS\_TAC}\index{EXISTS\_TAC}.
%
\begin{session}
\begin{verbatim}
- val it = !x. (!z. P x z) ==> ?a b.    Q a        b z : proof

> e( QUANT_INST_TAC [("z", `0`, []), ("a", `SUC a'`, [`a'`])] )
- val it = !x. (    P x 0) ==> ?  b a'. Q (SUC a') b z : proof
\end{verbatim}
\end{session}
%
This tactic is implemented using unjustified guesses. It normally
produces implications, which is fine when used as a tactic. There is
also a conversion called \holtxt{INST\_QUANT\_CONV} with the same
functionality. For a conversion, implications are
problematic. Therefore, the simplifier and Metis are used to prove
the validity of the explicitly given instantiations. This succeeds
only for simple examples.

\subsection{Simple Quantifier Heuristics}

The full quantifier heuristics described above are powerful and very flexible.
However, they are sometimes slow.
The unwind library\footnote{see \texttt{src/simp/src/Unwind.sml}} on the other hand is limited, but fast.
The simple version of the quantifier heuristics fills the gap in the middle.
They just search for gap guesses without any free variables.
Moreover, slow operations like recombining or automatically looking up datatype information is omitted.
As a result, the conversion \texttt{SIMPLE\_QUANT\_INSTANTIATE\_CONV} (and corresponding \texttt{SIMPLE\_QUANT\_INST\_ss}) is nearly as fast as the corresponding unwind conversions.
However, it supports more complicated syntax. Moreover, there is support for quantifiers, pairs, list and much more. 

\subsection{Quantifier Heuristic Parameters}\label{quantHeu-subsec-qps}

Quantifier heuristic parameters play a similar role for the quantifier
instantiation library as simpsets do for the simplifier. They contain
theorems, ML code and general configuration parameters that allow to configure
guess-search. There are predefined parameters that handle
common constructs and the user can define own parameters.

\subsubsection{Quantifier Heuristic Parameters for Common Datatypes}

There are \holtxt{option\_qp}, \holtxt{list\_qp}, \holtxt{num\_qp} and \holtxt{sum\_qp} for option types, lists,
natural numbers and sum types respectively.
Some examples are displayed in the following table:
%
\[\begin{array}{r@{\quad \Longleftrightarrow \quad}l}
\forall x.\ \holtxt{IS\_SOME}(x) \Rightarrow P(x) & \forall x'.\ P (\holtxt{SOME}(x')) \\
\forall x.\ \holtxt{IS\_NONE}(x)& \textit{false} \\
\forall l.\ l \neq [\,] \Rightarrow P(l)& \forall h, l'.\ P(h::l')  \\
\forall x.\ x = c + 3& \textit{false} \\
\forall x.\ x \neq 0 \Rightarrow P(x)& \forall x'.\ P(\holtxt{SUC}(x'))
\end{array}\]

\subsubsection{Quantifier Heuristic Parameters for Tuples}

For tuples the situation is peculiar, because each quantifier over a variable of a product type
can be instantiated. The challenge is to decide which quantifiers should be instantiated and
which new variable names to use for the components of the pair.
There is a quantifier heuristic parameter called \holtxt{pair\_default\_qp}. It first looks for
subterms of the form $(\lambda (x_1, \ldots, x_n).\ \ldots)\ x$. If such a term is found $x$ is instantiated with
$(x_1, \ldots, x_n)$. Otherwise, subterms of the form $\holtxt{FST}(x)$ and $\holtxt{SND}(x)$ are searched. If such a term
is found, $x$ is instantiated as well. This parameter therefore allows simplifications like:
%
\[\begin{array}{r@{\quad \Longleftrightarrow \quad}l}
\forall p.\ (x = \holtxt{SND}(p)) \Rightarrow P(p)& \forall p_1.\ P(p_1, x) \\
\exists p.\ (\lambda (p_a, p_b, p_c). P(p_a, p_b, p_c))\ p & \exists p_a, p_b, p_c.\ P(p_a, p_b, p_c)
\end{array}\]

\holtxt{pair\_default\_qp} is implemented in terms of the more general
quantifier heuristic parameter \holtxt{pair\_qp}, which allows the
user to provide a list of ML functions. These functions get the
variable and the term. If they return a tuple of variables, these
variables are used for the instantiation, otherwise the next function
in the list is called or - if there is no function left - the variable
is not instantiated. In the example of $\exists p.\ (\lambda (p_a,
p_b, p_c). P(p_a, p_b, p_c))\ p$ these functions are given the
variable $p$ and the term $(\lambda (p_a, p_b, p_c). P(p_a, p_b,
p_c))\ p$ and return $\holtxt{SOME} (p_a, p_b, p_c)$.  This simple
ML-interface gives the user full control over what quantifier over
product types to expand and how to name the new variables.

\subsubsection{Quantifier Heuristic Parameter for Records}

Records are similar to pairs, because they can always be instantiated. Here, it is interesting that the necessary
monochotomy lemma comes from HOL~4's \holtxt{Type\_Base} library. This means that \holtxt{record\_qp} is stateful.
If a new record type is defined, the automatically proven monochotomy lemma is then automatically used
by \holtxt{record\_qp}. In contrast to the pair parameter, the one for records gets only one function instead of a
list of functions to decide which variables to instantiate. However, this function is simpler, because it just needs
to return true or false. The names of the new variables are constructed from the field-names of the record.
The quantifier heuristic parameter \holtxt{default\_record\_qp} expands all records.

\subsubsection{Stateful Quantifier Heuristic Parameters}

The parameter for records is stateful, as it uses knowledge from
\holtxt{Type\_Base}. Such information is not only useful for records
but for general datatypes. The quantifier heuristic parameter
\holtxt{TypeBase\_qp} uses automatically proven theorems about new
datatypes to exploit mono- and dichotomies. Moreover, there is also a
stateful \holtxt{pure\_stateful\_qp} that allows the user to
explicitly add other parameters to it.  \holtxt{stateful\_qp} is a
combination of \holtxt{pure\_stateful\_qp} and \holtxt{TypeBase\_qp}.

\subsubsection{Standard Quantifier Heuristic Parameter}

The standard quantifier heuristic parameter \holtxt{std\_qp} combines
the parameters for lists, options, natural numbers, the default one
for pairs and the default one for records.


\subsection{User defined Quantifier Heuristic Parameters}\label{sec:qps-user}

The user is also able to define own parameters. There
is \holtxt{empty\_qp}, which does not contain any information. Several
parameters can be combined using
\holtxt{combine\_qps}. Together with the basic types of user defined
parameters that are explained below, these functions provide an
interface for user defined quantifier heuristic parameters.

\subsubsection{Rewrites / Conversions}

A very powerful, yet simple technique for teaching the guess search
about new constructs are rewrite rules. For example, the standard rules
for equations and basic logical operations
cannot generate guesses for the predicate \holtxt{IS\_SOME}. By
rewriting \holtxt{IS\_SOME(x)} to \holtxt{?x'. x =
SOME(x')}, however, these rules fire.

\holtxt{option\_qp} uses this rewrite to implement support for
\holtxt{IS\_SOME}. Similarly support for predicates like \holtxt{NULL} is
implemented using rewrites. Even adding
rewrites like $\textsf{append}(l_1, l_2) = [\,] \Longleftrightarrow (l_1 =
[\,]\ \wedge\ l_2 = [\,])$ for list-append turned out to be beneficial
in practice.
\bigskip

\holtxt{rewrite\_qp} allows to provide rewrites in the form of rewrite theorems.
For the example of \holtxt{IS\_SOME} this looks like:

\begin{session}
\begin{verbatim}
> val thm = QUANT_INSTANTIATE_CONV [] ``!x. IS_SOME x ==> P x``
Exception- UNCHANGED raised

> val IS_SOME_EXISTS = prove (``IS_SOME x = (?x'. x = SOME x')``,
   Cases_on `x` THEN SIMP_TAC std_ss []);
val IS_SOME_EXISTS = |- IS_SOME x <=> ?x'. x = SOME x': thm

> val thm = QUANT_INSTANTIATE_CONV [rewrite_qp[IS_SOME_EXISTS]]
    ``!x. IS_SOME x ==> P x``
val thm = |- (!x. IS_SOME x ==> P x) <=>
             !x'. IS_SOME (SOME x') ==> P (SOME x'): thm
\end{verbatim}
\end{session}

To clean up the result after instantiation, theorems used to rewrite the result after instantiation can be provided via
\holtxt{final\_rewrite\_qp}.
\begin{session}
\begin{verbatim}
> val thm = QUANT_INSTANTIATE_CONV [rewrite_qp[IS_SOME_EXISTS],
                                    final_rewrite_qp[option_CLAUSES]]
      ``!x. IS_SOME x ==> P x``
val thm = |- (!x. IS_SOME x ==> P x) <=> !x'. P (SOME x'): thm
\end{verbatim}
\end{session}

If rewrites are not enough, \holtxt{conv\_qp} can be used to add conversions:
\begin{session}
\begin{verbatim}
- val thm = QUANT_INSTANTIATE_CONV [] ``?x. (\y. y = 2) x``
Exception- UNCHANGED raised

- val thm = QUANT_INSTANTIATE_CONV [convs_qp[BETA_CONV]] ``?x. (\y. y = 2) x``
> val thm = |- (?x. (\y. y = 2) x) <=> T: thm
\end{verbatim}
\end{session}

\subsubsection{Strengthening / Weakening}

In rare cases, equivalences that can be used for rewrites are unavailable. There might be just implications that
can be used for strengthening or weakening. The function
\holtxt{imp\_qp} might be used to provide such implication.

\begin{session}
\begin{verbatim}
- val thm = QUANT_INSTANTIATE_CONV [list_qp] ``!l. 0 < LENGTH l ==> P l``
Exception- UNCHANGED raised

- val LENGTH_LESS_IMP = prove (``!l n. n < LENGTH l ==> l <> []``,
    Cases_on `l` THEN SIMP_TAC list_ss []);
> val LENGTH_LESS_IMP = |- !l n. n < LENGTH l ==> l <> []: thm

- val thm = QUANT_INSTANTIATE_CONV [imp_qp[LENGTH_LESS_IMP], list_qp]
    ``!l. 0 < LENGTH l ==> P l``
> val thm =
   |- (!l. 0 < LENGTH l ==> P l) <=>
      !l_t l_h. 0 < LENGTH (l_h::l_t) ==> P (l_h::l_t): thm

- val thm = SIMP_CONV (list_ss ++
              QUANT_INST_ss [imp_qp[LENGTH_LESS_IMP], list_qp]) []
              ``!l. SUC (SUC n) < LENGTH l ==> P l``
> val thm =
   |- (!l. SUC (SUC n) < LENGTH l ==> P l) <=>
      !l_h l_t_h l_t_t_t l_t_t_h. n < SUC (LENGTH l_t_t_t) ==>
                                  P (l_h::l_t_h::l_t_t_h::l_t_t_t): thm
\end{verbatim}
\end{session}


\subsubsection{Filtering}
Sometimes, one might want to avoid to instantiate certain quantifiers.
The function \holtxt{filter\_qp} allows to add ML-functions that filter the handled
quantifiers. These functions are given a variable $x$ and a term $P(x)$.
The tool only tries to instantiate $x$ in $P(x)$, if all filter functions
return \textit{true}.

\begin{session}
\begin{verbatim}
- val thm = QUANT_INSTANTIATE_CONV []
     ``?x y z. (x = 1) /\ (y = 2) /\ (z = 3) /\ P (x, y, z)``
> val thm = |- (?x y z. (x = 1) /\ (y = 2) /\ (z = 3) /\ P (x,y,z)) <=>
               P (1,2,3): thm

- val thm = QUANT_INSTANTIATE_CONV
     [filter_qp [fn v => fn t => (v = ``y:num``)]]
     ``?x y z. (x = 1) /\ (y = 2) /\ (z = 3) /\ P (x, y, z)``
> val thm = |- (?x y z. (x = 1) /\ (y = 2) /\ (z = 3) /\ P (x,y,z)) <=>
                ?x   z. (x = 1) /\            (z = 3) /\ P (x,2,z): thm
\end{verbatim}
\end{session}

\subsubsection{Satisfying and Contradicting Instantiations}

As the satisfy library demonstrates, it is often
useful to use unification and explicitly given theorems to
find instantiations. In addition to satisfying instantiations, the quantifier heuristics framework
is also able to use contradicting ones. The theorems used for finding instantiations usually come from
the context. However, \holtxt{instantiation\_qp} allows to add additional ones:

\begin{session}
\begin{verbatim}
> val thm = SIMP_CONV (std_ss++QUANT_INST_ss[]) []
    ``P n ==> ?m:num. n <= m /\ P m``
Exception- UNCHANGED raised

> val thm = SIMP_CONV (std_ss++
               QUANT_INST_ss[instantiation_qp[LESS_EQ_REFL]]) []
               ``P n ==> ?m:num. n <= m /\ P m``
> val thm = |- P n ==> ?m:num. n <= m /\ P m = T : thm
\end{verbatim}
\end{session}

\subsubsection{Di- and Monochotomies}

Dichotomies can be exploited for guess search.
\holtxt{distinct\_qp} provides an interface to add theorems
of the form $\forall x.\ c_1(x) \neq c_2(x)$.
\holtxt{cases\_qp} expects theorems of the form
$\forall x. \ (x = \exists \textit{fv}. c_1(\textit{fv}))\ \vee \ldots \vee (x = \exists \textit{fv}. c_n(\textit{fv}))$.
However, only theorems for $n = 2$ and $n = 1$ are used. All other cases are currently ignored.

\subsubsection{Oracle Guesses}

Sometimes, the user does not want to justify guesses. The tactic
\holtxt{QUANT\_TAC} is implemented using oracle guesses for example.
A simple interface to oracle guesses is provided by \holtxt{oracle\_qp}.
It expects a ML function that given a variable and a term returns
a pair of an instantiation and the free variables in this instantiation.

As an example, lets define a parameter that states that every list is non-empty:
\begin{verbatim}
   val dummy_list_qp = oracle_qp (fn v => fn t =>
     let
        val (v_name, v_list_ty) = dest_var v;
        val v_ty = listSyntax.dest_list_type v_list_ty;

        val x = mk_var (v_name ^ "_hd", v_ty);
        val xs = mk_var (v_name ^ "_tl", v_list_ty);
        val x_xs = listSyntax.mk_cons (x, xs)
     in
        SOME (x_xs, [x, xs])
     end)
\end{verbatim}

\noindent
Notice, that an option type is returned and that the function is
allowed to throw \holtxt{HOL\_ERR} exceptions.
With this definition, we get

\begin{session}
\begin{verbatim}
- NORE_QUANT_INSTANTIATE_CONSEQ_CONV [dummy_list_qp]
    CONSEQ_CONV_STRENGTHEN_direction ``?x:'a list y:'b. P (x, y)``
> val it = ?y x_hd x_tl. P (x_hd::x_tl,y)) ==> ?x y. P (x,y) : thm
\end{verbatim}
\end{session}

\subsubsection{Lifting Theorems}

The function \holtxt{inference\_qp} enables the
user to provide theorems that allow lifting guesses over
user defined connectives. As writing these lifting theorems requires
deep knowledge about guesses, it is not discussed here. Please have a
look at the detailed documentation of the quantifier heuristics library as
well as its sources. You might also want to contact
Thomas Tuerk (\url{tt291@cl.cam.ac.uk}).


\subsubsection{User defined Quantifier Heuristics}

At the lowest level, the tool searches guesses using ML-functions
called \emph{quantifier heuristics}. Slightly simplified, such a
quantifier heuristic gets a variable and a term and returns a set of
guesses for this variable and term. Heuristics allow full
flexibility. However, to write your own heuristics a lot of knowledge
about the ML-datastructures and auxiliary functions is
required. Therefore, no details are discussed here. Please have a look
at the source code and contact Thomas Tuerk
(\url{tt291@cl.cam.ac.uk}), if you have questions.
\holtxt{heuristics\_qp} and \holtxt{top\_heuristics\_qp} provide
interfaces to add user defined heuristics to a quantifier heuristics
parameter.

\index{quantHeuristicsLib|)}


%%% Local Variables:
%%% mode: latex
%%% TeX-master: "description"
%%% End:



%%% Local Variables:
%%% mode: latex
%%% TeX-master: "description"
%%% End:

%
\section{Learning HOL4 Tactics}

In the previous examples we have used Lassie to prove theorems in HOL4 in a
rigorous but still human readable format.
HOL4 itself supports a richer set of so-called tactics that are commonly used
when proving theorems.

We will next show how Lassie can be used to learn a first set of basic tactics
and explain some of their intricacies that a beginner may struggle with.
Instead of using function \lstinline{nltac} to parse textual descriptions into HOL4
tactics, we next use Lassie's proof-REPL \lstinline{nlexplain} to interactively
develop a first proof script by looking again at the proof of the gaussian sum
in \autoref{fig:gaussProof}.

Therefore we (re-)start the interactive proof with \lstinline{g `! n. sum n = (n * (n + 1)) DIV 2`}
and sending it to the REPL with \ekey{M-h M-r}.
The HOL4 REPL prints:
\begin{lstlisting}[frame=single, mathescape=true, deletekeywords={Proof}]
> > # # # val it =
   Proof manager status: 1 proof.
   1. Incomplete goalstack:
     Initial goal:
     $\forall$ n. sum n = n * (n + 1) DIV 2
   : proofs
\end{lstlisting}

As before the interactive proof shows us the current goal that we have to proof.
Next, we start the Lassie's proof-REPL with \lstinline{nlexplain()} and
sending it with \ekey{M-h M-r}.
Note that the REPL changes from \lstinline{>} to \lstinline{|>} to denote that
Lassie is capturing the input.
Instead of sending the full proofscript with \lstinline{nltac}, one can now send
each step separately and observe its output.
Sending the first step from the Lassie proof (\lstinline{Induction on 'n'.}) with
\ekey{M-h M-r} prints
%
\begin{lstlisting}[frame=single]
Induct_on ` n `
 >- (
 sum 0 = 0 * (0 + 1) DIV 2)
 >- (
  0.  sum n = n * (n + 1) DIV 2
 ------------------------------------
      sum (SUC n) = SUC n * (SUC n + 1) DIV 2)
\end{lstlisting}

This tells us that the HOL4 equivalent to Lassies ``Induction on 'n''' is the
tactic \lstinline{Induct_on `n`}.
The tactic takes a HOL4 expression, called a term as input and tries to perform
an induction on it using the underlying types induction scheme.

Next, we see a so-called tactic combinator that can be used to chain tactics
together.
In Lassie's proofs we relied on the ``.'' to do this for you, similar to
pen-and-paper proofs.
A HOL4 tactic proof however requires manually putting the tactics together.
Here we see \lstinline{>-} which combines a tactic (\lstinline{Induct_on})
with a tactic solving a single subgoal.

If we send the next step, \lstinline{use [sum_def] to simplify.}, from the Lassie proof, the REPL shows
%
\begin{lstlisting}[frame=single]
Induct_on ` n `
 >- (
 fs [ sum_def ])
 >- (
  0.  sum n = n * (n + 1) DIV 2
 ------------------------------------
      sum (SUC n) = SUC n * (SUC n + 1) DIV 2)
\end{lstlisting}

The tactic \lstinline{fs [ sum_def ]} has been used by Lassie to solve the first subgoal
of the proof, but the second subgoal remained unchanged.
If we want to also simplify the second goal, we have to send the Lassie step again,
leaving us with
%
\begin{lstlisting}[frame=single]
Induct_on ` n `
 >- (
 fs [ sum_def ])
 >- (
 fs [ sum_def ] >>
   0.  sum n = n * (n + 1) DIV 2
  ------------------------------------
       SUC n + n * (n + 1) DIV 2 = SUC n * (SUC n + 1) DIV 2)
\end{lstlisting}

Here, we also see another tactic combinator of HOL4, \lstinline{>>}, which can
alternatively be written as \lstinline{\\}.
Applying tactic \lstinline{t1 >> t2} applies tactic \lstinline{t2} to every
subgoal  generated by \lstinline{t1} and fails if \lstinline{t2} fails on one of
the subgoals.
Note that for all tactic combinators, their second argument can themselves be
tactics composed by applications of \lstinline{>>} and \lstinline{>-}.

We recommend sending each of steps of the Lassie proof separately to construct a
full proof script and getting an intuitive meaning of the tactics.
If a step should be undone one can send the command \lstinline{back.}, and
\lstinline{help.} prints a quick help message.
Sending \lstinline{exit.} ends the \lstinline{nlexplain} session.
The proof script generated by Lassie can then be copied and played with
interactively to get a feeling for the tactics themselves.
We recommend using the goal manager used before to start an interactive proof of
the closed form of the summation.
Tactics from the proof script can then be send separately or joined together by
marking them and pressing \ekey{M-h e}.
%
\section{Small Tips and Tricks}\label{subsec:tipsAndTricks}

We end this tutorial by giving some helpful tips and tricks, and giving pointers
where to find more information about HOL4.

\subsection{Extending the Simplifier}
While proving the gaussian sum in \autoref{sec:hol_ex1} we had to explicitly
tell the simplifier that it should also use the definition of function
\lstinline{sum} while proving properties about the function.
In a pen-and-paper proof, one would never explicitly write this down and as such
it is desirable to have the same convenience in HOL4.
We can do so by slightly changing the definition of \lstinline{sum}:
%
\begin{lstlisting}
Definition sum_def[simp]:
  sum 0 = 0 /\
  sum n = n + sum (n-1)
End
\end{lstlisting}

By appending \lstinline{[simp]} to the name of the function, HOL4 automatically
adds \lstinline{sum_def} to the list of theorems used by the simplifier.
A similar mechanism exists for adding theorems to the simplifier.
However, this mechanic has to be used with caution as it is very easy to make
the simplifier diverge.

As an example, suppose we used the old definition of \lstinline{sum} which
defines the function as a recursive function not in equational style:

\begin{lstlisting}
Definition sum_def[simp]:
  sum n = if (n = 0) then 0 else n + sum (n-1)
End
\end{lstlisting}

If we restart the proof for the gaussian sum now, and run through the first two
tactics only (\lstinline{nltac `Induction on 'n'. simplify.`}) HOL4 will just
keep running.
As a rule of thumb, it is recommended to be conservative and rather mention a
definition than adding it to the simplifier.
The machinery can be useful for (non-recursive) abbreviations.
For theorems, one should refrain from adding commutativity or associativity
theorems, but adding theorems of the form $\forall x. P x \rightarrow Q x$, where
$Q$ does not depend on $P$ should be fine.

\subsection{Making Proof Scripts More Robust}
The most cumbersome work once a proof has been developed is making sure that it
remains correct even when versions of HOL4 change.
We give some simple recommendations that have proven quite useful over time.

First, we recommend commenting larger case splits and induction proofs.
While it may seem obvious now which case is being worked on by the proofscript
this might not be the case in a month, or a year of time after writing the
initial version.

Second, we recommend using tactics like
\lstinline{first_x_assum, last_assum, qpat_x_assum}.
These tactics are independent of the specific order of assumptions and thus make
the proof more robust to additional assumptions, or their removal.

\subsection{Getting More Help}
This small tutorial has only covered the basics.
More reference material can be found on \url{https://hol-theorem-prover.org/#doc}.
We especially recommend looking at the documentation of the emacs mode
(\url{https://hol-theorem-prover.org/hol-mode.html}), and the description manual.

The help index located at \lstinline{<HOLDIR>/help/HOLindex.html} provides
documentation for a lot of tactics and contains signature files for all of the
HOL4 distributions libraries and theories.

Finally, the HOL-info mailing list
(\url{https://sourceforge.net/projects/hol/lists/hol-info}) is a good place to
ask further questions, as well as the \texttt{\#hol} channel of the Slack of the
CakeML project (\url{https://join.slack.com/t/cakeml/shared_invite/MjM1NjEyODgxODkzLTE1MDQzNjgwMTUtYjI4YTdlM2VmMQ}).

%
\clearpage
\appendix
\section{Setup Script}\label{sec:script}
\lstinputlisting[language=bash,mathescape=false]{setup_hol4.sh}

%%% Local Variables:
%%% mode: latex
%%% TeX-master: "main"
%%% End:

%
\printbibliography
%
\end{document}
%%% Local Variables:
%%% mode: latex
%%% TeX-master: t
%%% End:
