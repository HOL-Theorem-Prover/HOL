\chapter{Syntax}
{
 \newcommand{\term}      {\mbox{\it term}}
 \newcommand{\IMP}       {\mbox{\tt ==>}}
 \newcommand{\ALL}       {\mbox{\tt !}}
 \newcommand{\EXISTS}    {\mbox{\tt ?}}
 \newcommand{\CHOOSE}    {\mbox{\tt @}}
 \newcommand{\EXISTSONE} {\mbox{\tt ?!}}
 \newcommand{\LET}       {\mbox{\tt let}}
 \newcommand{\und}       {\mbox{\tt and}}
 \newcommand{\IN}        {\mbox{\tt in}}
 \newcommand{\CONS}      {\mbox{\tt CONS}}
 \newcommand{\INSERT}    {\mbox{\tt INSERT}}
 \newcommand{\SUC}       {\mbox{\tt SUC}}
 \newcommand{\vstr}      {\mbox{\it vstr}}
 \newcommand{\numeral}   {\mbox{\it numeral}}
 \newcommand{\charseq}   {\mbox{\it charseq}}

The HOL logic is a classical higher-order predicate calculus. Its
syntax enjoys two main differences from the syntax of standard first
order logic.\footnote{We assume the reader is familiar with first
  order logic.}  First, there is no distinction in HOL between terms
and formulas: HOL has only terms. Second, each term has a type: types
are used in order to build well-formed terms. There are two ways to
construct types and terms in HOL: by use of a parser, or by use of the
programmer's interface. In this chapter, we will focus on the concrete
syntax accepted by the parsers.


\section{Types}

A HOL type can be a variable, a constant, or a compound type, which is
a constant of arity $n$ applied to a list of $n$ types.
\[
\begin{array}{rclr}
  \type & ::= & \mbox{\bf '}\ident & \mbox{(type variable)} \\
  & | &  \verb+bool+ & \mbox{(type of truth values)} \\
  & | &  \verb+ind+ & \mbox{(type of individuals)} \\
  & | &  \type\ \verb+->+\ \type & \mbox{(function arrow)} \\
  & | &  \type\ \ident\ \type\ & \mbox{(binary compound type)}\\
  & | &  \ident & \mbox{(nullary type constant)} \\
  & | & \type\; \ident & \mbox{(unary compound type)} \\
  & | & (\type_1,\ldots,\type_n) \ident & \mbox{(compound type)}
\end{array}
\]
Type constants are also known as type operators.
Lexically, they either
\begin{itemize}
\item look like an \ML{} variable: beginning with an ASCII letter character (upper or lower case) or a Unicode letter (except for Greek lower-case letters), with the rest of the operator being composed of Unicode characters that are alphabetic characters, numeric digits, the apostrophe character or the underscore; or
\item are composed entirely of ASCII digits (appearing to be a number)
\end{itemize}
Type variables are alphanumerics written with a leading apostrophe character ('), also known as the prime character, or are a lower-case Greek letter ($\alpha$, $\beta$, \etc.).

In \HOL{}, the type constants {\tt bool}, {\tt fun}, and
{\tt ind} are primitive. The introduction of new type constants is
described in Section~\ref{type-defs}. {\tt bool} is the two element type
of truth values. The binary operator {\tt fun} is used to denote
function types; it can be written with an infix arrow. The nullary
type constant {\tt ind} denotes an infinite set of individuals; it is
used for a few highly technical developments in the system and can be
ignored by beginners.  Thus
\begin{verbatim}
     'a -> 'b
     (bool -> 'a) -> ind
\end{verbatim}
are both well-formed types. The function arrow is ``right associative'',
which means that ambiguous uses of the arrow in types are resolved by
adding parentheses in a right-to-left sweep: thus the type expression
\begin{verbatim}
     ind -> ind -> ind -> ind
\end{verbatim}
is identical to
\begin{verbatim}
     ind -> (ind -> (ind -> ind)).
\end{verbatim}
The product (\verb+#+) and sum (\verb!+!) are other infix type
operators, also right associative; however, they are not loaded by
default in \HOL{}. How to load in useful logical context is described
in Section~\ref{theoryfns}.

\section{Terms}

Ultimately, a HOL term can only be a variable, a constant, an
application, or a lambda term.
\[
\begin{array}{rclr}
  \term & ::= & \ident & \mbox{(variable or constant)} \\
  & | &  \term\  \term & \mbox{(combination)} \\
  & | &  \bs\ident.\  \term &
  \mbox{(lambda abstraction)}
\end{array}
\]
In the system, the usual logical operators have already been defined,
including truth (\verb+T+), falsity (\verb+F+), negation (\verb+~+),
equality (\verb+=+), conjunction (\verb+/\+), disjunction (\verb+\/+),
implication (\verb+==>+), universal (\verb+!+) and existential
(\verb+?+) quantification, and an indefinite description operator
(\verb+@+). As well, the basis includes conditional, lambda, and `let'
expressions. Thus the set of terms available is, in general, an
extension of the following grammar:
\[
\begin{array}{rclr}
  \mbox{\it term} & ::= & \term : \type & \mbox{(type constraint)} \\
  & | & \term\ \term & \mbox{(application)} \\
  & | & \verb+~+ \term & \mbox{(negation)} \\
  & | & \term\ =\ \term & \mbox{(equality)} \\
  & | & \term\ \IMP\ \term & \mbox{(implication)} \\
  & | & \term\ \verb+\/+\ \term & \mbox{(disjunction)} \\
  & | & \term\ \verb+/\+\ \term & \mbox{(conjunction)} \\
  & | & \texttt{if}\ \term\ \texttt{then}\ \term\ \texttt{else}\ \term &
  \mbox{(conditional)} \\
  & | & \bs\ident_1\ldots\ident_n.\  \term & \mbox{(lambda abstraction)} \\
  & | & \ALL \ident_1\ldots\ident_n.\ \term & \mbox{(forall)} \\
  & | & \EXISTS \ident_1\ldots\ident_n.\ \term & \mbox{(exists)} \\
  & | & \CHOOSE \ident_1\ldots\ident_n.\ \term & \mbox{(choose)} \\
  & | & \EXISTSONE \ident_1\ldots\ident_n.\ \term &\mbox{(exists-unique)} \\
  & | & \LET\; \ident = \term  & \\
  &   & [\und\ \ident = \term]^{*}\ \IN\ \term & \mbox{(let expression)} \\
  & | & \verb+T+ & \mbox{(truth)} \\
  & | & \verb+F+ & \mbox{(falsity)} \\
  & | & \ident & \mbox{(constant or variable)} \\
  & | & \verb+(+ \term \verb+)+ & \mbox{(parenthesized term)}
\end{array}
\]

Some examples may be found in Table \ref{syntaxExamples}. Term
application can be iterated. Application is left associative so that
$\term\ \term\ \term \ldots \term$ is equivalent in the eyes of the
parser to $(\ldots((\term\ \term)\ \term) \ldots)\ \term$.

The lexical structure for term identifiers is much like that for
ML: identifiers can be alphanumeric or symbolic. Variables must be
alphanumeric. A symbolic identifier is any concatenation of the characters
in the following list:
\begin{verbatim}
    #?+*/\=<>&%@!,:;_|~-
\end{verbatim}
with the exception of the keywords \verb+\+, \verb+;+, \verb+=>+,
\verb+|+, and \verb+:+ (colon). Any alphanumeric can be a constant except the
keywords \verb+let+, \verb+in+, \verb+and+, and \verb+of+.

 \begin{table}[h]
\begin{center}
 \begin{tabular}{|r|l|} \hline
 \verb+x = T+ & {\it x is equal to true.} \\
 \verb+!x. Person x ==> Mortal x+ & {\it All persons are mortal.} \\
 \verb+!x y z. (x ==> y) /\ (y ==> z) ==> x ==> z+ & {\it Implication is
 transitive.} \\
 \verb+!x. P x ==> Q x+ & {\it P is a subset of Q} \\
 \verb+S = \f g x. f x (g x)+ & {\it Definition of a famous combinator.} \\ \hline
 \end{tabular}
 \caption{Concrete Syntax Examples}\label{syntaxExamples}
\end{center}
 \end{table}


%%% Local Variables:
%%% mode: latex
%%% TeX-master: "description"
%%% End:
