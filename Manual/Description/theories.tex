\chapter{Commonly-used Theories}\label{HOLtheories}

\index{theories, in HOL logic@theories, in \HOL\ logic!hierarchies of}
The collection of theories distributed with the \HOL\ system
\index{ancestry, of HOL system theories@ancestry, of \HOL\ system
theories}
is listed in Table \ref{nativeTheories}. Some of the less commonly used
theories have been omitted.

\begin{table}
\begin{center}
\begin{tabular}{|l|l|} \hline
minTheory & the origin theory \\
boolTheory & definitions of logical operators and basic axioms \\
combinTheory & combinators \\
pairTheory & theory of pairs \\
sumTheory & disjoint sums \\
relationTheory & transitive closure and wellfoundedness \\
numTheory & Peano's axioms derived from the axiom of infinity \\
prim\_recTheory, & the primitive recursion theorem \\
arithmeticTheory & Peano arithmetic development \\
integerTheory & integers \\
setTheory & sets as a separate type (includes finite sets) \\
pred\_setTheory & sets as predicates (includes finite sets) \\
bagTheory & bags (also known as \emph{multisets})\\
listTheory & lists  \\
rich\_listTheory & extended theory of lists \\
optionTheory & the {\tt option} type \\
finite\_mapTheory & finite functions \\
ltreeTheory & polymorphic finitely branching trees \\
restr\_binderTheory & definitions of binder restrictions \\
res\_quanTheory & restricted quantifier support \\
asciiTheory & ascii \\
stringTheory & strings \\
wordTheory & ({\it plus several others}) theory of bitstrings \\
realTheory & ({\it plus several others}) real numbers and analysis \\
HOLTheory & equivalent to HOL theory from hol88/90\\  \hline
\end{tabular}
\caption{Native Theories}\label{nativeTheories}
\end{center}
\end{table}

% \begin{center}\index{BASIC-HOL@\ml{BASIC-HOL}}\index{bool, the HOL
% theory@\ml{bool}, the \HOL\ theory}\index{ind, the theory@\ml{ind},
% the theory}

\noindent In the rest of this section, each of these theories is briefly
described.  A complete list of all the definitions and theorems in each
theory is not given here; the sections that follow provide only an
overview of the contents of each theory.  For a complete list of all the
built-in axioms, definitions and theorems in \HOL, see \REFERENCE.

\section{Combinators and the theory {\tt combin}}

\index{function composition, in HOL logic@function composition, in \HOL\
logic|(}

The theory \ml{combin}\index{combin@\ml{combin}}\index{combinators, in HOL
logic@combinators, in \HOL\ logic} contains the definitions of function
composition (infixed \ml{o})\index{ function composition operator, in HOL
logic@\ml{o} (function composition operator, in \HOL\ logic)|(}
 and the combinators \ml{S}\index{S, constant in HOL logic@\ml{S}, constant in
\HOL\ logic}, \ml{K}\index{K, the constant in HOL logic@\ml{K}, the constant in
\HOL\ logic}
 and \ml{I}\index{I, constant in HOL logic@\ml{I}, constant in \HOL\ logic}.


\begin{hol} \index{K_DEF@\ml{K\_DEF}} \index{S_DEF@\ml{S\_DEF}}
\index{I_DEF@\ml{I\_DEF}} \begin{verbatim}
   o_DEF |- !f g. f o g = (\x. f(g x))

   K_DEF |- K = (\x y. x)

   S_DEF |- S = (\f g x. f x(g x))

   I_DEF |- I = S K K \end{verbatim}\end{hol}


\noindent The following elementary properties are pre-proved in the theory
\ml{combin}:

\begin{hol} \index{K_THM@\ml{K\_THM}} \index{S_THM@\ml{S\_THM}}
\index{I_THM@\ml{I\_THM}} \index{I_o_ID@\ml{I\_o\_ID}} \begin{verbatim}
   o_THM |- !f g x. (f o g)x = f(g x)

   o_ASSOC |- !f g h. f o (g o h) = (f o g) o h

   K_THM |- !x y. K x y = x

   S_THM |- !f g x. S f g x = f x (g x)

   I_THM |- !x. I x = x

   I_o_ID |- !f. (I o f = f) /\ (f o I = f) \end{verbatim}\end{hol}

Having the symbols \ml{o}, \ml{S}, \ml{K} and \ml{I} as built-in
constants\index{variables, in HOL logic@variables, in \HOL\ logic!with constant
names} is sometimes inconvenient because they are often wanted as mnemonic
names for variables (\eg\ \ml{S} to range over sets and \ml{o} to range over
outputs).  Variables
(though not constants) with these names can be used in the current system if
\ml{o}, \ml{S}, \ml{K} and \ml{I} are first hidden (see Section~\ref{hidden}).
\index{ function composition operator, in HOL logic@\ml{o} (function
composition operator, in \HOL\ logic)|)} \index{function composition, in HOL
logic@function composition, in \HOL\ logic|)}

\section{The theory {\tt relation}}\label{relation}

Mathematical relations can be directly represented in \HOL\ by the type
{\small\verb+:'a -> 'a -> bool+}. The theory {\small\verb+relation+}
inductively defines the transitive closure of a relation and also the
basic notion of a relation being wellfounded. 

\begin{hol} \begin{verbatim}
   TC_DEF |- !R a b.
               TC R a b =
                 !P.
                   (!x y. R x y ==> P x y) /\ 
                   (!x y z. P x y /\ P y z ==> P x z) 
                   ==>
                     P a b
   
   WF_DEF |- !R. WF R = !B. (?w. B w) ==> ?min. B min /\ !b. R b min ==> ~B b
\end{verbatim}\end{hol}

Wellfoundedness is then used to justify the principle of wellfounded
induction and also a general recursion theorem. The statement of the 
recursion theorem requires that the notion of a function restriction be
defined as well. 

\begin{hol} \begin{verbatim}
   WF_INDUCTION_THM
   |- !R:'a -> 'a -> bool. 
          WF R 
            ==> !P. (!x. (!y. R y x ==> P y) ==> P x) 
            ==> !x. P x
   
   RESTRICT_DEF |- !f R x. RESTRICT f R x = \y. if R y x then f y else ARB

   WFREC_COROLLARY
   |- !M R f. (f = WFREC R M) ==> WF R ==> !x. f x = M (RESTRICT f R x) x
   
   WF_RECURSION_THM |- !R. WF R ==> !M. ?!f. !x. f x = M (RESTRICT f R x) x

\end{verbatim}\end{hol}

\noindent A few basic combinators for wellfounded relations are also
provided in this theory.

\begin{hol} \begin{verbatim}
   Empty_def      |- !x y. Empty x y = F
   inv_image_def  |- !R f. inv_image R f = \x y. R (f x) (f y)

   WF_Empty       |- WF Empty
   WF_SUBSET      |- !R P. WF R /\ (!x y. P x y ==> R x y) ==> WF P
   WF_TC          |- !R. WF R ==> WF (TC R)
   WF_inv_image   |- !R f. WF R ==> WF (inv_image R f)

\end{verbatim}\end{hol}

\section{Pairs and the type {\tt prod}}\label{prod}

\index{representing types, in HOL logic@representing types, in \HOL\ logic!pair example of|(}
\index{pairs, in HOL logic@pairs, in \HOL\ logic|(}
\index{product types!in HOL logic@in \HOL\ logic|(}
The Cartesian  product  type  operator\index{type operators, in HOL
logic@type operators, in \HOL\ logic!for pairs}
\ml{prod}\index{prod@\ml{prod}} is  defined  in  the  theory \ml{pair}.
Values of type
{\small\verb%(%}$\sigma_1${\small\verb%,%}$\sigma_2${\small\verb%)prod%} are
ordered pairs  whose  first  component  has  type  $\sigma_1$  and whose second
component has type $\sigma_2$.  The \HOL\ parser\index{parsing, of HOL logic@parsing, of \HOL\ logic!of pairs}
converts type expressions of the
form \ml{`:}$\sigma_1${\small\verb%#%}$\sigma_2$\ml{`}\index{ product type operator, in HOL logic@{\small\verb+#+} (product
type operator, in \HOL\ logic)} into
\ml{(}$\sigma_1$\ml{,}$\sigma_2$\ml{)prod}\index{ pair constructor, in HOL
logic@\ml{,} (pair constructor, in \HOL\ logic)}, and   the   printer
inverts  this transformation. Pairs\index{pairing constructor, in HOL
logic@pairing constructor, in \HOL\ logic} are constructed with an infixed
comma symbol


\begin{hol}\begin{verbatim}
   $, : 'a -> 'b -> 'a # 'b
\end{verbatim}\end{hol}

\noindent so, for example, if $t_1$ and $t_2$ have types $\sigma_1$ and
$\sigma_2$
respectively, then $t_1$\ml{,}$t_2$ is a term with type
$\sigma_1${\small\verb%#%}$\sigma_2$. It is usual, but not necessary, to write
pairs within brackets:
\ml{(}$t_1$\ml{,}$t_2$\ml{)}. The comma symbol associates\index{pairing
constructor, in HOL logic@pairing constructor, in \HOL\
logic!associativity of} to the right, so  that
\ml{(}$t_1$\ml{,}$t_2$\ml{,}$\ldots$\ml{,}$t_n$\ml{)}
means
\ml{(}$t_1$\ml{,(}$t_2$\ml{,}$\ldots$\ml{,}$t_n$\ml{))}.

Cartesian products are defined by representing a pair
{\small\verb%(%}$t_1${\small\verb%,%}$t_2${\small\verb%)%} by the function

\begin{hol}\begin{alltt}
   \verb!\!a b. (a=\m{t\sb{1}}) /\verb!\! (b=\m{t\sb{2}})
\end{alltt}\end{hol}

\noindent The
representing type of $\sigma_1${\small\verb%#%}$\sigma_2$ is thus
$\sigma_1${\small\verb%->%}$\sigma_2${\small\verb%->bool%}.  To define
pairs this way, the constants \ml{MK\_PAIR} and
\ml{IS\_PAIR}\index{IS_PAIR@\ml{IS\_PAIR}} are first defined.


\begin{hol}
\index{MK_PAIR_DEF@\ml{MK\_PAIR\_DEF}}
\index{IS_PAIR_DEF@\ml{IS\_PAIR\_DEF}}
\begin{verbatim}
   MK_PAIR_DEF   |- !x y. MK_PAIR x y = (\a b. (a = x) /\ (b = y))

   IS_PAIR_DEF   |- !p. IS_PAIR p = (?x y. p = MK_PAIR x y)
\end{verbatim}\end{hol}

\noindent From these two definitions it is easy to prove that:

\begin{hol}\begin{verbatim}
   |- ?p:'a->'b->bool. IS_PAIR p
\end{verbatim}\end{hol}

\noindent since {\small\verb%|- IS_PAIR(MK_PAIR x y)%} follows  easily from the
definition of  \ml{IS\_PAIR}.    The  existence  theorem shown  above is called
{\small\verb%PAIR_EXISTS%}\index{PAIR_EXISTS@\ml{PAIR\_EXISTS}}.
 Given this theorem, the type operator
{\small\verb%prod%} is defined by evaluating:

\begin{hol}\begin{verbatim}
   new_type_definition(`prod`, "IS_PAIR:(*->**->bool)->bool", PAIR_EXISTS)
\end{verbatim}\end{hol}

\noindent which results in the definitional axiom\index{axioms!non-primitive, of HOL logic@non-primitive, of \HOL\ logic}\index{axioms!in bool theory@in \ml{bool} theory} \ml{prod\_TY\_DEF} shown
below being asserted in the theory \ml{bool}.

\begin{hol}\begin{verbatim}
   prod_TY_DEF  |- ?rep. TYPE_DEFINITION IS_PAIR rep
\end{verbatim}\end{hol}

Next, a new constant {\small\verb%REP_prod%} is defined, which
maps a pair to its representation as a function:

\begin{hol}
\index{REP_prod@\ml{REP\_prod}}
\begin{verbatim}
   REP_prod    |- REP_prod =
                  (@rep : 'a # 'b -> 'a -> 'b -> bool.
                    (!p' p''. (rep p' = rep p'') ==> (p' = p'')) /\
                    (!p. IS_PAIR p = (?p'. p = rep p')))
\end{verbatim}\end{hol}

The infix constructor `{\small\verb%,%}'
and the selectors
{\small\verb%FST:'a#'b->'a%} and {\small\verb%SND:'a#'b->'b%} are then
defined by the equations shown below.


\begin{hol}\index{COMMA_DEF@\ml{COMMA\_DEF}}
\index{FST_DEF@\ml{FST\_DEF}}
\index{SND_DEF@\ml{SND\_DEF}}
\index{pairing constructor, in HOL logic@pairing constructor, in \HOL\ logic!definition of}
\index{FST, the constant in HOL logic@\ml{FST}, the constant in \HOL\ logic!definition of}
\index{selectors, in HOL logic@selectors, in \HOL\ logic}
\begin{verbatim}
   COMMA_DEF  |- !x y. x,y = (@p. REP_prod p = MK_PAIR x y)

   FST_DEF    |- !p. FST p = (@x. ?y. MK_PAIR x y = REP_prod p)

   SND_DEF    |- !p. SND p = (@y. ?x. MK_PAIR x y = REP_prod p)
\end{verbatim}\end{hol}

The following standard theorems about pairs follow easily from these
definitions and the axiom \ml{prod\_TY\_DEF}.  Although these theorems could
be derived by formal proof, they are (for implementation reasons) asserted as
axioms in the theory \ml{bool} .

\begin{hol}
\index{PAIR@\ml{PAIR}}
\index{FST, the axiom in HOL logic@\ml{FST}, the axiom in \HOL\ logic}
\index{SND, the axiom in HOL logic@\ml{SND}, the axiom in \HOL\ logic}
\index{PAIR_EQ@\ml{PAIR\_EQ}}
\begin{verbatim}
   PAIR     |- !x. (FST x,SND x) = x

   FST      |- !x y. FST(x,y) = x

   SND      |- !x y. SND(x,y) = y

   PAIR_EQ  |- !x y a b. (x,y = a,b)  =  (x = a) /\ (y = b)
\end{verbatim}\end{hol}
\index{pairs, in HOL logic@pairs, in \HOL\ logic|)}
\index{product types!in HOL logic@in \HOL\ logic|)}
\index{representing types, in HOL logic@representing types, in \HOL\ logic!pair example of|)}
\index{bool, the HOL theory@\ml{bool}, the \HOL\ theory|)}


\subsection{Paired abstractions}
\label{HOL-varstruct}
\index{pairs, in HOL logic@pairs, in \HOL\ logic!in abstractions|(}
\index{UNCURRY@\ml{UNCURRY}|(}

\index{terms, in HOL logic@terms, in \HOL\ logic!pair|(}
\index{parsing, of HOL logic@parsing, of \HOL\ logic!of quotation syntax|(}
\index{function abstraction, in HOL logic@function abstraction, in \HOL\ logic!paired|(}
\index{function abstraction, in HOL logic@function abstraction, in \HOL\ logic!uncurrying, in paired|(}
The quotation parser\index{parsing, of HOL logic@parsing, of \HOL\ logic!of function abstractions}\index{function abstraction, in HOL logic@function abstraction, in \HOL\ logic!abbreviation for multiple}\index{terms, in HOL logic@terms, in \HOL\ logic!function abstraction}
 will convert\footnote{Only when the theory of pairs is loaded.}
{\small\verb%"\(%}$x_1${\small\verb%,%}$x_2${\small\verb%).%}$t${\small\verb%"%}
to {\small\verb%"UNCURRY(\%}$x_1\ x_2${\small\verb%.%}$t${\small\verb%)"%},
where the constant {\small\verb%UNCURRY%}
is defined by:

\begin{hol}\begin{verbatim}
   UNCURRY f (x,y)  =  f x y
\end{verbatim}\end{hol}

\noindent See Section~\ref{prod} for more details and an explanation
of pair terms \ml{($t_1$,$t_2$)}.
The transformation is done recursively so that, for example,


\begin{hol}\begin{alltt}
   "\verb%\%(\m{x\sb{1}},\m{x\sb{2}},\m{x\sb{3}}).\m{t}"
\end{alltt}\end{hol}

\noindent is converted to

\begin{hol}\begin{alltt}
   "UNCURRY \verb%\%\m{x\sb{1}}.UNCURRY(\verb%\%\m{x\sb{2}},\m{x\sb{3}}.\m{t}))"
\end{alltt}\end{hol}

\noindent More generally,
the quotation parser repeatedly applies the transformation:

\begin{hol}\begin{alltt}
   "\verb%\%(\m{v\sb{1}},\m{v\sb{2}}).\m{t}"\m{\quad \leadsto\quad }"UNCURRY(\verb%\%\m{v\sb{1}}.\verb%\%\m{v\sb{2}}.\m{t})"
\end{alltt}\end{hol}

\noindent until no more variable structures remain. For example:

\begin{flushleft}
\begin{tabular}{@{}ll}
{\small\verb%   "\(%}$x${\small\verb%,%}$y${\small\verb%).%}$t${\small\verb%"%}
&$\leadsto\ \ ${\small\verb%"UNCURRY(\%}$x\
y${\small\verb%.%}$t${\small\verb%)"%}\\
{\small\verb%   "\(%}$x_1${\small\verb%,%}$x_2${\small\verb%,%}$\ldots${\small\verb%,%}$x_n${\small\verb%).%}$t${\small\verb%"%}
&$\leadsto\ \ ${\small\verb%"UNCURRY(\%}$x_1${\small\verb%.\(%}$x_2${\small\verb%,%}$\ldots${\small\verb%,%}$x_n${\small\verb%).%}$t${\small\verb%)"%}\\
{\small\verb%   "\((%}$x_1${\small\verb%,%}$\ldots${\small\verb%,%}$x_n${\small\verb%),%}$y_1${\small\verb%,%}$\ldots${\small\verb%,%}$y_m${\small\verb%).%}t{\small\verb%"%}
&$\leadsto\ \ ${\small\verb%"UNCURRY(\(%}$x_1${\small\verb%,%}$\ldots${\small\verb%,%}$x_n${\small\verb%).\(%}$y_1${\small\verb%,%}$\ldots${\small\verb%,%}$y_m${\small\verb%).%}$t${\small\verb%)"%}\\
\end{tabular}
\end{flushleft}

\index{parsing, of HOL logic@parsing, of \HOL\ logic!of quotation syntax|)}
The \HOL\ top-level printer\index{printing, in HOL logic@printing, in \HOL\ logic!of function abstractions}
 inverts these transformations. For example:

\setcounter{sessioncount}{1}
\begin{session}\begin{verbatim}
#"\((a,b),(x,y)). a+b < x*y";;
"\((a,b),x,y). a + b < x * y" : term

#set_flag(`print_uncurry`,false);;
true : bool

#"\((a,b),(x,y)). (a+b) < (x*y)";;
"UNCURRY(UNCURRY(\a b. UNCURRY(\x y. (a + b) < (x * y))))" : term
\end{verbatim}\end{session}

Note that a variable structure like \ml{"(x,y)"} in
{\small\verb%"\(x,y).x+y"%}
is not a subterm
of the abstraction\index{function abstraction, in HOL logic@function abstraction, in \HOL\ logic!subterms of} in which it occurs; it disappears on parsing\index{binders, in HOL logic@binders, in \HOL\ logic!parsing of}\index{parsing, of HOL logic@parsing, of \HOL\ logic!of binders}. This can
lead to unexpected errors (accompanied by obscure error messages).
For example:

\begin{session}\begin{verbatim}
#"\(x,y).x+y";;
"\(x,y). x + y" : term

#let p = "(x:num,y:num)";;
p = "x,y" : term

#"\^p.x+y";;
evaluation failed     mk_abs in quotation
\end{verbatim}\end{session}

Furthermore, if the type checker complains, it may print out
diagnostic messages referring to the transformed term:

\begin{session}\begin{verbatim}
#"\(x,y). x+1";;
Indeterminate types:
    "UNCURRY:(num -> (?1 -> num)) -> (num * ?2 -> num)"

evaluation failed     types indeterminate in quotation
\end{verbatim}\end{session}

If $b$ is a binder, then
\ml{"}$b$\ml{(}$x_1${\small\verb%,%}$x_2${\small\verb%).%}$t${\small\verb%"%}
is parsed as
{\small\verb%"$%}$b${\small\verb%(\(%}$x_1${\small\verb%,%}$x_2${\small\verb%).%}$t${\small\verb%)"%}, and hence transformed as above.
For example, {\small\verb%"!(x,y).x>y"%} parses to
{\small\verb%"$!(UNCURRY(\x.\y.$> x y))"%} (where {\small{\tt >}} is an infixed
constant of the theory {\small\verb%num%} meaning `is greater than')\index{function abstraction, in HOL logic@function abstraction, in \HOL\ logic!paired|)}.
\index{function abstraction, in HOL logic@function abstraction, in \HOL\ logic!uncurrying, in paired|)}\index{pairs, in HOL logic@pairs, in \HOL\ logic!in abstractions|)}\index{terms, in HOL logic@terms, in \HOL\ logic!pair|)}\index{UNCURRY@\ml{UNCURRY}|)}


%A good exercise for the reader would be to understand why this use
%of \ml{UNCURRY} supports the intuitive meaning suggested by the surface
%notation.

Applications of paired abstraction to tuples can be $\beta$-reduced using
\ml{PAIRED\_BETA\_CONV} (see Section~\ref{genbeta}).


\subsection{{\tt let}-terms}
\label{let-exp}


The quotation parser\index{parsing, of HOL logic@parsing, of \HOL\ logic!of let-terms@of \ml{let}-terms}
 accepts \ml{let}-terms\index{terms, in HOL logic@terms, in \HOL\ logic!let-@\ml{let}-}\index{let-terms, in HOL logic@\ml{let}-terms, in \HOL\ logic!as abbreviations} superficially similar to those in
\ML. For example, the following term is allowed:

\begin{hol}\begin{verbatim}
   "let x = 1 and y = 2 in x+y"

   "let f(x,y) = (x*x)+(y*y) and a = 20*20 and b = 50*49 in f(a,b)"
\end{verbatim}\end{hol}

\ml{let}-terms are actually abbreviations for ordinary  terms which are
specially supported by the parser and pretty printer.
The constant \ml{LET}\index{LET@\ml{LET}} is defined (in the theory \ml{bool}) by:

\begin{hol}\index{function abstraction, in HOL logic@function abstraction, in \HOL\ logic!relation to let-terms@relation to \ml{let}-terms}
\begin{verbatim}
   LET = (\f x. f x)
\end{verbatim}\end{hol}

\noindent and is used to encode \ml{let}-terms in the logic. The parser
repeatedly applies the transformations:

\bigskip

\noindent
{\small\begin{tabular}{@{}ll}
{\small\verb%   "let %}$f\ v_1\ \ldots\ v_n${\small\verb% = %}$t_1${\small\verb% in %}$t_2${\small\verb%"%}
&$\leadsto\ \ ${\small\verb%"LET(\%}$f${\small\verb%.%}$t_2$
{\small\verb%)(\%}$v_1\ \ldots\ v_n${\small\verb%.%}$t_1${\small\verb%)"%}\\
{\small\verb%   "let (%}$v_1${\small\verb%,%}$\ldots${\small\verb%,%}$v_n${\small\verb%) = %}$t_1${\small\verb% in %}$t_2${\small\verb%"%}
&$\leadsto\ \ ${\small\verb%"LET(\(%}$v_1${\small\verb%,%}$\ldots${\small\verb%,%}$v_n${\small\verb%).%}$t_2${\small\verb%)%}$t_1${\small\verb%"%}\\
{\small\verb%   "let %}$v_1${\small\verb%=%}$t_1${\small\verb% and %}$\ldots${\small\verb% and %}$v_n${\small\verb%=%}$t_n${\small\verb% in %}$t${\small\verb%"%}
&$\leadsto\ \ ${\small\verb%"LET(%}$\ldots${\small\verb%(LET(LET(\%}$v_1\ldots v_n${\small\verb%.%}$t${\small\verb%)%}$t_1${\small\verb%)%}$t_2${\small\verb%)%}$\ldots${\small\verb%)%}$t_n${\small\verb%"%}\\
\end{tabular}}

\bigskip


\noindent The printer\index{printing, in HOL logic@printing, in \HOL\ logic!of let-terms@of \ml{let}-terms} inverts these transformations if the flag
\ml{print\_let} is \ml{true}.  For example:

\setcounter{sessioncount}{1}
\begin{session}\begin{verbatim}
#"let x = 1 and y = 2 in x+y";;
"let x = 1 and y = 2 in x + y" : term

#set_flag(`print_let`,false);;
true : bool

#"let x = 1 and y = 2 in x+y";;
"LET(LET(\x y. x + y)1)2"

#"let (x,y) = (1,2) in x+y";;
"LET(UNCURRY(\x y. x + y))(1,2)" : term
\end{verbatim}\end{session}

Note that, as with uncurried terms, the underlying representation in \HOL\ can
manifest itself in error messages. For example:

\setcounter{sessioncount}{1}
\begin{session}\begin{verbatim}
#"let x = y in z";;
Indeterminate types:  "LET:(?1 -> ?2) -> (?3 -> ?4)"

evaluation failed     types indeterminate in quotation
\end{verbatim}\end{session}

The reader is recommended to convince himself or herself that the
translations of \ml{let}-terms represent the intuitive meaning suggested by
the surface syntax.\index{quotation, in HOL logic@quotation, in \HOL\ logic|)}\index{quotation, in HOL logic@quotation, in \HOL\ logic!of non-primitive terms|)}\index{terms, in HOL logic@terms, in \HOL\ logic!syntax of|)}\index{type checking, in HOL logic@type checking, in \HOL\ logic!special forms in|)}

\ml{let}-terms can be simplified with \ml{let\_CONV} -- see Section~\ref{let-terms}.

\section{Disjoint sums}\label{sum}

The theory \ml{sum}\index{sum@\ml{sum}}
 defines the binary  disjoint union\index{disjoint union theory, in HOL
logic@disjoint union theory, in \HOL\ logic|(}  type operator \ml{sum}.
A type  {\small\verb%(%}$\sigma_1${\small\verb%,%}$\sigma_2${\small\verb%)sum%}
denotes the  disjoint  union  of  types  $\sigma_1$  and $\sigma_2$.   The type
operator {\small\verb%sum%} can be defined just as {\small\verb%prod%} was, but
the details are omitted here.\footnote{The definition of disjoint  unions in
the
HOL system is due to Tom Melham. The technical details of this definition can
be found in~\cite{Melham-banff}.}  The \HOL\ parser\index{parsing, of HOL logic@parsing, of \HOL\ logic!of sum types}
converts
\ml{":}$\sigma_1${\small\verb%+%}$\sigma_2$\ml{"}\index{ disjoint union
type operator, in HOL logic@\ml{+} (disjoint union
type operator, in HOL logic)} into
\ml{(}$\sigma_1$\ml{,}$\sigma_2$\ml{)sum}, and the printer inverts this.

The standard operations on sums are:


\begin{hol}
\index{disjoint union theory, in HOL logic@disjoint union theory, in \HOL\ logic|)}
\index{INL, the constant in HOL logic@\ml{INL}, the constant in \HOL\ logic}
\index{INR, the constant in HOL logic@\ml{INR}, the constant in \HOL\ logic}
\index{ISL, the constant in HOL logic@\ml{ISL}, the constant in \HOL\ logic}
\index{ISR, the constant in HOL logic@\ml{ISR}, the constant in \HOL\ logic}
\index{OUTL, the constant in HOL logic@\ml{OUTL}, the constant in \HOL\ logic}
\index{OUTR, the constant in HOL logic@\ml{OUTR}, the constant in \HOL\ logic}
\begin{verbatim}
   INL  : 'a      -> 'a + 'b
   INR  : 'b     -> 'a + 'b
   ISL  : 'a + 'b -> bool
   ISR  : 'a + 'b -> bool
   OUTL : 'a + 'b -> 'a
   OUTR : 'a + 'b -> 'b
\end{verbatim}\end{hol}

\noindent These are all defined as constants in the theory \ml{sum}.  The
constants \ml{INL} and \ml{INR} inject into the left and right summands,
respectively. The constants \ml{ISL} and \ml{ISR} test for membership of the
left and right summands, respectively. The constants \ml{OUTL} and \ml{OUTR}
project from a sum to the left and right summands, respectively.

The following two theorems, which are minor variants of each other, are
pre-proved in the built-in theory \ml{sum}. Each one, on its own, provides a
complete and abstract characterization of the disjoint sum type.

\begin{hol}
\index{sum_Axiom@\ml{sum\_Axiom}}
\index{sum_axiom@\ml{sum\_axiom}}
\begin{verbatim}
   sum_axiom   |- !f g. ?! h. (h o INL = f) /\ (h o INR = g)

   sum_Axiom = |- !f g. ?! h. (!x. h(INL x) = f x) /\ (!x. h(INR x) = g x)
\end{verbatim}\end{hol}

\noindent Also provided as built-in, are the following theorems having to
do with the discriminator functions \ml{ISL} and \ml{ISR}:

\begin{hol}
\index{ISL, the theorem in HOL logic@\ml{ISL}, the theorem in \HOL\ logic}
\index{ISR, the theorem in HOL logic@\ml{ISR}, the theorem in \HOL\ logic}
\index{ISL_OR_ISR@\ml{ISL\_OR\_ISR}}
\begin{verbatim}
   ISL         |- (!x. ISL(INL x)) /\ (!y. ~ISL(INR y))

   ISR         |- (!x. ISR(INR x)) /\ (!y. ~ISR(INL y))

   ISL_OR_ISR  |- !x. ISL x \/ ISR x
\end{verbatim}\end{hol}

\noindent The \ml{sum} theory also provides the following built-in theorems:

\begin{hol}
\index{OUTL, the theorem in HOL logic@\ml{OUTL}, the theorem in \HOL\ logic}
\index{OUTR, the theorem in HOL logic@\ml{OUTR}, the theorem in \HOL\ logic}
\index{INL, the theorem in HOL logic@\ml{INL}, the theorem in \HOL\ logic}
\index{INR, the theorem in HOL logic@\ml{INR}, the theorem in \HOL\ logic}
\begin{verbatim}
   OUTL        |- !x. OUTL(INL x) = x

   OUTR        |- !x. OUTR(INR x) = x

   INL         |- !x. ISL x ==> (INL(OUTL x) = x)

   INR         |- !x. ISR x ==> (INR(OUTR x) = x)
\end{verbatim}\end{hol}

\noindent which describe the projection functions \ml{OUTL} and \ml{OUTR}.

\section{The theory {\tt one}}%
\index{one, the HOL theory@\ml{one}, the \HOL\ theory}%
\index{one, the HOL type@\ml{one}, the \HOL\ type}%

The theory \ml{one} defines  the type  \ml{":one"} which  contains one element.
The constant  \ml{one}  is specified  to denote  this element.   The pre-proved
theorems in the theory \ml{one} are:

\begin{hol}
\index{one_axiom@\ml{one\_axiom}}
\index{one, the HOL theorem@\ml{one}, the \HOL\ theorem}
\index{one_Axiom@\ml{one\_Axiom}}
\begin{verbatim}
   one_axiom   |- !(f:'a -> one) (g:'a -> one). f = g
   one         |- !(v:one). v = one
   one_Axiom   |- !(e:'a). ?!(fn:one->'a). fn one = e
\end{verbatim}\end{hol}

\noindent These three theorems are equivalent characterizations of the type
with only one value.


\section{Natural numbers}

\subsection{The theory {\tt num}}

The theory \ml{num}\index{num, the theory in HOL logic@\ml{num}, the theory in \HOL\ logic}
defines the type \ml{num} of natural numbers to be
isomorphic to a countable subset of the primitive type \ml{ind}.  In this
theory, the constants \ml{0}\index{ zero, in HOL logic@\ml{0} ( zero, in \HOL\ logic)}
 and \ml{SUC} (the successor function) are defined
and Peano's axioms\index{axioms!in num theory@in \ml{num} theory}\index{Peano's axioms}\index{axioms!non-primitive, of HOL logic@non-primitive, of \HOL\ logic} pre-proved in the form:


\begin{hol}
\index{NOT_SUC@\ml{NOT\_SUC}}
\index{INV_SUC@\ml{INV\_SUC}}
\index{INDUCTION@\ml{INDUCTION}}
\begin{verbatim}
   NOT_SUC    |- !n. ~(SUC n = 0)
   INV_SUC    |- !m n. (SUC m = SUC n) ==> (m = n)
   INDUCTION  |- !P. P 0 /\ (!n. P n ==> P(SUC n)) ==> (!n. P n)
\end{verbatim}\end{hol}

In higher order logic, Peano's axioms are sufficient for developing number
theory because addition and multiplication can be defined. In first order
logic these must be taken as primitive.  Note also that
{\small\verb%INDUCTION%}\index{induction rule!for numbers, in HOL logic@for numbers, in \HOL\ logic} could not be stated as a single axiom in
first order logic because predicates (\eg\ {\small\verb%P%}) cannot be
quantified.

Uses of the theorem \ml{INDUCTION} are supported by the 
tactic \ml{numLib.INDUCT\_TAC}\index{INDUCT_TAC@\ml{INDUCT\_TAC}}
(see the documentation in \REFERENCE\ for details).

\subsection{The theory {\tt prim\_rec}}\label{prim_rec}

\index{primitive recursive definitions, in HOL logic@primitive recursive definitions, in \HOL\ logic!automated|(}
\index{primitive recursion theorem!for numbers|(}
\index{prim_rec@\ml{prim\_rec}|(}
In classical logic, unlike domain theory logics such as \PPL\index{PPlambda (same as PPLAMBDA), of LCF system@\ml{PP}$\lambda$ (same as \ml{PPLAMBDA}), of \ml{LCF} system},
arbitrary recursive definitions\index{recursive definitions, in classical logics} are not allowed. For example, there is no
function $f$ (of type \ml{num->num}) such that

\begin{hol}
{\small\verb%   !%}$x${\small\verb%. %}$f$ $x${\small\verb%  =  (%}$f$ $x${\small\verb%) + 1%}
\end{hol}

\noindent Certain restricted forms of recursive\index{primitive recursive
functions}
definition do, however, uniquely
define functions. An important example are the {\it primitive recursive\/}
functions.\footnote{In higher order logic, primitive recursion
is much more powerful than in first order logic;
for example, Ackermann's function can be defined
by primitive recursion in higher order logic.} For
any $x$ and $f$ the {\it primitive
recursion theorem\/} tells us that there is a unique function
{\small\verb%fn%} such that:

\begin{hol}
{\small\verb%   %}\ml{fn}{\small\verb% 0 = %}$x${\small\verb%) /\ (!%}$n${\small\verb%.%}\ml{fn}{\small\verb%(%}\ml{SUC} $n${\small\verb%) = %}$f${\small\verb% (%}\ml{fn} $n${\small\verb%)%} $n${\small\verb%)%}
\end{hol}

The primitive recursion theorem follows from Peano's\index{Peano's axioms}
 axioms. When the \HOL\ system is built, the following theorem is proved
 and stored in the theory {\small\verb%prim_rec%}:


\begin{hol}\index{num_Axiom@\ml{num\_Axiom}}
\index{characterizing theorem!for numbers}
\begin{verbatim}
   num_Axiom   |- !x f. ?!fn. (fn 0 = x) /\ (!n. fn(SUC n) = f (fn n) n)
\end{verbatim}\end{hol}

\noindent The theorem states the validity of primitive recursive
definitions on the natural numbers: for any \ml{x} and \ml{f} there exists a
corresponding total function \ml{fn} which satisfies
the primitive recursive definition whose form is determined by \ml{x} and
\ml{f}.

\subsubsection{Primitive recursive definitions}\label{num-prim-rec}

\index{type definitions, in HOL logic@type definitions, in \HOL\ logic!primitive recursive|(}
\index{recursive definitions, in HOL logic@recursive definitions, in \HOL\ logic!automated, for numbers|(}
\index{primitive recursion theorem!automated use of, in HOL system@automated use of, in \HOL\ system|(}
The primitive\index{primitive recursive definitions, in HOL logic@primitive recursive definitions, in \HOL\ logic!justification of} recursion theorem
can be used to justify any  definition of a
function on the  natural numbers  by primitive  recursion.   For example, a
primitive recursive definition in higher order
logic of the form

\begin{hol}\begin{alltt}
   fun 0       x\(\sb{1}\) \m{\dots} x\(\sb{i}\) = \m{f\sb{1}[}x\(\sb{1}\)\m{,\ldots,\,} x\(\sb{i}]\)
   fun (SUC n) x\(\sb{1}\) \m{\dots} x\(\sb{i}\) = \m{f\sb{2}[}fun n \m{t\sb{1} \dots t\sb{i},} n\m{,} x\(\sb{1}\)\m{,\ldots,\,}x\(\sb{i}]\)
\end{alltt}\end{hol}

\noindent where all the free variables in the  terms $t_1$,
\dots, $t_i$ are contained in $\{$\ml{n}, $\ml{x}_1$, \dots, $\ml{x}_i\}$,
is logically equivalent to:

\begin{hol}\begin{alltt}
   fun 0       = \verb!\!x\(\sb{1}\) \m{\dots} x\(\sb{i}\).\m{f\sb{1}[}x\(\sb{1}\)\m{,\ldots,\,}x\(\sb{i}]\)
   fun (SUC n) = \verb!\!x\(\sb{1}\) \m{\dots} x\(\sb{i}\).\m{f\sb{2}[}fun n \m{t\sb{1} \dots t\sb{i},} n\m{,}x\(\sb{1}\)\m{,\ldots,\,}x\(\sb{i}]\)
               = (\verb!\!f n x\(\sb{1}\) \m{\dots} x\(\sb{i}\).\m{f\sb{2}[}f \m{t\sb{1} \dots t\sb{i},} n\m{,} x\(\sb{1}\)\m{,\ldots,\,}x\(\sb{i}]\)) (fun n) n
\end{alltt}\end{hol}

The existence  of  a  recursive  function  \ml{fun} which  satisfies these two
equations follows directly from the primitive recursion theorem
\ml{num\_Axiom} shown above.   Specializing the  quantified variables \verb!x!
and \verb!f!  in  a suitably  type-instantiated version  of \ml{num\_Axiom} so
that 

\begin{hol}\begin{alltt}
   x\m{=}\verb!\!x\(\sb{1}\) \(\dots\) x\(\sb{i}\).\m{f\sb{1}[}x\(\sb{1}\)\(,\ldots,\,\)x\(\sb{i}]\)  {\rm and}  f\(=\)\verb!\!f n x\(\sb{1}\) \(\dots\) x\(\sb{i}\).\m{f\sb{2}[}f \m{t\sb{1} \dots t\sb{i},} n\(,\) x\(\sb{1}\)\(,\ldots,\,\)x\(\sb{i}]\))
\end{alltt}\end{hol}

\noindent yields (ignoring the uniqueness of \ml{fn})
the existence theorem shown below:

\begin{hol}\begin{alltt}
   |- ?fn. fn 0       = \verb!\!x\(\sb{1}\) \(\dots\) x\(\sb{i}\).\m{f\sb{1}[}x\(\sb{1}\)\(,\ldots,\,\)x\(\sb{i}]\) /\verb!\!
           fn (SUC n) = (\verb!\!f n x\(\sb{1}\) \(\dots\) x\(\sb{i}\).\m{f\sb{2}[}f \m{t\sb{1} \dots t\sb{i},} n\(,\) x\(\sb{1}\)\(,\ldots,\,\)x\(\sb{i}]\)) (fn n) n
\end{alltt}\end{hol}

\noindent This theorem allows a constant \ml{fun} to be introduced (via the
definitional mechanism of constant specifications---see Section~\ref{conspec})
to denote the recursive function that satisfies the two equations in the body
of the theorem. Introducing a constant \ml{fun} to name the function asserted
to exist by the theorem shown above, and simplifying using $\beta$-reduction,
yields the following theorem:

\begin{hol}\begin{alltt}
   |- fun 0       = \verb!\!x\(\sb{1}\) \(\dots\) x\(\sb{i}\).\m{f\sb{1}[}x\(\sb{1}\)\(,\ldots,\,\)x\(\sb{i}]\) /\verb!\!
      fun (SUC n) = \verb!\!x\(\sb{1}\) \(\dots\) x\(\sb{i}\).\m{f\sb{2}[}fun n \m{t\sb{1} \dots t\sb{i},} n\(,\) x\(\sb{1}\)\(,\ldots,\,\)x\(\sb{i}]\)
\end{alltt}\end{hol}

\noindent It follows immediately from this theorem that the constant \ml{fun}
satisfies the primitive recursive defining equations given by the theorem shown
below:

\begin{hol}\begin{alltt}
   |- fun 0 x\(\sb{1}\) \(\dots\) x\(\sb{i}\) = \m{f\sb{1}[}x\(\sb{1}\)\(,\ldots,\,\)x\(\sb{i}]\)
      fun (SUC n) x\(\sb{1}\) \(\dots\) x\(\sb{i}\) = \m{f\sb{2}[}fun n \m{t\sb{1} \dots t\sb{i},} n\(,\) x\(\sb{1}\)\(,\ldots,\,\)x\(\sb{i}]\)
\end{alltt}\end{hol}

To automate the use of the primitive recursion theorem in deriving recursive
definitions of this kind, the \HOL\ system provides two functions which,
together, first do automatic proofs of the existence of primitive recursive
functions and then make constant specifications to introduce constants that
denote such functions:

\begin{boxed}
\index{new_prim_rec_definition@\ml{new\_prim\_rec\_definition}|pin}
\index{new_infix_prim_rec_definition@\ml{new\_infix\_prim\_rec\_definition}|pin}
\begin{verbatim}
   new_prim_rec_definition       : (string # term) -> thm
   new_infix_prim_rec_definition : (string # term) -> thm
\end{verbatim}\end{boxed}

\noindent Evaluating

\begin{hol}\begin{alltt}
   new_prim_rec_definition
    (`fun_DEF`,
     "(fun 0 x\(\sb{1}\) \(\dots\) x\(\sb{i}\) = \m{f\sb{1}[}x\(\sb{1}\)\(,\ldots,\,\)x\(\sb{i}]\) /\verb!\!
      (fun (SUC n) x\(\sb{1}\) \(\dots\) x\(\sb{i}\) = \m{f\sb{2}[}fun n \m{t\sb{1} \dots t\sb{i},} n\(,\) x\(\sb{1}\)\(,\ldots,\,\)x\(\sb{i}]\))");;
\end{alltt}\end{hol}

\noindent automatically proves the theorem:

\begin{hol}\begin{alltt}
   |-  ?fun. !x\(\sb{1}\) \(\dots\) x\(\sb{i}\). fun 0 x\(\sb{1}\) \(\dots\) x\(\sb{i}\) = \m{f\sb{1}[}x\(\sb{1}\)\(,\ldots,\,\)x\(\sb{i}]\) /\verb!\!
             !x\(\sb{1}\) \(\dots\) x\(\sb{i}\). fun (SUC n) x\(\sb{1}\) \(\dots\) x\(\sb{i}\) = \m{f\sb{2}[}fun n \m{t\sb{1} \dots t\sb{i},} n\(,\) x\(\sb{1}\)\(,\ldots,\,\)x\(\sb{i}]\)
\end{alltt}\end{hol}

\noindent and then declares a new constant \ml{fun} with this property as its
specification. This constant specification is returned as a theorem by
\ml{new\_prim\_rec\_definition} and is saved with name \ml{fun\_DEF}\index{DEF@$\ldots$\ml{\_DEF}}
in the current theory segment. Failure occurs if:

\begin{myenumerate}
\item \HOL\ cannot prove there is a function satisfying the specification
(\ie\ if the term supplied to \ml{new\_prim\_rec\_definition}
 is not a well-formed primitive recursive definition);
\item any other condition for making a constant specification is violated
(see the failure conditions for \ml{new\_specification} in
Section~\ref{conspec}).
\end{myenumerate}

The \ML\  function \ml{new\_prim\_rec\_definition}  is, in  fact, slightly more
general than  is  indicated  by  the  example  application  shown  above.    In
particular, a  curried\index{currying, in ML@currying, in \ML!in primitive recursive definitions}
primitive  recursive  function defined  using this \ML\
function can be defined  by primitive  recursion on  any one  of its arguments.
For example, a curried  addition function  \ml{plus:num->num->num} can be
defined by primitive recursion on its first argument:

\setcounter{sessioncount}{1}
\begin{session}\begin{verbatim}
#let PLUS = new_prim_rec_definition
#           (`PLUS`,
#            "(plus 0 n = n) /\
#             (plus (SUC m) n = SUC(plus m n))");;
PLUS = |- (!n. plus 0 n = n) /\ (!m n. plus(SUC m)n = SUC(plus m n))
\end{verbatim}\end{session}

\noindent or by primitive recursion on its second argument:

\setcounter{sessioncount}{1}
\begin{session}\begin{verbatim}
#let PLUS = new_prim_rec_definition
#            (`PLUS`,
#             "(plus m 0 = m) /\
#              (plus m (SUC n) = SUC(plus m n))");;
PLUS = |- (!m. plus m 0 = m) /\ (!m n. plus m(SUC n) = SUC(plus m n))
\end{verbatim}\end{session}

The \ML\ function \ml{new\_prim\_rec\_definition} also allows the user to
partially specify the value of a function defined (possibly recursively) on the
natural numbers by giving its value for only one of \ml{0} or \ml{SUC n}.  For
example, a decrement function \ml{DEC}, whose value is specified for only
positive natural numbers, can be defined using \ml{new\_prim\_rec\_definition}
as follows

\setcounter{sessioncount}{1}
\begin{session}\begin{verbatim}
#let DEC = new_prim_rec_definition
#            (`DEC`, "DEC (SUC n) = n");;
DEC = |- !n. DEC(SUC n) = n
\end{verbatim}\end{session}

\noindent This definition specifies the value of the function \ml{DEC} only for
positive natural numbers. In particular, the value of \ml{DEC 0} is left
unspecified, and the only non-trivial property that can be proved to hold of
the constant \ml{DEC} is the property stated by the theorem returned by the
call to \ml{new\_prim\_rec\_definition} shown in the session above.

The function  \ml{new\_infix\_prim\_rec\_definition}\index{new_infix_prim_rec_definition@\ml{new\_infix\_prim\_rec\_definition}}
 can  be used  to define an
infixed\index{infixes, in HOL logic@infixes, in \HOL\ logic!primitive recursive definitions}
 function by primitive recursion on the natural numbers.  It takes its
arguments in the same form as \ml{new\_prim\_rec\_definition} and has similar
failure conditions; the only difference is that the resulting function constant
has infix status.  Here, for example, is the recursive definition of the
constant \ml{+} used by the system:

\begin{hol}\begin{verbatim}
   new_infix_prim_rec_definition
    (`ADD`,
     "($+ 0 n = n) /\
      ($+ (SUC m) n = SUC($+ m n))")
\end{verbatim}\end{hol}

\noindent The  {\small\verb%$%}'s  are  there (as documentation)
  to  indicate  that  the constant
{\small\verb%+%}\index{ addition, in HOL logic@\ml{+} (addition, in \HOL\ logic)}
is  being  declared  to  be  an  infix.   Evaluating this \ML\
expression will  create  the  following  constant specification  in the current
theory:

\begin{hol}\index{ADD@\ml{ADD}}
\begin{verbatim}
   ADD   |- (!n. 0 + n = n) /\ (!m n. (SUC m) + n = SUC(m + n))
\end{verbatim}\end{hol}

For further details about recursive definitions, see
Section~\ref{prim-rec-defs}, or the \REFERENCE\ documentation
on
\ml{new\_prim\_rec\_definition} and \ml{new\_infix\_prim\_rec\_definition}.
\index{prim_rec@\ml{prim\_rec}|)}
\index{primitive recursion theorem!automated use of, in HOL system@automated use of, in \HOL\ system|)}
\index{primitive recursion theorem!for numbers|)}
\index{primitive recursive definitions, in HOL logic@primitive recursive definitions, in \HOL\ logic!automated|)}
\index{recursive definitions, in HOL logic@recursive definitions, in \HOL\ logic!automated, for numbers|)}
\index{type definitions, in HOL logic@type definitions, in \HOL\ logic!primitive recursive|)}

\subsubsection{The less-than relation}

The less-than relation `{\small{\tt\verb+<+}}'\index{ less than, in HOL
  logic@\ml{<} (less than, in \HOL\ logic)} is most naturally defined by
  primitive recursion. However, it is needed for the proof of the
  primitive recursion theorem, so it must be defined before definition
  by primitive recursion is available. The theory \ml{prim\_rec}
  therefore contains the following non-recursive definition\index{less
  than, in HOL logic@less than, in \HOL\ logic} of \ml{<}:

\begin{hol}
\index{LESS@\ml{LESS}}
\begin{verbatim}
   LESS  |- !m n. m < n = (?P. (!n. P(SUC n) ==> P n) /\ P m /\ ~P n)
\end{verbatim}\end{hol}

\noindent
This definition says that {\small\verb%m < n%} if there exists a set (with
characteristic function {\small\verb%P%}) that is downward
closed\footnote{A set of numbers is {\it downward closed\/} if whenever it
contains the successor of a number, it also contains the number.} and
contains {\small\verb%m%} but not {\small\verb%n%}.

\subsubsection{Consequences of primitive recursion}

Once the primitive recursion theorem is available, other useful theorems
can be proved. The theory \ml{prim\_rec} supplies the Axiom of Dependent
Choice, which is a theorem in \HOL\ because it follows from
{\small\verb+SELECT_AX+}:

\begin{hol}
\index{Axiom of Dependent Choice@\ml{DC}}
\begin{verbatim}
   DC  |- !P R a.
            P a /\ (!x. P x ==> ?y. P y /\ R x y)
             ==>
           ?f. (f 0 = a) /\ !n. P (f n) /\ R (f n) (f (SUC n))
\end{verbatim}\end{hol}

The theorem {\small\verb+DC+} is useful when one wishes to build a
function having a certain property from a relational
characterization. An alternate characterization of wellfoundedness is
the absence of infinite decreasing $R$ chains. By use of {\small\verb+DC+},
this can be proved to be equal to the the notion of wellfoundedness
(namely, that every set has an $R$-minimal element) defined in the
theory {\small\verb+relation+}.

\begin{hol}
\index{Wellfounded@\ml{Wellfounded}}
\begin{verbatim}
   wellfounded_def    |- wellfounded (R:'a->'a->bool) 
                            = 
                         ~?f. !n. R (f (SUC n)) (f n)

   WF_IFF_WELLFOUNDED |- !R. WF R = wellfounded R
\end{verbatim}\end{hol}

The theory \ml{prim\_rec} also provides theorems asserting the
wellfoundedness of the predecessor relation and the less-than relation,
as well as the wellfoundedness of measure functions.

\begin{hol}
\index{WF_PRED\ml{WF_PRED}}
\index{WF_LESS@\ml{WF_LESS}}
\index{measure_def@\ml{measure_def}}
\index{WF_measure@\ml{WF_measure}}
\begin{verbatim}
   WF_PRED     |- WF (\x y. y = SUC x)

   WF_LESS     |- WF $<

   measure_def |- measure = inv_image $<

   WF_measure  |- !m. WF (measure m)
\end{verbatim}\end{hol}


\subsection{The theory {\tt arithmetic}}

The built-in theory {\small\verb%arithmetic%}\index{number theory, in HOL logic@number theory, in \HOL\ logic}
\index{arithmetic@\ml{arithmetic}} contains primitive recursive
definitions of following standard arithmetic operators.

\begin{hol}
\index{ADD@\ml{ADD}}
\index{SUB@\ml{SUB}}
\index{MULT@\ml{MULT}}
\index{EXP@\ml{EXP}}
\index{ subtraction, in HOL logic@\ml{-} (subtraction, in \HOL\ logic)}
\index{ multiplication, in HOL logic@\ml{*} (multiplication, in \HOL\ logic)}
\begin{verbatim}
   ADD            |- (!n. 0 + n = n) /\
                     (!m n. (SUC m) + n = SUC(m + n))

   SUB            |- (!m. 0 - m = 0) /\
                     (!m n. (SUC m) - n = (m < n => 0 | SUC(m - n)))

   MULT           |- (!n. 0 * n = 0) /\
                     (!m n. (SUC m) * n = (m * n) + n)

   EXP            |- (!m. m EXP 0 = 1) /\
                     (!m n. m EXP (SUC n) = m * (m EXP n))
\end{verbatim}\end{hol}

\noindent It also contains the following non-recursive definitions.

\begin{hol}
\index{arithmetic, in HOL logic@arithmetic, in \HOL\ logic}
\index{ greater than, in HOL logic@\ml{>} (greater than, in \HOL\ logic)}
\index{ less or equal, in HOL logic@\ml{<=} (less or equal, in \HOL\ logic)}
\index{ greater or equal, in HOL logic@\ml{>=} (greater or equal, in \HOL\ logic)}
\index{MOD@\ml{MOD}}
\index{DIV@\ml{DIV}}
\begin{verbatim}
   GREATER        |- !m n. m > n = n < m

   LESS_OR_EQ     |- !m n. m <= n = m < n \/ (m = n)

   GREATER_OR_EQ  |- !m n. m >= n = m > n \/ (m = n)

   DIVISION       |- !n. 0 < n ==> (!k. (k = ((k DIV n) * n) + (k MOD n)) /\
                                        (k MOD n) < n)
\end{verbatim}\end{hol}

An \adhoc\ but useful collection of over a hundred elementary theorems of
arithmetic are pre-proved when \HOL\ is built and stored in the theory
{\small\verb%arithmetic%}.  Each theorem will be autoloaded when its name is
first mentioned during any \HOL\ session. For a complete list of available
theorems, see \REFERENCE.

\subsection{The theory {\tt numeral}}\label{numeral}\index{numeral,
the construction of in HOL logic@\ml{numeral}, the construction of in \HOL\ logic}

The type \ml{:num}\index{num, the type in \HOL\
  logic@\ml{num}, the type in \HOL\ logic}, is typically thought of as
  being supplied with an infinite collection of numerals: \ml{1},
  \ml{2}, \ml{3}, \etc.  However, the \HOL\ logic has no way to define
  such infinite families of constants, thus all numerals other than $0$ 
  are actually built up from the constants introduced by the following 
  definitions:
\begin{itemize}
\item $\vdash \forall x.\, \ml{NUMERAL}(x) = x$
\item $\vdash \forall x.\, \ml{NUMERAL\_BIT1}(x) = 2x + 1$
\item $\vdash \forall x.\, \ml{NUMERAL\_BIT2}(x) = 2x + 2$
\item $\vdash \ml{ALT\_ZERO} = 0$
\end{itemize}

\noindent For example, the numeral $5$ is represented by the term
\[
   \ml{NUMERAL}(\ml{NUMERAL\_BIT1}(\ml{NUMERAL\_BIT2}(\ml{ALT\_ZERO})))
\]
but the \HOL\ parser and pretty-printer make such terms
appear as numerals. This binary representation for numerals allows for
asymptotically efficient calculation. Theorems supporting arithmetic 
calculations involving numerals can be found in the \ml{numeral} theory;
these are mechanized by the \ml{reduce} library.

Numerals are related to numbers via the derived inference rule
\ml{num\_CONV}, found in the \ml{numLib} library.

\begin{boxed}
\index{num_CONV@\ml{num\_CONV}|pin}
\begin{verbatim}
   numLib.num_CONV : term -> thm
\end{verbatim}\end{boxed}

\noindent \ml{num\_CONV} can be used to generate the ``\ml{SUC}''
equation for any non-zero numeral.  For example:

\setcounter{sessioncount}{1}
\begin{session}\begin{verbatim}
- load "numLib"; open numLib;

- num_CONV (Term `1`);
> val it = |- 1 = SUC 0 : thm

- num_CONV (Term `2`);
> val it = |- 2 = SUC 1 : thm

- num_CONV (Term `3141592653`);
> val it = |- 3141592653 = SUC 3141592652 : thm
\end{verbatim}\end{session}

\noindent The \ml{num\_CONV} function works purely by inference, using
the definitions provided above.\footnote{In previous versions of \HOL,
\ml{num\_CONV} would not prove its result, which was not in keeping with
the \LCF\ approach, and which moreover made \ml{num\_CONV} dependent on
the underlying implementation of numbers. This made \ml{num\_CONV}
incomplete on \ML\ systems without arbitrary-sized numbers.}

When other numeric theories are loaded (such as those for the reals or
integers), numerals are overloaded so that the numeral {\small\verb+1+} can
actually stand for a natural number, an integer or a real value.  In
order to precisely specify the desired type, the user can use single
character suffixes (`\ml{n}' for the natural numbers, and `\ml{i}' for
the integers): 

\begin{session}
\begin{verbatim}
- load "integerTheory";
> val it = () : unit

- Term`2`;
<<HOL message: more than one resolution of overloading was possible.>>
> val it = `2` : term

- type_of it;
> val it = `:int` : hol_type

- Term`2n`;
> val it = `2` : term

- type_of it;
> val it = `:num` : hol_type

- type_of (Term `42i`);
> val it = `:int` : hol_type
\end{verbatim}\end{session}

Question: what does $42i$ mean? There's a different map for the
integers. Need to discuss it in the theory of integers.

Question: are the character suffixes recognized by the direct
construction routines?

Question: how are new character suffixes created?

\section{Integers}
\section{Real numbers and analysis}

\section{Lists}

\subsection{The theory {\tt list}}\label{avra_list}\index{list, the type operator in HOL logic@\ml{list}, the type operator in \HOL\ logic}

\index{recursive definitions, in HOL logic@recursive definitions, in \HOL\ logic!automated for lists|(}
\index{types, in HOL logic@types, in \HOL\ logic!tools for construction of}
\index{lists, in HOL logic@lists, in \HOL\ logic|(}
\index{list theory, in HOL logic@\ml{list} theory, in \HOL\ logic|(}
\index{ lists, in HOL logic@\ml{[} $\cdots$ \ml{;} $\cdots$ \ml{]} (lists, in \HOL\ logic)|(}
The theory \ml{list} introduces the unary type operator \ml{list} by a type
definition.\footnote{For details of the definition,
see~\cite{HOL-paper,Melham-banff}.}  The standard list processing functions
are then defined on this type:

\begin{hol}
\index{NIL@\ml{NIL}}
\index{CONS@\ml{CONS}}
\index{HD, the constant in HOL logic@\ml{HD}, the constant in \HOL\ logic}
\index{TL, the constant in HOL logic@\ml{TL}, the constant in \HOL\ logic}
\index{NULL, the constant in HOL logic@\ml{NULL}, the constant in \HOL\ logic}
\begin{verbatim}
   NIL  : 'a list
   CONS : 'a -> 'a list -> 'a list
   HD   : 'a list -> 'a
   TL   : 'a list -> 'a list
   NULL : 'a list -> bool
\end{verbatim}\end{hol}

The \HOL\ parser\index{parsing, of HOL logic@parsing, of \HOL\ logic!of list expressions} has been specially  modified to  parse the expression
 {\small\verb%[]%} into
{\small\verb%NIL%} and to parse the expression
{\small\verb%[%}$t_1${\small\verb%;%}$t_2${\small\verb%;%}$\ldots${\small\verb%;%}$t_n${\small\verb%]%}
into {\small\verb%CONS %}$t_1${\small\verb% (CONS %}$t_2 \cdots\
${\small\verb%(CONS %}$t_n${\small\verb%  NIL)%}$\  \cdots\ ${\small\verb%)%}.
The \HOL\ printer\index{printing, in HOL logic@printing, in \HOL\ logic!of list expressions}
 reverses these transformations.

The functions \ml{NIL} and \ml{CONS} are defined in terms of the representing
type of lists. From their definitions, the following fundamental theorems about
lists\index{list theorems, in HOL logic@list theorems, in \HOL\ logic|(}
are proved and stored in the theory \ml{list}.

\begin{hol}
\index{list_Axiom@\ml{list\_Axiom}}
\index{axioms!non-primitive, of HOL logic@non-primitive, of \HOL\ logic}
\index{axioms!in list theory@in \ml{list} theory}
\index{list_INDUCT@\ml{list\_INDUCT}}
\index{list_CASES@\ml{list\_CASES}}
\index{CONS_11@\ml{CONS\_{11}}}
\index{NOT_NIL_CONS@\ml{NOT\_NIL\_CONS}}
\index{NOT_CONS_NIL@\ml{NOT\_CONS\_NIL}}
\index{characterizing theorem!for lists}
\begin{verbatim}
   list_Axiom    |- !x f. ?!fn.(fn[] = x) /\ (!h t. fn(CONS h t) = f(fn t)h t)

   list_INDUCT   |- !P. P[] /\ (!t. P t ==> (!h. P(CONS h t))) ==> (!l. P l)

   list_CASES    |- !l. (l = []) \/ (?t h. l = CONS h t)

   CONS_11       |- !h t h' t'. (CONS h t = CONS h' t') = (h = h') /\ (t = t')

   NOT_NIL_CONS  |- !h t. ~([] = CONS h t)

   NOT_CONS_NIL  |- !h t. ~(CONS h t = [])
\end{verbatim}\end{hol}

The theorem  \ml{list\_Axiom}  shown  above  is  analogous   to  the  primitive
recursion theorem\index{primitive recursion theorem!for lists} on the
natural numbers discussed above in
Section~\ref{num-prim-rec}.  It  states  the  validity  of  primitive recursive
definitions on lists, and can be used to justify any such definition.  The \ML\
function

\begin{boxed}
\index{new_recursive_definition@\ml{new\_recursive\_definition}|pin}
\begin{verbatim}
   new_recursive_definition : fixity -> thm -> string -> term -> thm
\end{verbatim}\end{boxed}

\noindent uses this theorem to do automatic\index{primitive recursion theorem!automated use of, in HOL system@automated use of, in \HOL\ system}
 proofs of the existence of primitive
recursive functions on lists and then make constant specifications to introduce
constants that  denote  such  functions.  For example, the \HOL\ system defines
a length function, \ml{LENGTH}, on lists by the primitive recursive
definition on lists shown below:

\begin{hol}\begin{verbatim}
   new_recursive_definition Prefix list_Axiom "LENGTH"
      (Term `(LENGTH NIL = 0) /\
             (!h t. LENGTH (h::t) = SUC (LENGTH t))`)
\end{verbatim}\end{hol}

\noindent When this \ML\
expression is evaluated, \HOL\ uses \ml{list\_Axiom} to prove existence
of a function that satisfies the given primitive recursive definition,
introduces a constant to name this function using a constant specification, and
stores the resulting theorem:

\begin{hol}\begin{verbatim}
    LENGTH   |- (LENGTH[] = 0) /\ (!h t. LENGTH(h::t) = SUC(LENGTH t))
\end{verbatim}\end{hol}

\noindent in the current theory (in this case, the theory \ml{list}).

The predicate \ml{NULL}  and  the
selectors \ml{HD} and  \ml{TL} are  defined\index{list definitions, in HOL logic@list definitions, in \HOL\ logic}
 in  the theory  \ml{list} by the
specifications:


\begin{hol}
\index{NULL, the definition in HOL logic@\ml{NULL}, the definition in \HOL\ logic}
\index{HD, the definition in HOL logic@\ml{HD}, the definition in \HOL\ logic}
\index{TL, the definition in HOL logic@\ml{TL}, the definition in \HOL\ logic}
\begin{verbatim}
   NULL |- NULL[] /\ (!h t. ~NULL(CONS h t))

   HD   |- !h t. HD(CONS h t) = h

   TL   |- !h t. TL(CONS h t) = t
\end{verbatim}\end{hol}

\noindent The following primitive recursive definitions of functions on lists
are also made in the theory \ml{list}:


\begin{hol}
\index{SUM, the theorem in HOL logic@\ml{SUM}, the theorem in \HOL\ logic}
\index{APPEND, the theorem in HOL logic@\ml{APPEND}, the theorem in \HOL\ logic}
\index{concatenation, of lists!in HOL logic@in \HOL\ logic}
\index{FLAT, the theorem in HOL logic@\ml{FLAT}, the theorem in \HOL\ logic}
\index{LENGTH, the theorem in HOL logic@\ml{LENGTH}, the theorem in \HOL\ logic}
\index{MAP, the theorem in HOL logic@\ml{MAP}, the theorem in \HOL\ logic}
\index{EL, the theorem in HOL logic@\ml{EL}, the theorem in \HOL\ logic}

\index{SUM, the constant in HOL logic@\ml{SUM}, the constant in \HOL\ logic}
\index{APPEND, the constant in HOL logic@\ml{APPEND}, the constant in \HOL\ logic}
\index{FLAT, the constant in HOL logic@\ml{FLAT}, the constant in \HOL\ logic}
\index{LENGTH, the constant in HOL logic@\ml{LENGTH}, the constant in \HOL\ logic}
\index{MAP, the constant in HOL logic@\ml{MAP}, the constant in \HOL\ logic}
\index{EL, the constant in HOL logic@\ml{EL}, the constant in \HOL\ logic}
\index{EVERY, the HOL constant@\ml{EVERY}, the \HOL\ constant}
\index{EVERY_DEF@\ml{EVERY\_DEF}}
\begin{verbatim}
   SUM        |- (SUM [] = 0) /\ (!h t. SUM(h::t) = h + SUM t)

   APPEND     |- (!l. APPEND [] l = l) /\
                 (!l1 l2 h. APPEND (h::l1) l2 = h::APPEND l1 l2)

   FLAT       |- (FLAT[] = []) /\ (!h t. FLAT(h::t) = APPEND h (FLAT t))

   LENGTH     |- (LENGTH [] = 0) /\ (!h t. LENGTH (h::t) = SUC(LENGTH t))

   MAP        |- (!f. MAP f [] = []) /\
                 (!f h t. MAP f (h::t) = f h::MAP f t)

   EL         |- (!l. EL 0 l = HD l) /\ (!l n. EL (SUC n)l = EL n (TL l))

   EVERY_DEF  |- (!P. EVERY P [] = T) /\
                 (!P h t. EVERY P (h::t) = P h /\ EVERY P t)
\end{verbatim}\end{hol}

\noindent For a complete list of available theorems in
{\small\verb+listTheory+}, see \REFERENCE. 

\index{list theorems, in HOL logic@list theorems, in \HOL\ logic|)} \index{
lists, in HOL logic@\ml{[} $\cdots$ \ml{;} $\cdots$ \ml{]} (lists, in \HOL\
logic)|)} \index{list theory, in HOL logic@\ml{list} theory, in \HOL\ logic|)}
\index{lists, in HOL logic@lists, in \HOL\ logic|)} \index{recursive
definitions, in HOL logic@recursive definitions, in \HOL\ logic!automated for
lists|)}

\subsection{The theory {\tt rich\_list}}\label{rich_list}

The theory {\small\verb+list+} is relatively compact, largely because of
how the system developed. An extensive theory of lists can be found in 
{\small\verb+rich_listTheory+}, in which the following
well-known functions over lists are defined:

\begin{hol}
\begin{verbatim}
\end{verbatim}\end{hol}

A gazillion theorems are proved about them too.

\section{Trees}

The theories {\tt tree} and {\tt ltree} \index{recursive
definitions, in HOL logic@recursive definitions, in \HOL\ logic!automated, for
trees|(}

\index{labelled tree theory, in HOL logic@labelled \ml{tree} theory, in \HOL\
logic|(} The theories \ml{tree}\index{tree, the HOL theory@\ml{tree}, the \HOL\
theory}
 and \ml{ltree}\index{ltree, the HOL theory@\ml{ltree}, the \HOL\ theory|(}
 contain the definitions of two structurally-isomorphic types of
finitely-branching ordered trees.  The types defined in these theories are used
by Tom Melham's\index{Melham, T.} type definition package (see
Section~\ref{types-package}) to construct representations for arbitrary
concrete recursive types.  The following is a summary of the main
theorems which are available in the theories \ml{tree} and \ml{ltree},
and which may be of use in certain specialized applications.  For full
details of the logical basis for these two theories, see~\cite{Melham-banff}.

\subsection{The theory {\tt tree}}

In the theory \ml{tree}, a type \ml{tree}\index{tree, the HOL type@\ml{tree},
the \HOL\ type} is defined to denote the set of all ordered trees whose nodes
can branch any (finite) number of times.  A constructor function

\index{node@\ml{node}} \begin{hol}\begin{verbatim}
   node : (tree)list -> tree \end{verbatim}\end{hol}

\noindent is then defined in the theory \ml{tree}.  This function can be used
to construct any tree-structured value of type \ml{tree}.  The expression
\ml{"node []"} denotes the tree consisting of a single leaf node with no
subtrees.  If $tl\ml{:(tree)list}$ is a non-empty list of trees, then the term
$\ml{"node }tl\ml{"}$ denotes the tree whose immediate subtrees are the trees
in the list $tl$.  Using \ml{node}, it is possible to construct a tree of any
shape.  For example, the tree

\begin{center}
{\setlength{\unitlength}{0.75mm}
\begin{picture}(60,25)
\thicklines
\put(10,10){\makebox(0,0){$\bullet$}}
\put(30,10){\makebox(0,0){$\bullet$}}
\put(50,10){\makebox(0,0){$\bullet$}}
\put(30,20){\makebox(0,0){$\bullet$}}
\put(44,0){\makebox(0,0){$\bullet$}}
\put(56,0){\makebox(0,0){$\bullet$}}
\put(30,20){\line(0,-1){10}}
\put(30,20){\line(2,-1){20}}
\put(30,20){\line(-2,-1){20}}
\put(50,10){\line(3,-5){6}}
\put(50,10){\line(-3,-5){6}}
\end{picture}}
\end{center}

\noindent is denoted by the term
\ml{"node[node[]; node[]; node[node[]; node[]]"}.

The next two theorems follow from the formal definition of \ml{node} and
are stored in the theory \ml{tree}:

\begin{hol}
\index{node_11@\ml{node\_{11}}}
\index{tree_Induct@\ml{tree\_Induct}}
\index{induction rule!for trees, in HOL logic@for \ml{trees}, in \HOL\ logic}
\begin{verbatim}
   node_11      |- !tl1 tl2. (node tl1 = node tl2) = (tl1 = tl2)
   tree_Induct  |- !P. (!tl. EVERY P tl ==> P(node tl)) ==> (!t. P t)
\end{verbatim}\end{hol}

\noindent These  two  theorems are  analogous to  the Peano\index{Peano's axioms}  postulates for the
natural numbers, and are used to prove the  following abstract characterization
of the defined type \ml{tree}.


\begin{hol}
\index{tree_Axiom@\ml{tree\_Axiom}}
\index{characterizing theorem!for finitely-branching ordered trees}
\begin{verbatim}
   tree_Axiom  |- !f. ?! fn. !tl. fn(node tl) = f(MAP fn tl)tl
\end{verbatim}\end{hol}

\noindent This theorem states the validity of general `primitive recursive'
definitions of functions over finitely-branching ordered trees.

\subsection{The theory {\tt ltree}}

In the theory \ml{ltree}\index{ltree@$\ldots$\ml{ltree}}, a type of {\it
labelled\/} trees (called \ml{'a ltree}) is defined. Labelled trees have the
same sort of structure as values of the defined type \ml{tree} discussed above.
The only difference is that a tree of type \ml{'a ltree} has a value or `label'
of type \ml{'a} associated with each of its nodes.  A constructor

\begin{hol}
\index{Node@\ml{Node}}
\begin{verbatim}
   Node : 'a -> 'a ltree list -> 'a ltree
\end{verbatim}\end{hol}

\noindent is defined in the theory \ml{ltree}.  The function \ml{Node}
constructs labelled trees by mapping a label of type \ml{'a} and a list of
labelled subtrees to a labelled tree of type \ml{'a ltree}.
The following theorems  about labelled trees are pre-proved and stored in the
theory \ml{ltree}.


\begin{hol}
\index{Node_11@\ml{Node\_{11}}}
\index{ltree_Induct@\ml{ltree\_Induct}}
\index{ltree_Axiom@\ml{ltree\_Axiom}}
\begin{verbatim}
   Node_11       |- !v1 v2 trl1 trl2.
                    (Node v1 trl1 = Node v2 trl2) = (v1 = v2) /\ (trl1 = trl2)
   ltree_Induct  |- !P. (!t. EVERY P t ==> (!h. P(Node h t))) ==> (!l. P l)
   ltree_Axiom   |- !f. ?! fn. !v tl. fn(Node v tl) = f(MAP fn tl)v tl
\end{verbatim}\end{hol}

\noindent These  theorems  are  analogous to  their counterparts  in the theory
\ml{tree} discussed  above.    The theorems  \ml{Node\_{11}}
and \ml{ltree\_Induct} amount to a Peano-type characterization of labelled
trees, and the theorem \ml{ltree\_Axiom} is a primitive recursion theorem for
labelled trees.
\index{labelled tree theory, in HOL logic@labelled \ml{tree} theory, in \HOL\ logic|)}
\index{ltree, the HOL theory@\ml{ltree}, the \HOL\ theory|)}
\index{recursive definitions, in HOL logic@recursive definitions, in \HOL\ logic!automated, for trees|)}


\section{Partial functions and the {\tt option} type}
\section{Finite maps}
\section{Strings}
\subsection{The theory {\tt ascii}}
\subsection{The theory {\tt string}}
\section{Sets}
\subsection{The theory {\tt set}}
\subsection{The theory {\tt pred\_set}}
\subsection{The theory {\tt finset}}
\subsection{Syntax for sets}\index{set theory notation}

The special purpose set-theoretic notations
{\small\verb%""%} and
{\small\verb%"{%}$t${\small\verb%|%}$p${\small\verb%}"%} are available.
The normal interpretation of the former is the finite set containing
$t_1,t_2,\ldots, t_n$ and the normal interpretation of the latter
is the set of all $t$s such that $p$.

For example,
\begin{hol}\begin{verbatim}
   "{1,2,3,4}"
\end{verbatim}\end{hol}

\noindent parses to

\begin{hol}\begin{verbatim}
   "1 INSERT (2 INSERT (3 INSERT (4 INSERT EMPTY)))"
\end{verbatim}\end{hol}

\noindent Similarly, {\small\verb%"{%}$t${\small\verb%|%}$p${\small\verb%}"%}
parses to:

\medskip

\noindent{\small
{\verb%   `%}}{\small{\verb%GSPEC(\(%}$x_1${\verb%,%}$\ldots${\verb%,%}$x_n${\verb%).(%}$t${\verb%,%}$p${\verb%))"%}
}

\medskip

\noindent where $x_1$, $\ldots$ , $x_n$ are the free variables occurring in both $t$
and $p$.  If there are no such free variables then an error results.
The order in which the variables are listed in the variable structure
of the paired abstraction is an unspecified function of the structure
of $t$ (it is approximately left to right).

For example,

\begin{hol}\begin{verbatim}
   "{x+y | (x < y) /\ (y < z)}"
\end{verbatim}\end{hol}

\noindent parses to:

\begin{hol}\begin{verbatim}
   "GSPEC(\(x,y). ((x+y), (x < y) /\ (y < z)))"
\end{verbatim}\end{hol}

\noindent where \ml{GSPEC} is defined by:

\begin{hol}\begin{verbatim}
   |- !f. GSPEC f = SPEC(\x. ?y. x,T = f y)
\end{verbatim}\end{hol}

\noindent and \ml{SPEC} abstracts a predicate to a set (it is the abstraction
bijection used in the definition of the type operator \ml{set}).
Other examples are:

\begin{hol}\begin{verbatim}
   "{x+y+z | (x < y) /\ (y < z)}"
\end{verbatim}\end{hol}

\noindent will parse to:

\begin{hol}\begin{verbatim}
   "GSPEC(\(x,y,z). (x+(y+z), (x < y /\ y < z)))"
\end{verbatim}\end{hol}

\noindent and

\begin{hol}\begin{verbatim}
   "{x+y+w | (x < y) /\ (y < z)}"
\end{verbatim}\end{hol}

\noindent will parse to:

\begin{hol}\begin{verbatim}
   "GSPEC(\(x,y). (x+(y+w), (x < y /\ y < z)))"
\end{verbatim}\end{hol}

\section{Multisets}

\section{Restricted binders}\label{res-quant}

\index{types, in HOL logic@types, in \HOL\ logic!dependent}
\index{dependent types in HOL logic@dependent types in \HOL\ logic}
\index{quantifiers!restricted}

The theory \ml{res\_quan} contains the definitions of the constants
{\small\verb!RES_FORALL!}, {\small\verb!RES_EXISTS!},
{\small\verb!RES_SELECT!}, and {\small\verb!RES_ABSTRACT!},
which are used to support restricted
quantification\index{quantifiers!restricted} in the \HOL\ logic.  This
implements a method of simulating subtypes and dependent types with
predicates. The definitions are:

\begin{hol}
\index{RES_FORALL@\ml{RES\_FORALL}}
\index{RES_EXISTS@\ml{RES\_EXISTS}}
\index{RES_SELECT@\ml{RES\_SELECT}}
\index{RES_ABSTRACT@\ml{RES\_ABSTRACT}}
\index{ARB@\ml{ARB}}
\begin{verbatim}
   RES_FORALL   |- !P B. RES_FORALL P B = (!x. P x ==> B x)

   RES_EXISTS   |- !P B. RES_EXISTS P B = (?x. P x /\ B x)

   RES_SELECT   |- !P B. RES_SELECT P B = (@x. P x /\ B x)

   RES_ABSTRACT |- !P B. RES_ABSTRACT P B = (\x. if P x then B x else ARB)
\end{verbatim}\end{hol}


The new syntax automatically translates as follows:

\begin{hol}
{\small\verb%   \%}$v${\small\verb%::%}$P${\small\verb%.%}$B${\small\verb%    <---->   RES_ABSTRACT %}$P${\small\verb% (\%}$v${\small\verb%.%}$B${\small\verb%)%}\\
{\small\verb%   !%}$v${\small\verb%::%}$P${\small\verb%.%}$B${\small\verb%    <---->   RES_FORALL   %}$P${\small\verb% (\%}$v${\small\verb%.%}$B${\small\verb%)%}\\
{\small\verb%   ?%}$v${\small\verb%::%}$P${\small\verb%.%}$B${\small\verb%    <---->   RES_EXISTS   %}$P${\small\verb% (\%}$v${\small\verb%.%}$B${\small\verb%)%}\\
{\small\verb%   @%}$v${\small\verb%::%}$P${\small\verb%.%}$B${\small\verb%    <---->   RES_SELECT   %}$P${\small\verb% (\%}$v${\small\verb%.%}$B${\small\verb%)%}
\end{hol}

Anything can be written between the binder and `\ml{::}' that can be
written between the binder and `\ml{.}`. See the examples below.

Currently no derived rules are provided to support this notation, so
any inferences will need to work on the underlying semantic
representation.

User-defined binders can also have restricted forms, which are set up
with the function:

\begin{boxed}\index{associate_restriction@\ml{associate\_restriction}|pin}
\begin{verbatim}
   associate_restriction : (string * string) -> *
\end{verbatim}\end{boxed}


\noindent If \m{c} is the name
of a binder and \ml{RES\_}\m{c} is the name of a suitable constant (which
must be explicitly defined), then executing:

\begin{hol}
{\small\verb%   associate_restriction(`%}$c${\small\verb%`, `RES_%}$c${\small\verb%`)%}
\end{hol}

\noindent will cause the parser and pretty-printer to support:

\begin{hol}
{\small\verb%   %}$c$ $v${\small\verb%::%}$P${\small\verb%. %}$B${\small\verb%    <---->   RES_%}$c$ $P${\small\verb% (\%}$v${\small\verb%. %}$B${\small\verb%)%}
\end{hol}

Here is an example session:

\setcounter{sessioncount}{1}
\begin{session}\begin{verbatim}
#"!x y::P. x<y";;
"!x y :: P. x < y" : term

#set_flag(`print_restrict`, false);;
true : bool

#"!x y::P. x<y";;
"RES_FORALL P(\x. RES_FORALL P(\y. x < y))" : term

#"?(x,y) p::(\(m,n).m<n). p=(x,y)";;
"RES_EXISTS
 (\(m,n). m < n)
 (\(x,y). RES_EXISTS(\(m,n). m < n)(\p. p = x,y))"
: term

#"\x y z::P.[0;x;y;z]";;
"RES_ABSTRACT P(\x. RES_ABSTRACT P(\y. RES_ABSTRACT P(\z. [0;x;y;z])))"
: term
\end{verbatim}\end{session}

A conversion that rewrites away the constants \ml{RES\_ABSTRACT},
\ml{RES\_FORALL}, \ml{RES\_EXISTS} and \ml{RES\_SELECT} is:

\begin{hol}\begin{verbatim}
   let RESTRICT_CONV =
        (PURE_REWRITE_CONV [definition `bool` `RES_ABSTRACT`;
                            definition `bool` `RES_FORALL`;
                            definition `bool` `RES_EXISTS`;
                            definition `bool` `RES_SELECT`])
        THENC (DEPTH_CONV BETA_CONV)
\end{verbatim}\end{hol}

\noindent This is a bit unsatisfactory, as is shown by the artificial
example below:

\setcounter{sessioncount}{1}
\begin{session}\begin{verbatim}
#let t = "!x y::P.?f:num->num::Q. f(@n::R.T) = (x+y)";;
t = "!x y :: P. ?f :: Q. f(@n :: R. T) = x + y" : term

#RESTRICT_CONV t;;
|- (!x y :: P. ?f :: Q. f(@n :: R. T) = x + y) =
   (!x. P x ==> (!x'. P x' ==> (?x. Q x /\ (x(@x. R x /\ T) = x + x'))))
\end{verbatim}\end{session}

The variable $x$ in the definitions of the constants
\ml{RES\_ABSTRACT}, \ml{RES\_FORALL}, \ml{RES\_EXISTS} and
\ml{RES\_SELECT} gets confused with the variable in the supplied term.
This can be avoided by changing \ml{RESTRICT\_CONV} to perform
explicit alpha-conversion. For example, by implementing a conversion:

\begin{hol}
{\small\verb%   RES_FORALL %}$P${\small\verb% (\%}$v$\ml{.}$B$\ml{[}$v$\ml{])  ---->  !}$v$\ml{. }$P$ $v$\ml{ ==> }$B$\ml{[}$v$\ml{]}
\end{hol}

\noindent Dealing with the case when
$v$ is a variable structure is also desirable. For example:

\begin{session}\begin{verbatim}
#let t1 = "!(m,n)::P. m<n";;
t1 = "!(m,n) :: P. m < n" : term

#RESTRICT_CONV t1;;
|- (!(m,n) :: P. m < n) = (!x. P x ==> (\(m,n). m < n)x)

\end{verbatim}\end{session}

\noindent If anyone writes the desired conversions please let us know!

Here is an example of a user-defined restriction:

\begin{session}\begin{verbatim}
#new_binder_definition(`DURING`, "DURING(p:num#num->bool) = $!p");;
|- !p. $DURING p = $! p

#"DURING x::(m,n). p x";;
no restriction constant associated with DURING
skipping: x " ;; parse failed

#new_definition
# (`RES_DURING`, "RES_DURING(m,n)p = !x. m<=x /\ x<=n ==> p x");;
|- !m n p. RES_DURING(m,n)p = (!x. m <= x /\ x <= n ==> p x)

#associate_restriction(`DURING`,`RES_DURING`);;
() : void

#"DURING x::(m,n). p x";;
"DURING x :: (m,n). p x" : term

#set_flag(`print_restrict`,false);;
true : bool

#"DURING x::(m,n). p x";;
"RES_DURING(m,n)(\x. p x)" : term
\end{verbatim}\end{session}

