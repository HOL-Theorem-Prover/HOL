\section{\texorpdfstring{Embedding \HOL{} in \LaTeX{}}{Embedding HOL in \LaTeX{}}}

\index{LaTeX@\LaTeX!embedding HOL@embedding in \HOL{}|(}
%
When writing documents in \LaTeX{} about one's favourite \HOL{} development, one frequently wants to include pretty-printed terms, types and theorems from that development.
%
Done manually, this will typically require involved use of the \texttt{alltt} environment, and cutting and pasting from a HOL session or theory file.
%
The result is that one must also keep two copies of \HOL{} texts synchronised: if the \HOL{} development changes, the \LaTeX{} document should change as well.

\newcommand{\munge}{\texttt{munge.exe}}
\index{munging (producing LaTeX from HOL)@munging (producing \LaTeX{} from \HOL{})}
%
This manual, and error-prone process is not necessary: the standard \HOL{} distribution comes with a tool called \munge{} to automate the process, and to remove the duplicate copies of \HOL{} text.
%
(Strictly speaking, the distribution comes with a tool that itself creates \munge{}; see Section~\ref{sec:munger-creation} below.)
%
The basic philosophy is that a \LaTeX{} document can be written ``as normal'', but that three new \LaTeX{}-like commands are available to the author.

The commands are not really processed by \LaTeX{}: instead the source file must first be passed through the \munge{} filter.
%
For example, one might write a document called \texttt{article.htex}.
%
This document contains instances of the new commands, and cannot be processed as is by \LaTeX{}.
%
Instead one first runs
\begin{alltt}
   \munge < article.htex > article.tex
\end{alltt}
and then runs \LaTeX{} on \texttt{article.tex}.
%
One would probably automate this process with a makefile of course.

\subsection{Munging commands}
\label{sec:munging-commands}
% need to get backslashes conveniently obscures the real names,
% preventing the munger from seeing them, which will be useful when we
% run the munger over this document!
\newcommand{\holtm}{\texttt{\bs{}HOLtm}}
\newcommand{\holty}{\texttt{\bs{}HOLty}}
\newcommand{\holthm}{\texttt{\bs{}HOLthm}}
\paragraph{Before starting} In order to use the munger, one must ``include'' (use the \texttt{\bs{}usepackage} command) the \texttt{holtexbasic.sty} style-file, which is found in the HOL source directory \texttt{src/TeX}.

\bigskip
There are then three commands for inserting text corresponding to \HOL{} entities into \LaTeX{} documents: \holtm, \holty{} and \holthm.
%
Each takes one argument, specifying something of the corresponding \HOL{} type.
%
In addition, options can be specified in square brackets, just as would be done with a genuine \LaTeX{} command.
%
For example, one can write
\begin{alltt}
   \holtm{}[tt]\{P(SUC n) /\bs{} q\}
\end{alltt}
and one will get \[
  \texttt{$P$ (SUC $n$) $\land$ $q$}
\]
or something very close to it, appearing in the resulting document.\footnote{The output is a mixture of typewriter font and math-mode characters embedded in a \texttt{\bs{}texttt} block within an \texttt{\bs{}mbox}.}
%
Note how the spacing in the input (nothing between the \texttt{P} and the \texttt{SUC n}) is \emph{not} reflected in the output; this is because the input is parsed and pretty-printed with \HOL{}.
%
This means that if the \HOL{} input is malformed, the \munge{} program will report errors.
%
Note also how the system knows that \texttt{P}, \texttt{n} and \texttt{q} are variables, and that \texttt{SUC} is not.
%
This analysis would not be possible without having \HOL{} actually parse and print the term itself.

The default behaviours of each command are as follows:
\begin{description}
\item[\holty\{\textit{\mdseries{string}}\}]
%
\index{HOLty (munging command)@\holty{} (munging command)}
%
Parses the string argument as a type (the input must include the leading colon), and prints it.
%
The output is suited for inclusion in the normal flow of \LaTeX{} (it is an \texttt{\bs{}mbox}).
\item[\holtm\{\textit{\mdseries{string}}\}]
%
\index{HOLtm (munging command)@\holtm{} (munging command)}
%
Parses the string argument as a term, and prints it.
%
Again, the output is wrapped in an \texttt{\bs{}mbox}.

\paragraph{Important:} If the string argument includes a right-brace character
(\ie, the character \rb, which has ASCII code 125), then it must be
escaped by preceding it with a backslash~(\bs).  Otherwise, the
munger's lexer will incorrectly determine that the argument ends at
that right-brace character rather than at a subsequent one.
\item[\holthm\{\textit{\mdseries{thmspecifier}}\}]
%
\index{HOLthm (munging command)@\holthm{} (munging command)}
%
The argument should be of the form $\langle\mbox{\textit{theory}}\rangle\texttt{.}\langle{\mbox{\textit{theorem-name}}}\rangle$.
%
For example, \verb|\HOLthm{bool.AND_CLAUSES}|.
%
This prints the specified theorem with a leading turnstile.
%
\index{Datatype@\ml{Datatype}!printing in LaTeX@printing in \LaTeX{}}
However, as a special case, if the theorem specified is a ``datatype theorem'' (with a name of the form \texttt{datatype\_}$\langle$\textit{type-name}$\rangle$), a BNF-style description of the given type (one that has been defined with \ml{Datatype}) will be printed.
%
Datatype theorems with these names are automatically generated when \ml{Datatype} is run.
If the trace \texttt{EmitTeX:~print~datatypes~compactly} is set to 1 (see the \texttt{tr} option below) the description is printed in a more compact form.
Also, if the type is a collection of nullary constants (a type consisting of only ``enumerated constants''), then it will always be printed compactly.
When not compact, all of a type's constructors will appear on the same line, or each will be on a separate line.

By default, the output is \emph{not} wrapped in an \texttt{\bs{}mbox}, making it best suited for inclusion in an environment such as \texttt{alltt}.
(The important characteristics of the \texttt{alltt} environment are that it respects layout in terms of newlines, while also allowing the insertion of \LaTeX{} commands.
The \texttt{verbatim} environment does the former, but not the latter.)
\end{description}

\paragraph{Munging command options}
\index{munging (producing LaTeX from HOL)@munging (producing \LaTeX{} from \HOL{})!command options}
There are a great many options for controlling the behaviour of each of these commands.
%
Some apply to all three commands, others are specific to a subset.
%
\newcommand{\indentoption}{\gt\gt}
If multiple options are desired, they should be separated by commas. For example: \texttt{\holthm{}[nosp,p/t,\indentoption]\{bool.AND\_CLAUSES\}}.

\begin{description}
\item[\texttt{alltt}] Makes the argument suitable for inclusion in an \texttt{alltt} environment.
%
This is the default for \holthm.
\item[\texttt{case}] (Only for use with \holtm.)
%
Causes the string to be parsed in such a way that any embedded \texttt{case} terms are only partly parsed, allowing their input form to appear when they are output.
%
This preserves underscore-patterns, for example.

\item[\texttt{conj}$n$] (Only for use with \holthm.)
Extracts the $n^{\mbox{\scriptsize th}}$ conjunct of a theorem.
The conjuncts are numbered starting at $1$, not $0$.
For example,
\begin{verbatim}
  \HOLthm[conj3]{bool.AND_CLAUSES}
\end{verbatim}
extracts the conjunct $\vdash \texttt{F} \land t \iff \texttt{F}$.

\item[\texttt{def}] (Only for use with \holthm.)
%
Causes the theorem to be split into its constituent conjuncts, for each conjunct to have any outermost universal quantifiers removed, and for each to be printed on a line of its own.
%
The turnstiles usually printed in front of theorems are also omitted, and a special form of equality is printed for the top-level (``defining'') equality in each clause.
%
This works well with definitions (or characterising theorems) over multiple data type constructors, changing
\begin{alltt}
\(\vdash\) (FACT 0 = 1) \(\land\) (\(\forall\)\ensuremath{n}. FACT (SUC \ensuremath{n}) = SUC \ensuremath{n} * FACT \ensuremath{n})
\end{alltt}
into
\begin{alltt}
   FACT 0 \HOLTokenDefEquality{} 1
   FACT (SUC \ensuremath{n}) \HOLTokenDefEquality{} SUC \ensuremath{n} * FACT \ensuremath{n}
\end{alltt}
If the special equality is not desired, the option \texttt{nodefsym} can be used to turn this off.
The special equality symbol can also be redefined by changing the \LaTeX{} definition of the macro \ml{\bs{}HOLTokenDefEquality}.

There are two variations on the \texttt{def} option:
\begin{description}
\item[\texttt{spaceddef}:]%
\index{spaceddef, HOL in LaTeX option@\ml{spaceddef}, \HOL{} in \LaTeX{} option}%
This option adds extra blank lines between successive conjuncts of a definition.
\item[\ml{aligneddef}:]%
\index{aligneddef, HOL in LaTeX option@\ml{aligneddef}, \HOL{} in \LaTeX{} option}%
This option puts ampersands around the \ml{\bs{}HOLTokenDefEquality} macro call in the emitted \LaTeX{}.
If the \LaTeX{} environment is an \texttt{array} or similar, this can ensure a nice column-based layout for one's definitions.
\end{description}
Both options can be used together, but \ml{aligneddef} will not work if \ml{nodefsym} is also used; if this combination is really required, it would be better to temporarily redefine (use \ml{\bs{}renewcommand}) \ml{\bs{}HOLTokenDefEquality}.


\item[\texttt{depth}=$n$]
Causes printing to be done with a maximum print depth of $n$; see Section~\ref{sec:pretty-print-depth}.

\item[\texttt{K}] (Only for use with \holtm.)
%
The argument must be the name of a theorem (as per the \holthm{} command), and the theorem should be of the form
\[
\vdash f\;x\;t
\]
for some term $t$.
%
The command prints the term $t$.
%
\index{combinators, in HOL logic@combinators, in \HOL{} logic}%
\index{K, the HOL constant@\ml{K}, the \HOL{} constant}%
The expectation is that $f$ will be the combinator \holtxt{K} from \theoryimp{combin} (see Section~\ref{sec:combinTheory}), and that $x$ will be truth~(\holtxt{T}), allowing $t$ to be anything at all.
%
In this way, large complicated terms that are not themselves theorems (or even of boolean type), can be stored in \HOL{} theories, and then printed in \LaTeX{} documents.

\index{munging (producing LaTeX from HOL)@munging (producing \LaTeX{} from \HOL{})!math mode}
\index{math mode (in the LaTeX munger)@math mode (in the \LaTeX{} munger)}
\item[\texttt{m}$\mathit{space}$,\texttt{nomath}] The \texttt{m} option makes \HOL{} material be typeset in ``math-mode''.
In particular, the output of the pretty-printer will be modified so that newline characters are replaced by \texttt{\bs\bs} commands.
This then requires that the surrounding \LaTeX{} environment be array-like, so that the \texttt{\bs\bs} command will have the desired effect.

In addition, because raw spaces have minimal effect in math-mode (something like \texttt{f\textvisiblespace{}x} will be typeset as $f x$), math-mode munging also replaces spaces with math-mode macros.
By default, the command \texttt{\bs;\bs;} is used, but if the \texttt{m} option is followed by some characters, each is interpreted as a single-letter macro name, with each macro concatenated together to provide the space command that will be used.

For example, if the option is \texttt{m;}, then the spacing command will be \texttt{\bs;}.
If the option is \texttt{m;!}, then the spacing command will be \texttt{\bs;\bs!}.
The comma character cannot be used because it conflicts with parsing the list of options, but one can use \texttt{c} instead, so that the option \texttt{mc} will make the spacing command be \texttt{\bs,}.

The \texttt{m} option can be installed globally with the \texttt{-m} command-line option.
If this option is enabled globally, it can be cancelled on a case-by-case basis by using the \texttt{nomath} option.
The \texttt{nomath} option also takes precedence over any \texttt{m} options that might occur.

See also the discussion about math-mode munging in Section~\ref{sec:math-mode-munging} below.

\item[\texttt{merge}, \texttt{nomerge}] (For use with \holtm{} and \holthm.)
By default, the HOL pretty-printer is paranoid about token-merging, and will insert spaces between the tokens it emits to try to ensure that what is output can be read in again without error.
%
This behaviour can be frustrating when getting one's \LaTeX{} to look ``just so'', so it can be turned off with the \texttt{nomerge} option.

Additionally, this behaviour can be turned off globally with the \texttt{--nomergeanalysis} option to the munger.
%
If this has been made the default, it may be useful to occasionally turn the merge analysis back on for a particular term or theorem; this is done with the \texttt{merge} option.
%
(In interactive HOL, the token-merging analysis is controlled by a trace variable called \texttt{"pp\_avoids\_symbol\_merges"}.)


\item[\texttt{nodollarparens}] (For use with \holtm{} and \holthm.) Causes the default escaping of syntactic sugar to be suppressed.
\index{ dollar sign, in HOL logic parser@\ml{\$} (dollar sign, in \HOL{} logic parser)!as escape character}%
\index{tokens!suppressing parsing behaviour of}%
The default behaviour is to use parentheses, so that
% encode all the special characters below so that emacs font-locking has a reasonable chance of working
\begin{alltt}
   \holtm\lb\dol/\bs p\rb
\end{alltt}
would get printed as $(\land)\;\,p$.
Note that this doesn't reflect the default behaviour in the interactive loop, which is to use dollar-signs (as in the input above); see Section~\ref{sec:parser:architecture}.
However, with the \texttt{nodollarparens} option specified, nothing at all is printed to indicate that the special syntax has been ``escaped''.

\item[\texttt{nosp}] (Only for use with \holthm.)
%
By default, arguments to \holthm{} are fully specialised (\ie, they have \ml{SPEC\_ALL} applied to them), removing outermost universal quantifiers.
%
The \texttt{nosp} option prevents this.

\item[\texttt{nostile}] (Only for use with \holthm.)
%
By default, arguments to \holthm{} are printed with a turnstile~($\vdash$).
%
If this option is present, the turnstile is not printed (and the theorem will have its left margin three spaces further left).
%
For controlling how the turnstile is printed when this option is not present, see the paragraph on Overrides in Section~\ref{sec:running-munger}.

\item[\texttt{of}] (Only for use with \holty.)
%
The argument is a string that parses to a \emph{term}, not a type.
%
The behaviour is to print the type of this term.
%
Thus \texttt{\holty{}[of]\{p /\bs{} q\}} will print \texttt{bool}.

If the string includes right-braces, they must be escaped with
back-slashes, just as with the arguments to \holtm.

\item[\texttt{rule}] (Only for use with \holtm{} and \holthm.)
Prints a term (or a theorem's conclusion) using the \texttt{\bs{}infer} command (available as part of the \texttt{proof.sty} package).
This gives a nice, ``natural deduction'' presentation.
\index{natural deduction!presentation style for the LaTeX munger@presentation style for the \LaTeX{} munger}
For example, the term
\begin{alltt}
   (p \bs{}/ q) /\bs{} (p ==> r) /\bs{} (q ==> r) ==> r
\end{alltt}
will print as
\[
\infer{r}{p \lor q & p \Rightarrow r & q \Rightarrow r}
\]
Conjuncts to the left of the outermost implication (if any) will be split into hypotheses separated by whitespace.
For large rules, this style of presentation breaks down, as there may not be enough horizontal space on the page to fit in all the hypotheses.
In this situation, the \texttt{stackedrule} option is appropriate.

The term or theorem must be within a \LaTeX{} math-environment (it is typeset as if inline, with the \texttt{tt} option).

For adding a name to the rule, see the \texttt{rulename} option below.

\item[\texttt{rulename=}$\mathit{name}$] (Only has an effect with \texttt{rule} or \texttt{stackedrule}.)
Adds \textit{name} as the optional argument to the \texttt{\bs{}infer} command when typesetting the rule.
The name is wrapped with \texttt{\bs{}HOLRuleName}, which by default is the same as \texttt{\bs{}textsf}.
For ease of parsing options, \textit{name} should not contain braces, brackets, or commas.
(A name including such special characters could be typeset by renewing the \texttt{\bs{}HOLRuleName} command.)

\item[\texttt{showtypes}$n$] (For use with \holthm{} and \holtm.)
%
Causes the term or theorem to be printed with the \texttt{types} trace set to level~$n$.
The $n$ is optional and defaults to $1$ if omitted (equivalent to having the \ml{show\_types} reference set to \ml{true}).

\item[\texttt{stackedrule}] (For use with \holthm{} and \holtm.)
This is similar to the \texttt{rule} option, but causes implication hypotheses to be presented as a ``stack'', centered in a \LaTeX{} array on top of one another.
Thus,
\begin{alltt}
   (p \bs{}/ q) /\bs{} (p ==> r) /\bs{} (q ==> r) ==> r
\end{alltt}
will print as
\[
\infer{r}{\begin{array}{c}p \lor q \\ p \Rightarrow r \\ q \Rightarrow r\end{array}}
\]
For this purely propositional example with single-letter variable names, the result looks a little odd, but if the hypotheses are textually larger, this option is indispensable.

For adding a name to the rule, see the \texttt{rulename} option.

\item[\texttt{tr\apost}$\mathit{tracename}$\texttt{\apost=}$n$]
This option allows the temporary setting of the provided trace to the integer value $n$.
For example, one can set \texttt{pp\_unambiguous\_comprehensions} to $1$ to ensure that set comprehensions are printed with bound variables explicitly identified.
See Section~\ref{sec:set-syntax} for more on set comprehensions, and Section~\ref{sec:traces} for more on traces.
\index{traces, controlling HOL feedback@traces, controlling \HOL{} feedback!when munging to LaTeX@when munging to \LaTeX{}}

\item[\texttt{tt}] %
Causes the term to be type-set as the argument to a \LaTeX{} command \texttt{\bs{}HOLinline}.
%
The default definition for \texttt{\bs{}HOLinline} is
\begin{verbatim}
   \newcommand{\HOLinline}[1]{\mbox{\textup{\texttt{#1}}}}
\end{verbatim}
This makes the argument suitable for inclusion in standard \LaTeX{} positions.
%
This is the default for \holtm{} and \holty.
%
(The \texttt{\bs{}HOLinline} command is defined in the \texttt{holtexbasic.sty} style file.)

\item[\texttt{width=}$n$] Causes the argument to be typeset in lines of width $n$.
%
The default width is $63$, which seems to work well with 11pt fonts.
%
This default can also be changed at the time the \munge{} command is
run (see Section~\ref{sec:running-munger} below).

\item[\texttt{-}$\mathit{name}$]
%
\index{parsing, of HOL logic@parsing, of \HOL{} logic!overloading}
This option causes the printing of the term or theorem to be done with respect to a grammar that has all overloading for $\mathit{name}$ removed.
When used with \holty, prints the type with all type abbreviations for $\mathit{name}$ removed.
For example, the command \texttt{\holtm[-+]\lb{}x + y\rb} will print as
\[
\texttt{arithmetic\dol+}\;x\;y
\]
because the underlying constant will no longer map to the string \texttt{"+"} and, in the absence of any other mappings for it, will be printed as a fully qualified name.

\index{integers, the HOL theory of@integers, the \HOL{} theory of}
If the theory of integers is loaded, then the command \texttt{\holtm[-+]\lb{}x + y:int\rb} will print as \[
\texttt{int\_add}\;x\;y
\]
because the mapping from the integer addition constant to \texttt{"+"} is removed, but the mapping to \texttt{"int\_add"} remains, allowing that form to be what is printed.

The \texttt{-} option can be useful when complicated notation involving overloads is first introduced in a document.

\item[\texttt{\indentoption} and \texttt{\indentoption\td}] Indents the argument.
%
  These options only make sense when used with the \texttt{alltt} option (the additional spaces will have no effect when inside an \texttt{\bs{}mbox}).
%
  The default indentation is two spaces; if a different indentation is desired, the option can be followed by digits specifying the number of space characters desired.
%
  For example, \texttt{\holthm{}[\indentoption10,...]\{...\}} will indent by 10 spaces.

  Note that simply placing a command such as \holthm{} within its \texttt{alltt} block with a given indentation, for example
\begin{alltt}
   \bs{}begin\{alltt\}
      \holthm\{bool.AND_CLAUSES\}
   \bs{}end\{alltt\}
\end{alltt}
will not do the right thing if the output spans multiple lines.
%
Rather the first line of \HOL{} output will be indented, and the subsequent lines will not.
%
The \texttt{\indentoption} option lets the pretty-printer know that it is printing with a given indentation, affecting all lines of its output.

The version with the tilde character~(\td) does not add indentation to the first line of output, but adds the specified amount (again 2, if no number is provided) to subsequent lines.
This allows one to achieve suitable alignment when other non-HOL text has been put onto the same line.
For example,
\begin{alltt}
  AND_CLAUSES \bs{}HOLthm[width=46,>>~12]\lb{}bool.AND_CLAUSES\rb
  TRUTH       \bs{}HOLthm[>>~12]\lb{}bool.TRUTH\rb
  MAP         \bs{}HOLthm[>>~12,width=50]\lb{}list.MAP\rb
\end{alltt}
ensures correct vertical alignment when extra lines are printed, as they will be with the printing of \ml{bool.AND_CLAUSES} and \ml{list.MAP}.

\item[$\mathit{nm}_1\mathtt{/}\mathit{nm}_2$] (For use with \holtm{}
  and \holthm{}.)
%
Causes name $\mathit{nm}_1$ to be substituted for name $\mathit{nm}_2$ in the term or theorem.
%
This will rename both free and bound variables, wherever they occur throughout a term.
%
Because it uses instantiation, free variables in theorem hypotheses will get renamed, but bound variables in hypotheses are not affected.
%
(Hypotheses are not printed by default anyway of course.)

If $\mathit{nm}_1$ and $\mathit{nm}_2$ both begin with the colon character then they are parsed as types, and type instantiation is performed on the term or theorem argument instead of variable substitution.

\item[$s\mathtt{//}t$] (For use with \holtm{}, \holthm, and \holty) Causes \LaTeX{} string $s$ to be substituted for token $t$.
This allows one-off manipulation of the override map (see Section~\ref{sec:running-munger} below).
The difference between this operation and the ``normal substitution'' done with a single slash (as above) is that it happens as the \HOL{} entity is printed, whereas normal substitution happens before pretty-printing is done.
If printing depends on particular variable name choices, the ``last minute'' manipulations possible with this form of substitution may be preferable.
The width of the \LaTeX{} string is taken to be the width of the original token $t$.
\end{description}

\subsection{Math-mode munging}
\label{sec:math-mode-munging}
\index{munging (producing LaTeX from HOL)@munging (producing \LaTeX{} from \HOL{})!math mode}
\index{math mode (in the LaTeX munger)@math mode (in the \LaTeX{} munger)}

There are a few steps needed to make math-mode munging a relatively painless affair.
First, there are two \LaTeX{} macros from \texttt{holtexbasic.sty} that should probably be overridden:
\begin{description}
\item[\texttt{\bs{}HOLConst}] By default this will print names in typewriter font.
In math mode, this will probably look better in sans serif, suggesting
\begin{verbatim}
   \renewcommand{\HOLConst}[1]{\textsf{#1}}
\end{verbatim}
Depending on personal taste, the \texttt{\bs{}HOLKeyword} macro might be redefined similarly.
This macro is used for keywords such as \texttt{if}.
\item[\texttt{\bs{}HOLinline}] This macro, used to wrap standard \texttt{\holtm} arguments, puts text into typewriter font.
One possibility for its redefinition would be
\begin{verbatim}
   \renewcommand{\HOLinline}[1]{\ensuremath{#1}}
\end{verbatim}
Note that if the term being typeset causes the pretty-printer to break over multiple lines, \LaTeX{} will complain because of the appearance of \texttt{\bs\bs} commands.
If necessary, this can be avoided on a case-by-case basis by setting the \texttt{width} option to a larger than normal width.
\end{description}

When using math-mode munging, one also has to be aware of how larger pieces of text will appear.
In non-math-mode munging, material is usually put into \texttt{alltt} environments.
The recommended alternative for math-mode is to use the \texttt{\bs{}HOLmath} environment:
\begin{alltt}
   \textit{article text}

   \bs{}begin\lb{}HOLmath\rb
   \holthm\lb{}bool.AND_CLAUSES\rb
   \bs{}end\lb{}HOLmath\rb
\end{alltt}
This uses a standard \texttt{array} environment within a \texttt{displaymath}.

Occasionally, one will want to arrange blocks of \HOL{} material within a larger math context.
The \texttt{HOLarray} environment is a simple alias for a single-column left-aligned array that one can use in these situations.


\subsection{Creating a munger}
\label{sec:munger-creation}

\newcommand{\mkmunge}{\texttt{mkmunge.exe}}
\index{munging (producing LaTeX from HOL)@munging (producing \LaTeX{} from \HOL{})!creating a munger}
%
The \HOL{} distribution comes with a tool called \mkmunge.
%
This executable is used to create munge executables that behave as described in this section.
%
A typical invocation of \mkmunge{} is
\begin{alltt}
   \mkmunge \(\langle\mathit{thy}\sb{1}\rangle\)Theory ... \(\langle\mathit{thy}\sb{n}\rangle\)Theory
\end{alltt}
Each commandline argument to \mkmunge{} is the name of a \HOL{} object file, so in addition to theory files, one can also include special purpose SML such as \texttt{monadsyntax}.

The \mkmunge{} program can also take an optional \texttt{-o} argument that is used to specify the name of the output munger (the default is \munge).  For example
\begin{alltt}
   \mkmunge -o bagtexprocess bagTheory
\end{alltt}

The theories specified as arguments to \mkmunge{} determine what theorems are in scope for calls to \holthm, and also determine the grammars that will govern the parsing and printing of the \HOL{} types, terms and theorems.

Under Poly/ML, the \mkmunge{} executable also takes an optional \texttt{-b} option that can be used to specify a heap (see Section~\ref{sec:polyml-heaps}) to use as a base.
Doing so allows for the incorporation of many theories at once, and will be more efficient than loading the heap's theories separately on top of the default HOL heap.
The use of a base heap argument to \mkmunge{} doesn't affect the efficiency of the resulting munging tool.

Building and running a munger is typically something that one would want to do under the control of \holmake{}.
A miniature example of such a file is presented in Figure~\ref{fig:holmake-munger}.
See Section~\ref{Holmake} for more on \holmake{}'s capabilities.
\begin{figure}
\begin{alltt}
INCLUDES = dir1 dir2

all: paper.pdf
.PHONY: all

munge.exe: prettyPrintingTheory.uo
        $(HOLDIR)/bin/mkmunge.exe -o $@ $<

paper.pdf: paper.tex paper.bib
        latexmk -pdf $<

paper.tex: paper.htex overrides holtexbasic.sty munge.exe Holmakefile
        ./munge.exe -w60 -m\; overrides < $< > $@

holtexbasic.sty: $(dprot $(HOLDIR)/src/TeX/holtexbasic.sty)
        cp $< $@

EXTRA_CLEANS = holtexbasic.sty munge.exe paper.pdf paper.tex
\end{alltt}
\caption{%
A sample Holmakefile for building a paper that incorporates \HOL{} material, where the user-written material is in the file \texttt{paper.htex} and a bibliography file \texttt{paper.bib}.
This sample assumes that \texttt{prettyPrintingTheory} incorporates ancestor theories (presumably drawing from directories \texttt{dir1} and \texttt{dir2}), as well as paper-specific grammar changes (\eg, calls to \texttt{add\_rule}, \texttt{set\_fixity}).
As with all Holmakefiles, the indentation under each target (\texttt{munge.exe}, \texttt{paper.pdf}, \etc) must be achieved with a TAB character.
For more on the options passed to \texttt{munge.exe}, see Section~\ref{sec:running-munger} below.
%
}
\label{fig:holmake-munger}
\end{figure}

\subsection{Running a munger}
\label{sec:running-munger}

\index{munging (producing LaTeX from HOL)@munging (producing \LaTeX{} from \HOL{})!running a munger}
Once created, a munger can be run as a filter command, consuming its
standard input, and writing to standard output.
%
It may also write error messages and warnings to its standard error.

Thus, a standard pattern of use is something like
\begin{alltt}
   \munge < article.htex > article.tex
\end{alltt}

However, there are a number of ways of further modifying the behaviour of the munger with command-line options.

\paragraph{Overrides}
Most importantly, one can specify an ``overrides file'' to provide
token-to-\LaTeX{} replacements of what is pretty-printed.
%
The command-line would then look like
\begin{alltt}
   \munge overrides_file < article.htex > article.tex
\end{alltt}
The overrides file is a text file containing lines of the form
\begin{alltt}
   tok width tex
\end{alltt}
where \texttt{tok} is a \HOL{} token, \texttt{width} is a number
giving the width of the \LaTeX{}, and \texttt{tex} is a \LaTeX{}
string.

As a very simple example, an overrides file might consist of just one
line:
\begin{alltt}
   pi1 2 \bs{}ensuremath\{\bs{}pi_1\}
\end{alltt}
This would cause the string \texttt{pi1} (presumably occurring in the
various \HOL{} entities as a variable name) to be replaced with the
rather prettier $\pi_1$.
%
The \texttt{2} records the fact that the
printer should record the provided \LaTeX{} as being 2 characters
wide.
%
This is important for the generation of reasonable line-breaks.

Overrides for \HOL{} tokens can also be provided within \HOL{}
theories, using the
\texttt{TeX\_notation} command (see
Section~\ref{sec:holtheories-tex-ready} below).

By overriding the special token \texttt{\$Turnstile\$}, one can control the printing of the turnstile produced by \holthm{}.
The default setup is roughly equivalent to overriding \texttt{\$Turnstile\$} to \texttt{\textbackslash{}HOLTokenTurnstile\{\}} followed by a space, giving a total width of \texttt{3}.
Overriding the turnstile in this way will probably be necessary in math-mode printing, where the turnstile character is typically of the same width as 5 \texttt{\bs;} invocations.
Providing the correct width is important in order to get lines past the first to line up with the left edge of the mathematical text rather than the turnstile.

\paragraph{Default width}
A munger can specify the default width in which \HOL{} will print its
output with a \texttt{-w} option.
%
For example,
\begin{alltt}
   \munge -w70 < article.htex > article.tex
\end{alltt}
This default width can be overridden on a case-by-case basis with the
\texttt{width=} option to any of the commands within a \LaTeX{}
document.

\paragraph{Preventing Merge Analysis}  As mentioned above in the description of the \texttt{merge} and \texttt{nomerge} options to the \holtm{} and \holthm{} commands, the munger can be configured to not do token-merging avoidance by passing the \texttt{--nomergeanalysis} option to the munger.

\smallskip \noindent The \texttt{-w}, \texttt{--nomergeanalysis} and
overrides file options can be given in any order.

\paragraph{Setting Math-mode Spacing} If one expects to include all of the various \texttt{\bs{}HOL} commands in \LaTeX{} math contexts (as described above), then the \texttt{-m} option both sets the default width for math-mode spaces, and also
enables math-mode typesetting by default.

The specification of spacing is with a string of characters, as already described.
Note that if the command-line option includes any semi-colons or exclamation marks (\eg, \texttt{-mc;}), then they need to be quoted to prevent the shell from getting confused.
If the \texttt{-m} option appears without any additional characters, the default math-mode spacing will be \texttt{\bs;\bs;}.


\subsection{Holindex}

\index{munging (producing LaTeX from HOL)@munging (producing \LaTeX{} from \HOL{})!Holindex}
Till now, it has been explained how the munger can be used as a preprocessor of \LaTeX{} sources.
Sometimes a tighter interaction with \LaTeX{} is beneficial.
Holindex is a \LaTeX{} package that provides genuine \LaTeX{} commands for inserting \HOL{}-theorems, types and terms as well as many related commands.
This allows it to generate an index of all \HOL{}-theorems, types and terms that occur in the document as well as providing citation commands for \HOL{} entities in this index.
Holindex can be found in \texttt{src/TeX/}.
There is also a demonstration file available in this directory.


\paragraph{Using Holindex}
To use Holindex add \texttt{\bs{}usepackage\{holindex\}} to the header
of the \LaTeX{} source file \texttt{article.tex}. Holindex loads the
\texttt{underscore} package which might cause trouble with references
and citations. In order to avoid problems,
\texttt{holindex} should be included after packages
like \texttt{natbib}. Holindex is used like BibTex or
MakeIndex. A run of \LaTeX{} on \texttt{jobname.tex} creates an
auxiliary file called \texttt{article.hix}. The munger is used to
process this file via
\begin{alltt}
   \munge -index article
\end{alltt}
This call generates two additional auxiliary files,
\texttt{article.tde} and \texttt{article.tid}.  The following runs of
\LaTeX{} use these files. After modifying the source file, the munger
can be rerun to update \texttt{article.tde} and \texttt{article.tid}.
If you are using emacs with AUCTeX to write your latex files, you might
want to add
\begin{verbatim}
(eval-after-load "tex" '(add-to-list 'TeX-command-list
   '("Holindex" "munge.exe -index %s"
     TeX-run-background t t :help "Run Holindex") t))
\end{verbatim}
to your emacs configuration file. This will allow
you to run Holindex using AUCTeX.

\paragraph{Holindex commands}
\begin{description}
\item[\texttt{\bs{}blockHOLthm\{id\}}, \texttt{\bs{}blockHOLtm\{id\}}, \texttt{
    \bs{}blockHOLty\{id\}}] These commands typeset the theorem, term
  or type with the given \texttt{id} as the argument to a \LaTeX{}
  command \texttt{\bs{}HOLblock}. They are intended for
  typesetting multiple lines in a new block.
  For theorem ids of the form \texttt{theory.thm} are predefined. All
  other ids have to be defined before usage as explained below.
%
\item[\texttt{\bs{}inlineHOLthm\{id\}}, \texttt{\bs{}inlineHOLtm\{id\}}, \texttt{\bs{}inlineHOLty\{id\}}]
   These commands are similar to \texttt{\bs{}blockHOLthm\{id\}}, \texttt{\bs{}blockHOLtm\{id\}} and \texttt{
    \bs{}blockHOLty\{id\}}. However, they are intended for inline typesetting and
   use \texttt{\bs{}HOLinline} instead of \texttt{\bs{}HOLblock}.
%
\item[\texttt{\bs{}citeHOLthm\{id\}}, \texttt{\bs{}citeHOLtm\{id\}}, \texttt{\bs{}citeHOLty\{id\}}]
   These commands cite a theorem, term or type.
%
\item[\texttt{\bs{}mciteHOLthm\{id,id,...id\}}, \texttt{\bs{}mciteHOLtm\{ids\}}, \texttt{\bs{}mciteHOLty\{ids\}}]
   These commands cite multiple theorems, terms or types.
%
\item[\texttt{\bs{}citePureHOLthm\{id\}}, \texttt{\bs{}citePureHOLtm\{id\}},
   \texttt{\bs{}citePureHOLty\{id\}}] These commands\linebreak cite a theorems, terms or types.
   They just typeset the number instead of the
   verbose form used by the \texttt{citeHOL} and \texttt{mciteHOL} commands.

\item[\texttt{\bs{}citeHiddenHOLthm\{id\}},
   \texttt{\bs{}citeHiddenHOLtm\{id\}},
   \texttt{\bs{}citeHiddenHOLty\{id\}}] These commands cite a
   theorems, terms or types, but not typeset anything. These commands
   can be used to add a page to the list of pages a theorem, term or
   type is cited.

 \item[\texttt{\bs{}printHOLIndex}, \texttt{\bs{}printHOLShortIndex},
   \texttt{\bs{}printHOLLongIndex}] These commands typeset the index
   of all theorems, terms and types cited in the document.  There are
   two types of entries in the index: long and short ones. Short
   entries contain a unique number, the label of the theorem, term or
   type and the pages it is cited.  Long entries contain additionally
   a representation as it would be inserted by
   \texttt{\bs{}blockHOL...} as well as an optional description.
   Theorems use by default short entries, while terms and types use
   long ones.  It is possible to change for each item whether a long
   or short entry should be used. \texttt{\bs{}printHOLIndex} prints
   the default index with mixed long and short entries.
   \texttt{\bs{}printHOLLongIndex} typesets just long entries and
   \texttt{\bs{}printHOLShortIndex} just short ones.
\end{description}


\paragraph{Defining and formatting terms, types and theorems}

  Most of the Holindex commands require an identifier of a theorem,
  term or type as arguments. Theorem identifiers of the form
  \texttt{theory.theorem} are predefined. All other identifiers need
  defining. Additionally one might want to change the default
  formatting options for these new identifiers as well as the old ones.
  \HOL{} definition files can be used for defining and
  setting the formatting options of identifiers. They are used by
  putting the command \texttt{\bs{}useHOLfile\{\textit{filename.hdf}\}} in
  the header of your latex source file. These file use a syntax similar to
  BibTex. They consist of a list of entries of the form
  \begin{verbatim}
@EntryType{id,
  option = value,
  boolFlag,
  ...
}
\end{verbatim}
\noindent
There are the following entry types
\begin{description}
\item[\texttt{Thm}, \texttt{Theorem}] used to define and format a
  theorem. If the identifier is of the form \texttt{theory.theorem},
  the \texttt{content} option can be skipped. Otherwise, the
  \texttt{content} option should be of this form and a new identifier
  is defined for the given theorem. This is for example useful if the
  theorem name contains special characters or if a theorem should
  be printed with different formatting options.
  \item[\texttt{Term}]
    used to define and format a term.
  \item[\texttt{Type}]
    used to define and format a type.
  \item[\texttt{Thms}, \texttt{Theorems}] used to set formatting options for
    a list of theorems. For example one might want to print long index entries
    for all theorems in a specific theory. For the \texttt{Theorems} entry
    the \texttt{id} part of the entry is given in the form
    \texttt{ids = [id,id,...]}. These \texttt{ids} may be theorem ids or special
    ids of the form \texttt{theorem.thmprefix*}.
    For example, the id
    \texttt{arithmetic.LESS\_EQ*} represents all theorems in
    theory \texttt{arithmetic} whose name starts with \texttt{LESS\_EQ}.
\end{description}
Options are name/value pairs. The value has to be quoted using
quotation marks or \HOL{}'s quotation syntax. There are the following
option names available:
\begin{description}
  \item[\texttt{content}]
    the content. For a term or type that's its \HOL{}\ definition.
    For theorems it is of the form \texttt{theory.theorem}.
  \item[\texttt{options}]
    formatting options for the munger as described in section~\ref{sec:munging-commands}.
    Please use the Holindex commands for typesetting inline or as a block instead of the options \texttt{tt} or \texttt{alltt}.
  \item[\texttt{label}] the label that will appear in the index. For
    theorems the label is by default its name and the label given here
    will be added after the name.
  \item[\texttt{comment}] \LaTeX{} code that gets typeset as a comment / description
    for long index entries.
  \item[\texttt{latex}] the \LaTeX{} code for the item. There are very rare cases,
    when it might be useful to provide handwritten \LaTeX{} code instead of the one
    generated by the munger. This option overrides the \LaTeX{} produced by the munger.
    It is recommended to use it very carefully.
\end{description}
Besides options, there are also boolean flags that change the formatting of entries:
\begin{description}
  \item[\texttt{force-index}]
    adds the entry to the index, even if it is not cited in the document.
  \item[\texttt{long-index}]
    use a long index-entry.
  \item[\texttt{short-index}]
    use a long index-entry.
\end{description}
Here is an example of such a \HOL{} definition file:

\begin{verbatim}
@Term{term_id_1,
   content = ``SOME_FUN = SUC a < 0 /\ 0 > SUC b``,
   options = "width=20",
   label = "a short description of term from external file",
   comment = "some lengthy\\comment

              with \textbf{formats} and newlines",
   force_index
}

@Type{type_id_1,
   content = ``:bool``
}

@Thm{arithmetic.LESS_SUCC_EQ_COR,
   force-index, long-index
}

@Thm{thm_1,
   label = "(second instance)",
   content = "arithmetic.LESS_SUC_EQ_COR"
}

@Theorems{
   ids = [arithmetic.LESS_ADD_SUC,
          arithmetic.LESS_EQ*],
   force-index
}
\end{verbatim}


\paragraph{Configuring Holindex}

There are some commands that can be used to change the overall behaviour
of Holindex. They should be used in the header directly after \texttt{holindex}
is included.
\begin{description}
\item[\texttt{\bs{}setHOLlinewidth}] sets the default line-width. This
   corresponds to the \texttt{-w} option of the munger.

\item[\texttt{\bs{}setHOLoverrides}] sets the ``overrides file'' to provide
token-to-\LaTeX{} replacements of what is pretty-printed.

\item[\texttt{\bs{}useHOLfile}]
is used to include a \HOL{} definition file. Several such files might be
included.
\end{description}


\paragraph{Additional documentation}
For more information about Holindex, please refer to
the demonstration file \texttt{src/TeX/holindex-demo.tex}. This file contains
documentation for rarely used commands as well as explanations of how to
customise Holindex.


\subsection{\texorpdfstring{Making \HOL{} theories \LaTeX{}-ready}{Making HOL theories \LaTeX{}-ready}}
\label{sec:holtheories-tex-ready}

Though one might specify all one's desired token-replacements in an \texttt{overrides} file, there is also support for specifying token replacements in the theory where tokens are first ``defined''.
%
(Of course, \emph{tokens} aren't defined \textit{per se}, but the definition of particular constants will naturally give rise to the generation of corresponding tokens when those constants appear in HOL terms, types or theorems.)

A token's printing form is given in a script-file with the \ml{TeX\_notation} command (from the \ml{TexTokenMap} module).
%
This function has type
\begin{alltt}
   \{ hol : string, TeX : string * int \} -> unit
\end{alltt}
The \ml{hol} field specifies the string of the token as \HOL{} prints it.
%
The \ml{TeX} field specifies both the string that should be emitted into the \LaTeX{} output, and the width that this string should be considered to have (as in the \texttt{overrides} file).

For example, in \texttt{boolScript.sml}, there are calls:
\begin{alltt}
   val _ = TeX_notation \{ hol = "!", TeX = ("\bs{}\bs{}HOLTokenForall\{\}", 1)\}
   val _ = TeX_notation \{ hol = UChar.forall,
                          TeX = ("\bs{}\bs{}HOLTokenForall\{\}", 1)\}
\end{alltt}
The \texttt{UChar} structure is a local binding in the script-file that points at the standard list of UTF8-encoded Unicode strings in the distribution (\ml{UnicodeChars}).
%
Note also how the backslashes that are necessary for the \LaTeX{} command have to be doubled because they are appearing in an SML string.

Finally, rather than mapping the token directly to the string \texttt{\bs{}forall} as one might expect, the mapping introduces another level of indirection by mapping to \texttt{\bs{}HOLTokenForall}.
%
Bindings for this, and a number of other \LaTeX{} commands are made in the file
\begin{alltt}
   src/TeX/holtexbasic.sty
\end{alltt}
which will need to be included in the \LaTeX{} source file.
%
(Such bindings can be overridden with the use of the command \texttt{\bs{}renewcommand}.)

Finally, all theory-bindings made with \ml{TeX\_notation} can be overridden with \texttt{overrides} files referenced at the time a munger is run.


\index{LaTeX@\LaTeX!embedding HOL@embedding in \HOL{}|)}

%%% Local Variables:
%%% mode: latex
%%% TeX-master: "description"
%%% End:
