\documentclass{article}

\usepackage{holindex}
\initHOLindex
\setHOLlinewidth{80}  %default is 80


\begin{document}


\section{Holindex usage}

Similar usage as bibtex, makeindex etc.

\begin{enumerate}
\item create jobname.tex with header
\begin{verbatim}
\usepackage{holindex}
\initHOLindex
\setHOLlinewidth{80}  %optional default is 80
\end{verbatim}
\item run latex on jobname.tex to generate jobname.hix
\item run \texttt{munge -index jobname} to create jobname.tix
\item rerun latex to use jobname.tix
\item rerun the munger whenever some significant HOL stuff changed
\end{enumerate}

\section{Defining}
\begin{verbatim}
   \defHOLtm{unique_id}{label}{def}
   \defHOLty{unique_id}{label}{def}
   \defHOLthm{unique_id}{label}{def}

   (Theorems don't need defining, since there ID is 
    theory.name, there label is there name and the def
    is stored in the theory. You can use this definition anyone, when
    - the theorem name is too lengthy / contains special characters
    - you want to use one theorem with different formating options
    - you want to add a theorem several times to the index)

   Use the same formating options as the munger
   \formatHOLtm{unique_id}{options}
   \formatHOLty{unique_id}{options}
   \formatHOLthm{unique_id}{options}

   Combine definition and formating
   \formatDefHOLtm{unique_id}{label}{options}{def}
   \formatDefHOLty{unique_id}{label}{options}{def}

   Add it explicitly to the index (citations are added automatically)
   \indexHOLtm{unique_id}
   \indexHOLty{unique_id}
   \indexHOLthm{unique_id}

   Print long version in the index
   \longIndexHOLthm{unique_id}
   \longIndexHOLty{unique_id}
   \longIndexHOLtm{unique_id}
   \shortIndexHOLthm{unique_id}
   \shortIndexHOLty{unique_id}
   \shortIndexHOLtm{unique_id}

   Use boolean flags to determine the default
   \setboolean{holIndexLongTermFlag}{true}
   \setboolean{holIndexLongTypeFlag}{true}
   \setboolean{holIndexLongThmFlag}{false}

   Add comments in the index
   \commentHOLtm{unique_id}{comment}
   \commentHOLty{unique_id}{comment}
   \commentHOLthm{unique_id}{comment}

   Definition like these inside a latex file are tedious,
   especially for long terms, so do it
   in an external file and use that one (see test.hdf)
   \useHOLfile{filename}
\end{verbatim}

%use external file
\useHOLfile{test.hdf}

%or define inline (recommended only for short, simple ones)
\defHOLtm{term_id_1}{The first term}{SUC a > 0 /\ X > 2}
\defHOLtm{term_id_2}{The second term}{SUC a < 0 /\ X > 3}
\defHOLty{type_id_1}{The first type}{:bool}
\defHOLty{type_id_2}{The second type}{:num}
\formatDefHOLtm{term_width_5}{Test term width=5}{width=5}{SUC a > 0 /\ X > 2}
\formatDefHOLtm{term_width_10}{Test term width=10}{width=10}{SUC a > 0 /\ X > 2}
\formatDefHOLtm{term_width_50}{Test term width=50}{width=50}{SUC a > 0 /\ X > 2}



\section{Printing}

\begin{verbatim}
   Print inline
   \inlineHOLtm{id}
   \inlineHOLty{id}
   \inlineHOLthm{id}

   Print as block
   \blockHOLtm{id}
   \blockHOLty{id}
   \blockHOLthm{id}

   Use def without encapsulation (seldom useful)
   \useHOLtm{id}
   \useHOLty{id}
   \useHOLthm{id}
\end{verbatim}

\subsection{Block Examples}
\begin{itemize}
\item Example 1 \blockHOLtm{term_width_5}
\item Example 2 \blockHOLtm{term_width_10}
\item Example 3 \blockHOLtm{term_width_50}
\item Example 4 \blockHOLthm{arithmetic.DIVMOD_THM}
\end{itemize}


\subsection{Inline Examples}
\begin{itemize}
\item Example 1 \inlineHOLtm{term_width_5}
\item Example 2 \inlineHOLtm{term_width_10}
\item Example 3 \inlineHOLtm{term_width_50}
\item Example 4 \inlineHOLthm{arithmetic.DIVMOD_THM}
\end{itemize}



\section{Citing}

\begin{verbatim}
   Pure numbers
   \citePureHOLtm{id}
   \citePureHOLty{id}
   \citePureHOLthm{id}

   Single citations
   \citeHOLtm{id}
   \citeHOLty{id}
   \citeHOLthm{id}

   Multiple citations 
   \mciteHOLtm{id,id,...}
   \mciteHOLty{id,id,...}
   \mciteHOLthm{id,id,...}

   Hidden citations (adds page to index, 
      but does not print anything)
   \citeHiddenHOLtm{id}
   \citeHiddenHOLty{id}
   \citeHiddenHOLthm{id}


   Printing Index 
   \printHOLIndex
   \printShortHOLIndex 
   \printLongHOLIndex 

   Sort the index by occurence in the text, not by names
   \sortHOLIndexOccurence
\end{verbatim}

\citeHOLthm{arithmetic.LESS_SUC_EQ_COR}
\citeHOLtm{term_id_1}
\citeHOLty{type_id_2}

\mciteHOLthm{arithmetic.LESS_SUC_EQ_COR,prim_rec.INV_SUC_EQ,arithmetic.LESS_SUC_EQ_COR}

\pagebreak

\citeHOLthm{prim_rec.INV_SUC_EQ}
\citeHOLthm{arithmetic.PRE_SUC_EQ}

\pagebreak

\citeHOLthm{prim_rec.INV_SUC_EQ}
\citeHOLthm{arithmetic.PRE_SUC_EQ}

\pagebreak

\citeHOLthm{prim_rec.INV_SUC_EQ}

\pagebreak

\citeHOLthm{prim_rec.INV_SUC_EQ}
\citeHOLthm{arithmetic.PRE_SUC_EQ}


\longIndexHOLthm{arithmetic.PRE_SUC_EQ}
\shortIndexHOLty{type_id_2}
\commentHOLty{type_id_4}{Some short comment}

%Sort index by occurence
%\sortHOLIndexOccurence

\setboolean{holIndexLongTermFlag}{false}
\printLongHOLIndex

\pagebreak

\printHOLIndex

\pagebreak

\printShortHOLIndex


\end{document}